\makeatletter
\def\@fnsymbol#1{
     \ensuremath{
         \ifcase#1
\or ~
         \fi
     }
}
\renewcommand{\thefootnote}{\fnsymbol{footnote}}
\makeatother

\newcommand\scieloDefineFootnotes{
%\footnotetext[1]{\scieloArticleUrl}
%\footnotetext[1]{EDVC é Mestre em Urbanisme et Aménagement, e-mail: eugeniadoria@gmail.com}
%\footnotetext[2]{\\\textbf{Endereço para correspondência}\\Vanessa de Paula Tiago.\\Universidade Federal do Triângulo Mineiro. Departamento de Pediatria da Universidade Federal do Triângulo Mineiro. Av. Frei Paulino, no 30, Bairro Abadia\\Uberaba - MG. Brasil. CEP: 38025-180.}
}

\newcommand{\scieloYear}{2015}
\newcommand\scieloArticleType{ORIGINAL ARTICLE}
\newcommand{\scieloArticleRef}{2012;1(2):41-6}
\newcommand\scieloArticleTitle{Prevalence of oral and maxillofacial diseases diagnosed in an Oral Medicine service during a 7-year period}
\newcommand\scieloArticleTitleTranslation{Admittance of Risk-Classified Cases: Assessment of Hospital Emergency Services\\Acogida con Clasificación de Riesgo: Evaluación de Servicios Hospitalarios de Emergencia}
\newcommand{\scieloAuthor}{Simone de Macedo Amaral\textsuperscript{1}\\Águida Maria Menezes Aguiar Miranda\textsuperscript{2}\\Juliana de Noronha Santos Netto\textsuperscript{3}\\Fábio Ramôa Pires\textsuperscript{4}}
\newcommand\scieloContributorsFootNotes{\textsuperscript{1}DDS, MSc, Post-graduation Program in Dentistry, Estácio de Sá University, Rio de Janeiro, Brazil (Oral Medicine, Brazilian Dental Association, Rio de Janeiro, Brazil).\\\textsuperscript{2}DDS, MSc, Stomatology and Oral Surgery, Estácio de Sá University, Rio de Janeiro, Brazil (DDS, MSc, Stomatology and Oral Surgery, Estácio de Sá University, Rio de Janeiro, Brazil).\\\textsuperscript{3}DDS, MSc, Oral Medicine, Brazilian Dental Association, Rio de Janeiro, Brazil (DDS, MSc, Oral Medicine, Brazilian Dental Association, Rio de Janeiro, Brazil).\\\textsuperscript{4}DDS, PhD, Post-graduation Program in Dentistry, Estácio de Sá University, Rio de Janeiro, Brazil (DDS, PhD, Post-graduation Program in Dentistry, Estácio de Sá University, Rio de Janeiro, Brazil).}
\newcommand{\scieloFooterText}{1 (2) April/June 2012}
\newcommand{\scieloCorrespondence}{\scieloCorrespondenceContainer{Send correspondence to:}{Fábio Ramôa Pires.\\Estácio de Sá University.\\Av. Alfredo Baltazar da Silveira, no 580,3 rd floor, Recreio dos Bandeirantes. Rio deJaneiro - RJ. Brazil.\\Zip code: 22790-701.\\Phone:/Fax: +55 (21) 2497-8988.\\E-mail: ramoafop@yahoo.com\\FAPERJ, Rio de Janeiro, Brazil.}}
\newcommand{\scieloSubmitted}{\scieloSubmittedContainer{Paper submitted to the JORDI-SGP (Publishing Management System - Journal of Oral Diagnosis) on July 15, 2012; and accepted on August 27, 2012. cod. 32.}}
\newcommand{\scieloRenderAbstracts}{
\scieloAbstractContainer{ABSTRACT}{Prevalence of oral and maxillofacial diseases is highly variable depending on the region, country and source of
the data. The aim of this study was to determine the prevalence of oral and maxillofacial diseases diagnosed in
an Oral Medicine service during a 7-year period. All clinical charts from patients attending the service in the
period were reviewed to retrieve demographic and clinical data; diagnosis were classified in groups, distributed
in absolute and relative values and analyzed with respect to their frequency. A total of 1075 clinical charts were
reviewed and females represented 60\% of the sample. Mean age of the patients was 41 years (ranging from 1
month to 94 years old) and most patients were in their fourties to fifties. A total of 1444 oral diseases were
diagnosed and the most prevalent groups were soft-tissue tumors (184 cases, 12.7\%), developmental defects
(161, 11.2\%) and epithelial diseases (127, 8.8\%). Individually, the most frequent diseases included fibrous hyper-
plasias (120 cases, 8.3\%), candidosis (77, 5.3\%), periradicular inflammatory lesions (72, 5.0\%) and potentially
malignant disorders (52, 3.6\%). Oral carcinomas represented 2.5\% of the sample (36 cases). The present results
reflect the frequency of oral diseases diagnosed in a specialized service in southeastern Brazil and will allow the
establishment of preventive strategies and the adequacy of the clinical services offered to the target population.}{Keywords:}{disease; epidemiology; mouth; oral medicine.}

\scieloAbstractContainer{Abstract}{\textbf{Background}: Water-specific 1470-nm lasers enable vein ablation at lower energy densities and with fewer side effects
because they target interstitial water in the vessel wall. \textbf{Conclusions}: The data from this study support the possibility that the incidence of complications can be
reduced without significantly affecting the clinical outcomes, by using lower energy density. However, this appears to
be at the cost of reduced efficacy in terms of GSV occlusion rates.}{Keywords:}{agend, bone desnsity, accidental falls; rehabilitation; mental recall}

\scieloAbstractContainer{Resumo}{\textbf{Contexto}: O laser de diodo 1470 nm, com comprimento de onda específico para água, tendo como alvo a água
intersticial da parede venosa, poderia causar ablação venosa a densidades de energia menores e com menos efeitos
colaterais. \textbf{Objetivos}: Determinar a taxa de obliteração da veia safena magna (VSM) após termoablação com laser
1470 nm utilizando 7 W de potência e avaliar a evolução clínica e as complicações. \textbf{Métodos}: Dezenove pacientes
(31 VSMs) foram submetidos a termoablação e reexaminados através de ecodoppler, avaliação clínica utilizando
o Venous Clinical Severity Score (VCSS) e avaliação das complicações do procedimento entre 3 e 5 dias e aos
30 e 180 dias de pós-operatório. \textbf{Resultados}: A média de idade dos pacientes foi de 46 anos; 17 eram mulheres
(89,47\%). De acordo com a classificação de Clinical-Etiology-Anatomy-Physiopathology (CEAP), 2 dos 31 membros
tratados eram C2, 19 eram C3, 9 eram C4 e 1 membro era C5. A densidade Cronos de
 energia de 33,53 J/cm. A taxa de obliteração da VSM foi de 93,5\% no pós-operatório imediato, de 100\% entre 3 e 5 dias e aos
30 dias, e de 87,1\% aos 180 dias. Houve uma redução significativa dos valores de VCSS em todos os momentos de
avaliação. \textbf{Conclusões}: Os dados deste estudo apoiam a possibilidade de que, utilizando baixa densidade de energia,
podemos reduzir a incidência de complicações sem afetar significativamente o resultado clínico. No entanto, isso
parece ocorrer às custas da diminuição da eficácia em termos de taxa de obliteração da VSM.}{Palavras-chave:}{técnicas de ablação; terapia a laser; varizes.}
}