\begin{multicols}{2}

\scieloSectionContainer{INTRODUCTION}
\par{}Studies reporting on the prevalence of oral diseases
in different countries from all continents have been reported
in the literature in a regular basis in the last years 1-6 . They are
essential in highlighting epidemiologic, geographic and racial
biases and similarities among different populations, offering
the possibility of comparing health service needs in different
countries and regions. In large countries, such as Brazil, even
local economic, cultural and social differences, for example, can
affect the prevalence of oral diseases in specific regions.
\par{}Most reports have focused on the prevalence of oral and
maxillofacial diseases in specific groups, including only soft-
-tissue lesions, lesions from a specific anatomic location, lesions
from a specific age group, lesions diagnosed in oral and ma-
xillofacial surgical pathology laboratories, lesions diagnosed in
convenience samples and lesions diagnosed in population-based
studies 1-6 . Although all studies have their specific and individual
importance when studying the epidemiology of oral diseases,
they do not reflect the real incidence of oral lesions in a regular
Stomatology service. For this reason, the aim of this study was to
report the prevalence of oral and maxillofacial diseases diagnosed
in consecutive patients attending on an Oral Medicine service
during a 7-year period.

\scieloSectionContainer{MATERIALS AND METHODS}

\par{}This is a descriptive cross-sectional retrospective study
based on data from the clinical records of consecutive patients
attending the Stomatology clinic, Dental School, Estácio de
Sá University, Rio de Janeiro, Brazil, from january 2003 to
december 2008. Demographic (gender and age) and clinical
(anatomical location of lesions, biopsies or other diagnostic
procedures and final diagnosis) data were retrieved from the
clinical charts from each individual patient. Patients whose
clinical charts were not found at the time of the review were
excluded from the study.

\lipsum

\scieloSectionContainer{RESULTS}

\scieloSubSectionContainer{SAMPLING AND ANALYSIS}

\par{}Surface water samples were taken using a Van
Dorn bottle and then stored in 500 to 1000 ml glass
vials and fixed with a Lugol solution (THRONDSEN,
1978). Sampling was carried out fortnightly (36.

\lipsum
\scieloSectionContainer{RESULTS}

\scieloSubSectionContainer{SPECIFIC COMPOSITION, RICHNESS AND ABUNDANCE}

\par{}A total of 110 taxa were identified, belonging to
4 Divisions (Figure 2; Table 1): Diatoms (55 taxa;
25 species), Dinoflagellates (51 taxa; 25 species),
Cyanobacteria (2 taxa) and Chlorophythes (2 taxa).
\par{}Richness, Abundance, Diversity Index and Evenness
results are presented as the mean values of each sampling
period, as the non-parametric tests revealed no differences
(p < 0.05) between sampling stations. Table 2 presents
information on these variables in each sampling period.


\lipsum

\scieloImageContainerOneCol{0.5\textwidth}{img1.jpg}{meu titulo}

%\begin{center}
%{
%\vspace{8mm}
%\centerline{
%\includegraphics[width=\maxwidth{0.5\textwidth}]{img1.jpg}
%}
%\vspace{8mm}
%}
%\end{center}

\lipsum

\scieloImageContainerTwoCol{img2.jpg}{\textbf{Figure 1:} A specimen of each species was embedded in paraffin in its entirety, longitudinally sectioned}

\lipsum
\lipsum

\scieloImageContainerTwoCol{img2.jpg}{\textbf{Figure 1:} A specimen of each species was embedded in paraffin in its entirety, longitudinally sectioned}

\lipsum
\lipsum
\lipsum[1]

\begin{scieloReferencesContainer}[REFERENCES]

\scieloReferencesItem{García-Pola Vallejo MJ, Martínez Díaz-Canel AI, García Martín JM,
González García M. Risk factors for oral soft tissue lesions in an adult Spa-
nish population. Community Dent Oral Epidemiol. 2002;30(4):277-85.
DOI: http://dx.doi.org/10.1034/j.1600-0528.2002.00048.x}

\scieloReferencesItem{Shulman JD, Beach MM, Rivera-Hidalgo F. The prevalence of oral
mucosal lesions in U.S. adults: Data from the Third National Health
and Nutrition Examination Survey, 1988-1994. J Am Dent Assoc.
2004;135(9):1279-86. PMID: 15493392}

\scieloReferencesItem{Mumcu G, Cimilli H, Sur H, Hayran O, Atalay T. Prevalence and
distribution of oral lesions: a cross-sectional study in Turkey. Oral
Dis. 2005;11(2):81-7. DOI: http://dx.doi.org/10.1111/j.1601-
0825.2004.01062.x}

\scieloReferencesItem{Parlak AH, Koybasi S, Yavuz T, Yesildal N, Anul H, Aydogan I, et
al. Prevalence of oral lesions in 13- to 16-year-old students in
Duzce, Turkey. Oral Dis. 2006;12(6):553-8. DOI: http://dx.doi.
org/10.1111/j.1601-0825.2006.01235.x}

\scieloReferencesItem{Pentenero M, Broccoletti R, Carbone M, Conrotto D, Gandolfo S. The
prevalence of oral mucosal lesions in adults from the Turin area. Oral
Dis. 2008;14(4):356-66. DOI: http://dx.doi.org/10.1111/j.1601-
0825.2007.01391.x}

\scieloReferencesItem{Cebeci AR, Gül ş ahi A, Kamburoglu K, Orhan BK, Ozta ş B. Prevalence
and distribution of oral mucosal lesions in an adult Turkish population.
Med Oral Patol Oral Cir Bucal 2009;14(6):E272-7.}

\scieloReferencesItem{Neville BW, Damm DD, Allen CM, Bouquot JE. Oral and Maxillofacial
Pathology. 3rd ed. St. Louis: Saunders Elsevier; 2009.}

\scieloReferencesItem{Carranza S, Baguñà J and Riutort M (1997)
Are the Platyhelminthes a monophyletic primitive
group? An assessment using 18S rDNA
sequences. Mol Biol Evol 14:485-497.}

\scieloReferencesItem{Cebrià F (2008) Organization of the nervous system in the model
planarian Schmidtea mediterranea: An immunocytochemical study. Neurosci Res 61:375-384.}

\scieloReferencesItem{Chien P and Koopowitz H (1972) The ultrastructure of neuromuscular systems in Notoplana acticola, a free-living polyclad flatworm. Z Zellforsch Mikroskop Anat 133:277-288.}


\scieloReferencesItem{Chien PK and Koopowitz H (1977) Ultrastructure of nerve plexus
in flatworms. III. The infra-epithelial nervous system. Cell
Tissue Res 176:335-347.}

\scieloReferencesItem{Day TA, Maule AG, Shaw C and Pax RA (1997) Structure-activity relationships of FMRFamide-related peptides contracting Schistosoma mansoni muscle. Peptides 18:917-921.}

\scieloReferencesItem{Egger B, Gschwentner R and Rieger R (2007) Free-living
flatworms under the knife: Past and present. Dev Genes Evol
217:89-104.}

\scieloReferencesItem{Egger B, Lapraz F, Tomiczek B, Müller S, Dessimoz C, Girstmair
J, Skunca N, Rawlinson KA, Cameron CB, Beli E et al.
(2015) A transcriptomic-phylogenomic analysis of the evolutionary relationships of flatworms. Curr Biol 25:1-7.}

\scieloReferencesItem{Ehlers U (1985) Das Phylogenetische System der Plathelminthes.
Gustav Fischer Verlag, Stuttgart, 317 pp.}

\scieloReferencesItem{Fernandes MC, Alvares EP, Gama P and Silveira M (2003) Serotonin in the nervous system of the head region of the land planarian Bipalium kewense. Tissue Cell 35:479-486.}

\scieloReferencesItem{Forest DL and Lindsay SM (2008) Observations of serotonin and
FMRFamide-like immunoreactivity in palp sensory structures and the anterior nervous system of spionid polychaetes. J Morphol 269:544-551.}

\scieloReferencesItem{Girstmair J, Schnegg R, Telford MJ and Egger B (2014) Cellular
dynamics during regeneration of the flatworm Monocelis sp. (Proseriata, Platyhelminthes). Evo Devo 5:e37.}

\scieloReferencesItem{Golding DM (1992) Polychaeta: Nervous system. In: Harrison
FW and Gardiner SL (eds) Microscopic Anatomy of Invertebrates. Vol. 7. Wiley-Liss, New York, pp 155-179.}

\scieloReferencesItem{Gustafsson MKS, Halton DW, Kreshchenko ND, Movsessian SO,
Raikova OI, Reuter M and Terenina NB (2002) Neuropeptides in flatworms. Peptides 23:2053-2061.}

\scieloReferencesItem{Hadenfeldt D (1929) Das Nervensystem von Stylochoplana
maculata und Notoplana atomata. Z wiss Zool 133:586638.}

\scieloReferencesItem{Halton DW and Gustafsson MKS (1996) Functional morphology
of the platyhelminth nervous system. Parasitology
113:S47-S72.}

\end{scieloReferencesContainer}

\scieloLicenseContainer{License information: This is an open-access article distributed under the terms of the Creative Commons Attribution License, which permits unrestricted use, distribution, and reproduction in any medium, provided the original work is properly cited.}



\end{multicols}

