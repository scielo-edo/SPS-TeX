% Generated by jats2tex@0.10.2.0:8927a9cc6d14b7897665c959b31cc7f822741feb
\documentclass[numberinsection,times,10pt,spreadimages]{memoir}
\usepackage[jvb]{scielo}

\makeatletter
\def\@fnsymbol#1{
     \ensuremath{
         \ifcase#1
\or 1
\or ~
\or ~
\or ~
         \fi
     }
}
\renewcommand{\thefootnote}{\fnsymbol{footnote}}
\makeatother

\newcommand\scieloDefineFootnotes{
\footnotetext[1]{Universidade Federal do Paraná – UFPR, Department of Surgery, Curitiba, PR, Brazil.}
\footnotetext[2]{Financial support: None.}
\footnotetext[3]{Conflicts of interest: No conflicts of interest declared concerning the publication of this article. }
\footnotetext[4]{Submitted: June 05, 2015. Accepted: June 26, 2015.}
}

\newcommand{\scieloYear}{2015}
\newcommand\scieloArticlePubIdType{DOI}
\newcommand\scieloArticleType{ORIGINAL ARTICLE}
\newcommand{\scieloArticleRef}{2015 Oct.-Dec.; 14(4):281-288}
\newcommand\scieloArticleTitle{Evaluation of great saphenous vein occlusion rate and clinical outcome in patients undergoing laser thermal ablation with a 1470-nm bare fiber laser with low linear endovenous energy density }
\newcommand\scieloArticleTitleTranslation{Avaliação da taxa de obliteração da veia safena magna e da evolução clínica de pacientes submetidos a termoablação com lase 1470 nm, fibra linear e baixa densidade de energia endovenosa linear\\Título traduzido 2}
\newcommand{\scieloArticleUrl}{http://dx.doi.org/10.1590/1677-5449.004015}
\newcommand{\scieloHeader}{Walter Junior Boim Araujo, Jorge Rufino Ribas Timi et al.}
\newcommand{\scieloAuthor}{Walter Junior Boim Araujo\scieloContributorFn{1}{xx}{1}{}, Jorge Rufino Ribas Timi\scieloContributorFn{1}{xx}{1}{}, Carlos Seme Nejm Júnior\scieloContributorFn{1}{xx}{1}{}, Filipe Carlos CaronCronos}



\begin{document}


\title{The clinical importance of air plethysmography in the assessment of
chronic venous disease}


% \author[\textsuperscript{th}
% ]{Dezotti, Nei Rodrigues Alves
% \textsuperscript{th}
% }
% \author[\textsuperscript{th}
% \textsuperscript{th}
% ]{Dalio, Marcelo Bellini
% \textsuperscript{th}
% \textsuperscript{th}
% }
% \author[\textsuperscript{th}
% ]{Ribeiro, Maurício Serra
% \textsuperscript{th}
% }
% \author[\textsuperscript{th}
% ]{Piccinato, Carlos Eli
% \textsuperscript{th}
% }
% \author[\textsuperscript{th}
% ]{Joviliano, Edwaldo Edner
% \textsuperscript{th}
% }\affil[aff01]{Universidade de São PauloFaculdade de Medicina de Ribeirão
% PretoUniversidade de São PauloUniversidade de São Paulo – USP, Faculdade
% de Medicina de Ribeirão Preto, Departamento de Cirurgia e Anatomia, Divisão
% de Cirurgia Vascular e Endovascular, Ribeirão Preto, SP,
% Brazil.}


\maketitle

\begin{multicols}{2}

Jornal Vascular BrasileiroJ. Vasc. Bras.1677-54491677-7301Sociedade
Brasileira de Angiologia e de Cirurgia Vascular
(SBACV)jvbAR2016002110.1590/1677-5449.00211601101ArticlesImportância clínica da
pletismografia a ar na avaliação da doença
venosa crônica
Conflicts of interest: No conflicts of interest declared concerning the
publication of this article.
*Correspondence Marcelo Bellini Dalio Departamento de Cirurgia e
Anatomia HCRP, Campus Universitário Av. Bandeirantes, 3900, 9º andar CEP
14049-900 - Ribeirão Preto (SP), Brazil Tel.: +55 (11) 8111-7424 E-mail:
mbdalio@usp.br
Author information NRAD and MBD - MD, PhD, and assistant physician at Divisão
de Cirurgia Vascular e Endovascular, Departamento de Cirurgia e Anatomia,
Faculdade de Medicina de Ribeirão Preto, Universidade de São Paulo (USP).
MSR - MD, PhD, and associate professor at Divisão de Cirurgia Vascular e
Endovascular, Departamento de Cirurgia e Anatomia, Faculdade de Medicina de
Ribeirão Preto, Universidade de São Paulo (USP). CEP - MD, PhD, and full
professor at Divisão de Cirurgia Vascular e Endovascular, Departamento de
Cirurgia e Anatomia, Faculdade de Medicina de Ribeirão Preto, Universidade
de São Paulo (USP). EEJ - MD, PhD, and associate professor and head of
Divisão de Cirurgia Vascular e Endovascular, Departamento de Cirurgia e
Anatomia, Faculdade de Medicina de Ribeirão Preto, Universidade de São Paulo
(USP).

Author contributions Conception and design: NRAD Analysis and interpretation:
NRAD, MBD, MSR Data collection: NRAD Writing the article: NRAD, MBD Critical
revision of the article: CEP, EEJ Final approval of the article*: NRAD, MBD,
MSR, CEP, EEJ Statistical analysis: NRAD Overall responsibility: NRAD *All
authors have read and approved of the final version of the article submitted
to J Vasc Bras.
\date{2017}{01}{05}
\newcommand{\volume}{00}
00\newcommand{\fpage}{000}
\newcommand{\lpage}{000}
2804201624082016This is an Open Access article distributed under the terms of
the
Creative Commons Attribution License, which permits unrestricted use,
distribution, and reproduction in any medium, provided the original work is
properly cited.\abstract{Abstract
Air plethysmography is a non-invasive test that can quantify venous reflux and
obstruction by measuring volume changes in the leg. Its findings correlate with
clinical and hemodynamic measures. It can quantitatively assess several
components of venous hemodynamics: valvular reflux, calf muscle pump function,
and venous obstruction. Although clinical uses of air plethysmography have been
validated, it is used almost exclusively for medical research. Air
plethysmography can be used to assess chronic venous disease, to evaluate
improvement after venous surgery, to diagnose acute and past episodes of deep
venous thrombosis, to evaluate compression stocking therapy, to study the
physiological implications of high-heeled shoes in healthy women, and even to
evaluate the probability of ulcer healing.
}
Resumo
A pletismografia a ar é um método não invasivo que pode quantificar refluxo e
obstrução venosa medindo alterações no volume das pernas. Seus achados se
correlacionam com parâmetros clínicos e hemodinâmicos. Ela pode fornecer
informações quantitativas dos diferentes componentes da hemodinâmica venosa:
refluxo valvular, função de bomba muscular da panturrilha e obstrução venosa.
Apesar de ter seu uso clínico validado, a pletismografia a ar é usada quase que
exclusivamente para pesquisa. Ela pode ser usada para avaliar a doença venosa
crônica, mensurar o ganho hemodinâmico após cirurgia venosa, diagnosticar
trombose venosa profunda atual ou prévia, avaliar os efeitos da
elastocompressão, estudar as implicações fisiológicas do uso de salto alto em
mulheres e também avaliar a probabilidade de cura de uma úlcera venosa.
Keywords: \textit{air plethysmography}
\textit{chronic venous disease}
\textit{varicose veins}
\textit{venous thrombosis}
\textit{leg ulcer}
Palavras-chave: \textit{pletismografia a ar}
\textit{doença venosa crônica}
\textit{varizes}
\textit{trombose venosa}
\textit{úlcera de perna}
\scieloSectionContainer{INTRODUCTION}

Chronic venous disease (CVD) includes a spectrum of clinical presentations
ranging
from uncomplicated telangiectasis and varicose veins to venous
ulceration.\textsuperscript{th}
It represents an important public
health problem with economic and social consequences.\textsuperscript{th}
\textsuperscript{th}
\textsuperscript{th}
Prevalence is about 20 to 73\% in females and 15 to 56\%
in males.\textsuperscript{th}
The combination of
skin abnormalities and sustained venous hypertension is referred to as chronic
venous insufficiency (CVI). Manifestations of CVD are the result of outflow
reflux,
obstruction, or a combination of both. These processes may be primary or
secondary
to other conditions, such as deep venous thrombosis (DVT).\textsuperscript{th}

Duplex scanning and venography can provide the anatomic and physiologic
information
necessary to diagnose and treat venous reflux and/or obstruction in the
superficial,
deep, and perforator systems.\textsuperscript{th}

Air plethysmography (APG) has been introduced as an additional tool for the
evaluation of venous hemodynamics.\textsuperscript{th}
\textsuperscript{th}
\textsuperscript{th}
APG is a non-invasive test that can quantify venous reflux
and obstruction by measuring volume changes in the leg.\textsuperscript{th}
Its findings correlate with clinical and
hemodynamic measures.\textsuperscript{th}

Although clinical uses of APG have been validated, it is almost exclusively used
for
medical research. The purpose of this review is to discuss the applications of
APG
for clinical assessment of chronic venous disease.

\scieloSectionContainer{STANDARD AIR PLETHYSMOGRAPHY TECHNIQUE}

Christopoulos et al.\textsuperscript{th}
have
described validation and the reproducibility and results of APG in detail, both
in
normal volunteers and in patients with superficial or deep venous disease. In
order
to evaluate venous reflux, APG is performed with a 35 cm-long polyvinyl chloride
air
chamber (5 L capacity) that surrounds the leg from knee to ankle and is
connected to
a pressure transducer and chart recorder. The pressure transducer is calibrated
with
100 mL of air after the air chamber is fitted around the leg, with the patient
supine. The leg is elevated to 45 degrees to empty the veins and a baseline
reading
is obtained (Figure 1A). The patient is then
asked to stand, putting body weight on the opposite leg. The increase in leg
volume
is observed until a plateau is reached, indicating that the veins are full
(Figure 1B). This plateau corresponds to the
functional venous volume (VV). The time taken to achieve 90 percent of venous
volume
has been defined as venous filling time 90 (VFT 90). The venous filling index
(VFI)
is calculated by dividing 90\% of VV by VFT90. The patient is then asked to
perform a
single heel-raise maneuver. The resultant decrease recorded is the ejected
volume
(EV), caused by contraction of the calf muscle (Figure 1C). The ejection
fraction (EF) is calculated by dividing the
ejection volume by the venous volume and multiplying by 100. After a new plateau
is
reached, 10 heel-raise maneuvers are performed to reach another plateau,
representing residual volume (Figure 1D). The
residual volume (RV) is the volume at the end of exercise. Finally, the patient
is
asked to remain standing (Figure 1E). The
residual volume fraction (RVF) is calculated by dividing the RV by the VV and
multiplying by 100. The VFI is an index of global venous reflux, the EF is a
reflection of calf muscle pump function, and the RVF is a reflection of
ambulatory
venous pressure.\textsuperscript{th}
\textsuperscript{th}
\textsuperscript{th}

In order to evaluate venous obstruction, APG is performed using an 11 cm
pneumatic
cuff with a 40 cm long bladder placed around the proximal thigh to act as a
partially occluding tourniquet. After calibration, with the patient in supine
position, the tourniquet is inflated to 70 mmHg and maintained until a maximal
venous volume is reached and the chart recorder traces a plateau. Upon rapid
deflation of the tourniquet, venous outflow is recorded. The outflow fraction
(OF)
is obtained by dividing the amount of venous volume emptied in 1 second by the
venous volume and multiplying by 100. OF values lower than 38\% suggest venous
obstruction.\textsuperscript{th}

\scieloSectionContainer{APPLICATIONS OF AIR PLETHYSMOGRAPHY IN THE CLINICAL
ASSESSMENT OF CHRONIC VENOUS
DISEASE}

Some important applications of APG in evaluation of lower extremity chronic
venous
disease and venous function are discussed below:

\scieloSectionContainer{DIFFICULTIES WITH USING AIR PLETHYSMOGRAPHY IN CLINICAL
SETTINGS}

Although it is a noninvasive and relatively inexpensive method, APG can be
technically difficult. It is highly dependent on accurate calibration and
because of
this it can be considered an examiner-dependent test. Minimal technical errors
during measurements invalidate the test and make it obligatory to repeat the
test
from the start. Calibration and frequent restarting mean that APG is often a
time-consuming procedure. Obesity can also influence results, making APG
parameters
inappropriate.

As stated above, APG is currently almost exclusively used for medical research.
Despite its low cost and non-invasive nature, APG is rarely found in clinical
vascular laboratories. In Brazil, this fact could be partly explained by the
absence
of a domestic APG device manufacturer. Until recently, there was an APG device
produced in Brazil that was available at a reasonable cost. Nowadays, all APG
devices in Brazil must be imported, which makes costs considerably higher.

\scieloSectionContainer{CONCLUSION}

Air plethysmography is a non-invasive test that can be used to quantitatively
assess

several different components of venous hemodynamics: valvular reflux, calf
muscle
pump function, and venous obstruction. It can be used to assess chronic venous
disease, to evaluate improvement after venous surgery, to diagnose acute and
past
episodes of deep venous thrombosis, to evaluate compression stocking therapy, to
study the physiological implications of high-heeled shoes in healthy women, and
even
to evaluate the probability of ulcer healing.

\end{multicols}

\end{document}
