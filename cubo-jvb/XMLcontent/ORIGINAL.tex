%{\noindent\fontsize{9}{10.8}\selectfont{Received: January 13, 2015; Accepted: April 19, 2015.}}

\begin{multicols}{2}
\scieloSectionContainer{INTRODUCTION}
\par{}Falls are events with a high prevalence\scieloBodyFn{1}{fn body}{1}{} among the elderly population, even among those who are active and healthy, and constitute one of the major preventable geriatric syndromes'. Among the community-dwelling elderly, approximately 30\% suffer a fall each year, and half experience recurrent falls2. Elderly women with osteoporosis and having a high risk of fractures exhibit an even higher frequency of falls (51.1\%)3. A significant portion of these falls results in injuries (36\%)4, fractures (3.4\% to 19\%)2,4,5, and the need for medical assistance (8 to 19\%)4,5 and affects lifestyle choices, creating a high socio-economic burden6. Additionally, experiencing one or more falls in the course of one year significantly increases the chances of the occurrence of new episodes in the following year among the community-dwelling elderly\textsuperscript{4,5} and postmenopausal women\textsuperscript{1'7}.
\par{}Thus, the surveillance of falls among the elderly represents a priority health issue\textsuperscript{6},  which is why uestioning the occurrence of previous falls has been used in clinical/scientific decision making". Several methods have been suggested for monitoring the occurrence of falls among the community-dwelling elderly, including questions asking individuals to recall these events at several intervals by means of telephone, face-to-face or mail interviews, information obtained from medical records, and/or prospective records using falls calendars or diaries\textsuperscript{8,9,11-13}. However, the elderly\\

\lipsum

\scieloSectionContainer{CONCLUSION}

\par{}The thermal ablation therapy investigated in this study, using a 1470-nm water-specific laser

\scieloSubSectionContainer{Animals}

\par{}

\scieloSubSectionContainer{Histological studies of polyclad nervous systems}

\par{}A specimen of each species was embedded in paraffin
in its entirety, longitudinally sectioned at 7-10 mm, and
stained with Milligan Trichrome technique which allows
differentiating well among connective tissue, muscles, and
nervous fibers (Presnell and Schreibman, 1997). Additional
cross and sagittal sections were also prepared. De-paraffi-
nized sections were treated with 3\% potassium dichro-
mate-hydrochloric acid solution for 5 min. Following a
distilled water rinse, sections were stained in acid fuchsin
for 8 min. After a second distilled water rinse, they were
placed into 1\% phosphomolybdic acid for 2 min and then
stained with 2\% solution of Orange G for 5 min. After a fi-
nal rinse with distilled water, the sections were treated with
1\% hydrochloric acid solution for 2 min stained in Fast
Green for 8 min, treated with 1\% acetic acid for 3 min and
then rinsed in 95\% alcohol and dehydrated. Finally, the sec-
tions were cleared with Histoclear (National Diagnostics)
and mounted in Permount (Fisher Scientific). Slides were
observed and photographed under an Axiostar Plus (Zeiss,
Thornwood, New York) light microscope. We applied the
morphological criteria of Reuter et al. (1998) to distinguish
main nerve cords from secondary nerve cords for the exam-
ination of the NS. Cross and sagittal sections were used for
interpretation. In addition, whole mounts of all species
were prepared, by dehydrating the specimens in a graded
alcohol series, cleared with Histoclear, and mounted in
Permount.

\lipsum

\scieloImageContainerOneCol{0.5\textwidth}{img1.jpg}{meu titulo}

%\begin{center}
%{
%\vspace{8mm}
%\centerline{
%\includegraphics[width=\maxwidth{0.5\textwidth}]{img1.jpg}
%}
%\vspace{8mm}
%}
%\end{center}

\lipsum

\scieloImageContainerTwoCol{img2.jpg}{\textbf{Figure 1:} A specimen of each species was embedded in paraffin in its entirety, longitudinally sectioned}

\lipsum
\lipsum

\scieloImageContainerTwoCol{img2.jpg}{\textbf{Figure 1:} A specimen of each species was embedded in paraffin in its entirety, longitudinally sectioned}

\lipsum
\lipsum

\begin{scieloReferencesContainer}[REFERENCES]

\scieloReferencesItem{Evans CJ, Fowkes FG, Ruckley CV, Lee AJ. Prevalence of varicose
veins and chronic venous insufficiency in men and women in the
general population: Edinburgh Vein Study. J Epidemiol Community
Health. 1999;53(3):149-53. http://dx.doi.org/10.1136/jech.53.3.149.
PMid:10396491.}

\scieloReferencesItem{Andreozzi GM, Cordova RM, Scomparin A, et al. Quality of life in
chronic venous insufficiency. An Italian pilot study of the Triveneto Region. Int Angiol. 2005;24(3):272-7. PMid:16158038.}

\scieloReferencesItem{Bailly X, Reichert H and Hartenstein V (2013)
The urbilaterian brain revisited: Novel insights
into old questions from new flatworm clades.
Dev Genes Evol 223:149-157.}

\scieloReferencesItem{Biserova NM, Dudicheva VA, Terenina NB,
Reuter M, Halton DW, Maule AG and Gustafsson
MK (2000) The nervous system of Amphilina
foliacea (Platyhelminthes, Amphilinidea),
an immunocytochemical, ultrastructural
and spectrofluorometrical study. Parasitology
121:441-453.}

\scieloReferencesItem{Bock S (1923) Boninia, a new polyclad genus
from the Pacific. Nov Act Reg Soc Uppsala
Ser 46:1-32.}

\scieloReferencesItem{Pannier F, Rabe E, Maurins U. First results with a new 1470-nm
diode laser for endovenous ablation of incompetent saphenous
veins. Phlebology. 2009;24(1):26-30. http://dx.doi.org/10.1258/
phleb.2008.008038. PMid:19155338.}

\scieloReferencesItem{Almeida J, Mackay E, Javier J, Mauriello J, Raines J. Saphenous
laser ablation at 1470 nm targets the vein wall, not blood.
Vasc Endovascular Surg. 2009;43(5):467-72. http://dx.doi.
org/10.1177/1538574409335916. PMid:19628516.}

\scieloReferencesItem{Proebstle TM, Gül D, Lehr HA, Kargl A, Knop J. Infrequent early
recanalization of greater saphenous vein after endovenous laser
treatment. J Vasc Surg. 2003;38(3):511-6. http://dx.doi.org/10.1016/
S0741-5214(03)00420-8. PMid:12947269.}

\scieloReferencesItem{Cebrià F (2008) Organization of the nervous system in the model
planarian Schmidtea mediterranea: An immunocytochemical study. Neurosci Res 61:375-384.}

\scieloReferencesItem{Chien P and Koopowitz H (1972) The ultrastructure of neuromuscular systems in Notoplana acticola, a free-living polyclad flatworm. Z Zellforsch Mikroskop Anat 133:277-288.}


\scieloReferencesItem{Chien PK and Koopowitz H (1977) Ultrastructure of nerve plexus
in flatworms. III. The infra-epithelial nervous system. Cell
Tissue Res 176:335-347.}

\scieloReferencesItem{Day TA, Maule AG, Shaw C and Pax RA (1997) Structure-activity relationships of FMRFamide-related peptides contracting Schistosoma mansoni muscle. Peptides 18:917-921.}

\scieloReferencesItem{Egger B, Gschwentner R and Rieger R (2007) Free-living
flatworms under the knife: Past and present. Dev Genes Evol
217:89-104.}

\scieloReferencesItem{Egger B, Lapraz F, Tomiczek B, Müller S, Dessimoz C, Girstmair
J, Skunca N, Rawlinson KA, Cameron CB, Beli E et al.
(2015) A transcriptomic-phylogenomic analysis of the evolutionary relationships of flatworms. Curr Biol 25:1-7.}

\scieloReferencesItem{Ehlers U (1985) Das Phylogenetische System der Plathelminthes.
Gustav Fischer Verlag, Stuttgart, 317 pp.}

\scieloReferencesItem{Fernandes MC, Alvares EP, Gama P and Silveira M (2003) Serotonin in the nervous system of the head region of the land planarian Bipalium kewense. Tissue Cell 35:479-486.}

\scieloReferencesItem{Forest DL and Lindsay SM (2008) Observations of serotonin and
FMRFamide-like immunoreactivity in palp sensory structures and the anterior nervous system of spionid polychaetes. J Morphol 269:544-551.}

\scieloReferencesItem{Girstmair J, Schnegg R, Telford MJ and Egger B (2014) Cellular
dynamics during regeneration of the flatworm Monocelis sp. (Proseriata, Platyhelminthes). Evo Devo 5:e37.}

\scieloReferencesItem{Golding DM (1992) Polychaeta: Nervous system. In: Harrison
FW and Gardiner SL (eds) Microscopic Anatomy of Invertebrates. Vol. 7. Wiley-Liss, New York, pp 155-179.}

\scieloReferencesItem{Gustafsson MKS, Halton DW, Kreshchenko ND, Movsessian SO,
Raikova OI, Reuter M and Terenina NB (2002) Neuropeptides in flatworms. Peptides 23:2053-2061.}

\scieloReferencesItem{Hadenfeldt D (1929) Das Nervensystem von Stylochoplana
maculata und Notoplana atomata. Z wiss Zool 133:586638.}

\scieloReferencesItem{Halton DW and Gustafsson MKS (1996) Functional morphology
of the platyhelminth nervous system. Parasitology
113:S47-S72.}

\end{scieloReferencesContainer}

\scieloLicenseContainer{License information: This is an open-access article distributed under the terms of the Creative Commons Attribution License, which permits unrestricted use, distribution, and reproduction in any medium, provided the original work is properly cited.}

\end{multicols}
