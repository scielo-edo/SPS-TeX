% Generated by jats2tex@0.10.2.0:8927a9cc6d14b7897665c959b31cc7f822741feb
\documentclass[numberinsection,times,10pt,spreadimages]{memoir}
\usepackage[jvb]{scielo}

\makeatletter
\def\@fnsymbol#1{
\ensuremath{
\ifcase#1
\or 1
\or ~
\or ~
\or ~
\fi
}
}

\begin{document}

\setcounter{footnote}{4}
\selectlanguage{english}

jvbJornal Vascular BrasileiroJ. Vasc. Bras.1677-54491677-7301Sociedade
Brasileira de Angiologia e de Cirurgia Vascular
(SBACV)jvbDT20160028\_{}PT10.1590/1677-5449.00281601102ArticlesUso de curativo a
vácuo como terapia adjuvante na cicatrização de
sítio cirúrgico infectadoCamargoPaula Angeleli Bueno
de1*BertanhaMatheus1MouraRegina1JaldinRodrigo Gibin1YoshidaRicardo de
Alvarenga1PimentaRafael Elias Farres1MariúbaJamil Victor de
Oliveira1SobreiraMarcone Lima11Universidade Estadual Paulista “Júlio de
Mesquita Filho”Faculdade de Medicina de
BotucatuUniversidade Estadual
PaulistaSão PauloSPBrasilUniversidade Estadual Paulista “Júlio de
Mesquita Filho” – UNESP, Faculdade de Medicina de Botucatu, Departamento de
Cirurgia e Ortopedia, São Paulo, SP, Brasil.Conflito de interesse: Os autores
declararam não haver conflitos de interesse
que precisam ser informados.*Correspondência Paula Angeleli Bueno de Camargo
Universidade
Estadual Paulista “Júlio de Mesquita Filho” – UNESP, Faculdade de Medicina de
Botucatu, Departamento de Cirurgia e Ortopedia Distrito de Rubião Jr., s/n CEP
18618-970 - Botucatu (SP), Brasil Tel.: (14) 3880-1447 E-mail:
paula.angeleli@gmail.comInformações sobre os autores PABC, RGJ, REFP e JVOM -
Médicos Assistentes,
Departamento de Cirurgia e Ortopedia, Faculdade de Medicina de Botucatu,
Universidade Estadual Paulista Júlio de Mesquita Filho (UNESP). MB, RM e MLS
- Professores Assistentes Doutores, Departamento de Cirurgia e Ortopedia,
Faculdade de Medicina de Botucatu, Universidade Estadual Paulista Júlio de
Mesquita Filho (UNESP). RAY - Professor Colaborador, Departamento de
Cirurgia e Ortopedia, Faculdade de Medicina de Botucatu, Universidade
Estadual Paulista Júlio de Mesquita Filho (UNESP).Contribuições dos autores
Concepção e desenho do estudo: PABC, MB, RM Análise
e interpretação dos dados: PABC, RGJ, RAY, JVOM Coleta de dados: PABC, RGJ,
RAY, JVOM Redação do artigo: PABC, MB, RM Revisão crítica do texto: MB, RGJ,
REFP, JVOM, MLS Aprovação final do artigo*: PABC, MB, RM, MLS, RGJ, RAY,
REFP, JVOM Análise estatística: N/A. Responsabilidade geral pelo estudo:
PABC, MLS *Todos os autores leram e aprovaram a versão final submetida ao J
Vasc Bras.0501201700000000001005201631082016Este é um artigo publicado em acesso
aberto (Open
Access) sob a licença Creative Commons
Attribution, que permite uso, distribuição e reprodução em
qualquer meio, sem restrições desde que o trabalho original seja
corretamente citado.ResumoInfecções de sítios cirúrgicos com envolvimento de
próteses sintéticas constituem
grande desafio para tratamento. Apresentamos o caso de uma paciente com
múltiplas comorbidades, histórico de enxerto aortobifemoral há 6 anos e
reabordagem das anastomoses femorais por reestenoses há 5 anos. Apresentou dor
inguinal esquerda e abaulamento súbitos com diagnóstico de pseudoaneurisma
femoral roto e instabilidade hemodinâmica. Foi submetida a correção emergencial
com interposição de prótese de dácron recoberta por prata e correção de grande
hérnia incisional abdominal com tela sintética ao mesmo tempo. No
pós-operatório, manteve-se por longo período sob terapia intensiva com
dificuldade de extubação. Nesse ínterim, apresentou deiscência das suturas e
fístula purulenta inguinal esquerda em contato com a prótese vascular. Optou-se
pelo tratamento conservador, com desbridamento das feridas e aplicação de
curativo a vácuo. A paciente evoluiu com melhora e cicatrização das feridas.
Essa pode se constituir em ferramenta importante em casos
similares.Palavras-chave: terapêuticacicatrizaçãotécnicas de fechamento de
ferimentos abdominaisinfecçãoprótese vascularINTRODUÇÃOAs deiscências de
incisões cirúrgicas, particularmente com envolvimento de próteses
sintéticas, constituem grande desafio para tratamento1. Nas próteses vasculares,
em particular, há um
microambiente favorável à produção de biofilme, que sustenta a colonização
bacteriana e encapsula os germes, protegendo-os contra as defesas naturais do
organismo e da terapia antibiótica1. O tratamento geralmente preconizado é a
remoção da prótese
infectada1. Porém, essas
cirurgias são geralmente maiores que as prévias, uma vez que apresentam forte
envolvimento inflamatório. Assim, o paciente deve ter boas condições clínicas
para
suportar uma cirurgia que pode envolver reconstruções vasculares
extra-anatômicas
complexas, que demandam maior tempo operatório e elevam as taxas de
morbimortalidade. Dessa forma, pretende-se evitar a progressão da infecção,
quadros
isquêmicos graves decorrentes da remoção isolada da prótese e o risco de
amputações.
Nesse contexto, apresentamos um caso de sucessivas complicações cirúrgicas
decorrentes de reintervenções emergenciais sobre um enxerto aortobifemoral
antigo,
realizado há 6 anos para o tratamento de isquemia crítica de membros inferiores.
Retornou com degeneração da artéria femoral comum esquerda e constituiu um
grande
desafio terapêutico.PARTE I - SITUAÇÃO CLÍNICAPaciente do sexo feminino, branca,
com 75 anos de idade, dislipidêmica, tabagista
ativa com doença pulmonar obstrutiva crônica, doença renal crônica, obesidade
mórbida e portadora de insuficiência cardíaca. Tinha histórico de cirurgia
convencional de enxerto aortobifemoral com prótese de dácron bifurcada,
realizada
para o tratamento de isquemia crítica de membros inferiores 6 anos atrás, quando
havia uma situação clínica mais favorável. O quadro de isquemia crítica inicial
caracterizava-se por claudicação intermitente, lesões tróficas em ambos os pés
(necrose puntiforme de pododáctilos) e dor em repouso. Como complicação daquela
cirurgia, no pós-operatório, teve deiscência da incisão abdominal com
consequente
hérnia incisional.Um ano após a primeira cirurgia, a paciente apresentou necrose
em região calcânea
esquerda, que foi relacionada a uma piora da perfusão desse membro. Isso foi
confirmado por meio de ultrassom dúplex colorido, o qual demonstrou estenose >
75\% nas anastomoses femorais. A correção foi realizada com reabordagem
cirúrgica das
anastomoses e confecção de remendo de dácron, com sucesso técnico. A evolução
pós-operatória trouxe compensação circulatória para os membros inferiores e
cicatrização da ferida.Após 5 anos de acompanhamento ambulatorial regular, deu
entrada no pronto-socorro de
nossa instituição queixando-se de dor forte e abaulamento súbito em região
inguinal
esquerda. Clinicamente, a paciente apresentou instabilidade hemodinâmica e, após
um
rastreamento emergencial com ultrassom dúplex colorido, visualizou-se
pseudoaneurisma na interface entre a artéria femoral e o remendo de dácron e
extravasamento de sangue para o retroperitônio. Foi indicado tratamento
cirúrgico de
emergência.Foi, então, realizada correção cirúrgica convencional, com remoção do
remendo e
interposição de segmento de dácron recoberto por prata entre a porção média do
ramo
esquerdo da prótese aortobifemoral e a bifurcação femoral. Além disso, a hérnia
incisional abdominal prévia foi corrigida por meio de colocação de tela
sintética no
mesmo ato cirúrgico, pela equipe de gastrocirurgia.Em decorrência das várias
comorbidades, a evolução pós-operatória foi desfavorável,
permanecendo a paciente em regime de terapia intensiva por longo período.
Ocorreu
deiscência de sutura das incisões inguinal e abdominal com drenagem de secreção
purulenta na inguinotomia esquerda. Detectou-se, por tomografia computadorizada,
uma
fístula de drenagem da coleção adjacente ao enxerto pela inguinotomia, sem
exposição
direta da prótese. Foi realizada a cultura das secreções, que se apresentou
positiva
para Staphylococcus epidermidis e Staphylococcus
aureus coagulase negativo, ambos sensíveis à vancomicina. A situação
clínica da paciente era agravada, ainda, por quadro de pneumonia, tratada com
imipenem, tazocin e polimixina B, e pelo antecedente de doença pulmonar
obstrutiva
crônica, o que a manteve em ventilação mecânica invasiva por 45 dias.Diante
desse quadro e sem condições cirúrgicas, as opções de tratamento
eram:aAntibióticos sistêmicos, desbridamentos e curativos locais;bLavagem
contínua dos sítios infectados com antissépticos e antibióticos;cRemoção
cirúrgica de todas as próteses, apesar das condições clínicas
desfavoráveis, seguida de reconstrução vascular extra-anatômica;dRemoção
cirúrgica de todas as próteses, apesar das condições clínicas
desfavoráveis, e espera por delimitação isquêmica para posterior
amputação;eCurativo a vácuo, antibioticoterapia e observação das condições
clínicas da
paciente.PARTE II – O QUE FOI FEITOFoi instalado curativo a vácuo nas
deiscências do abdome e região inguinal esquerda.
O equipamento usado foi o Sistema de Terapia V.A.C. ATS® (KCI Kinetic Concepts
Inc,
San Antonio, Texas, EUA) e antibioticoterapia (vancomicina e imipenem por 21
dias)
com observação das condições clínicas da paciente.Após desbridamento das
feridas, foi colocada esponja de poliuretano exclusivamente
dentro do leito das deiscências, com cobertura total com filme plástico (Figuras
1A e 1B). O tubo de sucção foi acoplado ao equipamento de pressão negativa,
mantendo-se uma pressão de -125 mmHg de forma contínua (Figuras 1C e 1D). A
troca dos kits de sucção foi feita a cada 3 dias. Houve drenagem de,
aproximadamente, 50 mL/dia de secreção purulenta. A paciente evoluiu com
estabilização das comorbidades e controle da infecção, extubação após 45 dias,
fechamento progressivo das deiscências incisionais, granulação total em 60 dias
com
suspensão do uso do curativo a vácuo, curativos convencionais com hidrogel
durante
internação por mais 25 dias e fechamento ambulatorial das feridas após 9 meses
de
tratamento (Figuras 2A e 2B).Figura 1Aplicação do curativo a vácuo após o
desbridamento das incisões
deiscentes. (A) Troca do curativo com a colocação da esponja de poliuretano
no leito das feridas; (B) Kit de sucção e filme plástico devidamente
instalados; (C) Equipamento de pressão negativa fixado no leito da paciente;
(D) Refil do frasco de drenagem do curativo a vácuo.Figura 2Processo de
cicatrização das feridas com uso do curativo a vácuo. (A)
Resultado intermediário do processo de cicatrização, ainda em uso de
curativo a vácuo; (B) Resultado final do processo de cicatrização das
feridas.DISCUSSÃOO uso de curativos com pressão negativa para tratamentos
diversos é conhecido desde a
antiguidade2. O tratamento de
feridas crônicas com terapia a vácuo padronizada teve início em 19973. Sua ação
é baseada nos conceitos
de: contração da ferida, eliminação do exsudado e do tecido inviável, estímulo à
mitose celular, manutenção de um ambiente úmido, redução do edema tecidual,
remoção
de bactérias, melhora da vascularização e aceleração da granulação4-7.Pode-se
considerar a indicação de tratamento com curativo a vácuo para feridas com
baixa resposta ao tratamento convencional, em que se prevê um longo período para
cicatrização, feridas profundas e com elevada quantidade de exsudado e como
coadjuvante a outras terapias ou intervenções. As contraindicações são: feridas
com
malignidade, fístulas para órgãos e cavidades, osteomielites e exposição de
vasos
sanguíneos com risco de sangramento.Pode-se, ainda, atribuir como vantagens
dessa terapia a redução da inflamação e da
dor resultantes da manipulação constante da ferida, isenção de contaminação por
contato e conforto para o paciente, uma vez que não deixa odores desagradáveis.
Em
contrapartida, apresenta custos imediatos mais elevados, principalmente
relacionados
às trocas do refil e do próprio curativo em condições assépticas ao menos uma
vez
por semana. Entretanto, quando se somam todos os benefícios do uso do curativo a
vácuo em comparação com os curativos convencionais, fica evidente que há uma
relação
de custo-efetividade com a adoção do curativo a vácuo. Com relação ao caso
previamente apresentado, provavelmente um curativo convencional teria
dificuldade em
manter um ambiente propício à cicatrização, levando-se em consideração a
presença de
uma fístula purulenta em contato com a prótese arterial implantada e as grandes
áreas deiscentes.Revisões sistemáticas8,9 e um estudo randomizado10 mostraram a
efetividade do curativo com pressão
negativa em várias situações, com relação à proporção de feridas cicatrizadas e
velocidade de fechamento das mesmas, sendo particularmente efetivo em pés
diabéticos11-13, enxertos de pele14 e infecções pós-cirúrgicas15,16. No entanto,
faltam ainda estudos randomizados
de boa qualidade e isentos de conflitos de interesse para melhor avaliar esse
método17.As complicações descritas como relacionadas ao uso do curativo a vácuo
são
infrequentes, sendo a maioria relacionada a dor local, hipertrofia do tecido de
granulação e danos aos vasos sanguíneos adjacentes18,19. Ressalta-se, então, que
a esponja do curativo a vácuo
não deve ser aplicada diretamente sobre vasos sanguíneos. Nessa situação, um
filme
de silicone não adesivo deve servir de anteparo como proteção de interface entre
a
esponja e o tecido, evitando a erosão do vaso15.A pressão negativa nas feridas é
geralmente usada de forma contínua, mas existem
equipamentos que podem atuar de forma intermitente ou variável, não sendo
possível
encontrar evidências clínicas de vantagem relacionada a essa variável19. Níveis
de pressão negativa
menores que 80 mmHg (pressão negativa) são recomendados para se obter algum
efeito
do tratamento20. A instilação de
fluidos no leito da ferida pode melhorar a eficiência do tratamento em alguns
casos21.Pode-se concluir que os curativos com pressão negativa têm recomendações
bem
estabelecidas para o tratamento de feridas com características variadas, sendo
que
podem apresentar redução no tempo de cicatrização de feridas, maior conforto
para o
paciente e raras complicações. No caso apresentado, para o qual houve aprovação
do
Comitê de Ética de nossa instituição e que não apresenta conflito de interesse,
o
curativo a vácuo foi uma importante ferramenta para o sucesso terapêutico em uma
situação de exceção, na qual a abordagem cirúrgica para remoção das próteses
arteriais estaria associada a um elevado risco cirúrgico e de amputação. A
conduta
adotada apresentou um resultado bastante satisfatório.Fonte de financiamento:
Nenhuma.O estudo foi realizado no Hospital das Clínicas da Faculdade de Medicina
de
Botucatu, Universidade Estadual Paulista “Júlio de Mesquita Filho” (UNESP), São
Paulo, SP, Brasil.REFERÊNCIAS1DonatoGSetacciFGalzeranoGet alProsthesis
infection: prevention and treatmentJ Cardiovasc Surg
(Torino)201455677979225017788.1 Donato G, Setacci F, Galzerano G, et al.
Prosthesis infection:
prevention and treatment. J Cardiovasc Surg (Torino). 2014;55(6):779-92.
PMid:25017788.2RezendeJMModismos na História da MedicinaGoiásUniversidade
Federal de Goiás2011citado 2016 mar 10Disponível em:
http://www.medicinabiomolecular.com.br/biblioteca/pdfs/Doencas/do-0516.pdf2
Rezende JM. Modismos na História da Medicina. Goiás: Universidade
Federal de Goiás; 2011. [citado 2016 mar 10]. Disponível em: http://www.medicina
biomolecular.com.br/biblioteca/pdfs/Doencas/do-0516.pdf3ArgentaLCMorykwasMJVacuu
m-assisted closure: a new method for wound control and
treatment: clinical experienceAnn Plast
Surg19973865635769188971.http://dx.doi.org/10.1097/00000637-199706000-000023
Argenta LC, Morykwas MJ. Vacuum-assisted closure: a new method for
wound control and treatment: clinical experience. Ann Plast Surg.
1997;38(6):563-76. PMid:9188971.
http://dx.doi.org/10.1097/00000637-199706000-00002.
4BallardKBaxterHDevelopments in wound care for difficult to manage
woundsBJN20009740540811111435.http://dx.doi.org/10.12968/bjon.2000.9.7.63194
Ballard K, Baxter H. Developments in wound care for difficult to
manage wounds. BJN. 2000;9(7):405-8. PMid:11111435.
http://dx.doi.org/10.12968/bjon.2000.9.7.6319. 5KaufmanMWPahlDWVacuum-assisted
closure therapy: wound care and nursing
implicationsDermatol Nurs200315431732014515610.5 Kaufman MW, Pahl DW.
Vacuum-assisted closure therapy: wound care
and nursing implications. Dermatol Nurs. 2003;15(4):317-20.
PMid:14515610.6BraakenburgAObdeijnMCFeitzRvan RooijIAvan
GriethuysenAJKlinkenbijlJHThe clinical efficacy and cost effectiveness of the
vacuum-assisted closure technique in the management of acute and chronic
wounds: a randomized controlled trialPlast Reconstr Surg200611823907, discussion
398-40016874208.http://dx.doi.org/10.1097/01.prs.0000227675.63744.af6
Braakenburg A, Obdeijn MC, Feitz R, van Rooij IA, van Griethuysen
AJ, Klinkenbijl JH. The clinical efficacy and cost effectiveness of the
vacuum-assisted closure technique in the management of acute and chronic wounds:
a randomized controlled trial. Plast Reconstr Surg. 2006;118(2):390-7,
discussion 398-400. PMid:16874208.
http://dx.doi.org/10.1097/01.prs.0000227675.63744.af.
7WadaAFerreiraMCTumaPJrArrunáteguiGExperience with local negative pressure
(vacuum method) in the
treatment of complex woundsSao Paulo Med
J2006124315015317119692.http://dx.doi.org/10.1590/S1516-318020060003000087 Wada
A, Ferreira MC, Tuma P Jr, Arrunátegui G. Experience with
local negative pressure (vacuum method) in the treatment of complex wounds. Sao
Paulo Med J. 2006;124(3):150-3. PMid:17119692.
http://dx.doi.org/10.1590/S1516-31802006000300008.
8BruhinAFerreiraFCharikerMSmithJRunkelNSystematic review and evidence based
recommendations for the use
of negative pressure wound therapy in the open abdomenInt J
Surg201412101105111425174789.http://dx.doi.org/10.1016/j.ijsu.2014.08.3968
Bruhin A, Ferreira F, Chariker M, Smith J, Runkel N. Systematic
review and evidence based recommendations for the use of negative pressure wound
therapy in the open abdomen. Int J Surg. 2014;12(10):1105-14. PMid:25174789.
http://dx.doi.org/10.1016/j.ijsu.2014.08.396. 9GuffantiANegative pressure wound
therapy in the treatment of diabetic foot
ulcers: a systematic review of the literatureJ Wound Ostomy Continence
Nurs.201441323323724805174.http://dx.doi.org/10.1097/WON.00000000000000219
Guffanti A. Negative pressure wound therapy in the treatment of
diabetic foot ulcers: a systematic review of the literature. J Wound Ostomy
Continence Nurs. 2014;41(3):233-7. PMid:24805174.
http://dx.doi.org/10.1097/WON.0000000000000021.
10MonsenCWann-HanssonCWictorssonCAcostaSVacuum-assisted wound closure versus
alginate for the treatment
of deep perivascular wound infections in the groin after vascular
surgeryJ Vasc
Surg201459114515124055512.http://dx.doi.org/10.1016/j.jvs.2013.06.07310 Monsen
C, Wann-Hansson C, Wictorsson C, Acosta S. Vacuum-assisted
wound closure versus alginate for the treatment of deep perivascular wound
infections in the groin after vascular surgery. J Vasc Surg. 2014;59(1):145-51.
PMid:24055512. http://dx.doi.org/10.1016/j.jvs.2013.06.073.
11GameFLHinchliffeRJApelqvistJet alA systematic review of interventions to
enhance the healing of
chronic ulcers of the foot in diabetesDiabetes Metab Res Rev201228Supl
111914122271737.http://dx.doi.org/10.1002/dmrr.224611 Game FL, Hinchliffe RJ,
Apelqvist J, et al. A systematic review
of interventions to enhance the healing of chronic ulcers of the foot in
diabetes. Diabetes Metab Res Rev. 2012;28(Supl 1):119-41. PMid:22271737.
http://dx.doi.org/10.1002/dmrr.2246. 12UbbinkDTWesterbosSJNelsonEAVermeulenHA
systematic review of topical negative pressure therapy for
acute and chronic woundsBr J
Surg200895668569218446777.http://dx.doi.org/10.1002/bjs.623812 Ubbink DT,
Westerbos SJ, Nelson EA, Vermeulen H. A systematic
review of topical negative pressure therapy for acute and chronic wounds. Br J
Surg. 2008;95(6):685-92. PMid:18446777. http://dx.doi.org/10.1002/bjs.6238.
13XieXMcGregorMDendukuriNThe clinical effectiveness of negative pressure wound
therapy: a
systematic reviewJ Wound
Care2010191149049521135797.http://dx.doi.org/10.12968/jowc.2010.19.11.7969713
Xie X, McGregor M, Dendukuri N. The clinical effectiveness of
negative pressure wound therapy: a systematic review. J Wound Care.
2010;19(11):490-5. PMid:21135797.
http://dx.doi.org/10.12968/jowc.2010.19.11.79697.
14LlanosSDanillaSBarrazaCet alEffectiveness of negative pressure closure in the
integration of
split thickness skin grafts: a randomized, double-masked, controlled
trialAnn Surg2006244570070517060762.http://dx.doi.org/10.1097/01.sla.0000217745.
56657.e514 Llanos S, Danilla S, Barraza C, et al. Effectiveness of negative
pressure closure in the integration of split thickness skin grafts: a
randomized, double-masked, controlled trial. Ann Surg. 2006;244(5):700-5.
PMid:17060762. http://dx.doi.org/10.1097/01.sla.0000217745.56657.e5.
15DamianiGPinnarelliLSommellaLet alVacuum-assisted closure therapy for patients
with infected
sternal wounds: a meta-analysis of current evidenceJ Plast Reconstr Aesthet
Surg20116491119112321256819.http://dx.doi.org/10.1016/j.bjps.2010.11.02215
Damiani G, Pinnarelli L, Sommella L, et al. Vacuum-assisted
closure therapy for patients with infected sternal wounds: a meta-analysis of
current evidence. J Plast Reconstr Aesthet Surg. 2011;64(9):1119-23.
PMid:21256819. http://dx.doi.org/10.1016/j.bjps.2010.11.022.
16SilvaCGCrossettiMGOCurativos para tratamento de feridas operatórias
abdominais: uma
revisão sistemáticaRGE201233318218916 Silva CG, Crossetti MGO. Curativos para
tratamento de feridas
operatórias abdominais: uma revisão sistemática. RGE.
2012;33(3):182-9.17MermerkayaUBekmezSAlkanEAyvazMTokgozogluMEvaluation of
vacuum-assisted closure in patients with wound
complications following tumour surgeryInt Wound J201613339439724976480.17
Mermerkaya U, Bekmez S, Alkan E, Ayvaz M, Tokgozoglu M.
Evaluation of vacuum-assisted closure in patients with wound complications
following tumour surgery. Int Wound J. 2016;13(3):394-7.
PMid:24976480.18LiZYuAComplications of negative pressure wound therapy: a mini
reviewWound Repair
Regen201422445746124852446.http://dx.doi.org/10.1111/wrr.1219018 Li Z, Yu A.
Complications of negative pressure wound therapy: a
mini review. Wound Repair Regen. 2014;22(4):457-61. PMid:24852446.
http://dx.doi.org/10.1111/wrr.12190.
19MalmsjöMGustafssonLLindstedtSGessleinBIngemanssonRThe effects of variable,
intermittent, and continuous negative
pressure wound therapy, using foam or gauze, on wound contraction,
granulation tissue formation, and ingrowth into the wound
fillerEplasty201212e522292101.19 Malmsjö M, Gustafsson L, Lindstedt S, Gesslein
B, Ingemansson R.
The effects of variable, intermittent, and continuous negative pressure wound
therapy, using foam or gauze, on wound contraction, granulation tissue
formation, and ingrowth into the wound filler. Eplasty. 2012;12:e5.
PMid:22292101.20BorgquistOIngemanssonRMalmsjoMIndividualizing the use of
negative pressure wound therapy for
optimal wound healing: a focused review of the literatureOstomy Wound
Manage2011574445421512192.20 Borgquist O, Ingemansson R, Malmsjo M.
Individualizing the use of
negative pressure wound therapy for optimal wound healing: a focused review of
the literature. Ostomy Wound Manage. 2011;57(4):44-54.
PMid:21512192.21JeromeDAdvances in negative pressure wound therapy: the VAC
instillJ Wound Ostomy Continence Nurs.200734219119417413837.http://dx.doi.org/10
.1097/01.WON.0000264834.18732.3b21 Jerome D. Advances in negative pressure wound
therapy: the VAC
instill. J Wound Ostomy Continence Nurs. 2007;34(2):191-4. PMid:17413837.
http://dx.doi.org/10.1097/01.WON.0000264834.18732.3b.
jvbDT20160028\_{}ENArticlesUsing vacuum therapy as an adjunctive treatment for
healing of
infected surgical sitesCamargoPaula Angeleli Bueno
de1*BertanhaMatheus1MouraRegina1JaldinRodrigo Gibin1YoshidaRicardo de
Alvarenga1PimentaRafael Elias Farres1MariúbaJamil Victor de
Oliveira1SobreiraMarcone Lima11Universidade Estadual Paulista “Júlio de
Mesquita Filho”Faculdade de Medicina de
BotucatuUniversidade Estadual
PaulistaSão PauloSPBrazilUniversidade Estadual Paulista “Júlio de
Mesquita Filho” – UNESP, Faculdade de Medicina de Botucatu, Departamento de
Cirurgia e Ortopedia, São Paulo, SP, Brazil.Conflicts of interest: No conflicts
of interest declared concerning the
publication of this article.*Correspondence Paula Angeleli Bueno de Camargo
Universidade
Estadual Paulista “Júlio de Mesquita Filho” – UNESP, Faculdade de Medicina de
Botucatu, Departamento de Cirurgia e Ortopedia Distrito de Rubião Jr., s/n CEP
18618-970 - Botucatu (SP), Brazil Tel.: +55 (14) 3880-1447 E-mail:
paula.angeleli@gmail.comAuthor information PABC, RGJ, REFP and JVOM - Assistant
physicians,
Departamento de Cirurgia e Ortopedia, Faculdade de Medicina de Botucatu,
Universidade Estadual Paulista Júlio de Mesquita Filho (UNESP). MB, RM and
MLS - PhDs, assistant professors, Departamento de Cirurgia e Ortopedia,
Faculdade de Medicina de Botucatu, Universidade Estadual Paulista Júlio de
Mesquita Filho (UNESP). RAY - Collaborating professor, Departamento de
Cirurgia e Ortopedia, Faculdade de Medicina de Botucatu, Universidade
Estadual Paulista Júlio de Mesquita Filho (UNESP).Author contributions
Conception and design: PABC, MB, RM Analysis and
interpretation: PABC, RGJ, RAY, JVOM Data collection: PABC, RGJ, RAY, JVOM
Writing the article: PABC, MB, RM Critical revision of the article: MB, RGJ,
REFP, JVOM, MLS Final approval of the article*: PABC, MB, RM, MLS, RGJ, RAY,
REFP, JVOM Statistical analysis: N/A. Overall responsibility: PABC, MLS *All
authors have read and approved of the final version of the article submitted
to J Vasc Bras.This is an Open Access article distributed under the terms of the
Creative Commons Attribution License, which permits unrestricted use,
distribution, and reproduction in any medium, provided the original work is
properly cited.AbstractInfections at the sites of surgery involving synthetic
prostheses are challenging
to treat. We present a case of a patient with multiple comorbidities who had
undergone an aortobifemoral bypass 6 years previously and then re-intervention
at the femoral anastomoses for restenosis 5 years previously. The patient
presented with acute left inguinal pain and swelling and was diagnosed with a
ruptured femoral pseudoaneurysm and hemodynamic instability. A repair was
conducted by interposition of a silver-coated Dacron graft in the emergency
room, and a large abdominal incisional hernia was repaired with synthetic mesh
during the same intervention. After surgery, the patient remained intubated in
intensive care for a long period. Meanwhile, she presented dehiscence of sutures
and a left inguinal purulent fistula that was in contact with the vascular
prosthesis. Conservative treatment was chosen, with debridement of wounds and
vacuum therapy. The patient improved and the wounds healed. This could be an
important tool in similar cases.Keywords: treatmenthealingabdominal injury
closure techniquesinfectionvascular prosthesisINTRODUCTIONDehiscence of surgical
incisions is a major challenge to treat, particularly when
they involve synthetic prostheses.1 Vascular prostheses in particular provide a
microenvironment that is conducive to production of biofilm, which supports
bacterial colonization and encapsulates the germs, protecting them from the
body’s natural defenses and from antibiotics.1 The treatment that is generally
recommended is
removal of the infected prosthesis.1 However, these operations are generally of
a larger
scale than the original surgery because of the significant inflammatory
involvement. The patient must therefore be in good clinical conditions to
withstand a surgical operation that may include complex extra-anatomic vascular
reconstruction, requiring longer operating times and elevating morbidity and
mortality rates. The objectives are to prevent continued development of the
infection, to avoid severe ischemia resulting from simply removing the
prosthesis, and to reduce the risk of amputations. Against this background, we
present a case of successive surgical complications triggered by emergency
reinterventions to repair a previous aortobifemoral bypass that had been
constructed 6 years previously to treat critical lower limb ischemia. The
patient presented with degeneration of the left common femoral artery and the
case was a significant therapeutic challenge.PART I – CLINICAL SITUATIONThe
patient was a white, 75-year-old, female, active smoker with dyslipidemia,
chronic obstructive pulmonary disease, chronic kidney disease, morbid obesity,
and heart failure. Six years previously she had undergone conventional surgery
to construct an aortobifemoral bypass using a bifurcated dacron graft to treat
critical lower limb ischemia, when her clinical situation had been less
unfavorable. Her initial critical ischemia had presented with intermittent
claudication, trophic ulcers on both feet (necrosis punctiform of the toes), and
pain at rest. The patient had already suffered a complication during the
postoperative period of that operation: dehiscence of the abdominal incision,
causing an incisional hernia.One year after the first operation, the patient
presented once more, with
necrosis of the left heel, which was related to deterioration of perfusion to
the left lower limb. This was confirmed with color duplex ultrasound, which
showed > 75\% stenosis of the femoral anastomoses. A surgical reintervention
was conducted to repair the anastomoses with a dacron patch and was a technical
success. Postoperative recovery was accompanied by compensation of the lower
limb circulation and the wound healed.After 5 years of regular outpatients
follow-up, the patient was admitted to the
emergency room at our institution complaining of severe pain and swelling in the
left inguinal region. Clinically, the patient was hemodynamically unstable and
emergency duplex ultrasound screening revealed a pseudoaneurysm at the interface
between the femoral artery and the dacron patch, with blood leaking into the
retroperitoneal space. The patient was prepared for emergency surgical
treatment.The pseudoaneurysm was repaired during a conventional surgical
operation, with
removal of the patch and interposition of a silver-coated dacron segment between
the medial portion of the left branch of the aortobifemoral graft and the
femoral bifurcation. During the same operation, the gastric surgery team
repaired the preexisting incisional hernia by placement of a synthetic mesh.The
multiple comorbidities caused unfavorable postoperative progress and the
patient was kept in intensive care for a long period of time. The inguinal and
abdominal sutures underwent dehiscence and there were purulent secretions
draining from the left inguinotomy. On computed tomography, a fistula was
detected draining pus adjacent to the graft via the inguinotomy, to which the
prosthesis was not directly exposed. The secretions were cultured and found to
be positive for Staphylococcus epidermidis and coagulase
negative Staphylococcus aureus, both sensitive to vancomycin.
The patient’s clinical situation was further aggravated by pneumonia, which was
treated with imipenem, tazocin and polymyxin B, and by the preexisting chronic
obstructive pulmonary disease, which kept her on invasive mechanical ventilation
for 45 days.Faced with this clinical status and without the necessary conditions
for surgery,
the treatment options were as follows:aSystemic antibiotics, debridement and
local dressings;bContinuous cleaning of the infected sites with antiseptics and
antibiotics;cSurgical removal of all prostheses, despite the unfavorable
clinical
conditions, followed by extra-anatomic vascular reconstruction;dSurgical removal
of all prostheses, despite the unfavorable clinical
conditions, and amputation after delimitation of ischemia;eVacuum dressing,
antibiotic therapy and observation of the patient’s
clinical condition.PART II – WHAT WAS DONEA vacuum dressing was applied to the
dehiscences at the abdomen and left inguinal
region. The V.A.C. ATS Therapy System® (KCI Kinetic Concepts Inc, San Antonio,
Texas, United States) was used in combination with antibiotic therapy
(vancomycin and imipenem for 21 days) and observation of the patient’s clinical
conditions.After debridement of the wounds, polyurethane foam was placed
exclusively within
the dehiscence bed and completely covered with plastic film (Figures 1A and 1B).
The suction tube was fitted to the negative
pressure unit, maintaining a pressure of -125 mmHg continuously (Figures 1C and
1D). Suction kits were changed every 3 days.
Approximately 50 mL/day of purulent secretions were drained. The patient’s
comorbidities stabilized and the infection was controlled. She was extubated
after 45 days, and the incisional dehiscences closed progressively, achieving
total granulation at 60 days, when the vacuum dressing was withdrawn, followed
by conventional hydrogel dressings for a further 25 days while still in
hospital, and the wounds closed during outpatients follow-up 9 months after
starting treatment (Figures 2A and
2B).Figure 1Application of vacuum dressing after debridement of dehiscent
incisions. (A) Dressing change, placing polyurethane foam on wound beds;
(B) Suction kit and plastic film correctly fitted; (C) Negative pressure
unit attched to patient’s bed; (D) Replacement vacuum dressing drainage
jar.Figure 2Wound healing process with vacuum dressing. (A) Intermediate stage
of
healing process, vacuum dressing still in use; (B) Final result of wound
healing process.DISCUSSIONUtilization of dressings with negative pressure has
been known since ancient
times for a range of treatments.2 Treatment of chronic wounds using standardized
vacuums
began in 1997.3 The effects
are based on the following concepts: contraction of the wound, removal of
exudate and non-viable tissues, stimulation of cell mitosis, maintenance of a
humid environment, reduction of tissue edema, removal of bacteria, improvement
of vascularization, and acceleration of granulation.4-7Treatment with vacuum
dressings may be indicated for wounds that do not respond
well to conventional treatment, when a long healing period is predicted, for
deep wounds and those with high quantities of exudate, and as a supplementary
technique in combination with other treatments or interventions.
Contraindications are: wounds with malignancies, fistulae into organs and
cavities, osteomyelitis, and exposure of blood vessels at risk of
bleeding.Advantages of this treatment include reduction of inflammation and pain
caused by
constant manipulation of the wound, exclusion of contamination by contact, and
improved patient comfort, since it does not produce unpleasant odors. On the
other hand, immediate costs are high, primarily related to changing the refill
and the dressing itself under aseptic conditions at least once a week. However,
when all the benefits of using a vacuum dressing compared with conventional
dressings are added up, it is clear that the choice of a vacuum dressing offers
a good cost-effectiveness ratio. With regard to the case described above, it is
probable that conventional dressings would have been unlikely to have
successfully maintained an environment conducive to healing, considering the
presence of a purulent fistula in contact with the prosthetic arterial graft and
the large areas of dehiscence.Systematic reviews8,9 and a randomized study10
show the effectiveness of negative pressure
dressings in a range of situations, in terms both of the proportion of wounds
healed and the speed with which they close, and they are particularly effective
for diabetic feet,11-13 skin grafts14 and infections after surgery.15,16
However, there is still a lack of good quality
randomized studies free from conflicts of interest that would allow the method
to be evaluated more thoroughly.17Complications that have been described in
relation to vacuum dressings are
uncommon and the majority are related to local pain, hypertrophy of granulation
tissue, and damage to adjacent blood vessels.18,19 It should however be stressed
that the dressing
foam should not be placed in direct contact with blood vessels. In such
situations, a non-adhesive silicone film should be used as a pre-preparation as
a protection for the interface between foam and tissue, preventing erosion of
the vessel.15In general the negative pressure is applied to wounds continuously,
but there are
systems that can provide intermittent or variable action, although there is no
clinical evidence that this variable offers advantages.19 Negative pressure
levels below 80 mmHg
(negative pressure) are recommended to obtain treatment effectiveness.20 In some
cases instillation
of fluids to the wound bed can improve efficacy.21It can be concluded that there
are well-established recommendations for using
negative pressure dressing to treat wounds with a variety of characteristics and
they can offer reductions in the time taken for wounds to heal, combined with
greater patient comfort and rare complications. In the case described here,
which was approved by our institutional Ethics Committee and does not involve
any conflicts of interests, the vacuum dressing was an important tool for
achieving therapeutic success in an exceptional situation, in which an
additional surgical operation to remove the arterial prostheses would have
incurred a high surgical risk and high risk of amputation. The conduct chosen
achieved very satisfactory results.Financial support: None.The study was carried
out at Hospital das Clínicas, Faculdade de Medicina de
Botucatu, Universidade Estadual Paulista “Júlio de Mesquita Filho” (UNESP),
São Paulo, SP, Brazil.

% \scieloAbstractContainer{ABSTRACT}{\textbf{Background}: The identification of the occurrence of falls is an important step for screening and for rehabilitation proccess for the elderly. The methods of monitoring these events are susceptible to recording biases, and the choice of the most accurate method remains challenging. \textbf{Objectives}: (i) To investigate the agreement between retrospective self-reporting and prospective monitoring of methods of recording falls, and (ii) to compare the retrospective self-reporting of falls and the prospective monitoring of falls and recurrent falls over a 12-month period among older women at high risk of falls and fractures. \textbf{Method}: A total of 118 community-dwelling older women with low bone density were recruited. The incidence of falls was monitored prospectively in 116 older women (2 losses) via monthly phone calls over the course of a year. At the end of this monitoring period, the older women were asked about their recall of falls in the same 12-month period.  The agreement between the two methods where analyzed and the sensitivity and specificity of self-reported previous falls in relation to the prospective monitoring weere calculated. \textbf{Results}: There was moderate agreement between the prospective monitoring and the retrospective self-reporting of falls in classifying fallers (Kappa=0.595) and recurrent fallers (Kappa=0.589). The limits of agreement were 0.35±1.66 falls. The self-reporting of prior falls had a 67.2\% sensitivity and a 94.2\% specificity in classifying fallers among older women and a 50\% sensitivity and a 98.9\% specificity in classifying recurrent fallers. Conclusion: Self-reporting of falls over a 12-month period underestimated 32.8\% of falls and 50\% of recurrent falls. The findings recommend caution if one is considering replacing monthly monitoring with annual retrospective questioning.}{Keywords:}{agend, bone desnsity, accidental falls; rehabilitation; mental recall}

%{\noindent\fontsize{9}{10.8}\selectfont{Received: January 13, 2015; Accepted: April 19, 2015.}}

\begin{multicols}{2}
\scieloSectionContainer{Introduction}
\par{}Falls are events with a high prevalence\scieloBodyFn{1}{fn body}{1}{} among the elderly population, even among those who are active and healthy, and constitute one of the major preventable geriatric syndromes'. Among the community-dwelling elderly, approximately 30\% suffer a fall each year, and half experience recurrent falls2. Elderly women with osteoporosis and having a high risk of fractures exhibit an even higher frequency of falls (51.1\%)3. A significant portion of these falls results in injuries (36\%)4, fractures (3.4\% to 19\%)2,4,5, and the need for medical assistance (8 to 19\%)4,5 and affects lifestyle choices, creating a high socio-economic burden6. Additionally, experiencing one or more falls in the course of one year significantly increases the chances of the occurrence of new episodes in the following year among the community-dwelling elderly\textsuperscript{4,5} and postmenopausal women\textsuperscript{1'7}.
\par{}Thus, the surveillance of falls among the elderly represents a priority health issue\textsuperscript{6},  which is why uestioning the occurrence of previous falls has been used in clinical/scientific decision making". Several methods have been suggested for monitoring the occurrence of falls among the community-dwelling elderly, including questions asking individuals to recall these events at several intervals by means of telephone, face-to-face or mail interviews, information obtained from medical records, and/or prospective records using falls calendars or diaries\textsuperscript{8,9,11-13}. However, the elderly\\

\lipsum

\scieloSectionContainer{Materials and Methods}

\scieloSubSectionContainer{Animals}

\par{}Specimens representing 12 species of polyclads were
collected from different sites in the Caribbean. In addition,
a deep-sea specimen of \textit{Anocellidus profundus} Quiroga,
Bolaños and Litvaitis, 2006, from the North Pacific was
also included in the analysis (specimen courtesy of Dr.
Janet Voight, Field Museum of Natural History, Chicago,
Illinois, USA). For details of collection, fixation, and spe-
cies identification see Litvaitis \textit{et al}. (2010).

\scieloSubSectionContainer{Histological studies of polyclad nervous systems}

\par{}A specimen of each species was embedded in paraffin
in its entirety, longitudinally sectioned at 7-10 mm, and
stained with Milligan Trichrome technique which allows
differentiating well among connective tissue, muscles, and
nervous fibers (Presnell and Schreibman, 1997). Additional
cross and sagittal sections were also prepared. De-paraffi-
nized sections were treated with 3\% potassium dichro-
mate-hydrochloric acid solution for 5 min. Following a
distilled water rinse, sections were stained in acid fuchsin
for 8 min. After a second distilled water rinse, they were
placed into 1\% phosphomolybdic acid for 2 min and then
stained with 2\% solution of Orange G for 5 min. After a fi-
nal rinse with distilled water, the sections were treated with
1\% hydrochloric acid solution for 2 min stained in Fast
Green for 8 min, treated with 1\% acetic acid for 3 min and
then rinsed in 95\% alcohol and dehydrated. Finally, the sec-
tions were cleared with Histoclear (National Diagnostics)
and mounted in Permount (Fisher Scientific). Slides were
observed and photographed under an Axiostar Plus (Zeiss,
Thornwood, New York) light microscope. We applied the
morphological criteria of Reuter et al. (1998) to distinguish
main nerve cords from secondary nerve cords for the exam-
ination of the NS. Cross and sagittal sections were used for
interpretation. In addition, whole mounts of all species
were prepared, by dehydrating the specimens in a graded
alcohol series, cleared with Histoclear, and mounted in
Permount.

\lipsum

\scieloImageContainerOneCol{0.5\textwidth}{img1.jpg}{meu titulo}

%\begin{center}
%{
%\vspace{8mm}
%\centerline{
%\includegraphics[width=\maxwidth{0.5\textwidth}]{img1.jpg}
%}
%\vspace{8mm}
%}
%\end{center}

\lipsum

\scieloImageContainerTwoCol{img2.jpg}{\textbf{Figure 1:} A specimen of each species was embedded in paraffin in its entirety, longitudinally sectioned}

\lipsum
\lipsum

\scieloImageContainerTwoCol{img2.jpg}{\textbf{Figure 1:} A specimen of each species was embedded in paraffin in its entirety, longitudinally sectioned}

\lipsum
\lipsum

\begin{scieloReferencesContainer}[References]

\scieloReferencesItem{Ganz DA, Bao Y, Shekelle PG, Rubenstein LZ. Will my patient fall? JAMA. 2007;297(1):77-86. http://dx.doi.org/10.1001/jama.297.1.77. PMid:17200478.}

\scieloReferencesItem{Baguñà J and Riutort M (2004) Molecular phylogeny
of the Platyhelminthes. Can J Zool
82:168-193.}

\scieloReferencesItem{Bailly X, Reichert H and Hartenstein V (2013)
The urbilaterian brain revisited: Novel insights
into old questions from new flatworm clades.
Dev Genes Evol 223:149-157.}

\scieloReferencesItem{Biserova NM, Dudicheva VA, Terenina NB,
Reuter M, Halton DW, Maule AG and Gustafsson
MK (2000) The nervous system of Amphilina
foliacea (Platyhelminthes, Amphilinidea),
an immunocytochemical, ultrastructural
and spectrofluorometrical study. Parasitology
121:441-453.}

\scieloReferencesItem{Bock S (1923) Boninia, a new polyclad genus
from the Pacific. Nov Act Reg Soc Uppsala
Ser 46:1-32.}

\scieloReferencesItem{Böckerman I, Reuter M and Timoshkin O
(1994) Ultrastructural study of the central
nervous system of endemic Geocentrophora
(Prorhynchida, Platyhelminthes) from Lake
Baikal. Acta Zool 75:47-55.}

\scieloReferencesItem{Bullock TH and Horridge GA (1965) Structure
and Function in the Nervous Systems of Invertebrates.
Vol. 2. W.H. Freeman and Co, San
Francisco, 1719 pp.}

\scieloReferencesItem{Carranza S, Baguñà J and Riutort M (1997)
Are the Platyhelminthes a monophyletic primitive
group? An assessment using 18S rDNA
sequences. Mol Biol Evol 14:485-497.}

\scieloReferencesItem{Cebrià F (2008) Organization of the nervous system in the model
planarian Schmidtea mediterranea: An immunocytochemical study. Neurosci Res 61:375-384.}

\scieloReferencesItem{Chien P and Koopowitz H (1972) The ultrastructure of neuromuscular systems in Notoplana acticola, a free-living polyclad flatworm. Z Zellforsch Mikroskop Anat 133:277-288.}


\scieloReferencesItem{Chien PK and Koopowitz H (1977) Ultrastructure of nerve plexus
in flatworms. III. The infra-epithelial nervous system. Cell
Tissue Res 176:335-347.}

\scieloReferencesItem{Day TA, Maule AG, Shaw C and Pax RA (1997) Structure-activity relationships of FMRFamide-related peptides contracting Schistosoma mansoni muscle. Peptides 18:917-921.}

\scieloReferencesItem{Egger B, Gschwentner R and Rieger R (2007) Free-living
flatworms under the knife: Past and present. Dev Genes Evol
217:89-104.}

\scieloReferencesItem{Egger B, Lapraz F, Tomiczek B, Müller S, Dessimoz C, Girstmair
J, Skunca N, Rawlinson KA, Cameron CB, Beli E et al.
(2015) A transcriptomic-phylogenomic analysis of the evolutionary relationships of flatworms. Curr Biol 25:1-7.}

\scieloReferencesItem{Ehlers U (1985) Das Phylogenetische System der Plathelminthes.
Gustav Fischer Verlag, Stuttgart, 317 pp.}

\scieloReferencesItem{Fernandes MC, Alvares EP, Gama P and Silveira M (2003) Serotonin in the nervous system of the head region of the land planarian Bipalium kewense. Tissue Cell 35:479-486.}

\scieloReferencesItem{Forest DL and Lindsay SM (2008) Observations of serotonin and
FMRFamide-like immunoreactivity in palp sensory structures and the anterior nervous system of spionid polychaetes. J Morphol 269:544-551.}

\scieloReferencesItem{Girstmair J, Schnegg R, Telford MJ and Egger B (2014) Cellular
dynamics during regeneration of the flatworm Monocelis sp. (Proseriata, Platyhelminthes). Evo Devo 5:e37.}

\scieloReferencesItem{Golding DM (1992) Polychaeta: Nervous system. In: Harrison
FW and Gardiner SL (eds) Microscopic Anatomy of Invertebrates. Vol. 7. Wiley-Liss, New York, pp 155-179.}

\scieloReferencesItem{Gustafsson MKS, Halton DW, Kreshchenko ND, Movsessian SO,
Raikova OI, Reuter M and Terenina NB (2002) Neuropeptides in flatworms. Peptides 23:2053-2061.}

\scieloReferencesItem{Hadenfeldt D (1929) Das Nervensystem von Stylochoplana
maculata und Notoplana atomata. Z wiss Zool 133:586638.}

\scieloReferencesItem{Halton DW and Gustafsson MKS (1996) Functional morphology
of the platyhelminth nervous system. Parasitology
113:S47-S72.}

\end{scieloReferencesContainer}

\scieloLicenseContainer{License information: This is an open-access article distributed under the terms of the Creative Commons Attribution License, which permits unrestricted use, distribution, and reproduction in any medium, provided the original work is properly cited.}

\end{multicols}


\end{document}
