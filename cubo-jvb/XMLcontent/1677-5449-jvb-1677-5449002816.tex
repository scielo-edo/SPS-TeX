% Generated by jats2tex@0.10.2.0:8927a9cc6d14b7897665c959b31cc7f822741feb
\documentclass[numberinsection,times,10pt,spreadimages]{memoir}

% jats2tex requirements
% \usepackage[T1]{fontenc}
\usepackage{amstext}
\usepackage{authblk}
\usepackage{unicode-math}
\usepackage{multirow}
\usepackage{graphicx}
\usepackage[jvb]{scielo}

\newcommand\scieloArticleTitle{Uso de curativo a vácuo como terapia adjuvante na
cicatrização de
sítio cirúrgico infectado}
\author[]{,
\textsuperscript{th}
*}
\author[]{,
\textsuperscript{th}
}
\author[]{,
\textsuperscript{th}
}
\author[]{,
\textsuperscript{th}
}
\author[]{,
\textsuperscript{th}
}
\author[]{,
\textsuperscript{th}
}
\author[]{,
\textsuperscript{th}
}
\author[]{,
\textsuperscript{th}
}\affil[aff01]{}\newcommand\scieloArticleTitle{Using vacuum therapy as an
adjunctive treatment for healing of
infected surgical sites}
\author[]{,
\textsuperscript{th}
*}
\author[]{,
\textsuperscript{th}
}
\author[]{,
\textsuperscript{th}
}
\author[]{,
\textsuperscript{th}
}
\author[]{,
\textsuperscript{th}
}
\author[]{,
\textsuperscript{th}
}
\author[]{,
\textsuperscript{th}
}
\author[]{,
\textsuperscript{th}
}\affil[aff0100]{}

\newcommand{\scieloYear}{2015}
\newcommand\scieloArticlePubIdType{DOI}
\newcommand\scieloArticleType{ORIGINAL ARTICLE}
\newcommand{\scieloArticleRef}{2015 Oct.-Dec.; 14(4):281-288}
\newcommand\scieloArticleTitleTranslation{Avaliação da taxa de obliteração da
veia safena magna e da evolução clínica de pacientes submetidos a termoablação
com lase 1470 nm, fibra linear e baixa densidade de energia endovenosa
linear\\Título traduzido 2}
\newcommand{\scieloArticleUrl}{http://dx.doi.org/10.1590/1677-5449.004015}
\newcommand{\scieloHeader}{Walter Junior Boim Araujo, Jorge Rufino Ribas Timi et
al.}
\newcommand{\scieloAuthor}{Walter Junior Boim
Araujo\scieloContributorFn{1}{xx}{1}{}, Jorge Rufino Ribas
Timi\scieloContributorFn{1}{xx}{1}{}, Carlos Seme Nejm
Júnior\scieloContributorFn{1}{xx}{1}{}, Filipe Carlos CaronCronos}

\begin{document}

\maketitle

\setcounter{footnote}{4}
\selectlanguage{english}

jvbJornal Vascular BrasileiroJ. Vasc. Bras.1677-54491677-7301Sociedade
Brasileira de Angiologia e de Cirurgia Vascular
(SBACV)jvbDT20160028\_{}PT10.1590/1677-5449.00281601102Articles
Conflito de interesse: Os autores declararam não haver conflitos de interesse
que precisam ser informados.
\label{*}
Correspondência Paula Angeleli Bueno de Camargo Universidade
Estadual Paulista “Júlio de Mesquita Filho” – UNESP, Faculdade de Medicina de
Botucatu, Departamento de Cirurgia e Ortopedia Distrito de Rubião Jr., s/n CEP
18618-970 - Botucatu (SP), Brasil Tel.: (14) 3880-1447 E-mail:
paula.angeleli@gmail.com
Informações sobre os autores PABC, RGJ, REFP e JVOM - Médicos Assistentes,
Departamento de Cirurgia e Ortopedia, Faculdade de Medicina de Botucatu,
Universidade Estadual Paulista Júlio de Mesquita Filho (UNESP). MB, RM e MLS
- Professores Assistentes Doutores, Departamento de Cirurgia e Ortopedia,
Faculdade de Medicina de Botucatu, Universidade Estadual Paulista Júlio de
Mesquita Filho (UNESP). RAY - Professor Colaborador, Departamento de
Cirurgia e Ortopedia, Faculdade de Medicina de Botucatu, Universidade
Estadual Paulista Júlio de Mesquita Filho (UNESP).

Contribuições dos autores Concepção e desenho do estudo: PABC, MB, RM Análise
e interpretação dos dados: PABC, RGJ, RAY, JVOM Coleta de dados: PABC, RGJ,
RAY, JVOM Redação do artigo: PABC, MB, RM Revisão crítica do texto: MB, RGJ,
REFP, JVOM, MLS Aprovação final do artigo*: PABC, MB, RM, MLS, RGJ, RAY,
REFP, JVOM Análise estatística: N/A. Responsabilidade geral pelo estudo:
PABC, MLS *Todos os autores leram e aprovaram a versão final submetida ao J
Vasc Bras.
\date{}{}{}
\newcommand{\volume}{00}
00\newcommand{\fpage}{000}
\newcommand{\lpage}{000}
1005201631082016Este é um artigo publicado em acesso aberto (Open
Access) sob a licença Creative Commons
Attribution, que permite uso, distribuição e reprodução em
qualquer meio, sem restrições desde que o trabalho original seja
corretamente citado.
\abstract{Resumo
Infecções de sítios cirúrgicos com envolvimento de próteses sintéticas
constituem
grande desafio para tratamento. Apresentamos o caso de uma paciente com
múltiplas comorbidades, histórico de enxerto aortobifemoral há 6 anos e
reabordagem das anastomoses femorais por reestenoses há 5 anos. Apresentou dor
inguinal esquerda e abaulamento súbitos com diagnóstico de pseudoaneurisma
femoral roto e instabilidade hemodinâmica. Foi submetida a correção emergencial
com interposição de prótese de dácron recoberta por prata e correção de grande
hérnia incisional abdominal com tela sintética ao mesmo tempo. No
pós-operatório, manteve-se por longo período sob terapia intensiva com
dificuldade de extubação. Nesse ínterim, apresentou deiscência das suturas e
fístula purulenta inguinal esquerda em contato com a prótese vascular. Optou-se
pelo tratamento conservador, com desbridamento das feridas e aplicação de
curativo a vácuo. A paciente evoluiu com melhora e cicatrização das feridas.
Essa pode se constituir em ferramenta importante em casos similares.
}
Palavras-chave: \textit{terapêutica}
\textit{cicatrização}
\textit{técnicas de fechamento de ferimentos abdominais}
\textit{infecção}
\textit{prótese vascular}
\section{}

\section{}

\section{}

\section{}

jvbDT20160028\_{}ENArticles
Conflicts of interest: No conflicts of interest declared concerning the
publication of this article.
\label{*}
Correspondence Paula Angeleli Bueno de Camargo Universidade
Estadual Paulista “Júlio de Mesquita Filho” – UNESP, Faculdade de Medicina de
Botucatu, Departamento de Cirurgia e Ortopedia Distrito de Rubião Jr., s/n CEP
18618-970 - Botucatu (SP), Brazil Tel.: +55 (14) 3880-1447 E-mail:
paula.angeleli@gmail.com
Author information PABC, RGJ, REFP and JVOM - Assistant physicians,
Departamento de Cirurgia e Ortopedia, Faculdade de Medicina de Botucatu,
Universidade Estadual Paulista Júlio de Mesquita Filho (UNESP). MB, RM and
MLS - PhDs, assistant professors, Departamento de Cirurgia e Ortopedia,
Faculdade de Medicina de Botucatu, Universidade Estadual Paulista Júlio de
Mesquita Filho (UNESP). RAY - Collaborating professor, Departamento de
Cirurgia e Ortopedia, Faculdade de Medicina de Botucatu, Universidade
Estadual Paulista Júlio de Mesquita Filho (UNESP).

Author contributions Conception and design: PABC, MB, RM Analysis and
interpretation: PABC, RGJ, RAY, JVOM Data collection: PABC, RGJ, RAY, JVOM
Writing the article: PABC, MB, RM Critical revision of the article: MB, RGJ,
REFP, JVOM, MLS Final approval of the article*: PABC, MB, RM, MLS, RGJ, RAY,
REFP, JVOM Statistical analysis: N/A. Overall responsibility: PABC, MLS *All
authors have read and approved of the final version of the article submitted
to J Vasc Bras.
This is an Open Access article distributed under the terms of the
Creative Commons Attribution License, which permits unrestricted use,
distribution, and reproduction in any medium, provided the original work is
properly cited.
\abstract{Abstract
Infections at the sites of surgery involving synthetic prostheses are
challenging
to treat. We present a case of a patient with multiple comorbidities who had
undergone an aortobifemoral bypass 6 years previously and then re-intervention
at the femoral anastomoses for restenosis 5 years previously. The patient
presented with acute left inguinal pain and swelling and was diagnosed with a
ruptured femoral pseudoaneurysm and hemodynamic instability. A repair was
conducted by interposition of a silver-coated Dacron graft in the emergency
room, and a large abdominal incisional hernia was repaired with synthetic mesh
during the same intervention. After surgery, the patient remained intubated in
intensive care for a long period. Meanwhile, she presented dehiscence of sutures
and a left inguinal purulent fistula that was in contact with the vascular
prosthesis. Conservative treatment was chosen, with debridement of wounds and
vacuum therapy. The patient improved and the wounds healed. This could be an
important tool in similar cases.
}
Keywords: \textit{treatment}
\textit{healing}
\textit{abdominal injury closure techniques}
\textit{infection}
\textit{vascular prosthesis}
\section{}

\section{}

\section{}

\section{}

\end{document}
