% Generated by jats2tex@0.11.0.0
\documentclass[numberinsection,times,10pt,spreadimages]{memoir}
\usepackage[jvb]{scielo}


\newcommand{\scieloYear}{2015}
\newcommand\scieloArticlePubIdType{DOI}
\newcommand\scieloArticleType{ORIGINAL ARTICLE}
\newcommand{\scieloArticleRef}{2015 Oct.-Dec.; 14(4):281-288}
\newcommand\scieloArticleTitle{Evaluation of great saphenous vein occlusion rate and clinical outcome in patients undergoing laser thermal ablation with a 1470-nm bare fiber laser with low linear endovenous energy density }
\newcommand\scieloArticleTitleTranslation{Avaliação da taxa de obliteração da veia safena magna e da evolução clínica de pacientes submetidos a termoablação com lase 1470 nm, fibra linear e baixa densidade de energia endovenosa linear\\Título traduzido 2}
\newcommand{\scieloArticleUrl}{http://dx.doi.org/10.1590/1677-5449.004015}
\newcommand{\scieloHeader}{Walter Junior Boim Araujo, Jorge Rufino Ribas Timi et al.}
\newcommand{\scieloAuthor}{Walter Junior Boim Araujo\scieloContributorFn{1}{xx}{1}{}, Jorge Rufino Ribas Timi\scieloContributorFn{1}{xx}{1}{}, Carlos Seme Nejm Júnior\scieloContributorFn{1}{xx}{1}{}, Filipe Carlos CaronCronos}


% jats2tex requirements
% \usepackage[document]{ragged2e}
% \usepackage{amssymb}
% \usepackage[T1]{fontenc}
% \usepackage{amstext}
% \usepackage{authblk}
%\usepackage{unicode-math}
% \usepackage{multirow}
% \usepackage{graphicx}

\begin{document}

Teste

% \title{Uso de curativo a vácuo como terapia adjuvante na cicatrização de
% sítio cirúrgico infectado}
% \author[\textsuperscript{th}
% *]{Camargo, Paula Angeleli Bueno de
% \textsuperscript{th}
% *}
% \author[\textsuperscript{th}
% ]{Bertanha, Matheus
% \textsuperscript{th}
% }
% \author[\textsuperscript{th}
% ]{Moura, Regina
% \textsuperscript{th}
% }
% \author[\textsuperscript{th}
% ]{Jaldin, Rodrigo Gibin
% \textsuperscript{th}
% }
% \author[\textsuperscript{th}
% ]{Yoshida, Ricardo de Alvarenga
% \textsuperscript{th}
% }
% \author[\textsuperscript{th}
% ]{Pimenta, Rafael Elias Farres
% \textsuperscript{th}
% }
% \author[\textsuperscript{th}
% ]{Mariúba, Jamil Victor de Oliveira
% \textsuperscript{th}
% }
% \author[\textsuperscript{th}
% ]{Sobreira, Marcone Lima
% \textsuperscript{th}
% }\affil[aff01]{Universidade Estadual Paulista “Júlio de
% Mesquita Filho”Faculdade de Medicina de
% BotucatuUniversidade Estadual
% PaulistaUniversidade Estadual Paulista “Júlio de
% Mesquita Filho” – UNESP, Faculdade de Medicina de Botucatu, Departamento de
% Cirurgia e Ortopedia, São Paulo, SP, Brasil.}\title{Using vacuum therapy as an
% adjunctive treatment for healing of
% infected surgical sites}
% \author[\textsuperscript{th}
% *]{Camargo, Paula Angeleli Bueno de
% \textsuperscript{th}
% *}
% \author[\textsuperscript{th}
% ]{Bertanha, Matheus
% \textsuperscript{th}
% }
% \author[\textsuperscript{th}
% ]{Moura, Regina
% \textsuperscript{th}
% }
% \author[\textsuperscript{th}
% ]{Jaldin, Rodrigo Gibin
% \textsuperscript{th}
% }
% \author[\textsuperscript{th}
% ]{Yoshida, Ricardo de Alvarenga
% \textsuperscript{th}
% }
% \author[\textsuperscript{th}
% ]{Pimenta, Rafael Elias Farres
% \textsuperscript{th}
% }
% \author[\textsuperscript{th}
% ]{Mariúba, Jamil Victor de Oliveira
% \textsuperscript{th}
% }
% \author[\textsuperscript{th}
% ]{Sobreira, Marcone Lima
% \textsuperscript{th}
% }\affil[aff0100]{Universidade Estadual Paulista “Júlio de
% Mesquita Filho”Faculdade de Medicina de
% BotucatuUniversidade Estadual
% PaulistaUniversidade Estadual Paulista “Júlio de
% Mesquita Filho” – UNESP, Faculdade de Medicina de Botucatu, Departamento de
% Cirurgia e Ortopedia, São Paulo, SP, Brazil.}

% \maketitle

% \setcounter{footnote}{4}
% \selectlanguage{english}

% jvbJornal Vascular BrasileiroJ. Vasc. Bras.1677-54491677-7301Sociedade
% Brasileira de Angiologia e de Cirurgia Vascular
% (SBACV)jvbDT20160028\_{}PT10.1590/1677-5449.00281601102Articles
% Conflito de interesse: Os autores declararam não haver conflitos de interesse
% que precisam ser informados.
% \label{*}
% Correspondência Paula Angeleli Bueno de Camargo Universidade
% Estadual Paulista “Júlio de Mesquita Filho” – UNESP, Faculdade de Medicina de
% Botucatu, Departamento de Cirurgia e Ortopedia Distrito de Rubião Jr., s/n CEP
% 18618-970 - Botucatu (SP), Brasil Tel.: (14) 3880-1447 E-mail:
% paula.angeleli@gmail.com
% Informações sobre os autores PABC, RGJ, REFP e JVOM - Médicos Assistentes,
% Departamento de Cirurgia e Ortopedia, Faculdade de Medicina de Botucatu,
% Universidade Estadual Paulista Júlio de Mesquita Filho (UNESP). MB, RM e MLS
% - Professores Assistentes Doutores, Departamento de Cirurgia e Ortopedia,
% Faculdade de Medicina de Botucatu, Universidade Estadual Paulista Júlio de
% Mesquita Filho (UNESP). RAY - Professor Colaborador, Departamento de
% Cirurgia e Ortopedia, Faculdade de Medicina de Botucatu, Universidade
% Estadual Paulista Júlio de Mesquita Filho (UNESP).

% Contribuições dos autores Concepção e desenho do estudo: PABC, MB, RM Análise
% e interpretação dos dados: PABC, RGJ, RAY, JVOM Coleta de dados: PABC, RGJ,
% RAY, JVOM Redação do artigo: PABC, MB, RM Revisão crítica do texto: MB, RGJ,
% REFP, JVOM, MLS Aprovação final do artigo*: PABC, MB, RM, MLS, RGJ, RAY,
% REFP, JVOM Análise estatística: N/A. Responsabilidade geral pelo estudo:
% PABC, MLS *Todos os autores leram e aprovaram a versão final submetida ao J
% Vasc Bras.
% \date{2017}{01}{05}
% \newcommand{\volume}{00}
% 00\newcommand{\fpage}{000}
% \newcommand{\lpage}{000}
% 1005201631082016Este é um artigo publicado em acesso aberto (Open
% Access) sob a licença Creative Commons
% Attribution, que permite uso, distribuição e reprodução em
% qualquer meio, sem restrições desde que o trabalho original seja
% corretamente citado.\abstract{Resumo
% Infecções de sítios cirúrgicos com envolvimento de próteses sintéticas
% constituem
% grande desafio para tratamento. Apresentamos o caso de uma paciente com
% múltiplas comorbidades, histórico de enxerto aortobifemoral há 6 anos e
% reabordagem das anastomoses femorais por reestenoses há 5 anos. Apresentou dor
% inguinal esquerda e abaulamento súbitos com diagnóstico de pseudoaneurisma
% femoral roto e instabilidade hemodinâmica. Foi submetida a correção emergencial
% com interposição de prótese de dácron recoberta por prata e correção de grande
% hérnia incisional abdominal com tela sintética ao mesmo tempo. No
% pós-operatório, manteve-se por longo período sob terapia intensiva com
% dificuldade de extubação. Nesse ínterim, apresentou deiscência das suturas e
% fístula purulenta inguinal esquerda em contato com a prótese vascular. Optou-se
% pelo tratamento conservador, com desbridamento das feridas e aplicação de
% curativo a vácuo. A paciente evoluiu com melhora e cicatrização das feridas.
% Essa pode se constituir em ferramenta importante em casos similares.
% }
% Palavras-chave: \textit{terapêutica}
% \textit{cicatrização}
% \textit{técnicas de fechamento de ferimentos abdominais}
% \textit{infecção}
% \textit{prótese vascular}
% \section{INTRODUÇÃO}

% As deiscências de incisões cirúrgicas, particularmente com envolvimento de
% próteses
% sintéticas, constituem grande desafio para tratamento\textsuperscript{th}
% . Nas próteses vasculares, em particular, há um
% microambiente favorável à produção de biofilme, que sustenta a colonização
% bacteriana e encapsula os germes, protegendo-os contra as defesas naturais do
% organismo e da terapia antibiótica\textsuperscript{th}
% . O tratamento geralmente preconizado é a remoção da prótese
% infectada\textsuperscript{th}
% . Porém, essas
% cirurgias são geralmente maiores que as prévias, uma vez que apresentam forte
% envolvimento inflamatório. Assim, o paciente deve ter boas condições clínicas
% para
% suportar uma cirurgia que pode envolver reconstruções vasculares
% extra-anatômicas
% complexas, que demandam maior tempo operatório e elevam as taxas de
% morbimortalidade. Dessa forma, pretende-se evitar a progressão da infecção,
% quadros
% isquêmicos graves decorrentes da remoção isolada da prótese e o risco de
% amputações.
% Nesse contexto, apresentamos um caso de sucessivas complicações cirúrgicas
% decorrentes de reintervenções emergenciais sobre um enxerto aortobifemoral
% antigo,
% realizado há 6 anos para o tratamento de isquemia crítica de membros inferiores.
% Retornou com degeneração da artéria femoral comum esquerda e constituiu um
% grande
% desafio terapêutico.

% \section{PARTE I - SITUAÇÃO CLÍNICA}

% Paciente do sexo feminino, branca, com 75 anos de idade, dislipidêmica,
% tabagista
% ativa com doença pulmonar obstrutiva crônica, doença renal crônica, obesidade
% mórbida e portadora de insuficiência cardíaca. Tinha histórico de cirurgia
% convencional de enxerto aortobifemoral com prótese de dácron bifurcada,
% realizada
% para o tratamento de isquemia crítica de membros inferiores 6 anos atrás, quando
% havia uma situação clínica mais favorável. O quadro de isquemia crítica inicial
% caracterizava-se por claudicação intermitente, lesões tróficas em ambos os pés
% (necrose puntiforme de pododáctilos) e dor em repouso. Como complicação daquela
% cirurgia, no pós-operatório, teve deiscência da incisão abdominal com
% consequente
% hérnia incisional.

% Um ano após a primeira cirurgia, a paciente apresentou necrose em região
% calcânea
% esquerda, que foi relacionada a uma piora da perfusão desse membro. Isso foi
% confirmado por meio de ultrassom dúplex colorido, o qual demonstrou estenose >
% 75\% nas anastomoses femorais. A correção foi realizada com reabordagem
% cirúrgica das
% anastomoses e confecção de remendo de dácron, com sucesso técnico. A evolução
% pós-operatória trouxe compensação circulatória para os membros inferiores e
% cicatrização da ferida.

% Após 5 anos de acompanhamento ambulatorial regular, deu entrada no
% pronto-socorro de
% nossa instituição queixando-se de dor forte e abaulamento súbito em região
% inguinal
% esquerda. Clinicamente, a paciente apresentou instabilidade hemodinâmica e, após
% um
% rastreamento emergencial com ultrassom dúplex colorido, visualizou-se
% pseudoaneurisma na interface entre a artéria femoral e o remendo de dácron e
% extravasamento de sangue para o retroperitônio. Foi indicado tratamento
% cirúrgico de
% emergência.

% Foi, então, realizada correção cirúrgica convencional, com remoção do remendo e
% interposição de segmento de dácron recoberto por prata entre a porção média do
% ramo
% esquerdo da prótese aortobifemoral e a bifurcação femoral. Além disso, a hérnia
% incisional abdominal prévia foi corrigida por meio de colocação de tela
% sintética no
% mesmo ato cirúrgico, pela equipe de gastrocirurgia.

% Em decorrência das várias comorbidades, a evolução pós-operatória foi
% desfavorável,
% permanecendo a paciente em regime de terapia intensiva por longo período.
% Ocorreu
% deiscência de sutura das incisões inguinal e abdominal com drenagem de secreção
% purulenta na inguinotomia esquerda. Detectou-se, por tomografia computadorizada,
% uma
% fístula de drenagem da coleção adjacente ao enxerto pela inguinotomia, sem
% exposição
% direta da prótese. Foi realizada a cultura das secreções, que se apresentou
% positiva
% para Staphylococcus epidermidis e Staphylococcus
% aureus coagulase negativo, ambos sensíveis à vancomicina. A situação
% clínica da paciente era agravada, ainda, por quadro de pneumonia, tratada com
% imipenem, tazocin e polimixina B, e pelo antecedente de doença pulmonar
% obstrutiva
% crônica, o que a manteve em ventilação mecânica invasiva por 45 dias.

% Diante desse quadro e sem condições cirúrgicas, as opções de tratamento eram:

% Antibióticos sistêmicos, desbridamentos e curativos locais;

% Lavagem contínua dos sítios infectados com antissépticos e antibióticos;

% Remoção cirúrgica de todas as próteses, apesar das condições clínicas
% desfavoráveis, seguida de reconstrução vascular extra-anatômica;

% Remoção cirúrgica de todas as próteses, apesar das condições clínicas
% desfavoráveis, e espera por delimitação isquêmica para posterior
% amputação;

% Curativo a vácuo, antibioticoterapia e observação das condições clínicas da
% paciente.

% \section{PARTE II – O QUE FOI FEITOAplicação do curativo a vácuo após o
% desbridamento das incisões
% deiscentes. (A) Troca do curativo com a colocação da esponja de poliuretano
% no leito das feridas; (B) Kit de sucção e filme plástico devidamente
% instalados; (C) Equipamento de pressão negativa fixado no leito da paciente;
% (D) Refil do frasco de drenagem do curativo a vácuo.Processo de cicatrização das
% feridas com uso do curativo a vácuo. (A)
% Resultado intermediário do processo de cicatrização, ainda em uso de
% curativo a vácuo; (B) Resultado final do processo de cicatrização das
% feridas.}

% Foi instalado curativo a vácuo nas deiscências do abdome e região inguinal
% esquerda.
% O equipamento usado foi o Sistema de Terapia V.A.C. ATS® (KCI Kinetic Concepts
% Inc,
% San Antonio, Texas, EUA) e antibioticoterapia (vancomicina e imipenem por 21
% dias)
% com observação das condições clínicas da paciente.

% Após desbridamento das feridas, foi colocada esponja de poliuretano
% exclusivamente
% dentro do leito das deiscências, com cobertura total com filme plástico (Figuras
% 1A e 1B). O tubo de sucção foi acoplado ao equipamento de pressão negativa,
% mantendo-se uma pressão de -125 mmHg de forma contínua (Figuras 1C e 1D). A
% troca dos kits de sucção foi feita a cada 3 dias. Houve drenagem de,
% aproximadamente, 50 mL/dia de secreção purulenta. A paciente evoluiu com
% estabilização das comorbidades e controle da infecção, extubação após 45 dias,
% fechamento progressivo das deiscências incisionais, granulação total em 60 dias
% com
% suspensão do uso do curativo a vácuo, curativos convencionais com hidrogel
% durante
% internação por mais 25 dias e fechamento ambulatorial das feridas após 9 meses
% de
% tratamento (Figuras 2A e 2B).

% \section{DISCUSSÃO}

% O uso de curativos com pressão negativa para tratamentos diversos é conhecido
% desde a
% antiguidade\textsuperscript{th}
% . O tratamento de
% feridas crônicas com terapia a vácuo padronizada teve início em
% 1997\textsuperscript{th}
% . Sua ação é baseada nos conceitos
% de: contração da ferida, eliminação do exsudado e do tecido inviável, estímulo à
% mitose celular, manutenção de um ambiente úmido, redução do edema tecidual,
% remoção
% de bactérias, melhora da vascularização e aceleração da
% granulação\textsuperscript{th}
% \textsuperscript{th}
% \textsuperscript{th}
% .

% Pode-se considerar a indicação de tratamento com curativo a vácuo para feridas
% com
% baixa resposta ao tratamento convencional, em que se prevê um longo período para
% cicatrização, feridas profundas e com elevada quantidade de exsudado e como
% coadjuvante a outras terapias ou intervenções. As contraindicações são: feridas
% com
% malignidade, fístulas para órgãos e cavidades, osteomielites e exposição de
% vasos
% sanguíneos com risco de sangramento.

% Pode-se, ainda, atribuir como vantagens dessa terapia a redução da inflamação e
% da
% dor resultantes da manipulação constante da ferida, isenção de contaminação por
% contato e conforto para o paciente, uma vez que não deixa odores desagradáveis.
% Em
% contrapartida, apresenta custos imediatos mais elevados, principalmente
% relacionados
% às trocas do refil e do próprio curativo em condições assépticas ao menos uma
% vez
% por semana. Entretanto, quando se somam todos os benefícios do uso do curativo a
% vácuo em comparação com os curativos convencionais, fica evidente que há uma
% relação
% de custo-efetividade com a adoção do curativo a vácuo. Com relação ao caso
% previamente apresentado, provavelmente um curativo convencional teria
% dificuldade em
% manter um ambiente propício à cicatrização, levando-se em consideração a
% presença de
% uma fístula purulenta em contato com a prótese arterial implantada e as grandes
% áreas deiscentes.

% Revisões sistemáticas\textsuperscript{th}
% \textsuperscript{th}
% \textsuperscript{th}
% e um estudo randomizado\textsuperscript{th}
% mostraram a efetividade do curativo com pressão
% negativa em várias situações, com relação à proporção de feridas cicatrizadas e
% velocidade de fechamento das mesmas, sendo particularmente efetivo em pés
% diabéticos\textsuperscript{th}
% \textsuperscript{th}
% \textsuperscript{th}
% , enxertos de pele\textsuperscript{th}
% e infecções pós-cirúrgicas\textsuperscript{th}
% \textsuperscript{th}
% \textsuperscript{th}
% . No entanto, faltam ainda estudos randomizados
% de boa qualidade e isentos de conflitos de interesse para melhor avaliar esse
% método\textsuperscript{th}
% .

% As complicações descritas como relacionadas ao uso do curativo a vácuo são
% infrequentes, sendo a maioria relacionada a dor local, hipertrofia do tecido de
% granulação e danos aos vasos sanguíneos adjacentes\textsuperscript{th}
% \textsuperscript{th}
% \textsuperscript{th}
% . Ressalta-se, então, que a esponja do curativo a vácuo
% não deve ser aplicada diretamente sobre vasos sanguíneos. Nessa situação, um
% filme
% de silicone não adesivo deve servir de anteparo como proteção de interface entre
% a
% esponja e o tecido, evitando a erosão do vaso\textsuperscript{th}
% .

% A pressão negativa nas feridas é geralmente usada de forma contínua, mas existem
% equipamentos que podem atuar de forma intermitente ou variável, não sendo
% possível
% encontrar evidências clínicas de vantagem relacionada a essa
% variável\textsuperscript{th}
% . Níveis de pressão negativa
% menores que 80 mmHg (pressão negativa) são recomendados para se obter algum
% efeito
% do tratamento\textsuperscript{th}
% . A instilação de
% fluidos no leito da ferida pode melhorar a eficiência do tratamento em alguns
% casos\textsuperscript{th}
% .

% Pode-se concluir que os curativos com pressão negativa têm recomendações bem
% estabelecidas para o tratamento de feridas com características variadas, sendo
% que
% podem apresentar redução no tempo de cicatrização de feridas, maior conforto
% para o
% paciente e raras complicações. No caso apresentado, para o qual houve aprovação
% do
% Comitê de Ética de nossa instituição e que não apresenta conflito de interesse,
% o
% curativo a vácuo foi uma importante ferramenta para o sucesso terapêutico em uma
% situação de exceção, na qual a abordagem cirúrgica para remoção das próteses
% arteriais estaria associada a um elevado risco cirúrgico e de amputação. A
% conduta
% adotada apresentou um resultado bastante satisfatório.

% jvbDT20160028\_{}ENArticles
% Conflicts of interest: No conflicts of interest declared concerning the
% publication of this article.
% \label{*}
% Correspondence Paula Angeleli Bueno de Camargo Universidade
% Estadual Paulista “Júlio de Mesquita Filho” – UNESP, Faculdade de Medicina de
% Botucatu, Departamento de Cirurgia e Ortopedia Distrito de Rubião Jr., s/n CEP
% 18618-970 - Botucatu (SP), Brazil Tel.: +55 (14) 3880-1447 E-mail:
% paula.angeleli@gmail.com
% Author information PABC, RGJ, REFP and JVOM - Assistant physicians,
% Departamento de Cirurgia e Ortopedia, Faculdade de Medicina de Botucatu,
% Universidade Estadual Paulista Júlio de Mesquita Filho (UNESP). MB, RM and
% MLS - PhDs, assistant professors, Departamento de Cirurgia e Ortopedia,
% Faculdade de Medicina de Botucatu, Universidade Estadual Paulista Júlio de
% Mesquita Filho (UNESP). RAY - Collaborating professor, Departamento de
% Cirurgia e Ortopedia, Faculdade de Medicina de Botucatu, Universidade
% Estadual Paulista Júlio de Mesquita Filho (UNESP).

% Author contributions Conception and design: PABC, MB, RM Analysis and
% interpretation: PABC, RGJ, RAY, JVOM Data collection: PABC, RGJ, RAY, JVOM
% Writing the article: PABC, MB, RM Critical revision of the article: MB, RGJ,
% REFP, JVOM, MLS Final approval of the article*: PABC, MB, RM, MLS, RGJ, RAY,
% REFP, JVOM Statistical analysis: N/A. Overall responsibility: PABC, MLS *All
% authors have read and approved of the final version of the article submitted
% to J Vasc Bras.
% This is an Open Access article distributed under the terms of the
% Creative Commons Attribution License, which permits unrestricted use,
% distribution, and reproduction in any medium, provided the original work is
% properly cited.\abstract{Abstract
% Infections at the sites of surgery involving synthetic prostheses are
% challenging
% to treat. We present a case of a patient with multiple comorbidities who had
% undergone an aortobifemoral bypass 6 years previously and then re-intervention
% at the femoral anastomoses for restenosis 5 years previously. The patient
% presented with acute left inguinal pain and swelling and was diagnosed with a
% ruptured femoral pseudoaneurysm and hemodynamic instability. A repair was
% conducted by interposition of a silver-coated Dacron graft in the emergency
% room, and a large abdominal incisional hernia was repaired with synthetic mesh
% during the same intervention. After surgery, the patient remained intubated in
% intensive care for a long period. Meanwhile, she presented dehiscence of sutures
% and a left inguinal purulent fistula that was in contact with the vascular
% prosthesis. Conservative treatment was chosen, with debridement of wounds and
% vacuum therapy. The patient improved and the wounds healed. This could be an
% important tool in similar cases.
% }
% Keywords: \textit{treatment}
% \textit{healing}
% \textit{abdominal injury closure techniques}
% \textit{infection}
% \textit{vascular prosthesis}
% \section{INTRODUCTION}

% Dehiscence of surgical incisions is a major challenge to treat, particularly
% when
% they involve synthetic prostheses.\textsuperscript{th}
% Vascular prostheses in particular provide a
% microenvironment that is conducive to production of biofilm, which supports
% bacterial colonization and encapsulates the germs, protecting them from the
% body’s natural defenses and from antibiotics.\textsuperscript{th}
% The treatment that is generally recommended is
% removal of the infected prosthesis.\textsuperscript{th}
% However, these operations are generally of a larger
% scale than the original surgery because of the significant inflammatory
% involvement. The patient must therefore be in good clinical conditions to
% withstand a surgical operation that may include complex extra-anatomic vascular
% reconstruction, requiring longer operating times and elevating morbidity and
% mortality rates. The objectives are to prevent continued development of the
% infection, to avoid severe ischemia resulting from simply removing the
% prosthesis, and to reduce the risk of amputations. Against this background, we
% present a case of successive surgical complications triggered by emergency
% reinterventions to repair a previous aortobifemoral bypass that had been
% constructed 6 years previously to treat critical lower limb ischemia. The
% patient presented with degeneration of the left common femoral artery and the
% case was a significant therapeutic challenge.

% \section{PART I – CLINICAL SITUATION}

% The patient was a white, 75-year-old, female, active smoker with dyslipidemia,
% chronic obstructive pulmonary disease, chronic kidney disease, morbid obesity,
% and heart failure. Six years previously she had undergone conventional surgery
% to construct an aortobifemoral bypass using a bifurcated dacron graft to treat
% critical lower limb ischemia, when her clinical situation had been less
% unfavorable. Her initial critical ischemia had presented with intermittent
% claudication, trophic ulcers on both feet (necrosis punctiform of the toes), and
% pain at rest. The patient had already suffered a complication during the
% postoperative period of that operation: dehiscence of the abdominal incision,
% causing an incisional hernia.

% One year after the first operation, the patient presented once more, with
% necrosis of the left heel, which was related to deterioration of perfusion to
% the left lower limb. This was confirmed with color duplex ultrasound, which
% showed > 75\% stenosis of the femoral anastomoses. A surgical reintervention
% was conducted to repair the anastomoses with a dacron patch and was a technical
% success. Postoperative recovery was accompanied by compensation of the lower
% limb circulation and the wound healed.

% After 5 years of regular outpatients follow-up, the patient was admitted to the
% emergency room at our institution complaining of severe pain and swelling in the
% left inguinal region. Clinically, the patient was hemodynamically unstable and
% emergency duplex ultrasound screening revealed a pseudoaneurysm at the interface
% between the femoral artery and the dacron patch, with blood leaking into the
% retroperitoneal space. The patient was prepared for emergency surgical
% treatment.

% The pseudoaneurysm was repaired during a conventional surgical operation, with
% removal of the patch and interposition of a silver-coated dacron segment between
% the medial portion of the left branch of the aortobifemoral graft and the
% femoral bifurcation. During the same operation, the gastric surgery team
% repaired the preexisting incisional hernia by placement of a synthetic mesh.

% The multiple comorbidities caused unfavorable postoperative progress and the
% patient was kept in intensive care for a long period of time. The inguinal and
% abdominal sutures underwent dehiscence and there were purulent secretions
% draining from the left inguinotomy. On computed tomography, a fistula was
% detected draining pus adjacent to the graft via the inguinotomy, to which the
% prosthesis was not directly exposed. The secretions were cultured and found to
% be positive for Staphylococcus epidermidis and coagulase
% negative Staphylococcus aureus, both sensitive to vancomycin.
% The patient’s clinical situation was further aggravated by pneumonia, which was
% treated with imipenem, tazocin and polymyxin B, and by the preexisting chronic
% obstructive pulmonary disease, which kept her on invasive mechanical ventilation
% for 45 days.

% Faced with this clinical status and without the necessary conditions for
% surgery,
% the treatment options were as follows:

% Systemic antibiotics, debridement and local dressings;

% Continuous cleaning of the infected sites with antiseptics and
% antibiotics;

% Surgical removal of all prostheses, despite the unfavorable clinical
% conditions, followed by extra-anatomic vascular reconstruction;

% Surgical removal of all prostheses, despite the unfavorable clinical
% conditions, and amputation after delimitation of ischemia;

% Vacuum dressing, antibiotic therapy and observation of the patient’s
% clinical condition.

% \section{PART II – WHAT WAS DONEApplication of vacuum dressing after debridement
% of dehiscent
% incisions. (A) Dressing change, placing polyurethane foam on wound beds;
% (B) Suction kit and plastic film correctly fitted; (C) Negative pressure
% unit attched to patient’s bed; (D) Replacement vacuum dressing drainage
% jar.Wound healing process with vacuum dressing. (A) Intermediate stage of
% healing process, vacuum dressing still in use; (B) Final result of wound
% healing process.}

% A vacuum dressing was applied to the dehiscences at the abdomen and left
% inguinal
% region. The V.A.C. ATS Therapy System® (KCI Kinetic Concepts Inc, San Antonio,
% Texas, United States) was used in combination with antibiotic therapy
% (vancomycin and imipenem for 21 days) and observation of the patient’s clinical
% conditions.

% After debridement of the wounds, polyurethane foam was placed exclusively within
% the dehiscence bed and completely covered with plastic film (Figures 1A and 1B).
% The suction tube was fitted to the negative
% pressure unit, maintaining a pressure of -125 mmHg continuously (Figures 1C and
% 1D). Suction kits were changed every 3 days.
% Approximately 50 mL/day of purulent secretions were drained. The patient’s
% comorbidities stabilized and the infection was controlled. She was extubated
% after 45 days, and the incisional dehiscences closed progressively, achieving
% total granulation at 60 days, when the vacuum dressing was withdrawn, followed
% by conventional hydrogel dressings for a further 25 days while still in
% hospital, and the wounds closed during outpatients follow-up 9 months after
% starting treatment (Figures 2A and
% 2B).

% \section{DISCUSSION}

% Utilization of dressings with negative pressure has been known since ancient
% times for a range of treatments.\textsuperscript{th}
% Treatment of chronic wounds using standardized vacuums
% began in 1997.\textsuperscript{th}
% The effects
% are based on the following concepts: contraction of the wound, removal of
% exudate and non-viable tissues, stimulation of cell mitosis, maintenance of a
% humid environment, reduction of tissue edema, removal of bacteria, improvement
% of vascularization, and acceleration of granulation.\textsuperscript{th}
% \textsuperscript{th}
% \textsuperscript{th}

% Treatment with vacuum dressings may be indicated for wounds that do not respond
% well to conventional treatment, when a long healing period is predicted, for
% deep wounds and those with high quantities of exudate, and as a supplementary
% technique in combination with other treatments or interventions.
% Contraindications are: wounds with malignancies, fistulae into organs and
% cavities, osteomyelitis, and exposure of blood vessels at risk of bleeding.

% Advantages of this treatment include reduction of inflammation and pain caused
% by
% constant manipulation of the wound, exclusion of contamination by contact, and
% improved patient comfort, since it does not produce unpleasant odors. On the
% other hand, immediate costs are high, primarily related to changing the refill
% and the dressing itself under aseptic conditions at least once a week. However,
% when all the benefits of using a vacuum dressing compared with conventional
% dressings are added up, it is clear that the choice of a vacuum dressing offers
% a good cost-effectiveness ratio. With regard to the case described above, it is
% probable that conventional dressings would have been unlikely to have
% successfully maintained an environment conducive to healing, considering the
% presence of a purulent fistula in contact with the prosthetic arterial graft and
% the large areas of dehiscence.

% Systematic reviews\textsuperscript{th}
% \textsuperscript{th}
% \textsuperscript{th}
% and a randomized study\textsuperscript{th}
% show the effectiveness of negative pressure
% dressings in a range of situations, in terms both of the proportion of wounds
% healed and the speed with which they close, and they are particularly effective
% for diabetic feet,\textsuperscript{th}
% \textsuperscript{th}
% \textsuperscript{th}
% skin grafts\textsuperscript{th}
% and infections after surgery.\textsuperscript{th}
% \textsuperscript{th}
% \textsuperscript{th}
% However, there is still a lack of good quality
% randomized studies free from conflicts of interest that would allow the method
% to be evaluated more thoroughly.\textsuperscript{th}

% Complications that have been described in relation to vacuum dressings are
% uncommon and the majority are related to local pain, hypertrophy of granulation
% tissue, and damage to adjacent blood vessels.\textsuperscript{th}
% \textsuperscript{th}
% \textsuperscript{th}
% It should however be stressed that the dressing
% foam should not be placed in direct contact with blood vessels. In such
% situations, a non-adhesive silicone film should be used as a pre-preparation as
% a protection for the interface between foam and tissue, preventing erosion of
% the vessel.\textsuperscript{th}

% In general the negative pressure is applied to wounds continuously, but there
% are
% systems that can provide intermittent or variable action, although there is no
% clinical evidence that this variable offers advantages.\textsuperscript{th}
% Negative pressure levels below 80 mmHg
% (negative pressure) are recommended to obtain treatment
% effectiveness.\textsuperscript{th}
% In some cases instillation
% of fluids to the wound bed can improve efficacy.\textsuperscript{th}

% It can be concluded that there are well-established recommendations for using
% negative pressure dressing to treat wounds with a variety of characteristics and
% they can offer reductions in the time taken for wounds to heal, combined with
% greater patient comfort and rare complications. In the case described here,
% which was approved by our institutional Ethics Committee and does not involve
% any conflicts of interests, the vacuum dressing was an important tool for
% achieving therapeutic success in an exceptional situation, in which an
% additional surgical operation to remove the arterial prostheses would have
% incurred a high surgical risk and high risk of amputation. The conduct chosen
% achieved very satisfactory results.

\end{document}
