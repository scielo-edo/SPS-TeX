\begin{multicols}{3}

\scieloSectionContainer{INTRODUÇÃO}
\par{}As disfunções sexuais femininas (DFSs) são
consideradas, pela Organização Mundial de
Saúde (OMS), um problema de saúde pública,
pois afetam, em curto ou longo prazo, a vida
social, psicológica, doméstica, ocupacional e
física das mulheres e de seus companheiros.\textsuperscript{1}
\par{}As DSFs consistem em múltiplas desor-
dens, como: distúrbio da excitação feminina,
distúrbio do desejo sexual hipoativo, transtor-
no sexual do orgasmo feminino, dispareunia e
vaginismo. 2 Estas desordens são classificadas
dentro de categorias diagnósticas que incluem
o desejo ou libido, a excitação, a dor ou des-
conforto e a inibição do orgasmo. 3,4 Elas são
caracterizadas como distúrbios multicausais e
multidimensionais, combinando determinan-
tes biológicos, psicológicos e interpessoais.\textsuperscript{2}
\par{}Estas disfunções têm alta prevalência,
atingindo cerca 67,9\% 5,6 das mulheres no mun-
do. Quanto aos índices em diferentes países,
estima-se que as DSFs estejam presentes em
cerca de 30 a 50\% das americanas, em mais de
50\% das asiáticas e em 30\% das brasileiras. 7-9
Em populações específicas, como mulheres
com diabetes, 10 com Parkinson 11 e que tive-
ram infarto agudo do miocárdio, 12 os valores
são mais expressivos, variando de 59\%,75% e
50\%, respectivamente.
\par{}A atuação da fisioterapia no tratamento
das DSFs é dirigida à melhora da mobilidade
da musculatura do assoalho pélvico e ao alí-
vio da dor pélvica e/ou abdominal. Para isso,
diversas terapêuticas são utilizadas como, por
exemplo, exercícios para os músculos do assoa-
lho pélvico, 9 eletroterapia 13 e terapia manual. 14
No entanto diante dessa abrangência, torna-se
necessária a busca por evidências científicas
sobre essas técnicas ou métodos para, poste-
riormente, determinar as condutas a serem uti-
lizadas no processo de redução de tais queixas.

\scieloSectionContainer{OBJETIVO}

\par{}A fim de proporcionar ao profissional fisio-
terapeuta conhecimento sintético, o objetivo
deste estudo é realizar uma revisão sistemáti-
ca sobre as diferentes técnicas de fisioterapia
utilizadas para tratamento das DSFs.

\scieloSectionContainer{MÉTODOS}

\scieloSubSectionContainer{Fonte de dados e pesquisa}

\par{}O planejamento para esta revisão foi
baseado nas orientações do Preferred Re-
porting Items for Systematic Reviews and
Meta-Analyses (recomendações PRISMA).
\par{}Os dados foram testados quanto à normalidade, usan-
do o teste de Kolmogorov-Smirnov, e expressos como
média±desvio padrão (DP). O limite superior da normali-
dade foi determinado como a média±2 DP. A concordân-
cia entre dois diferentes métodos e os observadores para
a mensuração do TRC foi calculada usando o método de
Bland-Altman. As correlações entre o TRC e os índices de
perfusão periférica foram avaliadas com a correlação de
Pearson. Considerando-se a falta de dados na literatura,
não se realizou cálculo do tamanho da amostra, porém foi
testada sua adequação. (10) Foi considerado como signifi-
cante um valor de p<0,05.

\lipsum
\scieloSectionContainer{RESULTS}

\scieloSubSectionContainer{SPECIFIC COMPOSITION, RICHNESS AND ABUNDANCE}

\par{}A total of 110 taxa were identified, belonging to
4 Divisions (Figure 2; Table 1): Diatoms (55 taxa;
25 species), Dinoflagellates (51 taxa; 25 species),
Cyanobacteria (2 taxa) and Chlorophythes (2 taxa).
\par{}Richness, Abundance, Diversity Index and Evenness
results are presented as the mean values of each sampling
period, as the non-parametric tests revealed no differences
(p < 0.05) between sampling stations. Table 2 presents
information on these variables in each sampling period.


\lipsum

\scieloImageContainerOneCol{0.5\textwidth}{img1.jpg}{meu titulo}

%\begin{center}
%{
%\vspace{8mm}
%\centerline{
%\includegraphics[width=\maxwidth{0.5\textwidth}]{img1.jpg}
%}
%\vspace{8mm}
%}
%\end{center}

\lipsum

\scieloImageContainerTwoCol{img2.jpg}{\textbf{Figure 1:} A specimen of each species was embedded in paraffin in its entirety, longitudinally sectioned}

\lipsum
\lipsum

\scieloImageContainerTwoCol{img2.jpg}{\textbf{Figure 1:} A specimen of each species was embedded in paraffin in its entirety, longitudinally sectioned}

\lipsum
\lipsum
\lipsum[1]



\begin{scieloReferencesContainer}[REFERÊNCIAS]

\scieloReferencesItem{Occhino JA, Trabuco EC, Heisler CA, Klingele CJ,
Gebhart JB. Validation of a visual analog scale form
of the pelvic organ prolapse/urinary incontinence
sexual function questionnaire 12. Female Pelvic Med
Reconstr Surg. 2011;17(5):246-8.}

\scieloReferencesItem{Rosen R, Brown C, Heiman J, Leiblum S, Meston C,
Shabsigh R, et al. The Female Sexual Function Index
(FSFI): a multidimensional self-report instrument
for the assessment of female sexual function. J Sex
Marital Ther. 2000;26(2):191-208. DOI: http://dx.doi.
org/10.1080/009262300278597}

\scieloReferencesItem{Basson R. The female sexual response: a different
model. J Sex Marital Ther. 2000;26(1):51-65. DOI:
http://dx.doi.org/10.1080/009262300278641}

\scieloReferencesItem{Safarinejad MR. Female sexual dysfunction in
a population-based study in Iran: prevalence
and associated risk factors. Int J Impot Res.
2006;18(4):382-95. DOI: http://dx.doi.org/10.1038/
sj.ijir.3901440}

\scieloReferencesItem{Cabral PU, Canário AC, Spyrides MH, Uchôa
SA, Eleutério J Jr, Amaral RL, et al. Influence of
menopausal symptoms on sexual function in
middle-aged women. Rev Bras Ginecol Obstet.
2012;34(7):329-34. DOI: http://dx.doi.org/10.1590/
S0100-72032012000700007}

\scieloReferencesItem{Oksuz E, Malhan S. Reliability and validity of the
Female Sexual Function Index in Turkish population.
Sendrom. 2005;17(7):54-60.}

\scieloReferencesItem{Lobos AT, Lee S, Menon K. Capillary refill time and cardiac output in
children undergoing cardiac catheterization. Pediatr Crit Care Med.
2012;13(2):136-40.}

\scieloReferencesItem{Otieno H, Were E, Ahmed I, Charo E, Brent A, Maitland K. Are bedside
features of shock reproducible between different observers? Arch Dis
Child. 2004;89(10):977-9.}

\scieloReferencesItem{Cebrià F (2008) Organization of the nervous system in the model
planarian Schmidtea mediterranea: An immunocytochemical study. Neurosci Res 61:375-384.}

\scieloReferencesItem{Chien P and Koopowitz H (1972) The ultrastructure of neuromuscular systems in Notoplana acticola, a free-living polyclad flatworm. Z Zellforsch Mikroskop Anat 133:277-288.}


\scieloReferencesItem{Chien PK and Koopowitz H (1977) Ultrastructure of nerve plexus
in flatworms. III. The infra-epithelial nervous system. Cell
Tissue Res 176:335-347.}

\scieloReferencesItem{Day TA, Maule AG, Shaw C and Pax RA (1997) Structure-activity relationships of FMRFamide-related peptides contracting Schistosoma mansoni muscle. Peptides 18:917-921.}

\scieloReferencesItem{Egger B, Gschwentner R and Rieger R (2007) Free-living
flatworms under the knife: Past and present. Dev Genes Evol
217:89-104.}

\scieloReferencesItem{Egger B, Lapraz F, Tomiczek B, Müller S, Dessimoz C, Girstmair
J, Skunca N, Rawlinson KA, Cameron CB, Beli E et al.
(2015) A transcriptomic-phylogenomic analysis of the evolutionary relationships of flatworms. Curr Biol 25:1-7.}

\scieloReferencesItem{Ehlers U (1985) Das Phylogenetische System der Plathelminthes.
Gustav Fischer Verlag, Stuttgart, 317 pp.}

\scieloReferencesItem{Fernandes MC, Alvares EP, Gama P and Silveira M (2003) Serotonin in the nervous system of the head region of the land planarian Bipalium kewense. Tissue Cell 35:479-486.}

\scieloReferencesItem{Forest DL and Lindsay SM (2008) Observations of serotonin and
FMRFamide-like immunoreactivity in palp sensory structures and the anterior nervous system of spionid polychaetes. J Morphol 269:544-551.}

\scieloReferencesItem{Girstmair J, Schnegg R, Telford MJ and Egger B (2014) Cellular
dynamics during regeneration of the flatworm Monocelis sp. (Proseriata, Platyhelminthes). Evo Devo 5:e37.}

\scieloReferencesItem{Golding DM (1992) Polychaeta: Nervous system. In: Harrison
FW and Gardiner SL (eds) Microscopic Anatomy of Invertebrates. Vol. 7. Wiley-Liss, New York, pp 155-179.}

\scieloReferencesItem{Gustafsson MKS, Halton DW, Kreshchenko ND, Movsessian SO,
Raikova OI, Reuter M and Terenina NB (2002) Neuropeptides in flatworms. Peptides 23:2053-2061.}

\scieloReferencesItem{Hadenfeldt D (1929) Das Nervensystem von Stylochoplana
maculata und Notoplana atomata. Z wiss Zool 133:586638.}

\scieloReferencesItem{Halton DW and Gustafsson MKS (1996) Functional morphology
of the platyhelminth nervous system. Parasitology
113:S47-S72.}

\end{scieloReferencesContainer}




\end{multicols}

