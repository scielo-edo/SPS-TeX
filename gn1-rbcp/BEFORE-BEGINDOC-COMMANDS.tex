\makeatletter
\def\@fnsymbol#1{
     \ensuremath{
         \ifcase#1
\or 1
\or 2
         \fi
     }
}
\renewcommand{\thefootnote}{\fnsymbol{footnote}}
\makeatother

\newcommand\scieloDefineFootnotes{
\footnotetext[1]{Universidade Federal Fluminense, Niterói, RJ, Brasil.}
\footnotetext[2]{Sociedade Brasileira de Cirurgia Plástica, Rio de Janeiro, RJ, Brasil.}
%\footnotetext[2]{\\\textbf{Endereço para correspondência}\\Vanessa de Paula Tiago.\\Universidade Federal do Triângulo Mineiro. Departamento de Pediatria da Universidade Federal do Triângulo Mineiro. Av. Frei Paulino, no 30, Bairro Abadia\\Uberaba - MG. Brasil. CEP: 38025-180.}
}

\newcommand{\scieloYear}{2015}
\newcommand\scieloArticleType{Artigo de Revisão}
\newcommand{\scieloArticleRef}{2015;30(4):00-00}
\newcommand{\scieloHeaderOdd}{Condutas atuais na prevenção da hipertrofia cicatricial pós-operatória}
\newcommand{\scieloHeaderEven}{Faveret PLS et al.}
\newcommand\scieloArticleTitle{Condutas atuais na prevenção da hipertrofia cicatricial pós-operatória}
\newcommand\scieloArticleTitleTranslation{Current management for the prevention of postoperative scar hypertrophy}
\newcommand{\scieloAuthor}{PEDRO LEONARDO SANCHES FAVERET\textsuperscript{1,2 *}\\KARIN SOARES GONÇALVES CUNHA\textsuperscript{1}}
\newcommand{\scieloArticleDOI}{10.5935/2177-1235.2014RBCP0070}
\newcommand{\scieloReceivedApproved}{Artigo submetido: 14/12/2014.\\Artigo aceito: 26/5/2015.\\}
\newcommand{\scieloCorrespondence}{\scieloCorrespondenceContainer{Autor correspondente:}{Pedro Leonardo Sanches Faveret\\Rua Marquês de Paraná, 303, Niterói, RJ, Brasil\\CEP 24033-900\\E-mail: pedrofaveret@id.uff.br}}

\newcommand{\scieloRenderAbstract}{
\scieloAbstractContainer{RESUMO}{\textbf{Introdução:} Considerando um número estimado de cerca de
51 milhões de cirurgias a cada ano apenas nos EUA, podemos
dizer que a hipertrofia cicatricial é um problema relevante,
já que uma cicatriz fina, de boa qualidade, pode ser a linha
divisória entre um bom resultado e uma cirurgia malsucedida.
O objetivo é fazer uma revisão bibliográfica acerca dos métodos
de tratamento não invasivos atualmente disponíveis para a
prevenção da hipertrofia cicatricial pós-cirúrgica e discutir
a sua eficácia baseada em evidências. \textbf{Método:} Foi realizada
uma pesquisa nas bases de dados Pubmed, Lilacs e SciELO,
utilizando os termos “scar prevention” and “hypertrophic
scars”, por ensaios clínicos, meta-análises e artigos de revisão
publicados a partir de 2004, em inglês ou português. \textbf{Resultados e Conclusões:} Foram encontrados vários trabalhos utilizando
o silicone, proporcionando alguma evidência acerca da sua
eficácia; foram encontrados apenas três ensaios clínicos
prospectivos relacionados ao uso do Contractubex ® ; dois
ensaios clínicos prospectivos, controlados, randomizados,
sendo apenas um deles duplo-cego, com o imiquimode a 5\%; foi
encontrado apenas um ensaio clínico bem desenhado utilizando
o esparadrapo microporoso e outro trabalho relacionado ao uso
da vitamina E, que não mostrou bons resultados; não foram
encontrados ensaios clínicos sobre o uso da massagem e da
pressão local. Apesar das deficiências dos estudos, o silicone
é considerado a primeira opção na prevenção da hipertrofia
cicatricial pós-cirúrgica. Não há evidências que comprovem a
eficácia do esparadrapo microporoso, da massagem, da pressão
local, do Contractubex, do imiquimode a 5\% e da vitamina E.}{Descritores:}{Cicatriz hipertrófica; Cicatriz hipertrófica/ Prevenção \& Controle; Cicatrização; Silicones; Vitamina E.}
}

\newcommand{\scieloRenderAbstractTranslations}{
\scieloAbstractContainerTranslation{ABSTRACT}{\textbf{Objective}: Current paper evaluates admittance of risk-classified cases in two hospital emergency services. \textbf{Methods}: The
exploratory, descriptive and quantitative research was undertaken between March and May 2013, with 47 nursing professionals at
two hospital emergency units in the state of Paraná, Brazil, who answered the questionnaire. \textbf{Results}: Acceptance of Risk-classified
Cases was reported hazardous at the two units; the lowest rates refer to issues on the place the accompanying person would
stay and to discussions on the flowchart. The best assessment occurred in the attendance of less serious cases. \textbf{Conclusion}: The hazardous assessment at the two health units was mainly due to the non-compliance with certain basic principles of its guidelines of the Acceptance of Risk-classified Cases.}{Keywords:}{User Embracement; Nursing; Emergency Service, Hospital.}
}

\newcommand{\scieloInstitution}{Instituição: Hospital Universitário Antônio Pedro. Departamento de Cirurgia Geral de Especializada, Niterói, RJ, Brasil.}