\begin{multicols}{2}

\scieloSectionContainer{INTRODUÇÃO}
\par{}Uma cicatriz hipertrófica, por definição, é uma
“cicatriz avermelhada, elevada, por vezes pruriginosa,
e confinada às bordas da incisão original.” 1 .Podem ser
dolorosas, causar limitação no movimento articular,
estresse psicológico e prejuízo ao convívio social.
Considerando que, apenas nos EUA, são realizadas
cerca de 51 milhões de cirurgias por ano 2 , podemos
dizer que a hipertrofia cicatricial pós-operatória é um
problema relevante, no qual uma cicatriz fina, de boa
qualidade, pode ser a linha divisória entre um bom
resultado e uma cirurgia malsucedida (Figura 1).
\par{}Diante da importância do problema, estão
disponíveis vários métodos não invasivos a fim de
prevenir a hipertrofia das cicatrizes lineares pós-
operatórias: a massagem, a pressão local, o esparadrapo
microporoso, o silicone, as formulações contendo
extrato de cebola (Allium cepa), a vitamina E e o
Imiquimode a 5%. Assim, o objetivo deste trabalho é
fazer uma revisão bibliográfica acerca dos métodos de
tratamento não invasivos atualmente disponíveis para
a prevenção da hipertrofia cicatricial pós-cirúrgica e
discutir a sua eficácia baseada em evidências.

\scieloSectionContainer{MÉTODO}

\par{}Foi realizada uma pesquisa nas bases de dados
Pubmed, Lilacs e SciELO, utilizando os termos “scar
prevention” and “hypertrophic scars”, por trabalhos
publicados a partir de 2004 em inglês e português. Os critérios utilizados para a seleção (inclusão)
dos artigos foram: ensaios clínicos, meta-análises e
artigos de revisão que abordassem terapias tópicas,
não invasivas. Os ensaios clínicos prospectivos mais
relevantes para o tema citados nos artigos de revisão,
mas não relacionados na pesquisa inicial por terem
data de publicação anterior a 2004, também foram
incluídos. Foram excluídos os trabalhos publicados em
outros idiomas, os experimentais, os que tratavam de
cicatrizes de queimaduras, os referentes a tratamentos
de hipertrofias cicatriciais, os infiltrativos e o laser.

\lipsum

\scieloSubSectionContainer{Massagem}

\par{}Um estudo publicado em 2010, que pretendia
validar a escala POSAS ® (Patient and Observer Scar
Assessment Scale) para a avaliação de cicatrizes
faciais, dividiu 30 pacientes em dois grupos, tratando
o primeiro grupo com silicone gel e o segundo grupo
com massagem local. Após dois meses, submeteu-
os à avaliação pela escala POSAS ® , obtendo boas
pontuações, de forma semelhante, nos dois grupos 3 .

\scieloSubSectionContainer{Esparadrapo microporoso}

\par{}Em 1995 foi publicado por Reiffel 5 um estudo
prospectivo com 64 pacientes onde o esparadrapo
microporoso foi aplicado no sentido longitudinal das
cicatrizes e mantido no local por dois meses ou mais,
sendo trocado sempre que necessário. O autor concluiu
que “a maioria dos pacientes demonstrou satisfação
completa”.
\par{}Em 2005 foi publicado um ensaio clínico,
randomizado, com grupo controle, com 70 pacientes
operadas de cesariana no qual o esparadrapo
microporoso foi aplicado na cicatriz logo que as suturas
foram removidas. O estudo concluiu que o esparadrapo
microporoso reduziu significativamente o volume da
cicatriz, avaliado por ultrassonografia, em média 0,16
cm 3 (95\% CI, 0,05 a 0,36). O efeito do tratamento foi maior
quando considerado apenas aquelas mulheres que
foram complacentes com o uso da fita de esparadrapo
nas 12 semanas de acompanhamento 6 . Quatro pacientes
(12\%) apresentaram reação adversa (rash cutâneo)
no local de contato da fita dentro nas primeiras 6
semanas de uso e foram excluídas do estudo. O uso do
esparadrapo foi descontinuado e a reação local cessou
espontaneamente. Ao final, nenhuma das 39 pacientes
que completaram as 12 semanas de tratamento e os
6 meses de acompanhamento desenvolveu cicatriz
hipertrófica, comparado com 12 pacientes (41\%) do
grupo controle.


\lipsum

\scieloImageContainerOneCol{0.5\textwidth}{img1.jpg}{meu titulo}

%\begin{center}
%{
%\vspace{8mm}
%\centerline{
%\includegraphics[width=\maxwidth{0.5\textwidth}]{img1.jpg}
%}
%\vspace{8mm}
%}
%\end{center}

\lipsum

\scieloImageContainerTwoCol{img2.jpg}{\textbf{Figure 1:} A specimen of each species was embedded in paraffin in its entirety, longitudinally sectioned}

\lipsum
\lipsum

\scieloImageContainerTwoCol{img2.jpg}{\textbf{Figure 1:} A specimen of each species was embedded in paraffin in its entirety, longitudinally sectioned}

\lipsum
\lipsum
\lipsum[1]

\begin{scieloReferencesContainer}[REFERÊNCIAS]

\scieloReferencesItem{Peacock EE Jr, Madden JW, Trier WC. Biologic basis for the
treatment of keloids and hypertrophic scars. South Med J.
1970;63(7):755-60. PMID: 5427162}

\scieloReferencesItem{Centers For Disease Control and Prevention. FastStats - Inpatient
Surgery [Citado 28 Out 2015]. Disponível em: http://www.cdc.gov/
nchs/fastats/inpatient-surgery.htm}

\scieloReferencesItem{Bianchi FA, Roccia F, Fiorini P, Berrone S. Use of Patient and Ob-
server Scar Assessment Scale for evaluation of facial scars treated
with self-drying silicone gel. J Craniofac Surg. 2010;21(3):719-23.
DOI: http://dx.doi.org/10.1097/SCS.0b013e3181d841af}

\scieloReferencesItem{Shin TM, Bordeaux JS. The role of massage in scar management:
a literature review. Dermatol Surg. 2012;38(3):414-23. DOI: http://
dx.doi.org/10.1111/j.1524-4725.2011.02201.x}

\scieloReferencesItem{Reiffel RS. Prevention of hypertrophic scars by long-term paper
tape application. Plast Reconstr Surg. 1995;96(7):1715-8. DOI:
http://dx.doi.org/10.1097/00006534-199512000-00031}

\scieloReferencesItem{Atkinson JA, McKenna KT, Barnett AG, McGrath DJ, Rudd M. A ran-
domized, controlled trial to determine the efficacy of paper tape in pre-
venting hypertrophic scar formation in surgical incisions that traverse
Langer’s skin tension lines. Plast Reconstr Surg. 2005;116(6):1648-56.
DOI: http://dx.doi.org/10.1097/01.prs.0000187147.73963.a5}

\scieloReferencesItem{Ahn ST, Monafo WW, Mustoe TA. Topical silicone gel for the
prevention and treatment of hypertrophic scar. Arch Surg.
1991;126(4):499-504. PMID: 2009067 DOI: http://dx.doi.org/10.1001/
archsurg.1991.01410280103016}

\scieloReferencesItem{Cruz-Korchin NI. Effectiveness of silicone sheets in the preven-
tion of hypertrophic breast scars. Ann Plast Surg. 1996;37(4):345-8.
DOI: http://dx.doi.org/10.1097/00000637-199610000-00001}

\scieloReferencesItem{Gold MH, Foster TD, Adair MA, Burlison K, Lewis T. Prevention
of hypertrophic scars and keloids by the prophylactic use of topi-
cal silicone gel sheets following a surgical procedure in an office
setting. Dermatol Surg. 2001;27(7):641-4.}

\scieloReferencesItem{Chien P and Koopowitz H (1972) The ultrastructure of neuromuscular systems in Notoplana acticola, a free-living polyclad flatworm. Z Zellforsch Mikroskop Anat 133:277-288.}


\scieloReferencesItem{Chien PK and Koopowitz H (1977) Ultrastructure of nerve plexus
in flatworms. III. The infra-epithelial nervous system. Cell
Tissue Res 176:335-347.}

\scieloReferencesItem{Day TA, Maule AG, Shaw C and Pax RA (1997) Structure-activity relationships of FMRFamide-related peptides contracting Schistosoma mansoni muscle. Peptides 18:917-921.}

\scieloReferencesItem{Egger B, Gschwentner R and Rieger R (2007) Free-living
flatworms under the knife: Past and present. Dev Genes Evol
217:89-104.}

\scieloReferencesItem{Egger B, Lapraz F, Tomiczek B, Müller S, Dessimoz C, Girstmair
J, Skunca N, Rawlinson KA, Cameron CB, Beli E et al.
(2015) A transcriptomic-phylogenomic analysis of the evolutionary relationships of flatworms. Curr Biol 25:1-7.}

\scieloReferencesItem{Ehlers U (1985) Das Phylogenetische System der Plathelminthes.
Gustav Fischer Verlag, Stuttgart, 317 pp.}

\scieloReferencesItem{Fernandes MC, Alvares EP, Gama P and Silveira M (2003) Serotonin in the nervous system of the head region of the land planarian Bipalium kewense. Tissue Cell 35:479-486.}

\scieloReferencesItem{Forest DL and Lindsay SM (2008) Observations of serotonin and
FMRFamide-like immunoreactivity in palp sensory structures and the anterior nervous system of spionid polychaetes. J Morphol 269:544-551.}

\scieloReferencesItem{Girstmair J, Schnegg R, Telford MJ and Egger B (2014) Cellular
dynamics during regeneration of the flatworm Monocelis sp. (Proseriata, Platyhelminthes). Evo Devo 5:e37.}

\scieloReferencesItem{Golding DM (1992) Polychaeta: Nervous system. In: Harrison
FW and Gardiner SL (eds) Microscopic Anatomy of Invertebrates. Vol. 7. Wiley-Liss, New York, pp 155-179.}

\scieloReferencesItem{Gustafsson MKS, Halton DW, Kreshchenko ND, Movsessian SO,
Raikova OI, Reuter M and Terenina NB (2002) Neuropeptides in flatworms. Peptides 23:2053-2061.}

\scieloReferencesItem{Hadenfeldt D (1929) Das Nervensystem von Stylochoplana
maculata und Notoplana atomata. Z wiss Zool 133:586638.}

\scieloReferencesItem{Halton DW and Gustafsson MKS (1996) Functional morphology
of the platyhelminth nervous system. Parasitology
113:S47-S72.}

\end{scieloReferencesContainer}



%\scieloLicenseContainer{License information: This is an open-access article distributed under the terms of the Creative Commons Attribution License, which permits unrestricted use, distribution, and reproduction in any medium, provided the original work is properly cited.}



\end{multicols}



\scieloCorrespondence