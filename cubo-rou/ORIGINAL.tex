\scieloAbstractContainer{Resumo}{\textbf{Introdução}: A indicação correta da época de tratamento de uma má oclusão de Classe II é essencial para o exercício
ético e eficiente da Ortodontia, mas os clínicos são resistentes em aceitar novos conceitos que contradizem seus
métodos preferidos de tratamento.\textbf{Objetivo}: Avaliar a concordância na indicação de tratamento interceptor das más
oclusões de Classe II entre um grupo de formadores de opinião em nível internacional e um grupo de ortodontistas
clínicos, e comparar a indicação de tratamento com os conceitos científicos contemporâneos. \textbf{Material e método}: Um
questionário eletrônico composto por fotografias representativas de diversos graus de gravidade no acometimento
da má oclusão de Classe II em crianças foi enviado a dois painéis de especialistas. Painel 1 (n=28) foi composto por ortodontistas internacionais autores artigos
científicos em revistas de elevado impacto, e o Painel 2 (n=261) foi composto por ortodontistas clínicos. Baseando-se em uma escala de Likert de 5 pontos, os ortodontistas indicaram
suas opções de tratamento para cada um dos 9 casos apresentados. \textbf{Resultado}: As indicações de tratamento do Painel
2 foram estatisticamente diferentes daquelas ofertadas pelo Painel 1, com pelo menos 1 ponto de divergência no
sentido de tratamento mais precoce. A indicação de tratamento ortodôntico interceptor do Painel 1 está de acordo
com os conceitos científicos atuais. \textbf{Conclusão}: Tratamento muito precoce parece ser a tendência de conduta entre
os ortodontistas clínicos, mas não entre os ortodontistas que estão academicamente envolvidos com a interceptação
ortodôntica. Existe uma lacuna entre o conhecimento científico e a prática da Ortodontia.}{Descritores:}{Má oclusão de Angle Classe II; terapêutica; questionários.}

%{\noindent\fontsize{9}{10.8}\selectfont{Received: January 13, 2015; Accepted: April 19, 2015.}}

\begin{multicols}{2}
\scieloSectionContainer{INTRODUCTION}
\par{}Falls are events with a high prevalence\scieloBodyFn{1}{fn body}{1}{} among the elderly population, even among those who are active and healthy, and constitute one of the major preventable geriatric syndromes'. Among the community-dwelling elderly, approximately 30\% suffer a fall each year, and half experience recurrent falls2. Elderly women with osteoporosis and having a high risk of fractures exhibit an even higher frequency of falls (51.1\%)3. A significant portion of these falls results in injuries (36\%)4, fractures (3.4\% to 19\%)2,4,5, and the need for medical assistance (8 to 19\%)4,5 and affects lifestyle choices, creating a high socio-economic burden6. Additionally, experiencing one or more falls in the course of one year significantly increases the chances of the occurrence of new episodes in the following year among the community-dwelling elderly\textsuperscript{4,5} and postmenopausal women\textsuperscript{1'7}.
\par{}Thus, the surveillance of falls among the elderly represents a priority health issue\textsuperscript{6},  which is why uestioning the occurrence of previous falls has been used in clinical/scientific decision making". Several methods have been suggested for monitoring the occurrence of falls among the community-dwelling elderly, including questions asking individuals to recall these events at several intervals by means of telephone, face-to-face or mail interviews, information obtained from medical records, and/or prospective records using falls calendars or diaries\textsuperscript{8,9,11-13}. However, the elderly
\par{}Health service delivery and clinical practice could be improved
through the introduction of novel interventions with efficacies that are
backed strong evidence 19 . However, the uptake and implementation
of innovations in healthcare have often proven challenging and
very slow in some cases. Consequently, research findings are not
always translated into changes in clinical practice. Some authors
have proposed that the adoption of new ideas is a process that is
far more dynamic and complex than previously suggested by the
classic innovation diffusion model of change, which proposes
that the adoption of innovations is a rational and linear process.
However, this model has been criticized for assuming a simplistic
rational view of change and ignoring the complexities of this process
including human cognitive limits and bounded rationality 20 , cognitive
dissonance, individual personalities and predispositions to change,
culture (values, beliefs, habits and assumptions) and attitudes, and
possible economic interests 21 . Economic reasons might motivate a
clinician outside of an academic practice setting to initiate Class II


\scieloSectionContainer{CONCLUSION}

\lipsum

\scieloSectionContainer{Materials and Methods}

\scieloSubSectionContainer{Animals}

\par{}Specimens representing 12 species of polyclads were
collected from different sites in the Caribbean. In addition,
a deep-sea specimen of \textit{Anocellidus profundus} Quiroga,
Bolaños and Litvaitis, 2006, from the North Pacific was
also included in the analysis (specimen courtesy of Dr.
Janet Voight, Field Museum of Natural History, Chicago,
Illinois, USA). For details of collection, fixation, and spe-
cies identification see Litvaitis \textit{et al}. (2010).

\scieloSubSectionContainer{Histological studies of polyclad nervous systems}

\par{}A specimen of each species was embedded in paraffin
in its entirety, longitudinally sectioned at 7-10 mm, and
stained with Milligan Trichrome technique which allows
differentiating well among connective tissue, muscles, and
nervous fibers (Presnell and Schreibman, 1997). Additional
cross and sagittal sections were also prepared. De-paraffi-
nized sections were treated with 3\% potassium dichro-
mate-hydrochloric acid solution for 5 min. Following a
distilled water rinse, sections were stained in acid fuchsin
for 8 min. After a second distilled water rinse, they were
placed into 1\% phosphomolybdic acid for 2 min and then
stained with 2\% solution of Orange G for 5 min. After a fi-
nal rinse with distilled water, the sections were treated with
1\% hydrochloric acid solution for 2 min stained in Fast
Green for 8 min, treated with 1\% acetic acid for 3 min and
then rinsed in 95\% alcohol and dehydrated. Finally, the sec-
tions were cleared with Histoclear (National Diagnostics)
and mounted in Permount (Fisher Scientific). Slides were
observed and photographed under an Axiostar Plus (Zeiss,
Thornwood, New York) light microscope. We applied the
morphological criteria of Reuter et al. (1998) to distinguish
main nerve cords from secondary nerve cords for the exam-
ination of the NS. Cross and sagittal sections were used for
interpretation. In addition, whole mounts of all species
were prepared, by dehydrating the specimens in a graded
alcohol series, cleared with Histoclear, and mounted in
Permount.

\lipsum

\scieloImageContainerOneCol{0.5\textwidth}{img1.jpg}{meu titulo}

%\begin{center}
%{
%\vspace{8mm}
%\centerline{
%\includegraphics[width=\maxwidth{0.5\textwidth}]{img1.jpg}
%}
%\vspace{8mm}
%}
%\end{center}

\lipsum

\scieloImageContainerTwoCol{img2.jpg}{\textbf{Figure 1:} A specimen of each species was embedded in paraffin in its entirety, longitudinally sectioned}

\lipsum
\lipsum

\scieloImageContainerTwoCol{img2.jpg}{\textbf{Figure 1:} A specimen of each species was embedded in paraffin in its entirety, longitudinally sectioned}

\lipsum
\lipsum

\end{multicols}

\begin{scieloReferencesContainer}[REFERENCES]

\scieloReferencesItem{Ackerman JL, Proffit WR. Preventive and interceptive orthodontics: a strong theory proves weak in practice. Angle Orthod. 1980 Apr;50(2):75-87. PMid:6929171.}

\scieloReferencesItem{Freeman JD. Preventive and interceptive orthodontics: a critical review and the results of a clinical study. J Prev Dent. 1977 Sept-Oct;4(5):7-14,
20-3. PMid:351175.}

\scieloReferencesItem{King GJ, Brudvik P. Effectiveness of interceptive orthodontic treatment in reducing malocclusions. Am J Orthod Dentofacial Orthop. 2010 Jan;137(1):18-25. http://dx.doi.org/10.1016/j.ajodo.2008.02.023. PMid:20122426.}

\scieloReferencesItem{King GJ, Keeling SD, Hocevar RA, Wheeler TT. The timing of treatment for Class II malocclusions in children: a literature review. Angle
Orthod. 1990;60(2):87-97. PMid:2111647.}

\scieloReferencesItem{Proffit WR. The timing of early treatment: an overview. Am J Orthod Dentofacial Orthop. 2006 Apr;129(4 Suppl):S47-9. http://dx.doi.
org/10.1016/j.ajodo.2005.09.014. PMid:16644417.}

\scieloReferencesItem{Franchi L, Baccetti T, De Toffol L, Polimeni A, Cozza P. Phases of the dentition for the assessment of skeletal maturity: a diagnostic performance
study. Am J Orthod Dentofacial Orthop. 2008 Mar;133(3):395-400. PMid: 18331939.}

\scieloReferencesItem{Bullock TH and Horridge GA (1965) Structure
and Function in the Nervous Systems of Invertebrates.
Vol. 2. W.H. Freeman and Co, San
Francisco, 1719 pp.}

\scieloReferencesItem{Carranza S, Baguñà J and Riutort M (1997)
Are the Platyhelminthes a monophyletic primitive
group? An assessment using 18S rDNA
sequences. Mol Biol Evol 14:485-497.}

\scieloReferencesItem{Cebrià F (2008) Organization of the nervous system in the model
planarian Schmidtea mediterranea: An immunocytochemical study. Neurosci Res 61:375-384.}

\scieloReferencesItem{Chien P and Koopowitz H (1972) The ultrastructure of neuromuscular systems in Notoplana acticola, a free-living polyclad flatworm. Z Zellforsch Mikroskop Anat 133:277-288.}


\scieloReferencesItem{Chien PK and Koopowitz H (1977) Ultrastructure of nerve plexus
in flatworms. III. The infra-epithelial nervous system. Cell
Tissue Res 176:335-347.}

\scieloReferencesItem{Day TA, Maule AG, Shaw C and Pax RA (1997) Structure-activity relationships of FMRFamide-related peptides contracting Schistosoma mansoni muscle. Peptides 18:917-921.}

\scieloReferencesItem{Egger B, Gschwentner R and Rieger R (2007) Free-living
flatworms under the knife: Past and present. Dev Genes Evol
217:89-104.}

\scieloReferencesItem{Egger B, Lapraz F, Tomiczek B, Müller S, Dessimoz C, Girstmair
J, Skunca N, Rawlinson KA, Cameron CB, Beli E et al.
(2015) A transcriptomic-phylogenomic analysis of the evolutionary relationships of flatworms. Curr Biol 25:1-7.}

\scieloReferencesItem{Ehlers U (1985) Das Phylogenetische System der Plathelminthes.
Gustav Fischer Verlag, Stuttgart, 317 pp.}

\scieloReferencesItem{Fernandes MC, Alvares EP, Gama P and Silveira M (2003) Serotonin in the nervous system of the head region of the land planarian Bipalium kewense. Tissue Cell 35:479-486.}

\scieloReferencesItem{Forest DL and Lindsay SM (2008) Observations of serotonin and
FMRFamide-like immunoreactivity in palp sensory structures and the anterior nervous system of spionid polychaetes. J Morphol 269:544-551.}

\scieloReferencesItem{Girstmair J, Schnegg R, Telford MJ and Egger B (2014) Cellular
dynamics during regeneration of the flatworm Monocelis sp. (Proseriata, Platyhelminthes). Evo Devo 5:e37.}

\scieloReferencesItem{Golding DM (1992) Polychaeta: Nervous system. In: Harrison
FW and Gardiner SL (eds) Microscopic Anatomy of Invertebrates. Vol. 7. Wiley-Liss, New York, pp 155-179.}

\scieloReferencesItem{Gustafsson MKS, Halton DW, Kreshchenko ND, Movsessian SO,
Raikova OI, Reuter M and Terenina NB (2002) Neuropeptides in flatworms. Peptides 23:2053-2061.}

\scieloReferencesItem{Hadenfeldt D (1929) Das Nervensystem von Stylochoplana
maculata und Notoplana atomata. Z wiss Zool 133:586638.}

\scieloReferencesItem{Halton DW and Gustafsson MKS (1996) Functional morphology
of the platyhelminth nervous system. Parasitology
113:S47-S72.}

\end{scieloReferencesContainer}

\scieloLicenseContainer{License information: This is an open-access article distributed under the terms of the Creative Commons Attribution License, which permits unrestricted use, distribution, and reproduction in any medium, provided the original work is properly cited.}
