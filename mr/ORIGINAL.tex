\scieloAbstractContainer{}{Polymer coatings have been used for the corrosion protection of metal surfaces acting as a physical
barrier against several corroding media. In spite of the good efficiency of these coatings their resistance
is limited due to the presence of localized defects which give place to localized corrosion. Aiming to
improve the barrier properties of these coatings this work has proposed the use of nanocomposites as
powder coatings based on a standard formulation of a commercial powder varnish. Nanocomposites
with 2 and 4 wt\% contents of organophilic montmorillonite (OMMT) were obtained in the molten
state through of a co-rotating twin-screw extruder. The application of the nanocomposite coatings
was performed by electrostatic pulverization on mild steel panels. The coatings were characterized
to determine their structure using X-ray diffraction (XRD). The morphologies of the coatings were
assessed using transmission electron microscopy (TEM). Gloss and adhesion measurements and the
flexibility and impact resistance of the coatings were included in the physical assessment of the coatings.
The corrosion performance was evaluated by the salt spray test and by electrochemical impedance
spectroscopy (EIS). The coatings with clays presented predominantly exfoliated structures, with good
dispersion of OMMT in the epoxy matrix. The addition of OMMT reduced the impact resistance,
flexibility and gloss but increased the barrier properties of the coatings. The best corrosion performance
in NaCl solution was achieved for 4 wt\% OMMT.}{Keywords:}{corrosion, epoxy resin, montmorillonite, nanocomposite, powder coatings}

\begin{multicols}{2}
\scieloSectionContainer{1. Introduction}
\par{}Recently, because of the improved physical and chemical properties provided by ceramic materials to polymers, lamellar silicate polymer nanocomposites have attracted the attention of both industry and the scientific community.
\par{}Several polymer matrices have been used in the production of nanocomposites, among which the epoxy resins are highlighted in view of their wide application in engineering.
\par{}Recently, lamellar silicate nanocomposites have been used
as anticorrosion reinforcement in liquid coatings, in special
to those based on an epoxy matrix. The improved barrier
properties of the coatings are based on the concept of the
tortuous path caused by the exfoliation of clay, which reduces
the diffusion of gas and/or liquid molecules throughout the
coating. Such reduced diffusion highlights the huge potential
of the manufacturing of nanocomposites applied to powder
coatings via intercalation in the molten state \textsuperscript{5,6}.
\par{}Powder coatings are considered to be widely available
and environmentally friendly. The significant increase in the
growth rate of the use of powder coatings is mainly due to
economic factors. Powder coatings have powerful advantages
compared to liquid coatings, such as zero solvent emission
during and/or after application and essentially no loss of
applied material because 98\% of the non-adhered material
to the substrate can be reused \textsuperscript{7,8}.
\par{}A fair anticorrosive performance of a powder coating was
observed by Piazza et al. 9 by incorporating 2 and 4 wt\% of
montmorillonite to the formulation of a polyester-based
powder coating. They concluded that the coatings barrier
properties are directly associated to the clay lamellar structure
and to the electrostatic forces existing between matrix and
filler, resulting in coatings with better resistance to the
permeability of gases and/or liquid molecules.
\par{}The corrosion protection is directly related to the
improvement in the coatings barrier properties, which influenced by the filler dispersion in the polymer matrix \textsuperscript{1,6,10-12}.
\par{}In this study, organically modified montmorillonite was
incorporated in the molten state into an epoxy-based powder
coating formulation. To evaluate the influence of the filler
addition, the morphology, physical properties, anticorrosive
performance and barrier properties of the nanocomposite
coatings were characterized.

\scieloSectionContainer{2. Experimental Procedures}

\scieloSubSectionContainer{2.1. Materials}

\par{}The materials used to obtain the powder coatings
included epoxy resin (Araldite GT 7004 ES – Hunstman),
curing agent (Casamid 710 Oiled (PR9990) – Thomas Swan
\& Co. Ltd.), benzoin (Datiquim Produtos Químicos Ltda),
spreading agent (Resiflow TM PV-60 – Estron Chemical Inc.),
and montmorillonite (OMMT)-type Cloisite  30B (Southern
Clay Products).
\par{}Carbon steel AISI 1010 substrates were used in the form
of 120 × 70 × 10 mm panels.

\scieloSubSectionContainer{2.2. Obtaining and applying nanocomposite epoxy powder coatings}

\par{}A commercial formulation of an epoxy-based powder
varnish, containing 2 and 4 wt\% OMMT was used.
The nanocomposite epoxy powder coating were obtained
through incorporation in the melt state in a co-rotating
twin-screw extruder, model MH-COR-LAB, L/D 32, thread
diameter of 20 mm, manufactured by MH Equipamentos
Ltda. The extruder is equipped with eight heating zones
with the load zone at 70 °C, the second zone at 80 °C, and
all the other zones at 90 °C. It was employed a speed screw
of 400 rpm, with feed rate of 5.5 kg.h -1 . After extrusion, the
materials were ground in a knife mill bench-top (Cuisinart
DCG-20) and sieved (200 mesh) to obtain an average particle
size of 40 μm. The average particle size distribution was
measured with a Scirocco 2000M laser granulometer using
Mastersizer 2000 software.
\par{}To provide improved anchorage for the coating, the
carbon steel panels were previously sanded (sandpaper \#320,
\#400 and \#600), degreased and submitted to a pretreatment
zinc-based phosphatizing. The coatings were applied on
the substrates with an electrostatic spray gun (type corona,
TCA ECO Tecnoavance, model 301). Curing of the coatings
was performed in an air circulation oven at 200 °C for 10 min.
\par{}The powder coatings were identified according to
the percentage of OMMT used, for example, EPC/2-30B
(EPC = epoxy-based powder coating) and 2-30B
(2 wt\% of Cloisite 30B OMMT).

\scieloSubSectionContainer{2.3. Characterization}

\par{}After curing the nanocomposite coatings were characterized
to determine their structure and morphology by X-ray
diffraction (XRD) in a Shimadzu – XRD 6000 diffractometer
using Cu-Kα radiation, a voltage of 40 kV, a current of
30 mA, scanning at 2θ from 1o to 12o with a fixed scanning
time with steps of 0.005o/5s and transmission electron
microscopy (TEM) using a Philips EM 208 S microscope
operating at an acceleration voltage of 100 kV. For TEM sections of 100 nm thickness were prepared using a Leica
Ultracut UcT ultramicrotome at 23 ± 2 °C.
\par{}The medium thickness of the coatings (40.5 ± 0.7 μm) was
determined by magnetic method according to ASTM D7378 13
using an Elcometer  345 gauge. The gloss measurement
was performed using a Zehntener Gloss 60o glossmeter
model ZGM 1020, with calibration for an angle of 60o in
92 gloss units according to ASTM D523 14 . The adhesion to
the substrate was measured following Method B of ASTM
D3359 15 . The flexibility test was done according to ASTM
D522 16 Method using the Gardner Conical Mandrel instrument
from BYK Gardner. The impact resistance measurements
were performed using a Heavy-Duty Impact Tester of BYK
Gardner at an impact force of 1 kg × 500 mm.
\par{}The corrosion performance of the coatings applied to the
metal substrate was evaluated by two methods. The coatings
were exposed to salt spray according to ASTM B117 17 for
a period of 504 h in a Bass Equipamentos chamber model
USX-6000/9000. The coatings were also characterized
by electrochemical impedance spectroscopy (EIS) using
a potentiostat/galvanostat ECOCHEMIE BV – Autolab
PGSTAT 302 assisted by FRA software. The impedance data
were measured periodically at an open circuit potential in
3.5 wt\% NaCl solution, in the frequency range of 10 6 to 1 Hz
and at an amplitude of sinusoidal perturbation of 10 mV AC.

\scieloSubSectionContainer{3. Results and Discussion}

The data provided by the X-ray diffraction analysis
performed on the cured epoxy resin-based powder coatings
are shown in the diffractograms of Figure 1. The spectra
exhibit two distinct peaks of different intensities associated
with the “d 001 ” basal spacing.


\lipsum

\scieloImageContainerOneCol{0.5\textwidth}{img1.jpg}{meu titulo}

%\begin{center}
%{
%\vspace{8mm}
%\centerline{
%\includegraphics[width=\maxwidth{0.5\textwidth}]{img1.jpg}
%}
%\vspace{8mm}
%}
%\end{center}

\lipsum

\scieloImageContainerTwoCol{img2.jpg}{\textbf{Figure 1:} A specimen of each species was embedded in paraffin in its entirety, longitudinally sectioned}

\lipsum
\lipsum

\scieloImageContainerTwoCol{img2.jpg}{\textbf{Figure 1:} A specimen of each species was embedded in paraffin in its entirety, longitudinally sectioned}

\lipsum
\lipsum


\end{multicols}

\begin{multicols}{2}

\begin{scieloReferencesContainer}[References]

\scieloReferencesItem{Carrasco F and Pagès P. Thermal degradation and stability
of epoxy nanocomposites: Influence of montmorillonite
content and cure temperature. \textit{Polymer Degradation \&
Stability.} 2008; 93(5):1000-1007. http://dx.doi.org/10.1016/j.
polymdegradstab.2008.01.018.}

\scieloReferencesItem{Bertuoli PT, Piazza D, Scienza LC and Zattera AJ. Preparation
and characterization of montmorillonite modified with
3-aminopropyltriethoxysilane. \textit{Applied Clay Science.} 2014;
87:46-51. http://dx.doi.org/10.1016/j.clay.2013.11.020.}

\scieloReferencesItem{Singh-Beemat J and Iroh JO. Iroh. Characterization of corrosion
resistant clay/epoxy ester composite coatings and thin films.
\textit{Progress in Organic Coatings.} 2012; 74(1):173-180. http://
dx.doi.org/10.1016/j.porgcoat.2011.12.006.}

\scieloReferencesItem{Piazza D, Lorandi NP, Pasqual CI, Scienza LC and Zattera
AJ. Influence of a microcomposite and a nanocomposite on
the properties of an epoxy-based powder coating. \textit{Materials
Science and Engineering A.} 2011; 528(22-23):6769-6775.
http://dx.doi.org/10.1016/j.msea.2011.05.062.}

\scieloReferencesItem{Baldissera AF, Freitas DB and Ferreira CA. Electrochemical
impedance spectroscopy investigation of chlorinated rubber-
based coatings containing polyaniline as anticorrosion agent.
Materials and Corrosion. 2010; 61(9):790-801. http://dx.doi.
org/10.1002/maco.200905254. 12.	 Zainuddin S, Hosur MV, Zhou Y, Kumar A and Jeelani S.
Durability studies of montmorillonite clay filled epoxy composites
under different environmental conditions. \textit{Materials Science
and Engineering A.} 2009; 507(1-2):117-123. http://dx.doi.
org/10.1016/j.msea.2008.11.058.}

\scieloReferencesItem{Sun L, Boo W-J, Clearfield A, Sue H-J and Pham HQ. Barrier
properties of model epoxy nanocomposites. \textit{Journal of Membrane
Science.} 2008; 318(1-2):129-136. http://dx.doi.org/10.1016/j.
memsci.2008.02.041.}

\end{scieloReferencesContainer}

\scieloLicenseContainer{License information: This is an open-access article distributed under the terms of the Creative Commons Attribution License, which permits unrestricted use, distribution, and reproduction in any medium, provided the original work is properly cited.}

\end{multicols}
