\begin{multicols}{2}

\scieloSectionContainer{INTRODUÇÃO}
\par{}Dentre as manifestações clínicas do choque, as alterações da perfusão perifé-
rica são um achado fundamental. Não apenas os sinais de má perfusão periférica
são indicadores precoces de instabilidade hemodinâmica, como também são
fortes preditores de complicações tardias e óbito. (1,2) A perfusão periférica pode
ser avaliada de diversas formas, sendo o tempo de reenchimento capilar (TRC)
a mais comum.\par{}O TRC é definido como o tempo necessário para que
um leito capilar distal recupere sua cor após ter recebido
pressão suficiente para causar seu clareamento. Ele pode
ser medido por meio de diferentes técnicas, sendo suscetí-
vel a fatores que podem afetar profundamente seus resul-
tados, como a temperatura ambiente, da pele e do corpo,
a idade, a iluminação ambiente e a duração, quantidade
e local da aplicação da pressão. (3) A despeito disso, essas
questões são raramente consideradas pelos médicos. (4)\par{}Uma fonte adicional de incerteza é a dependência que
essa avaliação tem do desempenho do observador. Quan-
do o método foi aplicado em crianças saudáveis (5) e neo-
natos, (6) em cirurgia cardíaca, (7) e em pacientes pediátricos
com choque, (8) foi encontrada uma acentuada variabilida-
de entre observadores. Apesar da relevância do TRC para
a avaliação clínica da perfusão tissular, sua reprodutibili-
dade ainda é insuficientemente estudada em adultos. Uma
baixa concordância entre observadores foi relatada em pa-
cientes adultos admitidos a um pronto-socorro, onde o
TRC foi, lamentavelmente, avaliado sem o uso de um dis-
positivo para mensuração do tempo. (9) Uma quantificação
adequada, como o uso de um cronômetro, teria produzido
resultados diferentes.

\scieloSectionContainer{MÉTODOS}

\par{}Escolar, 8 anos e 2 meses, com diagnóstico pregresso
de leucemia promielocítica aguda com tratamento completo
concluído há 1 ano e sem intercorrências durante este período.
Deu entrada no PSI do HC-UFTM, no dia 12 de abril de
2014, com queixa de sangramento gengival, equimoses em
membros inferiores e plaquetopenia. Em hemograma inicial,
visualizados 70% de blastos, levantada hipótese diagnóstica de
recidiva da doença, sendo então iniciado ATRA 25 mg/m 2 /dia
e suporte hemoterápico. Mielograma evidenciou medula
óssea hiperplásica e homogênea, com 90% de mieloblastos e
imunofenotipagem compatível com leucemia promielocítica
aguda, confirmada com identificação do gene de fusão
PML-RARA

\lipsum

\scieloSubSectionContainer{Análise estatística}

\par{}Os dados foram testados quanto à normalidade, usan-
do o teste de Kolmogorov-Smirnov, e expressos como
média±desvio padrão (DP). O limite superior da normali-
dade foi determinado como a média±2 DP. A concordân-
cia entre dois diferentes métodos e os observadores para
a mensuração do TRC foi calculada usando o método de
Bland-Altman. As correlações entre o TRC e os índices de
perfusão periférica foram avaliadas com a correlação de
Pearson. Considerando-se a falta de dados na literatura,
não se realizou cálculo do tamanho da amostra, porém foi
testada sua adequação. (10) Foi considerado como signifi-
cante um valor de p<0,05.

\lipsum
\scieloSectionContainer{RESULTS}

\scieloSubSectionContainer{SPECIFIC COMPOSITION, RICHNESS AND ABUNDANCE}

\par{}A total of 110 taxa were identified, belonging to
4 Divisions (Figure 2; Table 1): Diatoms (55 taxa;
25 species), Dinoflagellates (51 taxa; 25 species),
Cyanobacteria (2 taxa) and Chlorophythes (2 taxa).
\par{}Richness, Abundance, Diversity Index and Evenness
results are presented as the mean values of each sampling
period, as the non-parametric tests revealed no differences
(p < 0.05) between sampling stations. Table 2 presents
information on these variables in each sampling period.


\lipsum

\scieloImageContainerOneCol{0.5\textwidth}{img1.jpg}{meu titulo}

%\begin{center}
%{
%\vspace{8mm}
%\centerline{
%\includegraphics[width=\maxwidth{0.5\textwidth}]{img1.jpg}
%}
%\vspace{8mm}
%}
%\end{center}

\lipsum

\scieloImageContainerTwoCol{img2.jpg}{\textbf{Figure 1:} A specimen of each species was embedded in paraffin in its entirety, longitudinally sectioned}

\lipsum
\lipsum

\scieloImageContainerTwoCol{img2.jpg}{\textbf{Figure 1:} A specimen of each species was embedded in paraffin in its entirety, longitudinally sectioned}

\lipsum
\lipsum
\lipsum[1]

\end{multicols}

\scieloRenderAbstractTranslationsGroup
\clearpage
\newpage

\begin{multicols}{2}

\begin{scieloReferencesContainer}[REFERÊNCIAS]

\scieloReferencesItem{Lima A, Bakker J. Noninvasive monitoring of peripheral perfusion. Intensive
Care Med. 2005;31(10):1316-26.}

\scieloReferencesItem{Lima A, Jansen TC, van Bommel J, Ince C, Bakker J. The prognostic
value of the subjective assessment of peripheral perfusion in critically ill
patients. Crit Care Med. 2009;37(3):934-8.}

\scieloReferencesItem{Pickard A, Karlen W, Ansermino JM. Capillary refill time: is it still a useful
clinical sign? Anesth Analg. 2011;113(1):120-3.}

\scieloReferencesItem{Lobos AT, Menon K. A multidisciplinary survey on capillary refill time:
Inconsistent performance and interpretation of a common clinical test.
Pediatr Crit Care Med. 2008;9(4):386-91.}

\scieloReferencesItem{Gorelick MH, Shaw KN, Baker MD. Effect of ambient temperature on
capillary refill in healthy children. Pediatrics. 1993;92(5):699-702.}

\scieloReferencesItem{Raju NV, Maisels MJ, Kring E, Schwarz-Warner L. Capillary refill time
in the hands and feet of normal newborn infants. Clin Pediatr (Phila).
1999;38(3):139-44.}

\scieloReferencesItem{Lobos AT, Lee S, Menon K. Capillary refill time and cardiac output in
children undergoing cardiac catheterization. Pediatr Crit Care Med.
2012;13(2):136-40.}

\scieloReferencesItem{Otieno H, Were E, Ahmed I, Charo E, Brent A, Maitland K. Are bedside
features of shock reproducible between different observers? Arch Dis
Child. 2004;89(10):977-9.}

\scieloReferencesItem{Cebrià F (2008) Organization of the nervous system in the model
planarian Schmidtea mediterranea: An immunocytochemical study. Neurosci Res 61:375-384.}

\scieloReferencesItem{Chien P and Koopowitz H (1972) The ultrastructure of neuromuscular systems in Notoplana acticola, a free-living polyclad flatworm. Z Zellforsch Mikroskop Anat 133:277-288.}


\scieloReferencesItem{Chien PK and Koopowitz H (1977) Ultrastructure of nerve plexus
in flatworms. III. The infra-epithelial nervous system. Cell
Tissue Res 176:335-347.}

\scieloReferencesItem{Day TA, Maule AG, Shaw C and Pax RA (1997) Structure-activity relationships of FMRFamide-related peptides contracting Schistosoma mansoni muscle. Peptides 18:917-921.}

\scieloReferencesItem{Egger B, Gschwentner R and Rieger R (2007) Free-living
flatworms under the knife: Past and present. Dev Genes Evol
217:89-104.}

\scieloReferencesItem{Egger B, Lapraz F, Tomiczek B, Müller S, Dessimoz C, Girstmair
J, Skunca N, Rawlinson KA, Cameron CB, Beli E et al.
(2015) A transcriptomic-phylogenomic analysis of the evolutionary relationships of flatworms. Curr Biol 25:1-7.}

\scieloReferencesItem{Ehlers U (1985) Das Phylogenetische System der Plathelminthes.
Gustav Fischer Verlag, Stuttgart, 317 pp.}

\scieloReferencesItem{Fernandes MC, Alvares EP, Gama P and Silveira M (2003) Serotonin in the nervous system of the head region of the land planarian Bipalium kewense. Tissue Cell 35:479-486.}

\scieloReferencesItem{Forest DL and Lindsay SM (2008) Observations of serotonin and
FMRFamide-like immunoreactivity in palp sensory structures and the anterior nervous system of spionid polychaetes. J Morphol 269:544-551.}

\scieloReferencesItem{Girstmair J, Schnegg R, Telford MJ and Egger B (2014) Cellular
dynamics during regeneration of the flatworm Monocelis sp. (Proseriata, Platyhelminthes). Evo Devo 5:e37.}

\scieloReferencesItem{Golding DM (1992) Polychaeta: Nervous system. In: Harrison
FW and Gardiner SL (eds) Microscopic Anatomy of Invertebrates. Vol. 7. Wiley-Liss, New York, pp 155-179.}

\scieloReferencesItem{Gustafsson MKS, Halton DW, Kreshchenko ND, Movsessian SO,
Raikova OI, Reuter M and Terenina NB (2002) Neuropeptides in flatworms. Peptides 23:2053-2061.}

\scieloReferencesItem{Hadenfeldt D (1929) Das Nervensystem von Stylochoplana
maculata und Notoplana atomata. Z wiss Zool 133:586638.}

\scieloReferencesItem{Halton DW and Gustafsson MKS (1996) Functional morphology
of the platyhelminth nervous system. Parasitology
113:S47-S72.}

\end{scieloReferencesContainer}

\scieloLicenseContainer{License information: This is an open-access article distributed under the terms of the Creative Commons Attribution License, which permits unrestricted use, distribution, and reproduction in any medium, provided the original work is properly cited.}



\end{multicols}

