\makeatletter
\def\@fnsymbol#1{
     \ensuremath{
         \ifcase#1
\or ~
         \fi
     }
}
\renewcommand{\thefootnote}{\fnsymbol{footnote}}
\makeatother

\newcommand\scieloDefineFootnotes{
%\footnotetext[1]{\scieloArticleUrl}
%\footnotetext[1]{EDVC é Mestre em Urbanisme et Aménagement, e-mail: eugeniadoria@gmail.com}
%\footnotetext[2]{\\\textbf{Endereço para correspondência}\\Vanessa de Paula Tiago.\\Universidade Federal do Triângulo Mineiro. Departamento de Pediatria da Universidade Federal do Triângulo Mineiro. Av. Frei Paulino, no 30, Bairro Abadia\\Uberaba - MG. Brasil. CEP: 38025-180.}
}

\newcommand{\scieloYear}{2015}
\newcommand\scieloArticleType{ARTIGO ORIGINAL}
\newcommand{\scieloArticleRef}{2014;26(3):00-00}
\newcommand{\scieloHeaderOdd}{Falta de concordância entre diferentes observadores e métodos na mensuração do tempo de reenchimento capilar em voluntários saudáveis}
\newcommand{\scieloHeaderEven}{Valenzuela Espinoza ED, Welsh S, Dubin A}
\newcommand\scieloArticleTitle{Falta de concordância entre diferentes observadores e métodos na mensuração do tempo de reenchimento capilar em voluntários saudáveis: estudo observacional}
\newcommand\scieloArticleTitleTranslation{Lack of agreement between different observers and methods in the measurement of capillary refill time in healthy volunteers: an observational study}
\newcommand{\scieloAuthor}{Emilio Daniel Valenzuela Espinoza\textsuperscript{1}, Sebastián Welsh\textsuperscript{1}, Arnaldo Dubin\textsuperscript{1,2}}
\newcommand\scieloContributorsFootNotes{1. Serviço de Terapia Intensiva, Sanatorio Otamendi y Miroli - Buenos Aires, Argentina.\\2. Cátedra de Farmacologia Aplicada, Facultad de Ciencias Médicas, Universidad Nacional de La Plata - Buenos Aires, Argentina.}
\newcommand{\scieloArticleDOI}{10.1590/S1679-87592015086506303}
\newcommand{\scieloReceivedApproved}{\scieloReceivedApprovedContainer{Submetido em 24 de janeiro de 2014\\Aceito em 19 de julho de 2014}}
\newcommand{\scieloCorrespondence}{\scieloCorrespondenceContainer{Autor correspondente:}{Arnaldo Dubin\\Sanatorio Otamendi y Miroli - Servicio de Terapia Intensiva\\Azcuénaga, 870\\C1115AAB - Buenos Aires, Argentina\\E-mail: arnaldodubin@gmail.com}}
\newcommand{\scieloInterestsConflict}{\scieloInterestsConflictContainer{Conflitos de interesse:}{Nenhum.}}
\newcommand{\scieloRespEditor}{\scieloRespEditorContainer{Editor responsável:}{Luciano Cesar Pontes Azevedo}}

\newcommand{\scieloRenderAbstract}{
\scieloAbstractContainer{RESUMO}{\textbf{Objetivo}: As anomalias da perfusão periférica são manifestações importantes do choque, sendo o tempo de reenchimento capilar comumente utilizado em sua avaliação. Entretanto, a reprodutibilidade das mensurações do tempo de reenchimento capilar e sua correlação com outras variáveis da perfusão periférica não foram avaliadas de forma abrangente. Nosso objetivo foi determinar, em voluntários saudáveis, a concordância entre diferentes métodos e diferentes observadores na quantificação do tempo
de reenchimento capilar, assim como sua correlação com outros marcadores da perfusão periférica\par\textbf{Métodos:} Estudamos 63 voluntários saudáveis. Dois observadores mediram o tempo de reenchimento capilar por meio de dois métodos distintos: visão direta (TRC cronômetro ) e vídeo-análise (TRC vídeo ). Medimos também o índice de perfusão derivado de pletismografia de pulso e a temperatura da polpa digital (To periférica ). A concordância entre os observadores e os métodos foi avaliada utilizando o método de Bland-Altman. As correlações
foram calculadas utilizando a correlação de Pearson. Valor de p<0,05 foi considerado significante.\par\textbf{Resultados}: Os limites de concordância de 95\% entre ambos os observadores foram de 1,9 segundo para TRC cronômetro e 1,7 segundo para TRC vídeo . Os limites de concordância de 95\% entre TRC cronômetro e TRC vídeo foram de 1,7 segundo para o Observador 1 e 2,3 segundos para o Observador 2. As mensurações do TRC cronômetro realizadas pelos dois observadores se correlacionaram com a To periférica . As mensurações do TRC vídeo realizadas pelos dois observadores se correlacionaram com a To periférica e o índice de perfusão.\par\textbf{Conclusão}: As mensurações do tempo de reenchimento capilar realizadas por diferentes  bservadores ou diferentes métodos em voluntários saudáveis mostraram baixa concordância. Apesar disso, o tempo de reenchimento capilar ainda refletiu a perfusão periférica, conforme
mostrado por sua correlação com variáveis objetivas da perfusão periférica.}{Descritores:}{Choque/diagnóstico; Perfusão; Capilares/fisiologia}
}

\newcommand{\scieloRenderAbstractTranslations}{
\scieloAbstractContainer{ABSTRACT}{\textbf{Objective}: Current paper evaluates admittance of risk-classified cases in two hospital emergency services. \textbf{Methods}: The
exploratory, descriptive and quantitative research was undertaken between March and May 2013, with 47 nursing professionals at
two hospital emergency units in the state of Paraná, Brazil, who answered the questionnaire. \textbf{Results}: Acceptance of Risk-classified
Cases was reported hazardous at the two units; the lowest rates refer to issues on the place the accompanying person would
stay and to discussions on the flowchart. The best assessment occurred in the attendance of less serious cases. \textbf{Conclusion}: The hazardous assessment at the two health units was mainly due to the non-compliance with certain basic principles of its guidelines of the Acceptance of Risk-classified Cases.}{Keywords:}{User Embracement; Nursing; Emergency Service, Hospital.}
}