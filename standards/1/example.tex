% Generated by jats2tex@0.11.1.0
\documentclass{article}

\usepackage[T1]{fontenc}
\usepackage{amstext}
\usepackage{authblk}
\usepackage{unicode-math}
\usepackage{multirow}
\usepackage{graphicx}

%% METADATA %%%%%%%%%%%%%%%%%%%%%%%%%%%%%%%%%%%%%%%%%%%%%%%
\newcommand{\journalid}{urbe}
\newcommand{\journaltitle}{urbe. Revista Brasileira de Gestão Urbana} %%%%
\newcommand{\abbrevjournaltitle}{urbe, Rev. Bras. Gest.
					Urbana}
\newcommand{\issn}{2175-3369}
\newcommand{\publishername}{Pontifícia Universidade Católica do Paraná}
\newcommand{\subject}{Articles}
\title{Planejamento urbano participativo por meio da utilização de novas
					tecnologias: uma avaliação por especialistas}
\newcommand{\transtitle}{Participatory urban planning through the use of new technologies:
						an evaluation by experts}
%% METADATA %%%%%%%%%%%%%%%%%%%%%%%%%%%%%%%%%%%%%%%%%%%%%%%
\author[% <xref ref-type="aff" rid="aff01">
%tag
% <sup >
\textsuperscript{th}
%endelem
]{%tag
% <name >
% <surname >
Bugs, % <given-names >
Geisa
%endelem
%tag
% <xref ref-type="aff" rid="aff01">
%tag
% <sup >
\textsuperscript{th}
%endelem
%endelem
}
\author[% <xref ref-type="aff" rid="aff02">
%tag
% <sup >
\textsuperscript{th}
%endelem
]{%tag
% <name >
% <surname >
Reis, % <given-names >
Antônio Tarcísio da Luz
%endelem
%tag
% <xref ref-type="aff" rid="aff02">
%tag
% <sup >
\textsuperscript{th}
%endelem
%endelem
}\affil[aff01]{% <institution content-type="orgname">
Universidade Feevale% <institution content-type="normalized">
Universidade FEEVALE% <institution content-type="original">
Universidade Feevale, Novo Hamburgo, RS,
						Brasil.}\affil[aff02]{% <institution content-type="orgname">
Universidade Federal do Rio Grande do
						Sul% <institution content-type="orgdiv1">
Faculdade de Arquitetura% <institution content-type="normalized">
Universidade Federal do Rio Grande do
						Sul% <institution content-type="original">
Universidade Federal do Rio Grande do Sul –
						UFRGS, Faculdade de Arquitetura, Programa de Pós-graduação em Planejamento
						Urbano - PROPUR, Porto Alegre, RS, Brasil.}
\date{% <year >
2016}{% <month >
12}{% <day >
08}

\begin{document}

\maketitle

%tag
% <front >
%tag
% <journal-meta >
%tag
% <journal-id journal-id-type="publisher-id">
%endelem
%tag
% <journal-title-group >
%tag
% <journal-title >
%endelem
%tag
% <abbrev-journal-title abbrev-type="publisher">
%endelem
%endelem
%tag
% <issn pub-type="epub">
%endelem
%tag
% <publisher >
%tag
% <publisher-name >
%endelem
%endelem
%endelem
%tag
% <article-meta >
%tag
% <article-id pub-id-type="publisher-id">
urbe0901AO06%endelem
%tag
% <article-id pub-id-type="other">
08101%endelem
%tag
% <article-id pub-id-type="doi">
10.1590/2175-3369.009.001.AO06%endelem
%tag
% <article-categories >
%tag
% <subj-group subj-group-type="heading">
%tag
% <subject >
%endelem
%endelem
%endelem
%tag
% <title-group >
%tag
% <article-title >
%endelem
%tag
% <trans-title-group xml:lang="en">
%tag
% <trans-title >
%endelem
%endelem
%endelem
%tag
% <contrib-group >
%tag
% <contrib contrib-type="author">
%endelem
%tag
% <contrib contrib-type="author">
%endelem
%tag
% <aff id="aff01">
%endelem
%tag
% <aff id="aff02">
%endelem
%endelem
%tag
% <author-notes >
%tag
% <fn fn-type="other">
%tag
% <p >

GB é Arquiteta e Urbanista; Doutora pelo PROPUR/UFRGS, e-mail:
						geisa@feevale.br
%endelem
%endelem
%tag
% <fn fn-type="other">
%tag
% <p >

ATLR é Arquiteto e Urbanista; Doutor pela Pos Graduate Research School,
						Oxford Brookes University, e-mail: tarcisio.reis@ufrgs.br
%endelem
%endelem
%endelem
%tag
% <pub-date pub-type="epub">
%endelem
%tag
% <volume >
00%endelem
%tag
% <issue >
00%endelem
%tag
% <fpage >
000%endelem
%tag
% <lpage >
000%endelem
%tag
% <history >
%tag
% <date date-type="received">
%tag
% <day >
03%endelem
%tag
% <month >
09%endelem
%tag
% <year >
2015%endelem
%endelem
%tag
% <date date-type="accepted">
%tag
% <day >
12%endelem
%tag
% <month >
01%endelem
%tag
% <year >
2016%endelem
%endelem
%endelem
%tag
% <permissions >
%tag
% <license license-type="open-access" xlink:href="http://creativecommons.org/licenses/by/4.0/" xml:lang="pt">
%tag
% <license-p >
Este é um artigo publicado em acesso aberto (%tag
% <italic >
Open
							Access%endelem
) sob a licença %tag
% <italic >
Creative Commons
							Attribution%endelem
, que permite uso, distribuição e reprodução em
						qualquer meio, sem restrições desde que o trabalho original seja
						corretamente citado.%endelem
%endelem
%endelem
%tag
% <abstract >

\renewcommand{\abstractname}{% <title >
Resumo}
\begin{abstract}
% <p >

O objetivo deste artigo é avaliar a aceitação, por parte dos especialistas, da
					participação pública e da utilização de novas tecnologias, bem como investigar
					como eles avaliam metodologias alternativas, como a Participação Pública com
					Sistema de Informação Geográfica (PPSIG). Assume-se que novas abordagens
					metodológicas, que façam uso das Tecnologias da Informação e Comunicação (TIC) e
					dos Sistemas de Informação Geográfica (SIG), podem aperfeiçoar os processos de
					participação pública no planejamento urbano, tendo em vista a dificuldade de se
					incorporar a perspectiva da população e a necessidade de renovação das práticas
					para se lidar com os fluxos de informação da era digital. Os procedimentos
					metodológicos incluem a aplicação de questionários e a realização de entrevistas
					com arquitetos e urbanistas que atuam na área de planejamento urbano. Os
					resultados apontam que, apesar das barreiras institucionais e da necessidade de
					maior capacitação técnica, há aceitação quanto à participação pública e à
					utilização de novas ferramentas, e que a PPSIG pode auxiliar na difícil tarefa
					de incorporar a perspectiva do usuário do espaço urbano, na opinião dos
					respondentes.

\end{abstract}
%%%%%%%%%%%%%%%%%%%%%%%%%%%%%%%%%%%%%%%%%%%%%%%%%%%%%%%%%%%
%endelem
%tag
% <trans-abstract xml:lang="en">

\renewcommand{\abstractname}{% <title >
Abstract}
\begin{abstract}
% <p >

This paper aims to evaluate the experts acceptance of the public participation,
					the use of new technologies, and to investigate how they evaluate alternative
					methodologies such as the Public Participation Geographic Information Systems
					(PPGIS). We assume that new methodological approaches, using Information and
					Communication Technology (ICT) and Geographic Information Systems (GIS), can
					improve public participation in urban planning processes, given the difficulty
					to include the population perspective, and the need to renew the practices for
					dealing with the flow of information in the digital era. Questionnaires and
					interviews were conducted with architects working in the urban planning area, as
					part of the methodological procedures. The results show that despite
					institutional barriers and the necessity of technical training, the public
					participation and the use of new tools are accepted, and that PPSIG can help in
					the difficult task of incorporating the urban space users perspective, in the
					respondents opinion.

\end{abstract}
%%%%%%%%%%%%%%%%%%%%%%%%%%%%%%%%%%%%%%%%%%%%%%%%%%%%%%%%%%%
%endelem
%tag
% <kwd-group xml:lang="pt">
%tag
% <title >
Palavras-chave: %endelem
%tag
% <kwd >
\textit{Planejamento participativo}
%endelem
%tag
% <kwd >
\textit{PPSIG}
%endelem
%tag
% <kwd >
\textit{TIC}
%endelem
%endelem
%tag
% <kwd-group xml:lang="en">
%tag
% <title >
Keywords: %endelem
%tag
% <kwd >
\textit{Participatory planning}
%endelem
%tag
% <kwd >
\textit{PPGIS}
%endelem
%tag
% <kwd >
\textit{ICT}
%endelem
%endelem
%tag
% <counts >
%tag
% <fig-count count="1">
%endelem
%tag
% <table-count count="6">
%endelem
%tag
% <equation-count count="0">
%endelem
%tag
% <ref-count count="28">
%endelem
%tag
% <page-count count="1">
%endelem
%endelem
%endelem
%endelem
%tag
% <body >
%tag
% <sec sec-type="intro">
\section{% <title >
Introdução}
% <p >

O planejamento urbano normalmente tem sido baseado na coleta e na troca de
				informações entre diferentes partes interessadas. Contudo, uma mudança no modelo
				informacional, em consequência da revolução das tecnologias digitais, tem gerado um
				impacto significativo em como se desenvolve todo o processo (%tag
% <xref ref-type="bibr" rid="B011">
Haller \& Höffken, 2010%endelem
). Para acompanhar as mudanças
				tecnológicas, faz-se necessário adicionar novas ferramentas que permitam exibir e
				gerenciar novos fluxos de informações (%tag
% <xref ref-type="bibr" rid="B018">
Pereira et
					al., 2013%endelem
). Logo, tal mudança também vai exigir alterações no
					%tag
% <italic >
modus operandi%endelem
 do planejamento urbano, ainda muito influenciado
				pelo pensamento do modelo racional (%tag
% <xref ref-type="bibr" rid="B028">
Yigitcanlar,
					2006%endelem
; %tag
% <xref ref-type="bibr" rid="B013">
Horelli et al., 2013%endelem
),
				segundo o qual, os especialistas seriam capazes de considerar as melhores soluções
				para os cidadãos e para a sociedade como um todo, sem incluir, portanto, nenhuma
				forma de participação da sociedade civil nas discussões de propostas para a cidade
					(%tag
% <xref ref-type="bibr" rid="B026">
Villaça, 2005%endelem
).
% <p >

Na era da informação e da sociedade em rede, ferramentas digitais estão se tornando
				onipresentes na vida cotidiana e possuem alto potencial para a coleta de dados
				socioespaciais e temporais, o que é completamente diferente dos modelos estáticos de
				coleta de dados que o planejamento urbano comumente utiliza (%tag
% <xref ref-type="bibr" rid="B008">
Friedmann, 2007%endelem
; %tag
% <xref ref-type="bibr" rid="B001">
Batty et
					al., 2012%endelem
). A ubiquidade das Tecnologias da Informação e Comunicação
				(TIC) está produzindo ambientes urbanos que são completamente diferentes de tudo o
				que se experimentou até agora, nos quais um grupo muito maior de cidadãos pode
				aderir (%tag
% <xref ref-type="bibr" rid="B019">
Pfeffer et al., 2013%endelem
).
% <p >

Antigamente, os profissionais eram os únicos produtores e usuários das informações
				relativas às questões urbanas. Entretanto, hoje, qualquer pessoa pode facilmente
				produzir um mapa e publicá-lo on-line. Assim, mesmo que involuntariamente, está
				ocorrendo um aumento da consciência da informação geográfica pelo público, o que
				pode ser considerado uma revolução nesse campo, pois possibilita que ferramentas do
				Sistema de Informação Geográfica (SIG) possam ser compreendidas de maneira rápida e
				eficaz, sem haver a necessidade de imersão em atividades profissionais (%tag
% <xref ref-type="bibr" rid="B014">
Hudson-Smith \& Crooks, 2008%endelem
).
% <p >

Por conseguinte, o mundo digital está transformando a possibilidade de participação
				pública no planejamento urbano (%tag
% <xref ref-type="bibr" rid="B001">
Batty et al.,
					2012%endelem
). A capacidade de todos os cidadãos se comunicarem uns com os
				outros, até mesmo com seus representantes, além das novas formas de se relacionar
				com o espaço urbano, abre novas frentes para a ideia de que os cidadãos possam
				desempenhar um papel ativo no planejamento urbano. Mas, apesar disso, e dos mais de
				40 anos de história do planejamento participativo (considerando, por exemplo, os
				processos de planejamento e construção de habitações na Europa nos anos de 1960 e
				1970), a participação pública ainda não conseguiu entrar no
					%tag
% <italic >
mainstream%endelem
 do planejamento urbano de forma consistente (%tag
% <xref ref-type="bibr" rid="B012">
Horelli, 2002%endelem
).
% <p >

Em parte, tal acontecimento ocorre porque envolver o público não é uma tarefa simples
				e direta. Participação plena implica em responsabilização e poderes partilhados.
				Para se alcançar esse nível, existe uma série de barreiras que vão desde a
				relevância dada às opiniões e posições do público, passando pela credibilidade e
				idoneidade destas (%tag
% <xref ref-type="bibr" rid="B006">
Corburn, 2003%endelem
), até os
				entraves institucionais (%tag
% <xref ref-type="bibr" rid="B002">
Brown, 2012%endelem
). Nas
				palavras de %tag
% <xref ref-type="bibr" rid="B007">
Forester (2006%endelem
, p. 447): “Fácil
				de pregar, mas difícil de praticar, participação pública eficaz no planejamento e na
				gestão pública exige sensibilidade e técnica, imaginação e coragem”.
% <p >

De tal modo, ainda não se conseguiu abordar a perspectiva da população de forma
				consistente (%tag
% <xref ref-type="bibr" rid="B016">
Kahila \& Kyttä, 2009%endelem
). A
				literatura destaca, por exemplo, que há grande dificuldade em se passar de um
				sistema de planejamento dominado por especialistas para um que daria voz a
				diferentes interessados (%tag
% <xref ref-type="bibr" rid="B027">
Wallin \& Horelli,
					2012%endelem
) e que o planejamento urbano, em geral, utiliza somente dados
				oficiais, nos quais as pessoas são levadas em consideração apenas na forma de
				estatísticas agregadas (%tag
% <xref ref-type="bibr" rid="B008">
Friedmann,
				2007%endelem
).
% <p >

Distintos autores defendem a necessidade de uma reformulação no planejamento urbano
				por meio da utilização de tecnologias que permitam a criação de técnicas e
				metodologias mais interativas, emancipatórias e colaborativas (%tag
% <xref ref-type="bibr" rid="B028">
Yigitcanlar, 2006%endelem
; %tag
% <xref ref-type="bibr" rid="B013">
Horelli et
					al., 2013%endelem
). Nesse sentido, os recentes avanços tecnológicos em torno das
				TIC (em especial, a internet) e dos SIG envolvem novas perspectivas.
% <p >

Se a internet é o principal meio de troca de informações e comunicação da atualidade,
				os SIG são uma tecnologia que desempenha um papel importante na coleta, tratamento e
				disseminação de informações espaciais – e a maioria das informações necessárias à
				formulação de políticas urbanas contém um componente espacial (%tag
% <xref ref-type="bibr" rid="B023">
Sieber, 2006%endelem
). Para %tag
% <xref ref-type="bibr" rid="B028">
Yigitcanlar (2006)%endelem
, o uso efetivo das TIC e dos SIG no planejamento
				urbano representa a possibilidade de se estabelecer um sistema permanente de coleta
				das percepções do público sobre o espaço urbano, o que é apontado como essencial
				para que ele se torne mais colaborativo.
% <p >

Nessa direção, estudos recentes apontam que a Participação Pública com Sistema de
				Informação Geográfica (PPSIG) possui potencial para aprimorar a participação do
				público no planejamento urbano (%tag
% <xref ref-type="bibr" rid="B016">
Kahila \& Kyttä,
					2009%endelem
; %tag
% <xref ref-type="bibr" rid="B020">
Poplin, 2012%endelem
). A PPSIG
				utiliza ferramentas dos SIG para a participação pública, conectando a capacidade
				técnica deles ao conhecimento local (%tag
% <xref ref-type="bibr" rid="B023">
Sieber,
					2006%endelem
). O público participa produzindo mapas e/ou dados espaciais que
				representam a sua percepção do espaço urbano em questão (%tag
% <xref ref-type="bibr" rid="B003">
Bugs, 2014%endelem
).
% <p >

A PPSIG surgiu nos anos de 1990 a partir de reflexões sobre a interface do SIG com a
				sociedade. No início dos anos de 2000, chegou a ser considerada uma evolução dos
				métodos clássicos de participação (%tag
% <xref ref-type="bibr" rid="B004">
Carver,
					2001%endelem
). Mas o entusiasmo inicial começou a diminuir por volta de 2005
					(%tag
% <xref ref-type="bibr" rid="B020">
Poplin, 2012%endelem
), quando vários problemas
				foram levantados, tais como o reconhecimento de que termos e conceitos como
				“participação” e “empoderamento” foram utilizados de forma acrítica (%tag
% <xref ref-type="bibr" rid="B010">
Hall et al., 2010%endelem
). Então, quase que
				concomitantemente, os %tag
% <italic >
mashups%endelem
 de mapas e o mapeamento voluntário
				emergiram na web 2.0. Logo, surgiram experiências tirando proveito dos mapas on-line
				interativos e do aumento do conhecimento da informação geográfica pelo público.
				Esses projetos atraíram os cidadãos para o uso de ferramentas de participação
				baseadas em mapas on-line (%tag
% <xref ref-type="bibr" rid="B020">
Poplin, 2012%endelem
), e
				o interesse na PPSIG se renovou. Porém o tema ainda é, relativamente, pouco
				explorado a partir da perspectiva dos profissionais que trabalham com planejamento
				urbano, principalmente no contexto brasileiro. A maioria das experiências existentes
				foi desenvolvida por geógrafos, com foco na produção de cartografia, ou por
				profissionais da tecnologia da informação, com foco no desenvolvimento
				tecnológico.
% <p >

Em suma, estão em curso mudanças na sociedade contemporânea fomentadas pelas
				revoluções digital e geoespacial, sem que exista um maior conhecimento acerca do
				efeito de tais mudanças sobre o planejamento urbano, o que tem sido tema de debate
				entre os profissionais, conforme apontam %tag
% <xref ref-type="bibr" rid="B018">
Pereira et
					al. (2013)%endelem
. Logo, faz-se necessário entender como se dá o planejamento
				urbano participativo utilizando novas tecnologias, na opinião dos especialistas.
				Assim sendo, esta pesquisa tem como objetivo avaliar a aceitação, por parte dos
				planejadores, da participação do público no planejamento urbano e da utilização de
				novas tecnologias, assim como investigar como esses profissionais avaliam
				metodologias participativas alternativas, tal como a PPSIG.

%endelem
%tag
% <sec sec-type="methods">
\section{% <title >
Metodologia}
% <p >

Os dados foram coletados por meio de um questionário aplicado a arquitetos e
				urbanistas (AU) que trabalham com planejamento urbano no Rio Grande do Sul,
				especificamente em prefeituras, universidades e empresas privadas, e que desenvolvem
				tanto planos quanto projetos urbanos. Ainda que o planejamento urbano seja realizado
				por profissionais de diversas áreas, optou-se por trabalhar apenas com AU pela maior
				viabilidade de acesso a esses profissionais. Ainda, via de regra, o AU é o
				profissional mais identificado com a atividade de planejamento urbano, pois é o
				único habilitado para assinar a responsabilidade técnica de um plano diretor no
				Brasil.
% <p >

No total, foram distribuídos 100 questionários via e-mail, obtendo uma taxa de
				retorno de 42\% (42 de 100). O questionário continha 19 perguntas, das quais as três
				primeiras diziam respeito às características dos respondentes e outras duas questões
				abordavam o uso dos SIG. As demais perguntas ou afirmações retiradas da revisão da
				literatura tratavam da participação do público no planejamento urbano, das novas
				tecnologias e da PPSIG. A maior parte dos respondentes era do sexo feminino (64,3\%,
				ou 27 de 42), possuía entre 26 e 40 anos (59,5\%, ou 25 de 42) e tinha pós-graduação
				(69,1\%, ou 29 de 42). Adicionalmente, a maioria respondeu ter algum tipo de
				capacitação em SIG (78,6\%, ou 33 de 42) e utilizá-lo em suas atividades
				profissionais (62\%, ou 26 de 42).
% <p >

Ademais, a fim de interagir com os participantes e aprofundar a investigação, foram
				realizadas entrevistas com sete AU: dois servidores públicos, dois profissionais
				liberais, dois docentes e um estudante de pós-graduação.

%endelem
%tag
% <sec sec-type="results">
\section{% <title >
Resultados% <title >
Aceitação da participação do público e da utilização de novas
					tecnologias% <title >
- Os habitantes são peças fundamentais e o %tag
% <italic >
modus
								operandi%endelem
 não responde às demandas% <title >
- Os planejadores tendem a desconsiderar os dados produzidos pelos
							cidadãos, e estes, a partir das transformações socioculturais e
							tecnológicas, estão capacitados% <title >
- Faz-se obrigatório adicionar novas ferramentas e não se dispõe da
							capacitação técnica necessária% <title >
Avaliação do método PPSIG% <title >
- Fatores de qualidade individualmente significativos podem ser
							facilmente analisados, e as informações coletadas com a PPSIG podem
							levar a soluções diferentes% <title >
- As informações coletadas com a PPSIG podem dificultar ainda mais a
							tarefa do planejamento urbano, e as informações coletadas são difíceis
							de interpretar% <title >
- Mapa apresentado no questionário% <title >
- Como você avalia a legibilidade e a utilidade deste mapa}
% <p >

A expressiva maioria dos AU (88,1\%, ou 37 de 42) concorda totalmente (40,5\%, ou
					17 de 42) ou concorda (47,6\%, ou 20 de 42) com a afirmação de que “%tag
% <italic >
Os
						habitantes são peças fundamentais no planejamento urbano, pois conhecem a
						realidade e os problemas locais melhor do que ninguém%endelem
” (%tag
% <xref ref-type="table" rid="t01">
Tabela 1%endelem
). Logo, pode-se afirmar que,
					segundo os respondentes, os habitantes devem fazer parte do planejamento urbano,
					pois o seu conhecimento sobre a cidade é fundamental. Contudo, nos comentários,
					parte dos AU discordou da expressão “melhor do que ninguém”. Por exemplo:
						“%tag
% <italic >
Conhecem de forma cotidiana e extensa, mas não ‘melhor do que
						ninguém’. Esse exagero coloca todo e qualquer saber diverso do local em
						xeque, o que não é apropriado%endelem
” (AU 52).
% <p >

Fonte: Elaborado pelos autores. Nota: AU = arquitetos e urbanistas;
								CT = concordo totalmente; C = concordo; NCND = nem concordo nem
								discordo; D = discordo; DT = discordo totalmente.
% <p >

Destaca-se também a recorrência de “poréns” nos comentários, o que, por vezes,
					transmite certa conotação de hierarquia: “%tag
% <italic >
Eu diria que eles conhecem
						muito bem uma faceta dos problemas, mas, sem um embasamento técnico, eles
						podem ter uma visão distorcida e parcial dos problemas%endelem
” (AU
					29).
% <p >

Sem embargo, prevalece a visão de que a opinião do público deve considerar uma
					camada de informação tão importante e necessária quanto todas as outras.
					Conforme comentários: “%tag
% <italic >
São peças importantes não porque sabem mais do
						que ninguém, mas porque são decisores tão importantes quanto qualquer
						outro%endelem
” (AU 35). O entrevistado D sintetiza a questão da seguinte
					forma: 
% <p >

Teríamos que discutir esta questão da validade da opinião das pessoas, porque
						muitas pessoas não acreditam que seja válido o olhar do cidadão
							%tag
% <italic >
[...]%endelem
 e o olhar do cidadão é verdadeiro porque é o
						olhar do cidadão.
% <p >

Quanto à afirmação “%tag
% <italic >
Em muitos aspectos o modus operandi do planejamento
						urbano não responde às demandas dos cidadãos e aos desafios da complexidade
						urbana%endelem
”, a expressiva maioria dos respondentes (93,1\%, ou 39 de 42)
					concorda totalmente (45,2\%, ou 19 de 42) ou concorda (47,6\%, ou 20 de 42) (%tag
% <xref ref-type="table" rid="t01">
Tabela 1%endelem
). Alguns respondentes consideram
					que isso ocorre “%tag
% <italic >
justamente por faltar a visão do cidadão sobre o
						tema%endelem
” (AU 41). Para muitos, entretanto, isso ocorre porque a
					vontade política, em geral, prevalece sobre as indicações técnicas: 
% <p >

É preciso discernir o modo como se opera o planejamento urbano de fato do
						modo como deveria ser ou se espera que seja. Pois as propostas e diretrizes,
						enfim, todo o produto do trabalho dos planejadores é descartado para atender
						a desejos políticos %tag
% <italic >
(AU 29).%endelem

% <p >

Logo, as barreiras institucionais e políticas ainda são grandes entraves para a
					prática do planejamento urbano participativo, conforme aponta %tag
% <xref ref-type="bibr" rid="B002">
Brown (2012)%endelem
.
% <p >

Com relação à afirmação de que “%tag
% <italic >
Os planejadores tendem a desconsiderar os
						dados produzidos pelos cidadãos em suas atividades cotidianas%endelem
”,
					embora a maioria dos AU (57,1\%, ou 24 de 42) concorde totalmente (21,4\%, ou 9 de
					42) ou concorde (35,7\%, ou 15 de 42), o número daqueles que nem concordam nem
					discordam também é relevante (38,1\%, ou 16 de 42) (%tag
% <xref ref-type="table" rid="t02">
Tabela 2%endelem
).
% <p >

Nota: Elaborado pelos autores. Legenda: AU = arquitetos e urbanistas;
								CT = concordo totalmente; C = concordo; NCND = nem concordo nem
								discordo; D = discordo; DT = discordo totalmente.
% <p >

Nos comentários, os AU ressaltaram a dificuldade de acesso a esses dados. Outros
					defenderam: “%tag
% <italic >
Mesmo que desconexos de instrumentos e da lógica dos SIG,
						planejadores estão se utilizando de informações disponíveis pelos usuários
						na internet, como fotos, vídeos etc%endelem
.” (AU 26). Mas também houve uma
					crítica mais direta: 
% <p >

Diferente de muitas áreas que já se utilizam dos dados gerados pelos cidadãos
						de forma voluntária ou não, o planejamento está longe disso
							%tag
% <italic >
[...]%endelem
 não temos o costume e a técnica para buscar estes
						dados e torná-lo parte do processo %tag
% <italic >
(AU 41).%endelem

% <p >

Observa-se, portanto, que não há consenso entre os AU respondentes, confirmando
					que ainda não se tem um entendimento de como o planejamento urbano acompanha as
					mudanças tecnológicas, conforme apontam %tag
% <xref ref-type="bibr" rid="B018">
Pereira
						et al. (2013)%endelem
, e evidenciando a necessidade de renovação das práticas
					de coleta e de interpretação de dados, corroborando a literatura (%tag
% <xref ref-type="bibr" rid="B019">
Pfeffer et al., 2013%endelem
; %tag
% <xref ref-type="bibr" rid="B013">
Horelli et al., 2013%endelem
).
% <p >

Quanto à afirmação de que “%tag
% <italic >
Os cidadãos, a partir das transformações
						socioculturais e tecnológicas, estão capacitados a atuar e influenciar o
						modo de pensar e agir sobre as cidades%endelem
”, a maioria dos AU (73,8\%,
					ou 31 de 42) concorda totalmente (19\%, ou 8 de 42) ou concorda (54,8\%, ou 23 de
					42) (%tag
% <xref ref-type="table" rid="t02">
Tabela 2%endelem
). No entanto, por um lado,
					alguns fizeram ressalvas: 
% <p >

Não penso que sejam as ‘transformações socioculturais e tecnológicas’ que
						tenham mudado as coisas [...] as tecnologias são mero instrumento.
						Participei de momentos muito intensos de planejamento urbano participativo
						na década de 80 %tag
% <italic >
(AU 24); e%endelem

% <p >

Uma pessoa ter acesso a smartphones, tablets etc., não significa que esteja
						apto %tag
% <italic >
[...].%endelem
 Porém, utilizando-se de uma interface de fácil
						entendimento e manuseio, facilitaria. Ainda, acho que a inovação tecnológica
						é uma ferramenta muito interessante %tag
% <italic >
[...],%endelem
 mas acredito que
						abrange apenas uma parcela da população %tag
% <italic >
(AU 37).%endelem

% <p >

Por outro lado, outros questionados concordam que a ubiquidade das TIC está
					produzindo transformações significativas: 
% <p >

As novas ferramentas e, principalmente, sua acessibilidade crescente permitem
						que a interação dos cidadãos com o planejamento seja mais direta. Informa-se
						mais e assim tem mais capacidade de opinar ou tomar decisão %tag
% <italic >
(AU 41);
							e%endelem

% <p >

Eu acredito nesta ideia de que, do ponto de vista contemporâneo, as pessoas
						podem ser consideradas sensores e nós teríamos essa ferramenta de mapeamento
						capaz de capturar isso de um modo mais rápido %tag
% <italic >
(Entrevistado
							D).%endelem

% <p >

Pode-se discutir mais a fundo se os avanços tecnológicos estão empoderando ou não
					os cidadãos, mas, indiscutivelmente, estão produzindo mudanças nas relações do
					cidadão com o espaço urbano e com o governo. Assim, reforça-se o entendimento de
					que os avanços tecnológicos demandam mudanças na prática do planejamento urbano.
					O entrevistado I resume a questão: 
% <p >

Essa coisa de o usuário poder decidir coisas pela tecnologia vai até um certo
						ponto, mas ele não consegue decidir coisas mais estratégicas, então o
						planejamento não vai deixar de existir, mas ele tem que mudar, se
						reformular, ser mais rápido nas respostas.
% <p >

No mesmo viés, a expressiva maioria dos AU (92,9\%, ou 39 de 42) concorda
					totalmente (52,4\%, ou 22 de 42) ou concorda (40,5\%, ou 17 de 42) que
						“%tag
% <italic >
Atualmente se faz obrigatório adicionar novas ferramentas no
						planejamento urbano, as quais podem exibir e gerenciar novos fluxos de
						informações%endelem
” (%tag
% <xref ref-type="table" rid="t03">
Tabela 3%endelem
). Os
					comentários confirmam que os AU desejam utilizar novas tecnologias: “%tag
% <italic >
Não
						se pode planejar como se fazia anos atrás. Ter acesso aos dados disponíveis
						ou às fontes é fundamental para repostas mais consistentes%endelem
” (AU
					41).
% <p >

Fonte: Elaborado pelos autores. Nota: AU = arquitetos e urbanistas;
								TIC = Tecnologias da Informação e Comunicação; SIG = Sistemas de
								Informação Geográfica; CT = concordo totalmente; C = concordo; NCND
								= nem concordo nem discordo; D = discordo; DT = discordo
								totalmente.
% <p >

Quanto à afirmação de que “%tag
% <italic >
Ainda não se dispõe da capacitação técnica
						necessária para utilização das TIC e dos SIG no planejamento
					urbano%endelem
”, embora a maioria dos respondentes (54,7\%, ou 23 de 42)
					concorde totalmente (9,5\%, ou 4 de 42) ou concorde (45,2\%, ou 19 de 42), a
					porcentagem daqueles que nem concorda nem discorda não é desprezível (23,8\%, ou
					10 de 42) (%tag
% <xref ref-type="table" rid="t03">
Tabela 3%endelem
). Nos comentários,
					evidencia-se que, alguns dos que nem concordam nem discordam, o fazem por não
					terem conhecimento a respeito: “%tag
% <italic >
Não tenho conhecimento que me permita
						responder isso. Talvez os técnicos tenham a capacitação, mas não tenham
						infraestrutura e suporte tecnológico adequado nas instituições de
						atuação%endelem
” (AU 12).
% <p >

Recentemente, o Censo dos Arquitetos e Urbanistas do Brasil, realizado pelo
					Conselho de Arquitetura e Urbanismo (%tag
% <xref ref-type="bibr" rid="B005">
CAU,
						2014%endelem
), revelou que 28,04\% dos profissionais possuem conhecimento bom
					de %tag
% <italic >
softwares%endelem
 de geoprocessamento, 33,81\%, ruim, e 38,15\%
					desconhecem. Esse fato reforça o entendimento de que parte considerável dos AU
					ainda não dispõe de capacitação técnica adequada para utilização dos SIG.
					Aqueles respondentes que concordam com essa visão comentaram:
						“%tag
% <italic >
Realmente, na minha opinião, falta capacitação técnica, inclusive
						dentro das prefeituras. Poucos técnicos se interessam e se capacitam nesta
						área%endelem
” (AU 37); e “%tag
% <italic >
Está começando, mas não quer dizer que
						todas as pessoas que trabalhem com planejamento urbano vejam esse
						potencial%endelem
” (Entrevistado J). Logo, a necessidade da capacitação
					técnica em SIG existe.
% <p >

Nas entrevistas, foi perguntado se o AU está preparado para lidar com a opinião
					da população e qual a disposição e a capacidade de usarem essa camada de
					informação. Percebe-se que alguns AU ainda veem o público como tendo um déficit
					de conhecimento, conforme discutido por %tag
% <xref ref-type="bibr" rid="B006">
Corburn
						(2003)%endelem
. Por exemplo: “%tag
% <italic >
A população não consegue perceber toda
						a dimensão do urbano que é extremamente complexa. Nós, com conhecimento,
						temos dificuldades, imagina a população%endelem
” (Entrevistado M). De outra
					parte, na opinião do entrevistado O: “%tag
% <italic >
Tem a dimensão política%endelem

					[...], %tag
% <italic >
tem também o ranço de o arquiteto ser a única voz do processo,
						determinadas escolas formam muito nessa coisa autoral e transportam isso num
						discurso de cidade%endelem
”.
% <p >

Portanto, há divergências nas opiniões. Mas deve-se ter em mente que se trata de
					um processo de mudança. Antigamente era feito de um jeito e atualmente está se
					mostrando a necessidade de uma outra maneira. A princípio, com o passar do
					tempo, a tendência é que essa questão seja suplantada, tendo em vista a
					crescente importância atribuída ao conhecimento local, que pressiona os
					profisisonais a mudar sua postura. Corroborando, o entrevistado O acredita que o
					AU tem de adotar uma postura diferente – de ouvinte – e trabalhar mais em prol
					dos processos participativos. Isso porque, segundo ele, o planejamento
					participativo, tal qual praticado atualmente, não fomenta o debate de
					opiniões.
% <p >

Por fim, quanto ao papel do planejador urbano nos processos participativos, os
					entrevistados têm um entendimento semelhante àquele encontrado na literatura
					acerca da mediação dos diferentes interesses envolvidos no planejamento urbano,
					mas sem abdicar do seu conhecimento (%tag
% <xref ref-type="bibr" rid="B024">
Souza
						\& Rodrigues, 2004%endelem
; %tag
% <xref ref-type="bibr" rid="B025">
Staffans et
						al., 2010%endelem
; %tag
% <xref ref-type="bibr" rid="B017">
Mäntysalo et al.,
						2011%endelem
). De acordo com o entrevistado D: 
% <p >

Eu vejo o serviço dos planejadores como um direito das pessoas
							%tag
% <italic >
[...]%endelem
 eu como morador da cidade quero dizer as coisas
						que eu penso, quero que alguém me preste um serviço %tag
% <italic >
[...]%endelem

						uma equipe de profissionais que faça a análise dos resultados, que
						julgue.
% <p >

Os entrevistados também salientaram a necessidade de, nesse processo de mediação
					dos diferentes interesses, buscar-se a “consertação”, ou seja, um pacto ou
					acordo entre as partes. Essa visão corrobora o pensamento do planejamento
					agonístico (%tag
% <xref ref-type="bibr" rid="B017">
Mäntysalo et al., 2011%endelem
), o
					qual argumenta que o consenso, embora uma condição legítima, não deve ser
					mandatório, pois as diferentes racionalidades podem e devem coexistir. Assim,
					deve-se aprender a gerenciar o conflito intrínseco das diferentes racionalidades
					envolvidas no processo de planejamento urbano e buscar a pactuação. Segundo o
					entrevistado O, sendo a cidade uma manifestação física, se ela for produzida por
					meio de conflitos, vai expressar isto. Porém, a cidade também é um local de
					encontro, e, se não houver um projeto comum, abolir-se-á a cidade. Na sua visão,
					deve-se, no momento propositivo, interpretar esses conflitos e achar um ponto
					comum.
% <p >

Nas questões que abordaram especificamente o método PPSIG, foram considerados
					apenas os 19 AU (45,2\%, ou 19 de 42) que afirmaram ter conhecimento prévio do
					método. A totalidade desses respondentes concorda totalmente (52,6\%, ou 10 de
					19) ou concorda (47,4\%, ou 9 de 19) com a afirmação de que “%tag
% <italic >
A PPSIG
						possibilita a inclusão do conhecimento local de forma sistemática em um
						banco de dados espacial que pode alimentar um sistema de suporte à
						decisão%endelem
”.
% <p >

Nos comentários, um respondente destacou a necessidade de dados coletados serem
					transformados em informação útil, ou seja, que o dado seja analisado de forma
					consistente para ser utilizado no planejamento urbano. Em suas palavras: 
% <p >

O conhecimento local é fundamental, mas, se ele não estiver adequadamente
						estruturado (se o dado não for convertido em informação útil), ele pode
						acabar não contribuindo para um sistema de suporte à decisão eficiente
							%tag
% <italic >
(AU 55).%endelem

% <p >

Similarmente, outros respondentes chamaram a atenção para a necessidade de
					conhecimento técnico adequado para analisar os dados coletados de forma
					consistente: “%tag
% <italic >
Isso depende muito do modo como esse tal dado qualitativo
						é inserido, ou pode-se incorrer facilmente em falsas correlações%endelem
”
					(AU 29).
% <p >

Quanto à afirmação de que %tag
% <italic >
“Com o método PPSIG, fatores de qualidade
						individualmente significativos (por exemplo, níveis de satisfação com os
						espaços urbanos) podem ser facilmente analisados em relação a outras camadas
						de informação (por exemplo, dados censitários)”%endelem
, a expressiva
					maioria dos respondentes (89,4\%, ou 17 de 19) concorda totalmente (36,8\%, ou 7
					de 19) ou concorda (52,6\%, ou 10 de 19) (%tag
% <xref ref-type="table" rid="t04">
Tabela
						4%endelem
). O resultado confirma que a PPSIG possibilita a integração de
					informações qualitativas e quantitativas, de acordo com %tag
% <xref ref-type="bibr" rid="B016">
Kahila \& Kyttä (2009)%endelem
. Nesse sentido, um respondente
					afirmou que o cruzamento de informações só tem a agregar valor ao resultado
					final, já que haverá um cruzamento de dados que pode ajudar a se obter
					interpretações ou proposições projetais mais assertivas (AU 54).
% <p >

Fonte: Elaborado pelos autores. Nota: AU = arquitetos e urbanistas;
								PPSIG = Participação Pública com Sistema de Informação Geográfica;
								CT = concordo totalmente; C = concordo; NCND = nem concordo nem
								discordo; D = discordo; DT = discordo totalmente.
% <p >

A expressiva maioria dos respondentes (84,2\%, ou 16 de 19) também concorda
					totalmente (36,8\%, ou 7 de 19) ou concorda (47,4\%, ou 9 de 19) com a afirmação
					de que “%tag
% <italic >
As informações coletadas com a PPSIG podem levar a soluções
						diferentes das que teriam sido alcançadas utilizando-se apenas fontes
						oficias de dados e o conhecimento especialista%endelem
” (%tag
% <xref ref-type="table" rid="t04">
Tabela 4%endelem
).
% <p >

Nos comentários, entretanto, alguns AU expressaram ressalvas: “%tag
% <italic >
Não
						afirmaria que existiria uma dualidade entre a informação popular e a
						especializada%endelem
” (AU 26); e “%tag
% <italic >
Sim, mas creio que seja
						importante saber qual o público que está realmente participando%endelem
”
					(AU 46). De qualquer forma, a literatura (%tag
% <xref ref-type="bibr" rid="B016">
Kahila \& Kyttä, 2009%endelem
; %tag
% <xref ref-type="bibr" rid="B025">
Staffans
						et al., 2010%endelem
) já indica que os planejadores, inevitavelmente,
					precisarão desenvolver um entendimento mais profundo do conhecimento escondido
					nas experiências cotidianas dos indivíduos, bem como habilidades para lidar com
					ele. Isso implica na necessidade de reconsiderar suas formas de trabalho e os
					métodos por meio dos quais as informações de planejamento são criadas,
					distribuídas, processadas e usadas (%tag
% <xref ref-type="bibr" rid="B025">
Staffans et
						al., 2010%endelem
).
% <p >

Quanto à afirmação de que “%tag
% <italic >
As informações coletadas com a PPSIG podem
						dificultar ainda mais a tarefa do planejamento urbano, pois acrescenta uma
						camada extra de informação: a opinião do público%endelem
”, a maioria dos
					respondentes (73,7\%, ou 14 de 19) discorda (31,6\%, ou 6 de 19) ou discorda
					totalmente (42,1\%, ou 8 de 19) (%tag
% <xref ref-type="table" rid="t05">
Tabela
					5%endelem
).
% <p >

Fonte: Elaborado pelos autores. Nota: AU = arquitetos e urbanistas;
								PPSIG = Participação Pública com Sistema de Informação Geográfica;
								CT = concordo totalmente; C = concordo; NCND = nem concordo nem
								discordo; D = discordo; DT = discordo totalmente.
% <p >

Nota-se, pelos comentários, que, embora os AU reconheçam que adicionar a opinião
					do público traz ainda mais complexidade à prática do planejamento urbano, tal
					informação é bem-vinda, pois o qualifica e tende a tornar os resultados mais
					efetivos. Para o entrevistado N, é cada vez mais difícil lidar com tanta
					informação, mas é melhor ter um excesso de informação e poder fazer o filtro
					técnico. Segundo o entrevistado O, caso a opinião do público não seja
					considerada, posteriormente as soluções podem vir a ser questionadas, ou seja,
					perguntar antes significa antecipar-se aos problemas. Ainda, mesmo que a
					quantidade de informação que se tem antes, durante e inclusive ao longo da
					construção da cidade seja muito grande, o planejador tem que dar conta de todo
					esse montante, isto é, ter capacidade de filtrar o que interessa, abstrair o que
					não interessa e aplicar o que for possível. Nesse sentido, a utilização das
					informações coletadas com a PPSIG também se traduziria em ganhos de eficiência
					na gestão urbana por meio de uma maior aceitação social das propostas, em
					conformidade com %tag
% <xref ref-type="bibr" rid="B021">
Rantanen \& Kahila
						(2009)%endelem
.
% <p >

Na mesma direção, a maioria dos AU discorda totalmente (26,3\%, ou 5 de 19) ou
					discorda (36,8\%, ou 7 de 19) da afirmação de que “%tag
% <italic >
As informações
						coletadas com a PPSIG são difíceis de interpretar, pois são vagas%endelem
”
						(%tag
% <xref ref-type="table" rid="t05">
Tabela 5%endelem
). Diversos comentários dão
					conta de que tudo depende de como a ferramenta é elaborada, por exemplo: 
% <p >

Depende de como será a plataforma. Ela pode apontar questões pontuais para
						votação ou pode ser extremamente aberta para comentários (dessa forma, será
						necessário um agrupamento dos comentários por tema) %tag
% <italic >
(AU
							46).%endelem

% <p >

Ainda, destacam-se os seguintes comentários: “%tag
% <italic >
De fato são difíceis de
						interpretar, mas não porque ‘são vagas%endelem
’%tag
% <italic >
, talvez sejam bem
						precisas, mas não conseguem incluir a riqueza das discussões presenciais nem
						a profundidade das análises espaciais avançadas%endelem
” (AU 24);
						“%tag
% <italic >
Se a população não estiver capacitada e se o método utilizado não
						for muito objetivo, podem surgir informações muito vagas. O mesmo ocorre com
						informações coletadas através de métodos tradicionais
					(presenciais)%endelem
” (AU 43); e “%tag
% <italic >
As informações contêm, além da
						resposta do usuário, a variável espacializada, que é fundamental no processo
						de planejamento urbano%endelem
” (AU 26). Enquanto o primeiro relaciona a
					qualidade da opinião do público com os métodos presenciais, o segundo acredita
					que, se o público não estiver capacitado para participar, qualquer método pode
					ser falho. A primeira visão valida que a utilização de várias técnicas
					participativas é o ideal a ser buscado, já que a participação on-line não
					substitui a presencial, mas a complementa (%tag
% <xref ref-type="bibr" rid="B022">
Rojas \& Puig‐i‐Abril, 2009%endelem
). A segunda ilustra que, embora sejam
					minoria, alguns profissionais ainda veem o público como tendo um déficit de
					conhecimento, conforme discutido por %tag
% <xref ref-type="bibr" rid="B006">
Corburn
						(2003)%endelem
. Por fim, a última manifestação valoriza um importante
					diferencial do método PPSIG, apontado por %tag
% <xref ref-type="bibr" rid="B020">
Poplin (2012)%endelem
: a conveniência de se ter os dados coletados de forma
					automatizada e georreferenciados, o que, diferentemente do que ocorre nos
					encontros presenciais, agrega uma valiosa informação à opinião do público – a
					sua localização –, facilitando consideravelmente a sua interpretação, análise e,
					consequentemente, incorporação no planejamento urbano.
% <p >

Na sequência, há um questionamento, da parte dos respondentes, sobre a
					legibilidade e a utilidade do mapa para o planejamento urbano (%tag
% <xref ref-type="fig" rid="gf01">
Figura 1%endelem
). Ele foi produzido a partir de
					dados coletados em um experimento PPSIG realizado em Jaguarão, no Estado do Rio
					Grande do Sul (%tag
% <xref ref-type="bibr" rid="B003">
Bugs, 2014%endelem
). Ainda que
					parte dos respondentes tenha considerado a legibilidade do mapa satisfatória
					(47,4\%, ou 9 de 19), um número não desprezível a considerou nem satisfatória nem
					insatisfatória (21,1\%, ou 4 de 19) e mesmo insatisfatória (31,6\%, ou 6 de 19)
						(%tag
% <xref ref-type="table" rid="t06">
Tabela 6%endelem
). Foram feitas algumas
					críticas, principalmente, referente ao gradiente utilizado para representar as
					faixas de renda e à classificação por cores dos pontos marcados pelos
					participantes (branco: sem classificação; coloridos: classificação segundo os
					comentários dos respondentes).
% <p >

Fonte: Elaborado pelos autores.
% <p >

Assim, observa-se que, embora um exercício de sobreposição de informações, que
					objetiva mostrar relações entre varáveis e diferentes interpretações da questão
					em pauta, seja um instrumento fundamental de análise para o planejamento urbano,
					ele não é tão fácil de ser produzido, em sintonia com %tag
% <xref ref-type="bibr" rid="B016">
Kahila \& Kyttä (2009)%endelem
. No caso, foram sugeridas
					algumas alterações relativamente simples de se executar: atenção ao gradiente de
					cores e/ou transparência na sobreposição com imagens; e informação correta da
					fonte dos dados. Não obstante, entende-se que ainda se tem um longo caminho no
					tocante à visualização e interpretação da informação geográfica
					(geovisualização) para o planejamento urbano.
% <p >

Nas entrevistas, questionou-se como o dado coletado com a PPSIG pode ser
					utilizado no planejamento urbano, haja vista que essa é uma questão importante,
					mas, por vezes, negligenciada, pois não raramente se coletam dados que não são
					utilizados depois (%tag
% <xref ref-type="bibr" rid="B021">
Rantanen \& Kahila,
						2009%endelem
). Na opinião dos entrevistados, a questão passa, principalmente,
					pela correta incorporação da tecnologia SIG no planejamento urbano e pela
					capacitação técnica dos planejadores para utilizá-la.
% <p >

Na opinião de um AU que trabalha no poder público, a análise dos dados coletados
					deveria ser feita por terceiros: “%tag
% <italic >
A informação tem que chegar
						estruturada, em um nível pronto para a apreensão do técnico, senão o técnico
						não vai parar e ficar olhando resultado por resultado, porque o cliente dele
						não é um, mas são milhares%endelem
” (Entrevistado J). Essa visão se
					relaciona com o fato de que lidar com mais uma camada de informação, a qual,
					conforme destacado, não é simples de ser representada e analisada, e agrega
					dificuldades extras (por exemplo, disponibilidade de tempo) à complexa prática
					do planejamento urbano.
% <p >

Sendo assim, observa-se que, apesar dos avanços ferramentais e tecnológicos,
					ainda há uma certa dificuldade para se realizar análises estatísticas e
					trabalhar com SIG, o que se configura na principal barreira à incorporação dos
					dados coletados com a ferramenta PPSIG no planejamento urbano. Similarmente,
						%tag
% <xref ref-type="bibr" rid="B009">
Göçmen \& Ventura (2010)%endelem
 afirmam
					que nos Estados Unidos, o potencial dos SIG como ferramenta de planejamento não
					está sendo totalmente explorado. A capacitação dos técnicos, especificamente em
					aplicações de SIG no planejamento urbano, é a principal medida que deve ser
					tomada para alterar essa realidade, segundo os autores.
% <p >

Também questionou-se os entrevistados sobre “%tag
% <italic >
Como você vê as seguintes
						possibilidade de aplicação da PPSIG: (i) para coletar a percepção e/ou
						opiniões da população antes de um diagnóstico e/ou de uma etapa de
						desenvolvimento de projeto, e/ou (ii) como um sistema de monitoramento
						permanente?%endelem
”. No primeiro caso, a percepção da população sobre
					determinado tema seria coletada antes de uma etapa inicial de diagnóstico ou de
					projeto e, por conseguinte, incorporada na fase propositiva. O segundo caso
					seria a utilização constante da ferramenta para coletar a percepção da
					população.
% <p >

Em geral, para os entrevistados, ambas possibilidades são positivas e desejáveis.
					Todavia, alguns destacaram que a primeira é mais difícil, porque consultar a
					população antes de realizar um projeto ou proposta ainda não é uma prática
					comum. Assim, para alguns, a aplicação como uma ferramenta de monitoramento é
					mais factível. Nas palavras do entrevistado J: 
% <p >

É o requinte da participação o governo perguntar antes de fazer alguma coisa
						o que se quer. Então, ter esta ferramenta para monitoramento do que está
						sendo feito e colocar isso para o conhecimento do público já é muito válido.
					
% <p >

Já o entrevistado O destaca a importância da aplicação como uma ferramenta de
					monitoramento, pois não se dispõe de um instrumento de coleta de informações em
					longo prazo. Logo, na sua visão, a capacidade de armazenar informação de forma
					sistemática por um período maior é um potencial da PPSIG. No entanto, o
					entrevistado K levanta a questão de que uma ferramenta de coleta permanente da
					opinião da população pode tornar-se um repositório de queixas e reclamações, ao
					passo que para o entrevistado O: “%tag
% <italic >
Na cidade tem que ter alguém que vai
						pegar aquele monte de queixas e vai fazer alguma coisa
					propositiva%endelem
”.
% <p >

Há de se considerar ainda o interesse do público em utilizar a ferramenta PSSIG
					em cada um desses momentos. Para o entrevistado O, as pessoas em geral,
					mobilizam-se para enfrentar um problema, e, por isso, uma aplicação pontual da
					ferramenta atrairia maior interesse do público, uma vez que “%tag
% <italic >
ninguém
						liga para dizer que boa está a sua administração; ligam para dizer que ali
						tem um buraco%endelem
”. Então, para ele, a aplicação da PPISG para
					enfrentar um problema específico seria o mais adequado.
% <p >

Enfim, uma solução plausível foi apontada pelo entrevistado D, no sentido de
					utilizar a PPSIG como um canal permanente de coleta da percepção da população,
					mas que mudaria o tema em questão ao longo do tempo: 
% <p >

Acho que o fundamental é manter a ferramenta de mapeamento pela internet mais
						ou menos permanente. Por exemplo, nós vamos fazer o plano diretor, esta
						ferramenta está; se nós vamos fazer uma melhoria na praça, esta ferramenta
						está %tag
% <italic >
[...].%endelem
 Como sempre há alguma coisa, esta ferramenta
						estaria aplicada a cada momento.
% <p >

Portanto, conclui-se que aplicar a PPSIG de forma contínua no planejamento
					urbano, porém sempre voltada a questões objetivas e pontuais, ou seja, a
					projetos urbanos específicos, e não a problemas abrangentes, parece ser mais
					apropriado, pois, desse modo:
% <p >

O público se mobilizaria mais;
% <p >

As reclamações particularizadas seriam menos prováveis;
% <p >

O projeto proposto seria mais facilmente aceito;
% <p >

Seria possível comparar a opinião do público ao longo do tempo;
% <p >

O planejador teria o conhecimento local sempre à sua disposição.

%endelem
%tag
% <sec sec-type="conclusions">
\section{% <title >
Conclusões}
% <p >

Os resultados deste estudo indicam que, apesar da complexidade de se lidar com a
				opinião do público, o conhecimento local é considerado fundamental e, portanto,
				caracteriza-se como uma camada de informação indispensável para tornar o
				planejamento urbano mais efetivo. Contudo, por muito tempo o conhecimento
				especialista reinou de forma absoluta no planejamento urbano. Para diminuir a
				divisão entre expertos e leigos, e difundir a noção de que melhores soluções e
				ideias surgem a partir do envolvimento de pessoas leigas e seu conhecimento tácito
				do que aquelas desenvolvidas apenas por especialistas (%tag
% <xref ref-type="bibr" rid="B019">
Pfeffer et al., 2013%endelem
), assume-se, em conformidade com o
				planejamento agonístico (%tag
% <xref ref-type="bibr" rid="B017">
Mäntysalo et al.,
					2011%endelem
), que a racionalidade surge da interação entre uma série de atores,
				cada um defendendo os seus objetivos e interesses.
% <p >

A não aceitação e, por conseguinte, a não aplicação da opinião do público nas
				propostas são apontadas como sendo um dos motivos pelos quais o planejamento urbano
				não responde às demandas dos cidadãos e aos desafios da complexa realidade urbana.
				Nesse sentido, a efetiva incorporação do conhecimento local no planejamento passa,
				principalmente, por questões estruturais e ideológicas. Para que o conhecimento
				local seja incorporado adequadamente no planejamento urbano, as estruturas de poder
				existentes devem aceitar que segmentos leigos da sociedade têm valiosos
				conhecimentos e podem contribuir substancialmente para as decisões de planejamento e
				de gestão urbana. Em geral, há um medo de se perder o controle do processo e é
				difícil para o técnico compartilhar a definição das soluções com aqueles que não
				investiram tempo e esforço para obter formação técnica, conforme aponta %tag
% <xref ref-type="bibr" rid="B002">
Brown (2012)%endelem
.
% <p >

Contudo, percebe-se a existência de um entendimento o qual demonstra que essa postura
				prejudica o sucesso da atividade. Também, que está em curso um processo de mudança
				no modo de se fazer planejamento urbano, reforçada pela noção de o planejador urbano
				ter o dever de ser um mediador dos diferentes interesses envolvidos no planejamento
				urbano, isto é, aquele que busca a concertação entre as partes por meio da
				utilização dos seus conhecimentos técnicos, em conjunto com o conhecimento dos
				demais (%tag
% <xref ref-type="bibr" rid="B024">
Souza \& Rodrigues, 2004%endelem
; %tag
% <xref ref-type="bibr" rid="B025">
Staffans et al., 2010%endelem
). A experiência
				profissional dos planejadores não deixa de ser valorizada e suas competências são
				reforçadas com a inclusão das informações e ideias produzidas por vários atores.
				Assim, o saber do técnico/especialista não substitui a experiência dos cidadãos, mas
				tal experiência é incluída de forma colaborativa no planejamento (%tag
% <xref ref-type="bibr" rid="B015">
Innes \& Booher, 2004%endelem
).
% <p >

No entanto, a despeito de que os avanços tecnológicos estão produzindo um novo
				contexto em que um grupo maior de cidadãos está se habilitando a participar no
				planejamento urbano, evidenciou-se, neste estudo, que ainda não há consenso entre os
				profissionais sobre como considerar os dados produzidos pelos cidadãos nas suas
				atividades cotidianas, como os oriundos do uso de %tag
% <italic >
smartphones%endelem
 ou de
				atividades voluntarias e/ou colaborativas de mapeamento. Assim, o aproveitamento das
				informações produzidas pelos cidadãos de forma “espontânea” por meio das tecnologias
				digitais ainda é uma questão em debate, conforme salientado por %tag
% <xref ref-type="bibr" rid="B018">
Pereira et al. (2013)%endelem
.
% <p >

No entanto, mesmo considerando as limitações da amostra deste estudo, pode-se dizer
				que os AU que trabalham com planejamento urbano tendem a desejar utilizar novas
				tecnologias no planejamento urbano e estão abertos à participação do público. Nesse
				sentido, a ampliação da participação pública e a utilização de novas ferramentas
				passam pela adoção de novas metodologias de trabalho, bem como maior capacitação
				técnica para se lidar tanto com a opinião do público quanto com os novos fluxos de
				informação da era digital.
% <p >

Com relação à avaliação do método PPSIG, os resultados indicam que é consenso entre
				os especialistas participantes da amostra que tal método possibilita acessar e
				incorporar o conhecimento local no planejamento urbano de forma sistemática e que
				esse conhecimento, por ser uma fonte única de informações atualizadas, ajuda a
				melhorar a qualidade do conteúdo dos planos e/ou projetos urbanos. Além disso, os AU
				concordam que a percepção da população deveria formar uma camada adicional de
				informações e ser analisada em conjunto com as demais camadas necessárias ao
				planejamento urbano.
% <p >

Porém, para que isso ocorra, há a necessidade de se ampliar o conhecimento técnico
				para que os dados coletados com a PPSIG possam ser analisados e, por conseguinte,
				incorporados no planejamento urbano de forma consistente. Assim, a relativa falta de
				conhecimento sobre os SIG configura-se em barreira à incorporação dos dados
				coletados com a PPSIG no planejamento urbano, o que indica a necessidade de
				capacitação técnica.
% <p >

Também foi citado que a utilização de metodologias como a PPSIG se traduziria em
				ganhos de eficiência por meio de uma maior aceitação social das propostas. A PPSIG
				facilita que a opinião do público seja considerada na construção das propostas, uma
				vez que contém a variável espacializada. Mas, para que a interpretação das
				informações coletadas com a PPSIG seja satisfatória, a elaboração da ferramenta e
				das perguntas que preveem a marcação de lugares no mapa on-line interativo deve ser
				definida conforme os objetivos da pesquisa a fim de se coletar somente informações
				claras e úteis para o planejamento.
% <p >

Logo, mesmo que ainda existam barreiras para se trabalhar em um plano mais elevado de
				colaboração com a população, demonstrou-se, conforme as opiniões dos especialistas
				consultados neste estudo, que novas metodologias participativas, que tirem partido
				das TIC e dos SIG, tal como a PPSIG, podem auxiliar significativamente na difícil
				tarefa de acessar e incorporar o conhecimento local no planejamento urbano. O uso
				permanente da ferramenta PPSIG para coletar a percepção da população, sempre que
				aplicado a questões objetivas e pontuais, é apontado pelos AU como sendo desejável e
				positivo.
% <p >

Ainda, conforme a avaliação dos especialistas, conclui-se que o planejamento urbano
				participativo por meio da utilização de novas tecnologias, tal como a PPSIG, poderia
				permear todo o processo de planejamento, desde a coleta de informações sobre a
				percepção da população acerca de determinado tema, antes mesmo da elaboração das
				propostas, até o monitoramento da opinião da população ao longo do processo.
% <p >

Concluindo, espera-se que os resultados desta pesquisa auxiliem a reforçar a
				necessidade de uma postura, por parte daqueles envolvidos com o planejamento urbano,
				que esteja aberta à inclusão do conhecimento local por meio da utilização de novas
				tecnologias, tais como a PPSIG, necessidade esta já identificada em outros estudos
					(%tag
% <xref ref-type="bibr" rid="B001">
Batty et al., 2012%endelem
; %tag
% <xref ref-type="bibr" rid="B018">
Pereira et al., 2013%endelem
).

%endelem
%endelem
%tag
% <back >
%endelem

\end{document}%tag
% </ version="1.0" encoding="UTF-8" transfer-MimeType="text/xml" transfer-Status="200" transfer-Message="OK" transfer-URI="file:///C:/Users/DSoar/Desktop/Jorge/SPS-TeX/standards/1/example.xml" source="example.xml" transfer-Encoding="UTF-8">
%endelem
