% Generated by jats2tex@0.11.0.0
\documentclass{article}

\usepackage[T1]{fontenc}
\usepackage{amstext}
\usepackage{authblk}
\usepackage{unicode-math}
\usepackage{multirow}
\usepackage{graphicx}

\begin{document}

\title{Planejamento urbano participativo por meio da utilização de novas
tecnologias: uma avaliação por especialistas}
\author[\textsuperscript{th}
]{Bugs, Geisa
\textsuperscript{th}
}
\author[\textsuperscript{th}
]{Reis, Antônio Tarcísio da Luz
\textsuperscript{th}
}\affil[aff01]{Universidade FeevaleUniversidade FEEVALEUniversidade Feevale,
Novo Hamburgo, RS,
Brasil.}\affil[aff02]{Universidade Federal do Rio Grande do
SulFaculdade de ArquiteturaUniversidade Federal do Rio Grande do
SulUniversidade Federal do Rio Grande do Sul –
UFRGS, Faculdade de Arquitetura, Programa de Pós-graduação em Planejamento
Urbano - PROPUR, Porto Alegre, RS, Brasil.}

\maketitle

urbeurbe, Rev. Bras. Gest.
Urbana2175-3369Pontifícia Universidade Católica do
Paranáurbe0901AO060810110.1590/2175-3369.009.001.AO06ArticlesParticipatory urban
planning through the use of new technologies:
an evaluation by experts
GB é Arquiteta e Urbanista; Doutora pelo PROPUR/UFRGS, e-mail:
geisa@feevale.br

ATLR é Arquiteto e Urbanista; Doutor pela Pos Graduate Research School,
Oxford Brookes University, e-mail: tarcisio.reis@ufrgs.br
\date{2016}{12}{08}
\newcommand{\volume}{00}
00\newcommand{\fpage}{000}
\newcommand{\lpage}{000}
0309201512012016Este é um artigo publicado em acesso aberto (Open
Access) sob a licença Creative Commons
Attribution, que permite uso, distribuição e reprodução em
qualquer meio, sem restrições desde que o trabalho original seja
corretamente citado.\begin{abstract}
Resumo
O objetivo deste artigo é avaliar a aceitação, por parte dos especialistas, da
participação pública e da utilização de novas tecnologias, bem como investigar
como eles avaliam metodologias alternativas, como a Participação Pública com
Sistema de Informação Geográfica (PPSIG). Assume-se que novas abordagens
metodológicas, que façam uso das Tecnologias da Informação e Comunicação (TIC) e
dos Sistemas de Informação Geográfica (SIG), podem aperfeiçoar os processos de
participação pública no planejamento urbano, tendo em vista a dificuldade de se
incorporar a perspectiva da população e a necessidade de renovação das práticas
para se lidar com os fluxos de informação da era digital. Os procedimentos
metodológicos incluem a aplicação de questionários e a realização de entrevistas
com arquitetos e urbanistas que atuam na área de planejamento urbano. Os
resultados apontam que, apesar das barreiras institucionais e da necessidade de
maior capacitação técnica, há aceitação quanto à participação pública e à
utilização de novas ferramentas, e que a PPSIG pode auxiliar na difícil tarefa
de incorporar a perspectiva do usuário do espaço urbano, na opinião dos
respondentes.

\end{abstract}
\begin{abstract}
Abstract
This paper aims to evaluate the experts acceptance of the public participation,
the use of new technologies, and to investigate how they evaluate alternative
methodologies such as the Public Participation Geographic Information Systems
(PPGIS). We assume that new methodological approaches, using Information and
Communication Technology (ICT) and Geographic Information Systems (GIS), can
improve public participation in urban planning processes, given the difficulty
to include the population perspective, and the need to renew the practices for
dealing with the flow of information in the digital era. Questionnaires and
interviews were conducted with architects working in the urban planning area, as
part of the methodological procedures. The results show that despite
institutional barriers and the necessity of technical training, the public
participation and the use of new tools are accepted, and that PPSIG can help in
the difficult task of incorporating the urban space users perspective, in the
respondents opinion.

\end{abstract}
Palavras-chave: \textit{Planejamento participativo}
\textit{PPSIG}
\textit{TIC}
Keywords: \textit{Participatory planning}
\textit{PPGIS}
\textit{ICT}
\section{Introdução}

O planejamento urbano normalmente tem sido baseado na coleta e na troca de
informações entre diferentes partes interessadas. Contudo, uma mudança no modelo
informacional, em consequência da revolução das tecnologias digitais, tem gerado
um
impacto significativo em como se desenvolve todo o processo (Haller \& Höffken,
2010). Para acompanhar as mudanças
tecnológicas, faz-se necessário adicionar novas ferramentas que permitam exibir
e
gerenciar novos fluxos de informações (Pereira et
al., 2013). Logo, tal mudança também vai exigir alterações no
modus operandi do planejamento urbano, ainda muito influenciado
pelo pensamento do modelo racional (Yigitcanlar,
2006; Horelli et al., 2013),
segundo o qual, os especialistas seriam capazes de considerar as melhores
soluções
para os cidadãos e para a sociedade como um todo, sem incluir, portanto, nenhuma
forma de participação da sociedade civil nas discussões de propostas para a
cidade
(Villaça, 2005).

Na era da informação e da sociedade em rede, ferramentas digitais estão se
tornando
onipresentes na vida cotidiana e possuem alto potencial para a coleta de dados
socioespaciais e temporais, o que é completamente diferente dos modelos
estáticos de
coleta de dados que o planejamento urbano comumente utiliza (Friedmann, 2007;
Batty et
al., 2012). A ubiquidade das Tecnologias da Informação e Comunicação
(TIC) está produzindo ambientes urbanos que são completamente diferentes de tudo
o
que se experimentou até agora, nos quais um grupo muito maior de cidadãos pode
aderir (Pfeffer et al., 2013).

Antigamente, os profissionais eram os únicos produtores e usuários das
informações
relativas às questões urbanas. Entretanto, hoje, qualquer pessoa pode facilmente
produzir um mapa e publicá-lo on-line. Assim, mesmo que involuntariamente, está
ocorrendo um aumento da consciência da informação geográfica pelo público, o que
pode ser considerado uma revolução nesse campo, pois possibilita que ferramentas
do
Sistema de Informação Geográfica (SIG) possam ser compreendidas de maneira
rápida e
eficaz, sem haver a necessidade de imersão em atividades profissionais
(Hudson-Smith \& Crooks, 2008).

Por conseguinte, o mundo digital está transformando a possibilidade de
participação
pública no planejamento urbano (Batty et al.,
2012). A capacidade de todos os cidadãos se comunicarem uns com os
outros, até mesmo com seus representantes, além das novas formas de se
relacionar
com o espaço urbano, abre novas frentes para a ideia de que os cidadãos possam
desempenhar um papel ativo no planejamento urbano. Mas, apesar disso, e dos mais
de
40 anos de história do planejamento participativo (considerando, por exemplo, os
processos de planejamento e construção de habitações na Europa nos anos de 1960
e
1970), a participação pública ainda não conseguiu entrar no
mainstream do planejamento urbano de forma consistente (Horelli, 2002).

Em parte, tal acontecimento ocorre porque envolver o público não é uma tarefa
simples
e direta. Participação plena implica em responsabilização e poderes partilhados.
Para se alcançar esse nível, existe uma série de barreiras que vão desde a
relevância dada às opiniões e posições do público, passando pela credibilidade e
idoneidade destas (Corburn, 2003), até os
entraves institucionais (Brown, 2012). Nas
palavras de Forester (2006, p. 447): “Fácil
de pregar, mas difícil de praticar, participação pública eficaz no planejamento
e na
gestão pública exige sensibilidade e técnica, imaginação e coragem”.

De tal modo, ainda não se conseguiu abordar a perspectiva da população de forma
consistente (Kahila \& Kyttä, 2009). A
literatura destaca, por exemplo, que há grande dificuldade em se passar de um
sistema de planejamento dominado por especialistas para um que daria voz a
diferentes interessados (Wallin \& Horelli,
2012) e que o planejamento urbano, em geral, utiliza somente dados
oficiais, nos quais as pessoas são levadas em consideração apenas na forma de
estatísticas agregadas (Friedmann,
2007).

Distintos autores defendem a necessidade de uma reformulação no planejamento
urbano
por meio da utilização de tecnologias que permitam a criação de técnicas e
metodologias mais interativas, emancipatórias e colaborativas (Yigitcanlar,
2006; Horelli et
al., 2013). Nesse sentido, os recentes avanços tecnológicos em torno das
TIC (em especial, a internet) e dos SIG envolvem novas perspectivas.

Se a internet é o principal meio de troca de informações e comunicação da
atualidade,
os SIG são uma tecnologia que desempenha um papel importante na coleta,
tratamento e
disseminação de informações espaciais – e a maioria das informações necessárias
à
formulação de políticas urbanas contém um componente espacial (Sieber, 2006).
Para Yigitcanlar (2006), o uso efetivo das TIC e dos SIG no planejamento
urbano representa a possibilidade de se estabelecer um sistema permanente de
coleta
das percepções do público sobre o espaço urbano, o que é apontado como essencial
para que ele se torne mais colaborativo.

Nessa direção, estudos recentes apontam que a Participação Pública com Sistema
de
Informação Geográfica (PPSIG) possui potencial para aprimorar a participação do
público no planejamento urbano (Kahila \& Kyttä,
2009; Poplin, 2012). A PPSIG
utiliza ferramentas dos SIG para a participação pública, conectando a capacidade
técnica deles ao conhecimento local (Sieber,
2006). O público participa produzindo mapas e/ou dados espaciais que
representam a sua percepção do espaço urbano em questão (Bugs, 2014).

A PPSIG surgiu nos anos de 1990 a partir de reflexões sobre a interface do SIG
com a
sociedade. No início dos anos de 2000, chegou a ser considerada uma evolução dos
métodos clássicos de participação (Carver,
2001). Mas o entusiasmo inicial começou a diminuir por volta de 2005
(Poplin, 2012), quando vários problemas
foram levantados, tais como o reconhecimento de que termos e conceitos como
“participação” e “empoderamento” foram utilizados de forma acrítica (Hall et
al., 2010). Então, quase que
concomitantemente, os mashups de mapas e o mapeamento voluntário
emergiram na web 2.0. Logo, surgiram experiências tirando proveito dos mapas
on-line
interativos e do aumento do conhecimento da informação geográfica pelo público.
Esses projetos atraíram os cidadãos para o uso de ferramentas de participação
baseadas em mapas on-line (Poplin, 2012), e
o interesse na PPSIG se renovou. Porém o tema ainda é, relativamente, pouco
explorado a partir da perspectiva dos profissionais que trabalham com
planejamento
urbano, principalmente no contexto brasileiro. A maioria das experiências
existentes
foi desenvolvida por geógrafos, com foco na produção de cartografia, ou por
profissionais da tecnologia da informação, com foco no desenvolvimento
tecnológico.

Em suma, estão em curso mudanças na sociedade contemporânea fomentadas pelas
revoluções digital e geoespacial, sem que exista um maior conhecimento acerca do
efeito de tais mudanças sobre o planejamento urbano, o que tem sido tema de
debate
entre os profissionais, conforme apontam Pereira et
al. (2013). Logo, faz-se necessário entender como se dá o planejamento
urbano participativo utilizando novas tecnologias, na opinião dos especialistas.
Assim sendo, esta pesquisa tem como objetivo avaliar a aceitação, por parte dos
planejadores, da participação do público no planejamento urbano e da utilização
de
novas tecnologias, assim como investigar como esses profissionais avaliam
metodologias participativas alternativas, tal como a PPSIG.

\section{Metodologia}

Os dados foram coletados por meio de um questionário aplicado a arquitetos e
urbanistas (AU) que trabalham com planejamento urbano no Rio Grande do Sul,
especificamente em prefeituras, universidades e empresas privadas, e que
desenvolvem
tanto planos quanto projetos urbanos. Ainda que o planejamento urbano seja
realizado
por profissionais de diversas áreas, optou-se por trabalhar apenas com AU pela
maior
viabilidade de acesso a esses profissionais. Ainda, via de regra, o AU é o
profissional mais identificado com a atividade de planejamento urbano, pois é o
único habilitado para assinar a responsabilidade técnica de um plano diretor no
Brasil.

No total, foram distribuídos 100 questionários via e-mail, obtendo uma taxa de
retorno de 42\% (42 de 100). O questionário continha 19 perguntas, das quais as
três
primeiras diziam respeito às características dos respondentes e outras duas
questões
abordavam o uso dos SIG. As demais perguntas ou afirmações retiradas da revisão
da
literatura tratavam da participação do público no planejamento urbano, das novas
tecnologias e da PPSIG. A maior parte dos respondentes era do sexo feminino
(64,3\%,
ou 27 de 42), possuía entre 26 e 40 anos (59,5\%, ou 25 de 42) e tinha
pós-graduação
(69,1\%, ou 29 de 42). Adicionalmente, a maioria respondeu ter algum tipo de
capacitação em SIG (78,6\%, ou 33 de 42) e utilizá-lo em suas atividades
profissionais (62\%, ou 26 de 42).

Ademais, a fim de interagir com os participantes e aprofundar a investigação,
foram
realizadas entrevistas com sete AU: dois servidores públicos, dois profissionais
liberais, dois docentes e um estudante de pós-graduação.

\section{ResultadosAceitação da participação do público e da utilização de novas
tecnologias- Os habitantes são peças fundamentais e o modus
operandi não responde às demandas- Os planejadores tendem a desconsiderar os
dados produzidos pelos
cidadãos, e estes, a partir das transformações socioculturais e
tecnológicas, estão capacitados- Faz-se obrigatório adicionar novas ferramentas
e não se dispõe da
capacitação técnica necessáriaAvaliação do método PPSIG- Fatores de qualidade
individualmente significativos podem ser
facilmente analisados, e as informações coletadas com a PPSIG podem
levar a soluções diferentes- As informações coletadas com a PPSIG podem
dificultar ainda mais a
tarefa do planejamento urbano, e as informações coletadas são difíceis
de interpretar- Mapa apresentado no questionário- Como você avalia a
legibilidade e a utilidade deste mapa}

A expressiva maioria dos AU (88,1\%, ou 37 de 42) concorda totalmente (40,5\%,
ou
17 de 42) ou concorda (47,6\%, ou 20 de 42) com a afirmação de que “Os
habitantes são peças fundamentais no planejamento urbano, pois conhecem a
realidade e os problemas locais melhor do que ninguém” (Tabela 1). Logo, pode-se
afirmar que,
segundo os respondentes, os habitantes devem fazer parte do planejamento urbano,
pois o seu conhecimento sobre a cidade é fundamental. Contudo, nos comentários,
parte dos AU discordou da expressão “melhor do que ninguém”. Por exemplo:
“Conhecem de forma cotidiana e extensa, mas não ‘melhor do que
ninguém’. Esse exagero coloca todo e qualquer saber diverso do local em
xeque, o que não é apropriado” (AU 52).

Fonte: Elaborado pelos autores. Nota: AU = arquitetos e urbanistas;
CT = concordo totalmente; C = concordo; NCND = nem concordo nem
discordo; D = discordo; DT = discordo totalmente.

Destaca-se também a recorrência de “poréns” nos comentários, o que, por vezes,
transmite certa conotação de hierarquia: “Eu diria que eles conhecem
muito bem uma faceta dos problemas, mas, sem um embasamento técnico, eles
podem ter uma visão distorcida e parcial dos problemas” (AU
29).

Sem embargo, prevalece a visão de que a opinião do público deve considerar uma
camada de informação tão importante e necessária quanto todas as outras.
Conforme comentários: “São peças importantes não porque sabem mais do
que ninguém, mas porque são decisores tão importantes quanto qualquer
outro” (AU 35). O entrevistado D sintetiza a questão da seguinte
forma:

Teríamos que discutir esta questão da validade da opinião das pessoas, porque
muitas pessoas não acreditam que seja válido o olhar do cidadão
[...] e o olhar do cidadão é verdadeiro porque é o
olhar do cidadão.

Quanto à afirmação “Em muitos aspectos o modus operandi do planejamento
urbano não responde às demandas dos cidadãos e aos desafios da complexidade
urbana”, a expressiva maioria dos respondentes (93,1\%, ou 39 de 42)
concorda totalmente (45,2\%, ou 19 de 42) ou concorda (47,6\%, ou 20 de 42)
(Tabela 1). Alguns respondentes consideram
que isso ocorre “justamente por faltar a visão do cidadão sobre o
tema” (AU 41). Para muitos, entretanto, isso ocorre porque a
vontade política, em geral, prevalece sobre as indicações técnicas:

É preciso discernir o modo como se opera o planejamento urbano de fato do
modo como deveria ser ou se espera que seja. Pois as propostas e diretrizes,
enfim, todo o produto do trabalho dos planejadores é descartado para atender
a desejos políticos (AU 29).

Logo, as barreiras institucionais e políticas ainda são grandes entraves para a
prática do planejamento urbano participativo, conforme aponta Brown (2012).

Com relação à afirmação de que “Os planejadores tendem a desconsiderar os
dados produzidos pelos cidadãos em suas atividades cotidianas”,
embora a maioria dos AU (57,1\%, ou 24 de 42) concorde totalmente (21,4\%, ou 9
de
42) ou concorde (35,7\%, ou 15 de 42), o número daqueles que nem concordam nem
discordam também é relevante (38,1\%, ou 16 de 42) (Tabela 2).

Nota: Elaborado pelos autores. Legenda: AU = arquitetos e urbanistas;
CT = concordo totalmente; C = concordo; NCND = nem concordo nem
discordo; D = discordo; DT = discordo totalmente.

Nos comentários, os AU ressaltaram a dificuldade de acesso a esses dados. Outros
defenderam: “Mesmo que desconexos de instrumentos e da lógica dos SIG,
planejadores estão se utilizando de informações disponíveis pelos usuários
na internet, como fotos, vídeos etc.” (AU 26). Mas também houve uma
crítica mais direta:

Diferente de muitas áreas que já se utilizam dos dados gerados pelos cidadãos
de forma voluntária ou não, o planejamento está longe disso
[...] não temos o costume e a técnica para buscar estes
dados e torná-lo parte do processo (AU 41).

Observa-se, portanto, que não há consenso entre os AU respondentes, confirmando
que ainda não se tem um entendimento de como o planejamento urbano acompanha as
mudanças tecnológicas, conforme apontam Pereira
et al. (2013), e evidenciando a necessidade de renovação das práticas
de coleta e de interpretação de dados, corroborando a literatura (Pfeffer et
al., 2013; Horelli et al., 2013).

Quanto à afirmação de que “Os cidadãos, a partir das transformações
socioculturais e tecnológicas, estão capacitados a atuar e influenciar o
modo de pensar e agir sobre as cidades”, a maioria dos AU (73,8\%,
ou 31 de 42) concorda totalmente (19\%, ou 8 de 42) ou concorda (54,8\%, ou 23
de
42) (Tabela 2). No entanto, por um lado,
alguns fizeram ressalvas:

Não penso que sejam as ‘transformações socioculturais e tecnológicas’ que
tenham mudado as coisas [...] as tecnologias são mero instrumento.
Participei de momentos muito intensos de planejamento urbano participativo
na década de 80 (AU 24); e

Uma pessoa ter acesso a smartphones, tablets etc., não significa que esteja
apto [...]. Porém, utilizando-se de uma interface de fácil
entendimento e manuseio, facilitaria. Ainda, acho que a inovação tecnológica
é uma ferramenta muito interessante [...], mas acredito que
abrange apenas uma parcela da população (AU 37).

Por outro lado, outros questionados concordam que a ubiquidade das TIC está
produzindo transformações significativas:

As novas ferramentas e, principalmente, sua acessibilidade crescente permitem
que a interação dos cidadãos com o planejamento seja mais direta. Informa-se
mais e assim tem mais capacidade de opinar ou tomar decisão (AU 41);
e

Eu acredito nesta ideia de que, do ponto de vista contemporâneo, as pessoas
podem ser consideradas sensores e nós teríamos essa ferramenta de mapeamento
capaz de capturar isso de um modo mais rápido (Entrevistado
D).

Pode-se discutir mais a fundo se os avanços tecnológicos estão empoderando ou
não
os cidadãos, mas, indiscutivelmente, estão produzindo mudanças nas relações do
cidadão com o espaço urbano e com o governo. Assim, reforça-se o entendimento de
que os avanços tecnológicos demandam mudanças na prática do planejamento urbano.
O entrevistado I resume a questão:

Essa coisa de o usuário poder decidir coisas pela tecnologia vai até um certo
ponto, mas ele não consegue decidir coisas mais estratégicas, então o
planejamento não vai deixar de existir, mas ele tem que mudar, se
reformular, ser mais rápido nas respostas.

No mesmo viés, a expressiva maioria dos AU (92,9\%, ou 39 de 42) concorda
totalmente (52,4\%, ou 22 de 42) ou concorda (40,5\%, ou 17 de 42) que
“Atualmente se faz obrigatório adicionar novas ferramentas no
planejamento urbano, as quais podem exibir e gerenciar novos fluxos de
informações” (Tabela 3). Os
comentários confirmam que os AU desejam utilizar novas tecnologias: “Não
se pode planejar como se fazia anos atrás. Ter acesso aos dados disponíveis
ou às fontes é fundamental para repostas mais consistentes” (AU
41).

Fonte: Elaborado pelos autores. Nota: AU = arquitetos e urbanistas;
TIC = Tecnologias da Informação e Comunicação; SIG = Sistemas de
Informação Geográfica; CT = concordo totalmente; C = concordo; NCND
= nem concordo nem discordo; D = discordo; DT = discordo
totalmente.

Quanto à afirmação de que “Ainda não se dispõe da capacitação técnica
necessária para utilização das TIC e dos SIG no planejamento
urbano”, embora a maioria dos respondentes (54,7\%, ou 23 de 42)
concorde totalmente (9,5\%, ou 4 de 42) ou concorde (45,2\%, ou 19 de 42), a
porcentagem daqueles que nem concorda nem discorda não é desprezível (23,8\%, ou
10 de 42) (Tabela 3). Nos comentários,
evidencia-se que, alguns dos que nem concordam nem discordam, o fazem por não
terem conhecimento a respeito: “Não tenho conhecimento que me permita
responder isso. Talvez os técnicos tenham a capacitação, mas não tenham
infraestrutura e suporte tecnológico adequado nas instituições de
atuação” (AU 12).

Recentemente, o Censo dos Arquitetos e Urbanistas do Brasil, realizado pelo
Conselho de Arquitetura e Urbanismo (CAU,
2014), revelou que 28,04\% dos profissionais possuem conhecimento bom
de softwares de geoprocessamento, 33,81\%, ruim, e 38,15\%
desconhecem. Esse fato reforça o entendimento de que parte considerável dos AU
ainda não dispõe de capacitação técnica adequada para utilização dos SIG.
Aqueles respondentes que concordam com essa visão comentaram:
“Realmente, na minha opinião, falta capacitação técnica, inclusive
dentro das prefeituras. Poucos técnicos se interessam e se capacitam nesta
área” (AU 37); e “Está começando, mas não quer dizer que
todas as pessoas que trabalhem com planejamento urbano vejam esse
potencial” (Entrevistado J). Logo, a necessidade da capacitação
técnica em SIG existe.

Nas entrevistas, foi perguntado se o AU está preparado para lidar com a opinião
da população e qual a disposição e a capacidade de usarem essa camada de
informação. Percebe-se que alguns AU ainda veem o público como tendo um déficit
de conhecimento, conforme discutido por Corburn
(2003). Por exemplo: “A população não consegue perceber toda
a dimensão do urbano que é extremamente complexa. Nós, com conhecimento,
temos dificuldades, imagina a população” (Entrevistado M). De outra
parte, na opinião do entrevistado O: “Tem a dimensão política
[...], tem também o ranço de o arquiteto ser a única voz do processo,
determinadas escolas formam muito nessa coisa autoral e transportam isso num
discurso de cidade”.

Portanto, há divergências nas opiniões. Mas deve-se ter em mente que se trata de
um processo de mudança. Antigamente era feito de um jeito e atualmente está se
mostrando a necessidade de uma outra maneira. A princípio, com o passar do
tempo, a tendência é que essa questão seja suplantada, tendo em vista a
crescente importância atribuída ao conhecimento local, que pressiona os
profisisonais a mudar sua postura. Corroborando, o entrevistado O acredita que o
AU tem de adotar uma postura diferente – de ouvinte – e trabalhar mais em prol
dos processos participativos. Isso porque, segundo ele, o planejamento
participativo, tal qual praticado atualmente, não fomenta o debate de
opiniões.

Por fim, quanto ao papel do planejador urbano nos processos participativos, os
entrevistados têm um entendimento semelhante àquele encontrado na literatura
acerca da mediação dos diferentes interesses envolvidos no planejamento urbano,
mas sem abdicar do seu conhecimento (Souza
\& Rodrigues, 2004; Staffans et
al., 2010; Mäntysalo et al.,
2011). De acordo com o entrevistado D:

Eu vejo o serviço dos planejadores como um direito das pessoas
[...] eu como morador da cidade quero dizer as coisas
que eu penso, quero que alguém me preste um serviço [...]
uma equipe de profissionais que faça a análise dos resultados, que
julgue.

Os entrevistados também salientaram a necessidade de, nesse processo de mediação
dos diferentes interesses, buscar-se a “consertação”, ou seja, um pacto ou
acordo entre as partes. Essa visão corrobora o pensamento do planejamento
agonístico (Mäntysalo et al., 2011), o
qual argumenta que o consenso, embora uma condição legítima, não deve ser
mandatório, pois as diferentes racionalidades podem e devem coexistir. Assim,
deve-se aprender a gerenciar o conflito intrínseco das diferentes racionalidades
envolvidas no processo de planejamento urbano e buscar a pactuação. Segundo o
entrevistado O, sendo a cidade uma manifestação física, se ela for produzida por
meio de conflitos, vai expressar isto. Porém, a cidade também é um local de
encontro, e, se não houver um projeto comum, abolir-se-á a cidade. Na sua visão,
deve-se, no momento propositivo, interpretar esses conflitos e achar um ponto
comum.

Nas questões que abordaram especificamente o método PPSIG, foram considerados
apenas os 19 AU (45,2\%, ou 19 de 42) que afirmaram ter conhecimento prévio do
método. A totalidade desses respondentes concorda totalmente (52,6\%, ou 10 de
19) ou concorda (47,4\%, ou 9 de 19) com a afirmação de que “A PPSIG
possibilita a inclusão do conhecimento local de forma sistemática em um
banco de dados espacial que pode alimentar um sistema de suporte à
decisão”.

Nos comentários, um respondente destacou a necessidade de dados coletados serem
transformados em informação útil, ou seja, que o dado seja analisado de forma
consistente para ser utilizado no planejamento urbano. Em suas palavras:

O conhecimento local é fundamental, mas, se ele não estiver adequadamente
estruturado (se o dado não for convertido em informação útil), ele pode
acabar não contribuindo para um sistema de suporte à decisão eficiente
(AU 55).

Similarmente, outros respondentes chamaram a atenção para a necessidade de
conhecimento técnico adequado para analisar os dados coletados de forma
consistente: “Isso depende muito do modo como esse tal dado qualitativo
é inserido, ou pode-se incorrer facilmente em falsas correlações”
(AU 29).

Quanto à afirmação de que “Com o método PPSIG, fatores de qualidade
individualmente significativos (por exemplo, níveis de satisfação com os
espaços urbanos) podem ser facilmente analisados em relação a outras camadas
de informação (por exemplo, dados censitários)”, a expressiva
maioria dos respondentes (89,4\%, ou 17 de 19) concorda totalmente (36,8\%, ou 7
de 19) ou concorda (52,6\%, ou 10 de 19) (Tabela
4). O resultado confirma que a PPSIG possibilita a integração de
informações qualitativas e quantitativas, de acordo com Kahila \& Kyttä (2009).
Nesse sentido, um respondente
afirmou que o cruzamento de informações só tem a agregar valor ao resultado
final, já que haverá um cruzamento de dados que pode ajudar a se obter
interpretações ou proposições projetais mais assertivas (AU 54).

Fonte: Elaborado pelos autores. Nota: AU = arquitetos e urbanistas;
PPSIG = Participação Pública com Sistema de Informação Geográfica;
CT = concordo totalmente; C = concordo; NCND = nem concordo nem
discordo; D = discordo; DT = discordo totalmente.

A expressiva maioria dos respondentes (84,2\%, ou 16 de 19) também concorda
totalmente (36,8\%, ou 7 de 19) ou concorda (47,4\%, ou 9 de 19) com a afirmação
de que “As informações coletadas com a PPSIG podem levar a soluções
diferentes das que teriam sido alcançadas utilizando-se apenas fontes
oficias de dados e o conhecimento especialista” (Tabela 4).

Nos comentários, entretanto, alguns AU expressaram ressalvas: “Não
afirmaria que existiria uma dualidade entre a informação popular e a
especializada” (AU 26); e “Sim, mas creio que seja
importante saber qual o público que está realmente participando”
(AU 46). De qualquer forma, a literatura (Kahila \& Kyttä, 2009; Staffans
et al., 2010) já indica que os planejadores, inevitavelmente,
precisarão desenvolver um entendimento mais profundo do conhecimento escondido
nas experiências cotidianas dos indivíduos, bem como habilidades para lidar com
ele. Isso implica na necessidade de reconsiderar suas formas de trabalho e os
métodos por meio dos quais as informações de planejamento são criadas,
distribuídas, processadas e usadas (Staffans et
al., 2010).

Quanto à afirmação de que “As informações coletadas com a PPSIG podem
dificultar ainda mais a tarefa do planejamento urbano, pois acrescenta uma
camada extra de informação: a opinião do público”, a maioria dos
respondentes (73,7\%, ou 14 de 19) discorda (31,6\%, ou 6 de 19) ou discorda
totalmente (42,1\%, ou 8 de 19) (Tabela
5).

Fonte: Elaborado pelos autores. Nota: AU = arquitetos e urbanistas;
PPSIG = Participação Pública com Sistema de Informação Geográfica;
CT = concordo totalmente; C = concordo; NCND = nem concordo nem
discordo; D = discordo; DT = discordo totalmente.

Nota-se, pelos comentários, que, embora os AU reconheçam que adicionar a opinião
do público traz ainda mais complexidade à prática do planejamento urbano, tal
informação é bem-vinda, pois o qualifica e tende a tornar os resultados mais
efetivos. Para o entrevistado N, é cada vez mais difícil lidar com tanta
informação, mas é melhor ter um excesso de informação e poder fazer o filtro
técnico. Segundo o entrevistado O, caso a opinião do público não seja
considerada, posteriormente as soluções podem vir a ser questionadas, ou seja,
perguntar antes significa antecipar-se aos problemas. Ainda, mesmo que a
quantidade de informação que se tem antes, durante e inclusive ao longo da
construção da cidade seja muito grande, o planejador tem que dar conta de todo
esse montante, isto é, ter capacidade de filtrar o que interessa, abstrair o que
não interessa e aplicar o que for possível. Nesse sentido, a utilização das
informações coletadas com a PPSIG também se traduziria em ganhos de eficiência
na gestão urbana por meio de uma maior aceitação social das propostas, em
conformidade com Rantanen \& Kahila
(2009).

Na mesma direção, a maioria dos AU discorda totalmente (26,3\%, ou 5 de 19) ou
discorda (36,8\%, ou 7 de 19) da afirmação de que “As informações
coletadas com a PPSIG são difíceis de interpretar, pois são vagas”
(Tabela 5). Diversos comentários dão
conta de que tudo depende de como a ferramenta é elaborada, por exemplo:

Depende de como será a plataforma. Ela pode apontar questões pontuais para
votação ou pode ser extremamente aberta para comentários (dessa forma, será
necessário um agrupamento dos comentários por tema) (AU
46).

Ainda, destacam-se os seguintes comentários: “De fato são difíceis de
interpretar, mas não porque ‘são vagas’, talvez sejam bem
precisas, mas não conseguem incluir a riqueza das discussões presenciais nem
a profundidade das análises espaciais avançadas” (AU 24);
“Se a população não estiver capacitada e se o método utilizado não
for muito objetivo, podem surgir informações muito vagas. O mesmo ocorre com
informações coletadas através de métodos tradicionais
(presenciais)” (AU 43); e “As informações contêm, além da
resposta do usuário, a variável espacializada, que é fundamental no processo
de planejamento urbano” (AU 26). Enquanto o primeiro relaciona a
qualidade da opinião do público com os métodos presenciais, o segundo acredita
que, se o público não estiver capacitado para participar, qualquer método pode
ser falho. A primeira visão valida que a utilização de várias técnicas
participativas é o ideal a ser buscado, já que a participação on-line não
substitui a presencial, mas a complementa (Rojas \& Puig‐i‐Abril, 2009). A
segunda ilustra que, embora sejam
minoria, alguns profissionais ainda veem o público como tendo um déficit de
conhecimento, conforme discutido por Corburn
(2003). Por fim, a última manifestação valoriza um importante
diferencial do método PPSIG, apontado por Poplin (2012): a conveniência de se
ter os dados coletados de forma
automatizada e georreferenciados, o que, diferentemente do que ocorre nos
encontros presenciais, agrega uma valiosa informação à opinião do público – a
sua localização –, facilitando consideravelmente a sua interpretação, análise e,
consequentemente, incorporação no planejamento urbano.

Na sequência, há um questionamento, da parte dos respondentes, sobre a
legibilidade e a utilidade do mapa para o planejamento urbano (Figura 1). Ele
foi produzido a partir de
dados coletados em um experimento PPSIG realizado em Jaguarão, no Estado do Rio
Grande do Sul (Bugs, 2014). Ainda que
parte dos respondentes tenha considerado a legibilidade do mapa satisfatória
(47,4\%, ou 9 de 19), um número não desprezível a considerou nem satisfatória
nem
insatisfatória (21,1\%, ou 4 de 19) e mesmo insatisfatória (31,6\%, ou 6 de 19)
(Tabela 6). Foram feitas algumas
críticas, principalmente, referente ao gradiente utilizado para representar as
faixas de renda e à classificação por cores dos pontos marcados pelos
participantes (branco: sem classificação; coloridos: classificação segundo os
comentários dos respondentes).

Fonte: Elaborado pelos autores.

Assim, observa-se que, embora um exercício de sobreposição de informações, que
objetiva mostrar relações entre varáveis e diferentes interpretações da questão
em pauta, seja um instrumento fundamental de análise para o planejamento urbano,
ele não é tão fácil de ser produzido, em sintonia com Kahila \& Kyttä (2009). No
caso, foram sugeridas
algumas alterações relativamente simples de se executar: atenção ao gradiente de
cores e/ou transparência na sobreposição com imagens; e informação correta da
fonte dos dados. Não obstante, entende-se que ainda se tem um longo caminho no
tocante à visualização e interpretação da informação geográfica
(geovisualização) para o planejamento urbano.

Nas entrevistas, questionou-se como o dado coletado com a PPSIG pode ser
utilizado no planejamento urbano, haja vista que essa é uma questão importante,
mas, por vezes, negligenciada, pois não raramente se coletam dados que não são
utilizados depois (Rantanen \& Kahila,
2009). Na opinião dos entrevistados, a questão passa, principalmente,
pela correta incorporação da tecnologia SIG no planejamento urbano e pela
capacitação técnica dos planejadores para utilizá-la.

Na opinião de um AU que trabalha no poder público, a análise dos dados coletados
deveria ser feita por terceiros: “A informação tem que chegar
estruturada, em um nível pronto para a apreensão do técnico, senão o técnico
não vai parar e ficar olhando resultado por resultado, porque o cliente dele
não é um, mas são milhares” (Entrevistado J). Essa visão se
relaciona com o fato de que lidar com mais uma camada de informação, a qual,
conforme destacado, não é simples de ser representada e analisada, e agrega
dificuldades extras (por exemplo, disponibilidade de tempo) à complexa prática
do planejamento urbano.

Sendo assim, observa-se que, apesar dos avanços ferramentais e tecnológicos,
ainda há uma certa dificuldade para se realizar análises estatísticas e
trabalhar com SIG, o que se configura na principal barreira à incorporação dos
dados coletados com a ferramenta PPSIG no planejamento urbano. Similarmente,
Göçmen \& Ventura (2010) afirmam
que nos Estados Unidos, o potencial dos SIG como ferramenta de planejamento não
está sendo totalmente explorado. A capacitação dos técnicos, especificamente em
aplicações de SIG no planejamento urbano, é a principal medida que deve ser
tomada para alterar essa realidade, segundo os autores.

Também questionou-se os entrevistados sobre “Como você vê as seguintes
possibilidade de aplicação da PPSIG: (i) para coletar a percepção e/ou
opiniões da população antes de um diagnóstico e/ou de uma etapa de
desenvolvimento de projeto, e/ou (ii) como um sistema de monitoramento
permanente?”. No primeiro caso, a percepção da população sobre
determinado tema seria coletada antes de uma etapa inicial de diagnóstico ou de
projeto e, por conseguinte, incorporada na fase propositiva. O segundo caso
seria a utilização constante da ferramenta para coletar a percepção da
população.

Em geral, para os entrevistados, ambas possibilidades são positivas e
desejáveis.
Todavia, alguns destacaram que a primeira é mais difícil, porque consultar a
população antes de realizar um projeto ou proposta ainda não é uma prática
comum. Assim, para alguns, a aplicação como uma ferramenta de monitoramento é
mais factível. Nas palavras do entrevistado J:

É o requinte da participação o governo perguntar antes de fazer alguma coisa
o que se quer. Então, ter esta ferramenta para monitoramento do que está
sendo feito e colocar isso para o conhecimento do público já é muito válido.

Já o entrevistado O destaca a importância da aplicação como uma ferramenta de
monitoramento, pois não se dispõe de um instrumento de coleta de informações em
longo prazo. Logo, na sua visão, a capacidade de armazenar informação de forma
sistemática por um período maior é um potencial da PPSIG. No entanto, o
entrevistado K levanta a questão de que uma ferramenta de coleta permanente da
opinião da população pode tornar-se um repositório de queixas e reclamações, ao
passo que para o entrevistado O: “Na cidade tem que ter alguém que vai
pegar aquele monte de queixas e vai fazer alguma coisa
propositiva”.

Há de se considerar ainda o interesse do público em utilizar a ferramenta PSSIG
em cada um desses momentos. Para o entrevistado O, as pessoas em geral,
mobilizam-se para enfrentar um problema, e, por isso, uma aplicação pontual da
ferramenta atrairia maior interesse do público, uma vez que “ninguém
liga para dizer que boa está a sua administração; ligam para dizer que ali
tem um buraco”. Então, para ele, a aplicação da PPISG para
enfrentar um problema específico seria o mais adequado.

Enfim, uma solução plausível foi apontada pelo entrevistado D, no sentido de
utilizar a PPSIG como um canal permanente de coleta da percepção da população,
mas que mudaria o tema em questão ao longo do tempo:

Acho que o fundamental é manter a ferramenta de mapeamento pela internet mais
ou menos permanente. Por exemplo, nós vamos fazer o plano diretor, esta
ferramenta está; se nós vamos fazer uma melhoria na praça, esta ferramenta
está [...]. Como sempre há alguma coisa, esta ferramenta
estaria aplicada a cada momento.

Portanto, conclui-se que aplicar a PPSIG de forma contínua no planejamento
urbano, porém sempre voltada a questões objetivas e pontuais, ou seja, a
projetos urbanos específicos, e não a problemas abrangentes, parece ser mais
apropriado, pois, desse modo:

O público se mobilizaria mais;

As reclamações particularizadas seriam menos prováveis;

O projeto proposto seria mais facilmente aceito;

Seria possível comparar a opinião do público ao longo do tempo;

O planejador teria o conhecimento local sempre à sua disposição.

\section{Conclusões}

Os resultados deste estudo indicam que, apesar da complexidade de se lidar com a
opinião do público, o conhecimento local é considerado fundamental e, portanto,
caracteriza-se como uma camada de informação indispensável para tornar o
planejamento urbano mais efetivo. Contudo, por muito tempo o conhecimento
especialista reinou de forma absoluta no planejamento urbano. Para diminuir a
divisão entre expertos e leigos, e difundir a noção de que melhores soluções e
ideias surgem a partir do envolvimento de pessoas leigas e seu conhecimento
tácito
do que aquelas desenvolvidas apenas por especialistas (Pfeffer et al., 2013),
assume-se, em conformidade com o
planejamento agonístico (Mäntysalo et al.,
2011), que a racionalidade surge da interação entre uma série de atores,
cada um defendendo os seus objetivos e interesses.

A não aceitação e, por conseguinte, a não aplicação da opinião do público nas
propostas são apontadas como sendo um dos motivos pelos quais o planejamento
urbano
não responde às demandas dos cidadãos e aos desafios da complexa realidade
urbana.
Nesse sentido, a efetiva incorporação do conhecimento local no planejamento
passa,
principalmente, por questões estruturais e ideológicas. Para que o conhecimento
local seja incorporado adequadamente no planejamento urbano, as estruturas de
poder
existentes devem aceitar que segmentos leigos da sociedade têm valiosos
conhecimentos e podem contribuir substancialmente para as decisões de
planejamento e
de gestão urbana. Em geral, há um medo de se perder o controle do processo e é
difícil para o técnico compartilhar a definição das soluções com aqueles que não
investiram tempo e esforço para obter formação técnica, conforme aponta Brown
(2012).

Contudo, percebe-se a existência de um entendimento o qual demonstra que essa
postura
prejudica o sucesso da atividade. Também, que está em curso um processo de
mudança
no modo de se fazer planejamento urbano, reforçada pela noção de o planejador
urbano
ter o dever de ser um mediador dos diferentes interesses envolvidos no
planejamento
urbano, isto é, aquele que busca a concertação entre as partes por meio da
utilização dos seus conhecimentos técnicos, em conjunto com o conhecimento dos
demais (Souza \& Rodrigues, 2004; Staffans et al., 2010). A experiência
profissional dos planejadores não deixa de ser valorizada e suas competências
são
reforçadas com a inclusão das informações e ideias produzidas por vários atores.
Assim, o saber do técnico/especialista não substitui a experiência dos cidadãos,
mas
tal experiência é incluída de forma colaborativa no planejamento (Innes \&
Booher, 2004).

No entanto, a despeito de que os avanços tecnológicos estão produzindo um novo
contexto em que um grupo maior de cidadãos está se habilitando a participar no
planejamento urbano, evidenciou-se, neste estudo, que ainda não há consenso
entre os
profissionais sobre como considerar os dados produzidos pelos cidadãos nas suas
atividades cotidianas, como os oriundos do uso de smartphones ou de
atividades voluntarias e/ou colaborativas de mapeamento. Assim, o aproveitamento
das
informações produzidas pelos cidadãos de forma “espontânea” por meio das
tecnologias
digitais ainda é uma questão em debate, conforme salientado por Pereira et al.
(2013).

No entanto, mesmo considerando as limitações da amostra deste estudo, pode-se
dizer
que os AU que trabalham com planejamento urbano tendem a desejar utilizar novas
tecnologias no planejamento urbano e estão abertos à participação do público.
Nesse
sentido, a ampliação da participação pública e a utilização de novas ferramentas
passam pela adoção de novas metodologias de trabalho, bem como maior capacitação
técnica para se lidar tanto com a opinião do público quanto com os novos fluxos
de
informação da era digital.

Com relação à avaliação do método PPSIG, os resultados indicam que é consenso
entre
os especialistas participantes da amostra que tal método possibilita acessar e
incorporar o conhecimento local no planejamento urbano de forma sistemática e
que
esse conhecimento, por ser uma fonte única de informações atualizadas, ajuda a
melhorar a qualidade do conteúdo dos planos e/ou projetos urbanos. Além disso,
os AU
concordam que a percepção da população deveria formar uma camada adicional de
informações e ser analisada em conjunto com as demais camadas necessárias ao
planejamento urbano.

Porém, para que isso ocorra, há a necessidade de se ampliar o conhecimento
técnico
para que os dados coletados com a PPSIG possam ser analisados e, por
conseguinte,
incorporados no planejamento urbano de forma consistente. Assim, a relativa
falta de
conhecimento sobre os SIG configura-se em barreira à incorporação dos dados
coletados com a PPSIG no planejamento urbano, o que indica a necessidade de
capacitação técnica.

Também foi citado que a utilização de metodologias como a PPSIG se traduziria em
ganhos de eficiência por meio de uma maior aceitação social das propostas. A
PPSIG
facilita que a opinião do público seja considerada na construção das propostas,
uma
vez que contém a variável espacializada. Mas, para que a interpretação das
informações coletadas com a PPSIG seja satisfatória, a elaboração da ferramenta
e
das perguntas que preveem a marcação de lugares no mapa on-line interativo deve
ser
definida conforme os objetivos da pesquisa a fim de se coletar somente
informações
claras e úteis para o planejamento.

Logo, mesmo que ainda existam barreiras para se trabalhar em um plano mais
elevado de
colaboração com a população, demonstrou-se, conforme as opiniões dos
especialistas
consultados neste estudo, que novas metodologias participativas, que tirem
partido
das TIC e dos SIG, tal como a PPSIG, podem auxiliar significativamente na
difícil
tarefa de acessar e incorporar o conhecimento local no planejamento urbano. O
uso
permanente da ferramenta PPSIG para coletar a percepção da população, sempre que
aplicado a questões objetivas e pontuais, é apontado pelos AU como sendo
desejável e
positivo.

Ainda, conforme a avaliação dos especialistas, conclui-se que o planejamento
urbano
participativo por meio da utilização de novas tecnologias, tal como a PPSIG,
poderia
permear todo o processo de planejamento, desde a coleta de informações sobre a
percepção da população acerca de determinado tema, antes mesmo da elaboração das
propostas, até o monitoramento da opinião da população ao longo do processo.

Concluindo, espera-se que os resultados desta pesquisa auxiliem a reforçar a
necessidade de uma postura, por parte daqueles envolvidos com o planejamento
urbano,
que esteja aberta à inclusão do conhecimento local por meio da utilização de
novas
tecnologias, tais como a PPSIG, necessidade esta já identificada em outros
estudos
(Batty et al., 2012; Pereira et al., 2013).

\end{document}
