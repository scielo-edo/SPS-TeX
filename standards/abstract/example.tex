% Generated by jats2tex@0.11.1.0

\begin{abstract}
\newcommand{\abstracttitle}{ABSTRACT}
\section{OBJECTIVEMETHODSRESULTSCONCLUSIONS}
To estimate the prevalence of arterial hypertension and obesity and the
population attributable fraction of hypertension that is due to obesity in
Brazilian adolescents. Data from participants in the Brazilian Study of
Cardiovascular Risks in Adolescents (ERICA), which was the first national
school-based, cross-section study performed in Brazil were evaluated. The sample
was divided into 32 geographical strata and clusters from 32 schools and
classes, with regional and national representation. Obesity was classified using
the body mass index according to age and sex. Arterial hypertension was defined
when the average systolic or diastolic blood pressure was greater than or equal
to the 95th percentile of the reference curve. Prevalences and 95\% confidence
intervals (95\%CI) of arterial hypertension and obesity, both on a national
basis and in the macro-regions of Brazil, were estimated by sex and age group,
as were the fractions of hypertension attributable to obesity in the population.
We evaluated 73,399 students, 55.4\% female, with an average age of 14.7 years
(SD = 1.6). The prevalence of hypertension was 9.6\% (95\%CI 9.0-10.3); with the
lowest being in the North, 8.4\% (95\%CI 7.7-9.2) and Northeast regions, 8.4\%
(95\%CI 7.6-9.2), and the highest being in the South, 12.5\% (95\%CI 11.0-14.2).
The prevalence of obesity was 8.4\% (95\%CI 7.9-8.9), which was lower in the
North region and higher in the South region. The prevalences of arterial
hypertension and obesity were higher in males. Obese adolescents presented a
higher prevalence of hypertension, 28.4\% (95\%CI 25.5-31.2), than overweight
adolescents, 15.4\% (95\%CI 17.0-13.8), or eutrophic adolescents, 6.3\% (95\%CI
5.6-7.0). The fraction of hypertension attributable to obesity was 17.8\%. ERICA
was the first nationally representative Brazilian study providing prevalence
estimates of hypertension in adolescents. Regional and sex differences were
observed. The study indicates that the control of obesity would lower the
prevalence of hypertension among Brazilian adolescents by 1/5.
\ifdef{\kwdgroup}{\kwdgroup}{}
\end{abstract}
