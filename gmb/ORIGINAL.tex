\scieloAbstractContainer{Abstract}{The nervous systems of flatworms have diversified extensively as a consequence of the broad range of adaptations
in the group. Here we examined the central nervous system (CNS) of 12 species of polyclad flatworms belonging to
11 different families by morphological and histological studies. These comparisons revealed that the overall organi-
zation and architecture of polyclad central nervous systems can be classified into three categories (I, II, and III)
based on the presence of globuli cell masses -ganglion cells of granular appearance-, the cross-sectional shape of
the main nerve cords, and the tissue type surrounding the nerve cords. In addition, four different cell types were iden-
tified in polyclad brains based on location and size. We also characterize the serotonergic and FMRFamidergic ner-
vous systems in the cotylean Boninia divae by immunocytochemistry. Although both neurotransmitters were broadly
expressed, expression of serotonin was particularly strong in the sucker, whereas FMRFamide was particularly
strong in the pharynx. Finally, we test some of the major hypothesized trends during the evolution of the CNS in the
phylum by a character state reconstruction based on current understanding of the nervous system across different
species of Platyhelminthes and on up-to-date molecular phylogenies.}{Keywords:}{brain, FMRFamide, globuli cell masses, orthogon, serotonin.}

{\noindent\fontsize{9}{10.8}\selectfont{Received: January 13, 2015; Accepted: April 19, 2015.}}

\vspace*{5.5mm}

\begin{multicols}{2}
\scieloSectionContainer{Introduction}
\par{}Current phylogenetic views of Platyhelminthes divide the phylum into Catenulida and Rhabditophora (Laumer and Giribet, 2014; Laumer et al,. 2015; Egger et al.,
2015). Within the diverse Rhabditophora, strong morphological and molecular evidence suggests Macrostomorpha
as the basal clade; whereas Polycladida, often considered a
basal clade (for a review see Baguñà and Riutort, 2004;
Karling, 1967, 1974; Carranza et al,. 1997, Litvaitis and
Rhode, 1999; Laumer and Giribet, 2014), currently groups
together with the Prorhynchida in a more derived position
based on recent phylogenomic and transcriptomic analyses (Laumer et al., 2015; Egger et al., 2015). The main characteristic of the Polycladida is their highly branched intestine
(Hyman, 1951), from which they derive their name. Polyclads have few external traits; however, the presence or absence of clusters of eyespots and either true tentacles or
pseudotentacles, which form by folds of the anterior body
margin, can be used as systematic characters (Newman and
Cannon, 1994). The initial division of the order is based on
the presence or absence of a ventral sucker. This character
divides the polyclads into the two suborders Acotylea
(without sucker) and Cotylea (with sucker) (Lang, 1884).
\par{}Among invertebrates, nervous systems exhibit a wide
variety of organization, which can range from an unorganized diffuse nerve net (cnidarians) to complex systems
with highly specialized sensory receptors (arthropods,
some mollusks) (Ruppert and Barnes, 1994). The bilaterally flattened body of flatworms preserves a common organization of the central nervous system (CNS). The CNS of
flatworms consists of: (i) the orthogon, composed of main
longitudinal nerve cords and transverse commissures that
form a ladder-like network. The main cords have direct
connections to the brain, are generally multifibrillar, are
more strongly developed on the ventral side, and express
serotonin or cholinergic neuropeptides (Reuter and Gustafsson, 1995). The number, position, and arrangement of
cords vary drastically from one taxonomic group to the
next; (ii) an anterior brain consisting of two lobes connected by one or several commissures. The lobes may be either loosely aggregated or encapsulated by a sheath of
extracellular matrix (Reuter and Gustafsson, 1995); and
(iii) the plexus, composed of a network of sub- and infraepidermal, or submuscular nerves that expand throughout
the body of the worm. In addition, in free-living flatworms
the presence of pharyngeal and stomatogastric plexi, consisting of either one or two nerve rings, innervated by suband infra-epithelial nerve nets (Reuter and Gustafsson,
1995; Halton and Gustafsson, 1996), are considered plesiomorphic characters (Ehlers, 1985).
\par{}Reisinger (1925) defined a standard orthogonal pattern for many free-living forms. This pattern consists of
several pairs of ventral, lateral, and dorsal longitudinal
nerve cords that are connected at right angles along their
lengths by transverse commissures, forming a ladder-like
arrangement. Although the orthogon organization and occurrence does not show clear evolutionary trends (Reuter
and Gustafsson, 1995), a common origin of the CNS in
flatworms has been suggested with convergent evolution of
similar orthogonal organizations in different taxa (Kotikova, 1991). The orthogon varies in position and number of
cords, different points of contact with the brain, and thickness of cords. Reuter et al. (1998) determined that the main
cords derive from strong roots in the brain, are composed of
numerous neurons within wide fiber bundles, and generally
show immunoreactivity for serotonin and catecholamines.
It has been hypothesized that more derived clades often
have fewer but thicker nerve cords than more basal taxa
(Reisinger, 1972; Reuter and Gustafsson, 1995), and that
longitudinal cords may have evolved from the fusion of
plexal fibers of ancestral taxa (Joffe and Reuter, 1993).

\lipsum

\scieloSectionContainer{Materials and Methods}

\scieloSubSectionContainer{Animals}

\par{}Specimens representing 12 species of polyclads were
collected from different sites in the Caribbean. In addition,
a deep-sea specimen of \textit{Anocellidus profundus} Quiroga,
Bolaños and Litvaitis, 2006, from the North Pacific was
also included in the analysis (specimen courtesy of Dr.
Janet Voight, Field Museum of Natural History, Chicago,
Illinois, USA). For details of collection, fixation, and spe-
cies identification see Litvaitis \textit{et al}. (2010).

\scieloSubSectionContainer{Histological studies of polyclad nervous systems}

\par{}A specimen of each species was embedded in paraffin
in its entirety, longitudinally sectioned at 7-10 mm, and
stained with Milligan Trichrome technique which allows
differentiating well among connective tissue, muscles, and
nervous fibers (Presnell and Schreibman, 1997). Additional
cross and sagittal sections were also prepared. De-paraffi-
nized sections were treated with 3\% potassium dichro-
mate-hydrochloric acid solution for 5 min. Following a
distilled water rinse, sections were stained in acid fuchsin
for 8 min. After a second distilled water rinse, they were
placed into 1\% phosphomolybdic acid for 2 min and then
stained with 2\% solution of Orange G for 5 min. After a fi-
nal rinse with distilled water, the sections were treated with
1\% hydrochloric acid solution for 2 min stained in Fast
Green for 8 min, treated with 1\% acetic acid for 3 min and
then rinsed in 95\% alcohol and dehydrated. Finally, the sec-
tions were cleared with Histoclear (National Diagnostics)
and mounted in Permount (Fisher Scientific). Slides were
observed and photographed under an Axiostar Plus (Zeiss,
Thornwood, New York) light microscope. We applied the
morphological criteria of Reuter et al. (1998) to distinguish
main nerve cords from secondary nerve cords for the exam-
ination of the NS. Cross and sagittal sections were used for
interpretation. In addition, whole mounts of all species
were prepared, by dehydrating the specimens in a graded
alcohol series, cleared with Histoclear, and mounted in
Permount.

\lipsum

\scieloImageContainerOneCol{0.5\textwidth}{img1.jpg}{meu titulo}

%\begin{center}
%{
%\vspace{8mm}
%\centerline{
%\includegraphics[width=\maxwidth{0.5\textwidth}]{img1.jpg}
%}
%\vspace{8mm}
%}
%\end{center}

\lipsum

\scieloImageContainerTwoCol{img2.jpg}{\textbf{Figure 1:} A specimen of each species was embedded in paraffin in its entirety, longitudinally sectioned}

\lipsum
\lipsum

\scieloImageContainerTwoCol{img2.jpg}{\textbf{Figure 1:} A specimen of each species was embedded in paraffin in its entirety, longitudinally sectioned}

\lipsum
\lipsum

\begin{scieloReferencesContainer}[References]

\scieloReferencesItem{Åkesson B (1958) A study of the nervous system of the Sipunculoideae, with some remarks on the development of the two
species Phascolion strombi Montagu and Golfingia minuta
Keferstein. Undersokningar over Oresund 38:1-249.}

\scieloReferencesItem{Baguñà J and Riutort M (2004) Molecular phylogeny of the
Platyhelminthes. Can J Zool 82:168-193.}

\scieloReferencesItem{Bailly X, Reichert H and Hartenstein V (2013) The urbilaterian
brain revisited: Novel insights into old questions from new
flatworm clades. Dev Genes Evol 223:149-157.}

\scieloReferencesItem{Biserova NM, Dudicheva VA, Terenina NB, Reuter M, Halton
DW, Maule AG and Gustafsson MK (2000) The nervous
system of Amphilina foliacea (Platyhelminthes,
Amphilinidea), an immunocytochemical, ultrastructural and
spectrofluorometrical study. Parasitology 121:441-453.}

\scieloReferencesItem{Bock S (1923) Boninia, a new polyclad genus from the Pacific.
Nov Act Reg Soc Uppsala Ser 46:1-32.}

\scieloReferencesItem{Böckerman I, Reuter M and Timoshkin O (1994) Ultrastructural
study of the central nervous system of endemic
Geocentrophora (Prorhynchida, Platyhelminthes) from
Lake Baikal. Acta Zool 75:47-55.}

\scieloReferencesItem{Bullock TH and Horridge GA (1965) Structure and Function in
the Nervous Systems of Invertebrates. Vol. 2. W.H. Freeman and Co, San Francisco, 1719 pp.}

\scieloReferencesItem{Carranza S, Baguñà J and Riutort M (1997) Are the Platyhelminthes a monophyletic primitive group? An assessment using 18S rDNA sequences. Mol Biol Evol 14:485-497.}

\scieloReferencesItem{Cebrià F (2008) Organization of the nervous system in the model
planarian Schmidtea mediterranea: An immunocytochemical study. Neurosci Res 61:375-384.}

\scieloReferencesItem{Chien P and Koopowitz H (1972) The ultrastructure of neuromuscular systems in Notoplana acticola, a free-living polyclad flatworm. Z Zellforsch Mikroskop Anat 133:277-288.}


\scieloReferencesItem{Chien PK and Koopowitz H (1977) Ultrastructure of nerve plexus
in flatworms. III. The infra-epithelial nervous system. Cell
Tissue Res 176:335-347.}

\scieloReferencesItem{Day TA, Maule AG, Shaw C and Pax RA (1997) Structure-activity relationships of FMRFamide-related peptides contracting Schistosoma mansoni muscle. Peptides 18:917-921.}

\scieloReferencesItem{Egger B, Gschwentner R and Rieger R (2007) Free-living
flatworms under the knife: Past and present. Dev Genes Evol
217:89-104.}

\scieloReferencesItem{Egger B, Lapraz F, Tomiczek B, Müller S, Dessimoz C, Girstmair
J, Skunca N, Rawlinson KA, Cameron CB, Beli E et al.
(2015) A transcriptomic-phylogenomic analysis of the evolutionary relationships of flatworms. Curr Biol 25:1-7.}

\scieloReferencesItem{Ehlers U (1985) Das Phylogenetische System der Plathelminthes.
Gustav Fischer Verlag, Stuttgart, 317 pp.}

\scieloReferencesItem{Fernandes MC, Alvares EP, Gama P and Silveira M (2003) Serotonin in the nervous system of the head region of the land planarian Bipalium kewense. Tissue Cell 35:479-486.}

\scieloReferencesItem{Forest DL and Lindsay SM (2008) Observations of serotonin and
FMRFamide-like immunoreactivity in palp sensory structures and the anterior nervous system of spionid polychaetes. J Morphol 269:544-551.}

\scieloReferencesItem{Girstmair J, Schnegg R, Telford MJ and Egger B (2014) Cellular
dynamics during regeneration of the flatworm Monocelis sp. (Proseriata, Platyhelminthes). Evo Devo 5:e37.}

\scieloReferencesItem{Golding DM (1992) Polychaeta: Nervous system. In: Harrison
FW and Gardiner SL (eds) Microscopic Anatomy of Invertebrates. Vol. 7. Wiley-Liss, New York, pp 155-179.}

\scieloReferencesItem{Gustafsson MKS, Halton DW, Kreshchenko ND, Movsessian SO,
Raikova OI, Reuter M and Terenina NB (2002) Neuropeptides in flatworms. Peptides 23:2053-2061.}

\scieloReferencesItem{Hadenfeldt D (1929) Das Nervensystem von Stylochoplana
maculata und Notoplana atomata. Z wiss Zool 133:586638.}

\scieloReferencesItem{Halton DW and Gustafsson MKS (1996) Functional morphology
of the platyhelminth nervous system. Parasitology
113:S47-S72.}

\end{scieloReferencesContainer}

\scieloLicenseContainer{License information: This is an open-access article distributed under the terms of the Creative Commons Attribution License, which permits unrestricted use, distribution, and reproduction in any medium, provided the original work is properly cited.}

\end{multicols}
