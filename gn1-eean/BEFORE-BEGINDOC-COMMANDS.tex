\makeatletter
\def\@fnsymbol#1{
     \ensuremath{
         \ifcase#1
\or ~
         \fi
     }
}
\renewcommand{\thefootnote}{\fnsymbol{footnote}}
\makeatother

\newcommand\scieloDefineFootnotes{
%\footnotetext[1]{\scieloArticleUrl}
%\footnotetext[1]{EDVC é Mestre em Urbanisme et Aménagement, e-mail: eugeniadoria@gmail.com}
%\footnotetext[2]{\\\textbf{Endereço para correspondência}\\Vanessa de Paula Tiago.\\Universidade Federal do Triângulo Mineiro. Departamento de Pediatria da Universidade Federal do Triângulo Mineiro. Av. Frei Paulino, no 30, Bairro Abadia\\Uberaba - MG. Brasil. CEP: 38025-180.}
}

\newcommand{\scieloYear}{2015}
\newcommand\scieloArticleType{PESQUISA}
\newcommand\scieloArticleTypeTranslation{RESEARCH}
\newcommand{\scieloArticleRef}{2015;19(3):000-000}
\newcommand{\scieloHeaderLineOne}{Acolhimento com Classificação de Risco: Avaliação de SHE}
\newcommand{\scieloHeaderLineTwo}{Costa MAR, Versa GLGS, Bellucci Júnior JA, Inoue KC, Sales CA, Matsuda LM}
\newcommand{\scieloFooter}{19(3) JUL-SET 2015}
\newcommand\scieloArticleTitle{Acolhimento com Classificação de Risco: Avaliação de Serviços Hospitalares de Emergência}
\newcommand\scieloArticleTitleTranslation{Admittance of Risk-Classified Cases: Assessment of Hospital Emergency Services\\Acogida con Clasificación de Riesgo: Evaluación de Servicios Hospitalarios de Emergencia}
\newcommand{\scieloAuthor}{Maria Antonia Ramos Costa\textsuperscript{1}\\Gelena Lucinéia Gomes da Silva Versa\textsuperscript{2}\\José Aparecido Bellucci\textsuperscript{3}\\Kelly Cristina Inoue\textsuperscript{4}\\Catarina Aparecida Sales\textsuperscript{5}\\Laura Misue Matsuda\textsuperscript{5}}
\newcommand\scieloContributorsFootNotes{1. Universidade Estadual do Paraná-Campus Paranavaí - PR, Brasil.\\2. Hospital Universitário Oeste do Paraná. Cascavel - PR, Brasil.\\3. Universidade Estadual do Norte do Paraná. Bandeirantes - PR, Brasil.\\4. Faculdade Ingá. Maringá – PR, Brasil.\\5. Universidade Estadual de Maringá. Maringá - PR, Brasil.}
\newcommand{\scieloArticleDOI}{10.1590/S1679-87592015086506303}
\newcommand{\scieloReceivedApproved}{Recebido em 01/09/2014.\\Aprovado em 08/06/2015.}
\newcommand{\scieloCorrespondence}{\textbf{Autor correspondente:}\\Maria Antonia Ramos Costa.\\E-mail:enfunespar1982@hotmail.com\\}

\newcommand{\scieloRenderAbstracts}{
\scieloAbstractContainer{RESUMO}{\textbf{Objetivo}: Avaliar o Acolhimento com Classificação de Risco (ACCR) implantado em dois serviços hospitalares de emergência \textbf{Métodos}: Pesquisa exploratório-descritiva, de abordagem quantitativa, realizada entre março e maio de 2013. Participaram 47 profissionais de enfermagem de dois serviços hospitalares de emergência do Paraná que responderam a um questionário. \textbf{Resultados}: O ACCR foi considerado "Precário" nos dois Serviços; e as avaliações mais baixas se referiram à acomodação do acompanhante e discussão sobre o fluxograma. A melhor avaliação se relacionou ao atendimento de casos não graves. \textbf{Conclusão}: A avaliação precária nos dois Serviços, deveu-se principalmente, a não adequação de alguns princípios fundamentais da diretriz ACCR.}{Palavras-chave:}{Acolhimento; Enfermagem; Serviço Hospitalar de Emergência.}

\scieloAbstractContainer{ABSTRACT}{\textbf{Objective}: Current paper evaluates admittance of risk-classified cases in two hospital emergency services. \textbf{Methods}: The
exploratory, descriptive and quantitative research was undertaken between March and May 2013, with 47 nursing professionals at
two hospital emergency units in the state of Paraná, Brazil, who answered the questionnaire. \textbf{Results}: Acceptance of Risk-classified
Cases was reported hazardous at the two units; the lowest rates refer to issues on the place the accompanying person would
stay and to discussions on the flowchart. The best assessment occurred in the attendance of less serious cases. \textbf{Conclusion}: The hazardous assessment at the two health units was mainly due to the non-compliance with certain basic principles of its guidelines of the Acceptance of Risk-classified Cases.}{Keywords:}{User Embracement; Nursing; Emergency Service, Hospital.}
}