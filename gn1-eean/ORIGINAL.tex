\begin{multicols}{2}

\scieloSectionContainer{INTRODUÇÃO}
\par{}Com o lançamento da Política Nacional de Humanização
(PNH), tornou-se necessária a reorganização dos serviços
de saúde de modo a garantir à população, atendimento
resolutivo, humanizado e acolhedor. Para isso, foi proposta a
diretriz Acolhimento com Classificação de Risco (ACCR), que
representa um sistema dinâmico de identificação de pacientes
e ordenação do atendimento, em consonância ao grau de
complexidade e risco potencial de cada caso 1 .
\par{}Reconhece-se que, a priorização de pacientes graves
pode viabilizar o manejo da doença no tempo adequado, com
maiores chances de recuperação de casos agudos atendidos
em Serviço Hospitalar de Emergência (SHE) 2 . Para isso, foram
desenvolvidos sistemas de classificação de risco em diversos
países, dentre os quais se destacam: National Triage Scale
(NTS) da Austrália; Canadian Emergency Department Triage
and Acuity Scale (CTAS) do Canadá; Manchester Triage System

\scieloSectionContainer{RELATO DO CASO}

\par{}Escolar, 8 anos e 2 meses, com diagnóstico pregresso
de leucemia promielocítica aguda com tratamento completo
concluído há 1 ano e sem intercorrências durante este período.
Deu entrada no PSI do HC-UFTM, no dia 12 de abril de
2014, com queixa de sangramento gengival, equimoses em
membros inferiores e plaquetopenia. Em hemograma inicial,
visualizados 70% de blastos, levantada hipótese diagnóstica de
recidiva da doença, sendo então iniciado ATRA 25 mg/m 2 /dia
e suporte hemoterápico. Mielograma evidenciou medula
óssea hiperplásica e homogênea, com 90% de mieloblastos e
imunofenotipagem compatível com leucemia promielocítica
aguda, confirmada com identificação do gene de fusão
PML-RARA

\lipsum

\scieloSectionContainer{MATERIAL AND METHODS}

\scieloSubSectionContainer{SAMPLING AND ANALYSIS}

\par{}Surface water samples were taken using a Van
Dorn bottle and then stored in 500 to 1000 ml glass
vials and fixed with a Lugol solution (THRONDSEN,
1978). Sampling was carried out fortnightly (36.

\lipsum
\scieloSectionContainer{RESULTS}

\scieloSubSectionContainer{SPECIFIC COMPOSITION, RICHNESS AND ABUNDANCE}

\par{}A total of 110 taxa were identified, belonging to
4 Divisions (Figure 2; Table 1): Diatoms (55 taxa;
25 species), Dinoflagellates (51 taxa; 25 species),
Cyanobacteria (2 taxa) and Chlorophythes (2 taxa).
\par{}Richness, Abundance, Diversity Index and Evenness
results are presented as the mean values of each sampling
period, as the non-parametric tests revealed no differences
(p < 0.05) between sampling stations. Table 2 presents
information on these variables in each sampling period.


\lipsum

\scieloImageContainerOneCol{0.5\textwidth}{img1.jpg}{meu titulo}

%\begin{center}
%{
%\vspace{8mm}
%\centerline{
%\includegraphics[width=\maxwidth{0.5\textwidth}]{img1.jpg}
%}
%\vspace{8mm}
%}
%\end{center}

\lipsum

\scieloImageContainerTwoCol{img2.jpg}{\textbf{Figure 1:} A specimen of each species was embedded in paraffin in its entirety, longitudinally sectioned}

\lipsum
\lipsum

\scieloImageContainerTwoCol{img2.jpg}{\textbf{Figure 1:} A specimen of each species was embedded in paraffin in its entirety, longitudinally sectioned}

\lipsum
\lipsum
\lipsum[1]

\begin{scieloReferencesContainer}[REFERÊNCIAS]

\scieloReferencesItem{Ministério da Saúde (BR). Acolhimento e classificação de risco nos
serviços de urgência. Brasília (DF): Ministério da Saúde; 2009. 56p.}

\scieloReferencesItem{Buschhorn HM, Strout TD, Sholl JM, Baumann MR. Emergency Medical
Services Triage using the Emergency Severity Index: is it reliable and
valid? J Emerg Nurs. 2013;39(5):55-63.}

\scieloReferencesItem{Souza CC, Toledo AD, Tadeu LFR, Chianca TCM. Risk classification
in an emergency room: agreement level between a Brazilian
institutional and the Manchester Protocol. Rev Lat Am Enfermagem.
2011;19(1):26-33. Avaiable from: doi.org/10.1590/S0104-
11692011000100005.}

\scieloReferencesItem{Duro CLM, Lima MADS. O papel do enfermeiro nos sistemas de triagem
em Emergências: análise da literatura. Online Braz J Nurs. [on line].
2010;[citado 8 abr. 2014];9(3). Disponível em: http://www.objnursing.
uff.br/index.php/nursing/article/view/j.1676-4285.2010.3132/718.}

\scieloReferencesItem{Ministério da Saúde (BR). Manual instrutivo da Rede de Atenção às
Urgências e Emergências no Sistema Único de Saúde (SUS). Brasília
(DF): Editora do Ministério da Saúde; 2013. 84p.}

\scieloReferencesItem{Franchi L, Baccetti T, De Toffol L, Polimeni A, Cozza P. Phases of the dentition for the assessment of skeletal maturity: a diagnostic performance
study. Am J Orthod Dentofacial Orthop. 2008 Mar;133(3):395-400. PMid: 18331939.}

\scieloReferencesItem{Bullock TH and Horridge GA (1965) Structure
and Function in the Nervous Systems of Invertebrates.
Vol. 2. W.H. Freeman and Co, San
Francisco, 1719 pp.}

\scieloReferencesItem{Carranza S, Baguñà J and Riutort M (1997)
Are the Platyhelminthes a monophyletic primitive
group? An assessment using 18S rDNA
sequences. Mol Biol Evol 14:485-497.}

\scieloReferencesItem{Cebrià F (2008) Organization of the nervous system in the model
planarian Schmidtea mediterranea: An immunocytochemical study. Neurosci Res 61:375-384.}

\scieloReferencesItem{Chien P and Koopowitz H (1972) The ultrastructure of neuromuscular systems in Notoplana acticola, a free-living polyclad flatworm. Z Zellforsch Mikroskop Anat 133:277-288.}


\scieloReferencesItem{Chien PK and Koopowitz H (1977) Ultrastructure of nerve plexus
in flatworms. III. The infra-epithelial nervous system. Cell
Tissue Res 176:335-347.}

\scieloReferencesItem{Day TA, Maule AG, Shaw C and Pax RA (1997) Structure-activity relationships of FMRFamide-related peptides contracting Schistosoma mansoni muscle. Peptides 18:917-921.}

\scieloReferencesItem{Egger B, Gschwentner R and Rieger R (2007) Free-living
flatworms under the knife: Past and present. Dev Genes Evol
217:89-104.}

\scieloReferencesItem{Egger B, Lapraz F, Tomiczek B, Müller S, Dessimoz C, Girstmair
J, Skunca N, Rawlinson KA, Cameron CB, Beli E et al.
(2015) A transcriptomic-phylogenomic analysis of the evolutionary relationships of flatworms. Curr Biol 25:1-7.}

\scieloReferencesItem{Ehlers U (1985) Das Phylogenetische System der Plathelminthes.
Gustav Fischer Verlag, Stuttgart, 317 pp.}

\scieloReferencesItem{Fernandes MC, Alvares EP, Gama P and Silveira M (2003) Serotonin in the nervous system of the head region of the land planarian Bipalium kewense. Tissue Cell 35:479-486.}

\scieloReferencesItem{Forest DL and Lindsay SM (2008) Observations of serotonin and
FMRFamide-like immunoreactivity in palp sensory structures and the anterior nervous system of spionid polychaetes. J Morphol 269:544-551.}

\scieloReferencesItem{Girstmair J, Schnegg R, Telford MJ and Egger B (2014) Cellular
dynamics during regeneration of the flatworm Monocelis sp. (Proseriata, Platyhelminthes). Evo Devo 5:e37.}

\scieloReferencesItem{Golding DM (1992) Polychaeta: Nervous system. In: Harrison
FW and Gardiner SL (eds) Microscopic Anatomy of Invertebrates. Vol. 7. Wiley-Liss, New York, pp 155-179.}

\scieloReferencesItem{Gustafsson MKS, Halton DW, Kreshchenko ND, Movsessian SO,
Raikova OI, Reuter M and Terenina NB (2002) Neuropeptides in flatworms. Peptides 23:2053-2061.}

\scieloReferencesItem{Hadenfeldt D (1929) Das Nervensystem von Stylochoplana
maculata und Notoplana atomata. Z wiss Zool 133:586638.}

\scieloReferencesItem{Halton DW and Gustafsson MKS (1996) Functional morphology
of the platyhelminth nervous system. Parasitology
113:S47-S72.}

\end{scieloReferencesContainer}

\scieloLicenseContainer{License information: This is an open-access article distributed under the terms of the Creative Commons Attribution License, which permits unrestricted use, distribution, and reproduction in any medium, provided the original work is properly cited.}



\end{multicols}

