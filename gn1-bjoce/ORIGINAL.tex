\begin{multicols}{2}

\scieloAbstractContainer{ABSTRACT}{This study analyzed the temporal variability of
phytoplankton assemblages in the surface waters of
Guanabara Bay (RJ, Brazil), at six stations in front
of Icaraí Inlet from April/2011 to April/2012. Our
results highlight the great contribution of diatoms,
dinoflagellates and cyanobacteria, represented by
111 taxa typical of estuarine and coastal areas. The
coexistence of benthic and planktonic species suggests
considerable hydrodinamism in these waters. All
variables were homogeneous (p > 0.05) between the
stations, but differed between sampling periods. On
average, phytoplankton abundance (10 7 cells.L -1 ) was
higher than that of other estuaries and its temporal
behavior was closely correlated (p < 0.01) with
diatoms and cyanobacteria. The richness distribution
pattern (7 to 27 taxa) was closely correlated (p < 0.01)
with dinoflagellates and diatoms. Ninety per cent of
all samples presented a low diversity index (< 2.0
bits.cell -1 ), which indicated the unstable balance
of the system, typical of environments subjected to
eutrophication. The population structure analysis
revealed that 10\% of all taxa were resident, 12%
visitors and 78\% accidental, suggesting the influence
of continental and oceanic water influxes. Between
the "typical" taxa, the most common were the
cyanobacteria of the order Oscillatoriales, the diatoms
Ceratoneis closterium (=Cylindrotheca closterium)
and Leptocylindrus minimus and the dinoflagellate
Prorocentrum triestinum.}{Descriptors:}{Tropical estuary, Temporal variation, Species Diversity, Microphytoplankton.}

\scieloAbstractContainerTranslation{RESUMO}{Este trabalho analisou a variabilidade sazonal
da comunidade microfitoplanctônica em águas
superficiais da Baía da Guanabara (RJ, Brasil) em 6
estações em frente à Enseada de Icaraí, de abril de
2011 a abril de 2012. Os resultados destacaram a alta
representatividade de diatomáceas, dinoflagelados e
cianobactérias, representados por 111 táxons típicos
de ambientes estuarinos/costeiros. A coexistência
de espécies bentônicas e planctônicas indicou o alto
hidrodinamismo local. Houve homogeneidade (p >
0,05) entre as estações de coleta para todas as variáveis,
mas diferenças entre campanhas. A densidade
média (10 7 cel.L -1 ) foi superior à de outros sistemas
estuarinos e seu comportamento temporal esteve
altamente correlacionado (p < 0,01) com diatomáceas
e cianobactérias. Por sua vez o padrão de distribuição
da riqueza (7 a 27 táxons) apresentou alta correlação
positiva (p < 0,01) com dinoflagelados e diatomáceas.
Baixos índices de diversidade (< 2,0 bits.cel -1 ) em
90\% das amostras reafirmaram o equilíbrio instável
do sistema, típico de ambientes sujeitos à eutrofização.
A análise da estrutura das populações estabeleceu
que 10\% dos táxons são residentes, 12\% visitantes
e 78\% acidentais, reafirmando a influência do aporte
continental e/ou águas oceânicas. Entre os táxons
"típicos", destacaram-se cianobactérias da Ordem
Oscillatoriales, diatomáceas Ceratoneis closterium
(=Cylindrotheca closterium) e Leptocylindrus
minimus e o dinoflagelado Prorocentrum triestinum.}{Descritores:}{Estuário tropical, Variação temporal,
Diversidade específica, Microfitoplâncton.}

\scieloSectionContainer{INTRODUCTION}
\par{}The evaluation of the condition of an ecosystem
requires the assessment of its primary productivity and
trophic dynamics in view of the structure and ecological
functions of its communities. Phytoplankton represents
the basis of the main food webs in aquatic ecosystems,
and its taxonomic composition and abundance respond
to environmental disturbances (i.e. physical processes
such as advective currents and turbulence, and chemical
composition - nutrients), and to the interaction between
species (i.e. competition for resources: light and nutrients)

\scieloSectionContainer{RELATO DO CASO}

\par{}Escolar, 8 anos e 2 meses, com diagnóstico pregresso
de leucemia promielocítica aguda com tratamento completo
concluído há 1 ano e sem intercorrências durante este período.
Deu entrada no PSI do HC-UFTM, no dia 12 de abril de
2014, com queixa de sangramento gengival, equimoses em
membros inferiores e plaquetopenia. Em hemograma inicial,
visualizados 70% de blastos, levantada hipótese diagnóstica de
recidiva da doença, sendo então iniciado ATRA 25 mg/m 2 /dia
e suporte hemoterápico. Mielograma evidenciou medula
óssea hiperplásica e homogênea, com 90% de mieloblastos e
imunofenotipagem compatível com leucemia promielocítica
aguda, confirmada com identificação do gene de fusão
PML-RARA

\lipsum

\scieloSectionContainer{MATERIAL AND METHODS}

\scieloSubSectionContainer{SAMPLING AND ANALYSIS}

\par{}Surface water samples were taken using a Van
Dorn bottle and then stored in 500 to 1000 ml glass
vials and fixed with a Lugol solution (THRONDSEN,
1978). Sampling was carried out fortnightly (36.

\lipsum
\scieloSectionContainer{RESULTS}

\scieloSubSectionContainer{SPECIFIC COMPOSITION, RICHNESS AND ABUNDANCE}

\par{}A total of 110 taxa were identified, belonging to
4 Divisions (Figure 2; Table 1): Diatoms (55 taxa;
25 species), Dinoflagellates (51 taxa; 25 species),
Cyanobacteria (2 taxa) and Chlorophythes (2 taxa).
\par{}Richness, Abundance, Diversity Index and Evenness
results are presented as the mean values of each sampling
period, as the non-parametric tests revealed no differences
(p < 0.05) between sampling stations. Table 2 presents
information on these variables in each sampling period.


\lipsum

\scieloImageContainerOneCol{0.5\textwidth}{img1.jpg}{meu titulo}

%\begin{center}
%{
%\vspace{8mm}
%\centerline{
%\includegraphics[width=\maxwidth{0.5\textwidth}]{img1.jpg}
%}
%\vspace{8mm}
%}
%\end{center}

\lipsum

\scieloImageContainerTwoCol{img2.jpg}{\textbf{Figure 1:} A specimen of each species was embedded in paraffin in its entirety, longitudinally sectioned}

\lipsum
\lipsum

\scieloImageContainerTwoCol{img2.jpg}{\textbf{Figure 1:} A specimen of each species was embedded in paraffin in its entirety, longitudinally sectioned}

\lipsum
\lipsum
\lipsum[1]

\begin{scieloReferencesContainer}[REFERENCES]



\scieloReferencesItem{AMADOR, E. S. Baía de Guanabara e ecossistemas periféricos:
homem e natureza. Rio de Janeiro: Reproarte. 1997. 539 p.}

\scieloReferencesItem{BERGESCH, M.; ODEBRECHT, C.; ABREU, P. C. O.
Microalgas do estuário da Lagoa dos Patos: interação entre o
sedimento e a coluna de água. Oecol. Bras., v.1, p. 273-289,
1995.}

\scieloReferencesItem{Castaigne S, Chomienne C, Daniel MT, Ballerini P, Berger R, Fenaux P, et al.
All-trans retinoic acid as a differentiation therapy for acute promyelocytic
leukemia. I. Clinical results. Blood. 1990;76(9):1704-9. PMID: 2224119}

\scieloReferencesItem{Rego EM, Santana-Lemos BAA, Tamarozzi MB. Differentiation syndrome
in acute promyelocytic leukemia: pathogenesis and risk factors. Rev Bras
Hematol Hemoter. 2008;30(suppl.2):33-6. DOI: http://dx.doi.org/10.1590/
S1516-84842008000800009}

\scieloReferencesItem{Martín del Pozo M, Cisneros de La Fuente E, Solano F, Martín ML, De la Serna
J. El síndrome del ATRA como complicación del tratamiento en la leucemia
promielocítica aguda. An Med Interna (Madrid). 2001;18(4):195-200.}

\scieloReferencesItem{Franchi L, Baccetti T, De Toffol L, Polimeni A, Cozza P. Phases of the dentition for the assessment of skeletal maturity: a diagnostic performance
study. Am J Orthod Dentofacial Orthop. 2008 Mar;133(3):395-400. PMid: 18331939.}

\scieloReferencesItem{Bullock TH and Horridge GA (1965) Structure
and Function in the Nervous Systems of Invertebrates.
Vol. 2. W.H. Freeman and Co, San
Francisco, 1719 pp.}

\scieloReferencesItem{Carranza S, Baguñà J and Riutort M (1997)
Are the Platyhelminthes a monophyletic primitive
group? An assessment using 18S rDNA
sequences. Mol Biol Evol 14:485-497.}

\scieloReferencesItem{Cebrià F (2008) Organization of the nervous system in the model
planarian Schmidtea mediterranea: An immunocytochemical study. Neurosci Res 61:375-384.}

\scieloReferencesItem{Chien P and Koopowitz H (1972) The ultrastructure of neuromuscular systems in Notoplana acticola, a free-living polyclad flatworm. Z Zellforsch Mikroskop Anat 133:277-288.}


\scieloReferencesItem{Chien PK and Koopowitz H (1977) Ultrastructure of nerve plexus
in flatworms. III. The infra-epithelial nervous system. Cell
Tissue Res 176:335-347.}

\scieloReferencesItem{Day TA, Maule AG, Shaw C and Pax RA (1997) Structure-activity relationships of FMRFamide-related peptides contracting Schistosoma mansoni muscle. Peptides 18:917-921.}

\scieloReferencesItem{Egger B, Gschwentner R and Rieger R (2007) Free-living
flatworms under the knife: Past and present. Dev Genes Evol
217:89-104.}

\scieloReferencesItem{Egger B, Lapraz F, Tomiczek B, Müller S, Dessimoz C, Girstmair
J, Skunca N, Rawlinson KA, Cameron CB, Beli E et al.
(2015) A transcriptomic-phylogenomic analysis of the evolutionary relationships of flatworms. Curr Biol 25:1-7.}

\scieloReferencesItem{Ehlers U (1985) Das Phylogenetische System der Plathelminthes.
Gustav Fischer Verlag, Stuttgart, 317 pp.}

\scieloReferencesItem{Fernandes MC, Alvares EP, Gama P and Silveira M (2003) Serotonin in the nervous system of the head region of the land planarian Bipalium kewense. Tissue Cell 35:479-486.}

\scieloReferencesItem{Forest DL and Lindsay SM (2008) Observations of serotonin and
FMRFamide-like immunoreactivity in palp sensory structures and the anterior nervous system of spionid polychaetes. J Morphol 269:544-551.}

\scieloReferencesItem{Girstmair J, Schnegg R, Telford MJ and Egger B (2014) Cellular
dynamics during regeneration of the flatworm Monocelis sp. (Proseriata, Platyhelminthes). Evo Devo 5:e37.}

\scieloReferencesItem{Golding DM (1992) Polychaeta: Nervous system. In: Harrison
FW and Gardiner SL (eds) Microscopic Anatomy of Invertebrates. Vol. 7. Wiley-Liss, New York, pp 155-179.}

\scieloReferencesItem{Gustafsson MKS, Halton DW, Kreshchenko ND, Movsessian SO,
Raikova OI, Reuter M and Terenina NB (2002) Neuropeptides in flatworms. Peptides 23:2053-2061.}

\scieloReferencesItem{Hadenfeldt D (1929) Das Nervensystem von Stylochoplana
maculata und Notoplana atomata. Z wiss Zool 133:586638.}

\scieloReferencesItem{Halton DW and Gustafsson MKS (1996) Functional morphology
of the platyhelminth nervous system. Parasitology
113:S47-S72.}

\end{scieloReferencesContainer}

\scieloLicenseContainer{License information: This is an open-access article distributed under the terms of the Creative Commons Attribution License, which permits unrestricted use, distribution, and reproduction in any medium, provided the original work is properly cited.}



\end{multicols}

