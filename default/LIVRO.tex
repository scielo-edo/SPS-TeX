%%% LaTeX Template: Two column article
%%%
%%% Source: http://www.howtotex.com/
%%% Feel free to distribute this template, but please keep to referal to http://www.howtotex.com/ here.
%%% Date: February 2011

%%% Preamble
\documentclass[DIV=calc,%
              paper=a4,%
              fontsize=11pt%
              ]{scrartcl}            % KOMA-article class

% chinese fonts
\usepackage{xeCJK}
\setCJKmainfont{SimSun}
\setCJKsansfont{SimHei}
\setCJKmonofont{FangSong}

\usepackage[pdfpagelabels,implicit=false,hidelinks]{hyperref} % configurar pagina inicial do pdf
\usepackage{lipsum}
\usepackage{abstract}
\usepackage{penalidades, biblio}
\usepackage[xetex,chicagofootnotes]{tipografia}

\usepackage{amsmath,amsfonts,amsthm}          % Math packages
\usepackage[svgnames]{xcolor}                 % Enabling colors by their 'svgnames'
%\usepackage[hang, small,labelfont=bf,up,textfont=up]{caption}  % Custom captions under/above floats
\usepackage{caption}
\usepackage{hyphenat}
\usepackage{epstopdf}                       % Converts .eps to .pdf
\usepackage{subfig}                         % Subfigures
\usepackage{booktabs}                       % Nicer tables
\usepackage{fix-cm}                         % Custom fontsizes
\usepackage{multicol}
%
%% scielo
\usepackage{graphicx}
\usepackage{ctable}
\usepackage{balance}
\usepackage{verse}
\usepackage{seqsplit}
\usepackage{afterpage}
\usepackage{setspace} %espaçamento entre linhas

% START packages do ctable + longtable
% http://tex.stackexchange.com/questions/88304/multipage-tables-with-ctable-package
\usepackage[utf8]{inputenc}
%\usepackage{libertine}
%\usepackage[T1]{fontenc}
\usepackage{lmodern}
\usepackage[english]{babel}
\usepackage[babel]{csquotes}
\usepackage[backend=bibtex8, style=authoryear]{biblatex}
\usepackage{geometry}
\usepackage{amssymb}
\usepackage{mathtools}
\usepackage{enumitem}
\usepackage{array}
\usepackage{lscape}
\usepackage{multirow}
\usepackage{pbox}
\usepackage{longtable}
% END packages do ctable + longtable

% mathml to tex
\usepackage{bm} % bold
\usepackage{pmml-new} % math template
\usepackage{unicode-math}
\setmathfont{XITS Math}

\makeatletter


\makeatletter
\def\@fnsymbol#1{
     \ensuremath{
         \ifcase#1
\or 1
         \fi
     }
}
\renewcommand{\thefootnote}{\fnsymbol{footnote}}
\makeatother

\newcommand\scieloDefineFootnotes{
\footnotetext[1]{Departamento de Ciência Florestal, Faculdade de Ciências Agronômicas, Universidade Estadual Paulista - UNESP, Botucatu/SP, Brasil}
}

\newcommand{\scieloYear}{2015}
\newcommand\scieloArticlePubIdType{DOI}
\newcommand\scieloArticleType{Artigo Original}
\newcommand{\scieloArticleRef}{2015; 22(3): 355-367}
\newcommand\scieloArticleTitle{Estrutura Diamétrica e Arranjo Espacial das Espécies Mais Abundantes de um Fragmento de Floresta Estacional Semidecidual em Botucatu, SP} 
%\newcommand\scieloArticleTitleTranslation{Tradução do título 1\\Tradução do título 2}
\newcommand{\scieloArticleUrl}{http://dx.doi.org/10.1590/2179-8087.027713} 
\newcommand{\scieloHeader}{Estrutura Diamétrica e Arranjo Espacial...}
\newcommand{\scieloAuthor}{Luiz Alberto Blanco Jorge\scieloContributorFn{1}{xx}{1}{}, Thais Maria Millani\scieloContributorFn{1}{xx}{1}{}, Renata Cristina Batista Fonseca\scieloContributorFn{1}{xx}{1}{}, Aparecido Agostinho}
\newcommand\scieloContributorsFootNotes{\textsuperscript{1}Departamento de Ciência Florestal, Faculdade de Ciências Agronômicas, Universidade Estadual Paulista - UNESP, Botucatu/SP, Brasil}
\newcommand{\scieloISSNImpresso}{1415-0980}
\newcommand{\scieloISSNOnline}{2179-8087}

% table columns
\newcolumntype{L}[1]{>{\raggedright\let\newline\\\arraybackslash\hspace{0pt}}m{#1}}
\newcolumntype{C}[1]{>{\centering\let\newline\\\arraybackslash\hspace{0pt}}m{#1}}
\newcolumntype{R}[1]{>{\raggedleft\let\newline\\\arraybackslash\hspace{0pt}}m{#1}}

% attrib verse
\newcommand{\attrib}[1]{%
\nopagebreak{\raggedleft\footnotesize #1\par}}
\renewcommand{\poemtitlefont}{\normalfont\large\itshape\centering}

% START ctable + longtable
% http://tex.stackexchange.com/questions/88304/multipage-tables-with-ctable-package
\newif\if@ctbl@long
\define@key{ctbl}{long}[]{\@ctbl@longtrue}

\long\def\foo#1\sbox#2!!{%
\long\expandafter\def\csname\string\ctable\endcsname[##1]##2##3##4{%
#1\if@ctbl@long\ctbl@long\fi\sbox#2}}

\expandafter\expandafter\expandafter\foo
\csname\string\ctable\endcsname[#1]{#2}{#3}{#4}!!

\def\ctbl@long\fi\sbox#1#2#3\@ctblend{\fi
\ctbl@@long#2%
}

\long\def\ctbl@@long#1#2#3#4#5#6{%
\ctbl@@@long#6%
}

\def\ctbl@@@long#1\begin#2#3#4\end#5{%
\def\@tempa{#5}%
\def\@tempb{tabular}%
\ifx\@tempa\@tempb\else\show\@tempa\ERROR\fi
\begin{longtable}{#3}%
%\caption{\@ctblcaption\ifx\@ctbllabel\empty\else\label{\@ctbllabel}\fi}%
\caption{\@ctblcaption}
\endfirsthead
\caption*{\@ctblcaption\space(continued)}%
\endhead
#4\end{longtable}}
% END ctable + longtable

% remove label automatico das tabelas
\captionsetup{justification=raggedright, singlelinecheck=false, labelformat=empty}
\captionsetup[table]{labelformat=empty}

% remove padding entre colunas da tabela
\setlength\tabcolsep{0.005\textwidth}

% includegraphics maxwidth
\def\maxwidth#1{\ifdim\Gin@nat@width>#1 #1\else\Gin@nat@width\fi}

% footnotes
\newcommand{\scieloTitleFn}[4]{% {label}{description}{footnotemarkindex}{use footnotemark}
\footnotemark[#3]
}%

\newcommand{\scieloContributorFn}[4]{% {label}{description}{footnotemarkindex}{use footnotemark}
\footnotemark[#3]
}%

\newcommand{\scieloBodyFn}[4]{% {label}{description}{footnotemarkindex}{use footnotemark}
\ifx&#4&%
\protect\footnote{#2}
\else
\protect\footnotemark[#3]
\fi
}%

\newcommand{\scieloCorrespFn}[4]{% {label}{description}{footnotemarkindex}{use footnotemark}
\footnotemark[#3]
}%

% exibicao da seção
\newcommand{\scieloSectionContainer}[1]{%
\section*{#1}
}%
\newcommand{\scieloSubSectionContainer}[1]{%
\subsection*{#1}
}%

% exibicao dos creditos do documento
\newcommand{\scieloLicenseContainer}[1]{%
\medskip\par\noindent
\footnotesize{#1}
}%

% exibicao do abstract
\newcommand{\scieloAbstractContainer}[4]{%
\renewcommand{\abstractname}{#1}
\begin{abstract}
#2
\scieloKeywordsContainer{#3}{#4}
\end{abstract}
}%

% exibicao keywords abstract
\newcommand{\scieloKeywordsContainer}[2]{%
\medskip\par\noindent
\footnotesize{\textit{#1} #2}
}%

% image one col command
\newcommand{\scieloImageContainerOneCol}[3]{% {width}{url}{caption}
\par
\centerline{
\includegraphics[width=\maxwidth{#1}]{#2}
}
\ifx&#3&%
\else
\captionof{figure}{#3} 
\fi
\par
}%

% image two col command
\newcommand{\scieloImageContainerTwoCol}[2]{% {url}{caption}
\begin{figure*}
\centerfloat
\includegraphics[width=\maxwidth{\textwidth}]{#1}
\ifx&#2&%
\else
\caption{#2} 
\fi
\end{figure*}
}%

% image inline  command
\newcommand{\scieloImageInlineContainer}[1]{% {url}{caption}
{
\centering{
\includegraphics[width=\maxwidth{0.5\textwidth},height=0.7cm,keepaspectratio]{#1}
}
}
}%

%%% Headers and footers
\usepackage{fancyhdr}                       % Needed to define custom headers/footers
  \pagestyle{fancy}                           % Enabling the custom headers/footers
\usepackage{lastpage} 

% Header (empty)
\lhead{}
\chead{}
\rhead{}
% Footer (you may change this to your own needs)
\lfoot{\scieloFooterText \ifnum\thepage=1{\\\scieloArticleId}\fi }
\cfoot{}
\rfoot{\footnotesize \scieloFooterRight}  % "Page 1 of 2"
\renewcommand{\headrulewidth}{0.0pt}
\renewcommand{\footrulewidth}{0.4pt}


%%% Creating an initial of the very first character of the content
\usepackage{lettrine}
\newcommand{\initial}[1]{%
     \lettrine[lines=3,lhang=0.3,nindent=0em]{
            \color{DarkGoldenrod}
            {\textsf{#1}}}{}}



%%% Title, author and date metadata
\usepackage{titling}                              % For custom titles

\newcommand{\HorRule}{\color{DarkGoldenrod}%      % Creating a horizontal rule
                      \rule{\linewidth}{1pt}%
                    }

\pretitle{
  \vspace{-30pt} 
  \begin{flushright}{\small\scieloArticleType \\\HorRule}\end{flushright} 
  \begin{flushleft} % start flush
    \fontsize{\scieloTitleFontSize}{\scieloTitleFontSize} 
    \usefont{\encodingdefault}{phv}{b}{n} 
    \color{DarkRed} 
    \selectfont 
}

\title{\scieloArticleTitle}

\posttitle{
  \par
  \end{flushleft} % end flush
  \vskip 0.5em
  \@ifundefined{scieloArticleTitleTranslation}
  {}
  {
    \color{DarkRed}{\Large\scieloArticleTitleTranslation}
    \vskip 0.5em
  }
}

\preauthor{
  \begin{flushleft} % start flush
    \large 
    \color{DarkRed}
}

\author{\scieloContributorNames}

\postauthor
{
  \color{Black} 
  \vskip 0.5em
  { \scriptsize \setstretch{0.5} \scieloContributorsFootNotes \par\relax}
  \par
  \end{flushleft} % end flush
  \HorRule
}

\date{}                                       % No date

%
%
%%%% Begin document
\begin{document}

\scieloDefineFootnotes

\maketitle
\thispagestyle{fancy}       % Enabling the custom headers/footers for the first page 

\setcounter{footnote}{4}
\selectlanguage{english}

\scieloAbstractContainer{ABSTRACT}{\textbf{Background}: The identification of the occurrence of falls is an important step for screening and for rehabilitation proccess for the elderly. The methods of monitoring these events are susceptible to recording biases, and the choice of the most accurate method remains challenging. \textbf{Objectives}: (i) To investigate the agreement between retrospective self-reporting and prospective monitoring of methods of recording falls, and (ii) to compare the retrospective self-reporting of falls and the prospective monitoring of falls and recurrent falls over a 12-month period among older women at high risk of falls and fractures. \textbf{Method}: A total of 118 community-dwelling older women with low bone density were recruited. The incidence of falls was monitored prospectively in 116 older women (2 losses) via monthly phone calls over the course of a year. At the end of this monitoring period, the older women were asked about their recall of falls in the same 12-month period.  The agreement between the two methods where analyzed and the sensitivity and specificity of self-reported previous falls in relation to the prospective monitoring weere calculated. \textbf{Results}: There was moderate agreement between the prospective monitoring and the retrospective self-reporting of falls in classifying fallers (Kappa=0.595) and recurrent fallers (Kappa=0.589). The limits of agreement were 0.35±1.66 falls. The self-reporting of prior falls had a 67.2\% sensitivity and a 94.2\% specificity in classifying fallers among older women and a 50\% sensitivity and a 98.9\% specificity in classifying recurrent fallers. Conclusion: Self-reporting of falls over a 12-month period underestimated 32.8\% of falls and 50\% of recurrent falls. The findings recommend caution if one is considering replacing monthly monitoring with annual retrospective questioning.}{Keywords:}{agend, bone desnsity, accidental falls; rehabilitation; mental recall}

%{\noindent\fontsize{9}{10.8}\selectfont{Received: January 13, 2015; Accepted: April 19, 2015.}}

\begin{multicols}{2}
\scieloSectionContainer{Introduction}
\par{}Falls are events with a high prevalence\scieloBodyFn{1}{fn body}{1}{} among the elderly population, even among those who are active and healthy, and constitute one of the major preventable geriatric syndromes'. Among the community-dwelling elderly, approximately 30\% suffer a fall each year, and half experience recurrent falls2. Elderly women with osteoporosis and having a high risk of fractures exhibit an even higher frequency of falls (51.1\%)3. A significant portion of these falls results in injuries (36\%)4, fractures (3.4\% to 19\%)2,4,5, and the need for medical assistance (8 to 19\%)4,5 and affects lifestyle choices, creating a high socio-economic burden6. Additionally, experiencing one or more falls in the course of one year significantly increases the chances of the occurrence of new episodes in the following year among the community-dwelling elderly\textsuperscript{4,5} and postmenopausal women\textsuperscript{1'7}.
\par{}Thus, the surveillance of falls among the elderly represents a priority health issue\textsuperscript{6},  which is why uestioning the occurrence of previous falls has been used in clinical/scientific decision making". Several methods have been suggested for monitoring the occurrence of falls among the community-dwelling elderly, including questions asking individuals to recall these events at several intervals by means of telephone, face-to-face or mail interviews, information obtained from medical records, and/or prospective records using falls calendars or diaries\textsuperscript{8,9,11-13}. However, the elderly\\

\lipsum

\scieloSectionContainer{Materials and Methods}

\scieloSubSectionContainer{Animals}

\par{}Specimens representing 12 species of polyclads were
collected from different sites in the Caribbean. In addition,
a deep-sea specimen of \textit{Anocellidus profundus} Quiroga,
Bolaños and Litvaitis, 2006, from the North Pacific was
also included in the analysis (specimen courtesy of Dr.
Janet Voight, Field Museum of Natural History, Chicago,
Illinois, USA). For details of collection, fixation, and spe-
cies identification see Litvaitis \textit{et al}. (2010).

\scieloSubSectionContainer{Histological studies of polyclad nervous systems}

\par{}A specimen of each species was embedded in paraffin
in its entirety, longitudinally sectioned at 7-10 mm, and
stained with Milligan Trichrome technique which allows
differentiating well among connective tissue, muscles, and
nervous fibers (Presnell and Schreibman, 1997). Additional
cross and sagittal sections were also prepared. De-paraffi-
nized sections were treated with 3\% potassium dichro-
mate-hydrochloric acid solution for 5 min. Following a
distilled water rinse, sections were stained in acid fuchsin
for 8 min. After a second distilled water rinse, they were
placed into 1\% phosphomolybdic acid for 2 min and then
stained with 2\% solution of Orange G for 5 min. After a fi-
nal rinse with distilled water, the sections were treated with
1\% hydrochloric acid solution for 2 min stained in Fast
Green for 8 min, treated with 1\% acetic acid for 3 min and
then rinsed in 95\% alcohol and dehydrated. Finally, the sec-
tions were cleared with Histoclear (National Diagnostics)
and mounted in Permount (Fisher Scientific). Slides were
observed and photographed under an Axiostar Plus (Zeiss,
Thornwood, New York) light microscope. We applied the
morphological criteria of Reuter et al. (1998) to distinguish
main nerve cords from secondary nerve cords for the exam-
ination of the NS. Cross and sagittal sections were used for
interpretation. In addition, whole mounts of all species
were prepared, by dehydrating the specimens in a graded
alcohol series, cleared with Histoclear, and mounted in
Permount.

\lipsum

\scieloImageContainerOneCol{0.5\textwidth}{img1.jpg}{meu titulo}

%\begin{center}
%{
%\vspace{8mm}
%\centerline{
%\includegraphics[width=\maxwidth{0.5\textwidth}]{img1.jpg}
%}
%\vspace{8mm}
%}
%\end{center}

\lipsum

\scieloImageContainerTwoCol{img2.jpg}{\textbf{Figure 1:} A specimen of each species was embedded in paraffin in its entirety, longitudinally sectioned}

\lipsum
\lipsum

\scieloImageContainerTwoCol{img2.jpg}{\textbf{Figure 1:} A specimen of each species was embedded in paraffin in its entirety, longitudinally sectioned}

\lipsum
\lipsum

\begin{scieloReferencesContainer}[References]

\scieloReferencesItem{Ganz DA, Bao Y, Shekelle PG, Rubenstein LZ. Will my patient fall? JAMA. 2007;297(1):77-86. http://dx.doi.org/10.1001/jama.297.1.77. PMid:17200478.}

\scieloReferencesItem{Baguñà J and Riutort M (2004) Molecular phylogeny
of the Platyhelminthes. Can J Zool
82:168-193.}

\scieloReferencesItem{Bailly X, Reichert H and Hartenstein V (2013)
The urbilaterian brain revisited: Novel insights
into old questions from new flatworm clades.
Dev Genes Evol 223:149-157.}

\scieloReferencesItem{Biserova NM, Dudicheva VA, Terenina NB,
Reuter M, Halton DW, Maule AG and Gustafsson
MK (2000) The nervous system of Amphilina
foliacea (Platyhelminthes, Amphilinidea),
an immunocytochemical, ultrastructural
and spectrofluorometrical study. Parasitology
121:441-453.}

\scieloReferencesItem{Bock S (1923) Boninia, a new polyclad genus
from the Pacific. Nov Act Reg Soc Uppsala
Ser 46:1-32.}

\scieloReferencesItem{Böckerman I, Reuter M and Timoshkin O
(1994) Ultrastructural study of the central
nervous system of endemic Geocentrophora
(Prorhynchida, Platyhelminthes) from Lake
Baikal. Acta Zool 75:47-55.}

\scieloReferencesItem{Bullock TH and Horridge GA (1965) Structure
and Function in the Nervous Systems of Invertebrates.
Vol. 2. W.H. Freeman and Co, San
Francisco, 1719 pp.}

\scieloReferencesItem{Carranza S, Baguñà J and Riutort M (1997)
Are the Platyhelminthes a monophyletic primitive
group? An assessment using 18S rDNA
sequences. Mol Biol Evol 14:485-497.}

\scieloReferencesItem{Cebrià F (2008) Organization of the nervous system in the model
planarian Schmidtea mediterranea: An immunocytochemical study. Neurosci Res 61:375-384.}

\scieloReferencesItem{Chien P and Koopowitz H (1972) The ultrastructure of neuromuscular systems in Notoplana acticola, a free-living polyclad flatworm. Z Zellforsch Mikroskop Anat 133:277-288.}


\scieloReferencesItem{Chien PK and Koopowitz H (1977) Ultrastructure of nerve plexus
in flatworms. III. The infra-epithelial nervous system. Cell
Tissue Res 176:335-347.}

\scieloReferencesItem{Day TA, Maule AG, Shaw C and Pax RA (1997) Structure-activity relationships of FMRFamide-related peptides contracting Schistosoma mansoni muscle. Peptides 18:917-921.}

\scieloReferencesItem{Egger B, Gschwentner R and Rieger R (2007) Free-living
flatworms under the knife: Past and present. Dev Genes Evol
217:89-104.}

\scieloReferencesItem{Egger B, Lapraz F, Tomiczek B, Müller S, Dessimoz C, Girstmair
J, Skunca N, Rawlinson KA, Cameron CB, Beli E et al.
(2015) A transcriptomic-phylogenomic analysis of the evolutionary relationships of flatworms. Curr Biol 25:1-7.}

\scieloReferencesItem{Ehlers U (1985) Das Phylogenetische System der Plathelminthes.
Gustav Fischer Verlag, Stuttgart, 317 pp.}

\scieloReferencesItem{Fernandes MC, Alvares EP, Gama P and Silveira M (2003) Serotonin in the nervous system of the head region of the land planarian Bipalium kewense. Tissue Cell 35:479-486.}

\scieloReferencesItem{Forest DL and Lindsay SM (2008) Observations of serotonin and
FMRFamide-like immunoreactivity in palp sensory structures and the anterior nervous system of spionid polychaetes. J Morphol 269:544-551.}

\scieloReferencesItem{Girstmair J, Schnegg R, Telford MJ and Egger B (2014) Cellular
dynamics during regeneration of the flatworm Monocelis sp. (Proseriata, Platyhelminthes). Evo Devo 5:e37.}

\scieloReferencesItem{Golding DM (1992) Polychaeta: Nervous system. In: Harrison
FW and Gardiner SL (eds) Microscopic Anatomy of Invertebrates. Vol. 7. Wiley-Liss, New York, pp 155-179.}

\scieloReferencesItem{Gustafsson MKS, Halton DW, Kreshchenko ND, Movsessian SO,
Raikova OI, Reuter M and Terenina NB (2002) Neuropeptides in flatworms. Peptides 23:2053-2061.}

\scieloReferencesItem{Hadenfeldt D (1929) Das Nervensystem von Stylochoplana
maculata und Notoplana atomata. Z wiss Zool 133:586638.}

\scieloReferencesItem{Halton DW and Gustafsson MKS (1996) Functional morphology
of the platyhelminth nervous system. Parasitology
113:S47-S72.}

\end{scieloReferencesContainer}

\scieloLicenseContainer{License information: This is an open-access article distributed under the terms of the Creative Commons Attribution License, which permits unrestricted use, distribution, and reproduction in any medium, provided the original work is properly cited.}

\end{multicols}


\end{document}
