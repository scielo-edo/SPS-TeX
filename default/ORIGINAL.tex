\scieloAbstractContainer{Resumo}{Na ciência,\allowbreak{} modelos podem ser analogias provindas de outros campos.\allowbreak{} A narrativa literária da vivência de desencontro com a própria imagem no espelho é uma metáfora da subjetivação como um processo a partir de uma alteridade que a precede.\allowbreak{} Se Lacan estabeleceu uma analogia com um experimento da óptica a fim de elaborar um modelo desse processo psíquico,\allowbreak{} a literatura,\allowbreak{} por sua vez,\allowbreak{} dispõe das metáforas que funcionam como modelos de fenômenos psíquicos devido à sua função crítica.\allowbreak{}}{Palavras Chave:}{Psicanálise, literatura, modelo óptico, espelho}

\scieloAbstractContainer{Abstract}{In science,\allowbreak{} models can be built based on analogies from other fields.\allowbreak{} The literary narrative of the non-\allowbreak{}identification with one's own image in the mirror is a metaphor of subjectivation,\allowbreak{} in terms of a process starting from an otherness preceding it.\allowbreak{} If Lacan established an analogy with an optical experiment,\allowbreak{} devising a psychoanalytical model for that psychic process,\allowbreak{} literature,\allowbreak{} in turn,\allowbreak{} provides the metaphors that serve as models for psychic phenomena due to their critical function.\allowbreak{}}{Keywords:} {Psychoanalysis, literature, optical model, mirror}

\scieloAbstractContainer{Résumé}{En sciences,\allowbreak{} les modèles peuvent être construits à partir d'analogies provenant d’autres domaines.\allowbreak{} Le récit littéraire sur l’expérience du manque d'identification avec sa propre image que l'on voit dans le miroir est une métaphore de la subjectivation comme processus à partir d’une altérité qui la précède.\allowbreak{} Si Lacan a établi une analogie avec une expérience d’optique pour l’élaboration d’un modèle de ce processus psychique,\allowbreak{} la littérature,\allowbreak{} à son tour,\allowbreak{} fournit des métaphores qui servent de modèles pour les phénomènes psychiques en raison de leur fonction critique.\allowbreak{}}{Keywords:} {Psychanalyse, littérature, modèle optique, miroir}

\scieloAbstractContainer{Resumen}{En ciencias,\allowbreak{} los modelos pueden ser analogías provenientes de otros campos.\allowbreak{} La narrativa literaria de la vivencia de la discrepancia con la propia imagen en el espejo es una metáfora de la subjetivación como un proceso desde una alteridad que la precede.\allowbreak{} Si Lacan estableció una analogía con un experimento óptico con el fin de desarrollar un modelo de este proceso psíquico,\allowbreak{} la literatura,\allowbreak{} a su vez,\allowbreak{} proporciona metáforas que sirven como modelos de los fenómenos psíquicos debido a su función crítica.\allowbreak{}}{Keywords:} {Psicoanálisis, literatura, modelo óptico, espejo}

\scieloAbstractContainer{Zusammenfassung}{In der Wissenschaft können Modelle aufgrund von Analogien entwickelt werden,\allowbreak{} die aus anderen Wissensbereichen stammen.\allowbreak{} Die literarische Erzählung der Erfahrung der Nicht-\allowbreak{}identifizierung mit dem eigenen Bild im Spiegel ist eine Metapher der Subjektivierung als Prozess,\allowbreak{} die von einer Andersartigkeit ausgeht,\allowbreak{} die dieser Subjektivierung vorausgeht.\allowbreak{} Lacan wählte z.\allowbreak{} B.\allowbreak{} eine Analogie mit einem optischen Experiment,\allowbreak{} um das psychoanalytische Modell für diesen psychischen Prozess zu entwickeln.\allowbreak{} Die Literatur wiederum verfügt über Metaphern die aufgrund ihrer kritischen Funktion als Modelle für psychische Phänomene benutzt werden können.\allowbreak{}}{Keywords:} {Psychoanalyse, Literatur, Optisches Modell, Spiegel}

\begin{multicols}{2}
\scieloSectionContainer{Introdução}
\begin{quote}\par{}\textit{As ciências,\allowbreak{} e sobretudo as ciências\scieloImageInlineContainer{not-found.png} em gestação como a nossa,\allowbreak{} frequentemente tomam emprestado modelos a outras ciências}.\allowbreak{}\par{}(\allowbreak{}\textsuperscript{Lacan,\allowbreak{}1953-\allowbreak{}1954\fshyp{}1979},\allowbreak{} p.\allowbreak{} 91)\allowbreak{}.\allowbreak{}\par{}\textit{Models are expendable; theories are not.\allowbreak{} }\par{}(\allowbreak{}\textsuperscript{Bion,\allowbreak{} 1990},\allowbreak{} p.\allowbreak{} 25)\allowbreak{}.\allowbreak{}\end{quote}\par{}Em um processo de investigação científica,\allowbreak{} modelos costumam ser utilizados como analogias que podem provir do campo de outras ciências.\allowbreak{} Os modelos servem para teorizar,\allowbreak{} como destaca \textsuperscript{Bunge (\allowbreak{}1975)\allowbreak{}},\allowbreak{} para quem a construção de um modelo teórico faz parte do processo de investigação científica e inclui a invenção de hipóteses e sua tradução em linguagem matemática.\allowbreak{} Nesse sentido,\allowbreak{} \textsuperscript{Granger (\allowbreak{}1994)\allowbreak{}} indica o uso que se fez da topologia para postular propriedades formais de relações que não se traduzem na empiria; já o uso da estrutura de grafo serviria para representar fenômenos humanos como as relações de parentesco.\allowbreak{} Lembramos então do relato de \textsuperscript{Lévi-\allowbreak{}Strauss (\allowbreak{}2005)\allowbreak{}} sobre a busca de modelos matemáticos para a elaboração das \textit{As estruturas elementares do parentesco} (\allowbreak{}2009)\allowbreak{}.\allowbreak{} Dada complexidade dos sistemas de parentesco,\allowbreak{} o antropólogo francês procurara um matemático.\allowbreak{} Se o primeiro matemático consultado não admitiu que fosse possível trabalhar matematicamente o sistema de casamento,\allowbreak{} o segundo matemático entendeu que não era necessário definir o casamento sob um ponto de vista matemático.\allowbreak{} O que interessava,\allowbreak{} para a formalização,\allowbreak{} eram as relações entre os casamentos.\allowbreak{} Eis a definição da função dos modelos:\allowbreak{} dar forma à estrutura de relações entre os elementos.\allowbreak{}\par{}Por conseguinte,\allowbreak{} um modelo é uma ficção e um instrumento.\allowbreak{} Ficção,\allowbreak{} porque é uma criação que não tem existência,\allowbreak{} não é um referente.\allowbreak{} Instrumento,\allowbreak{} porque permite um modo de acesso ao real.\allowbreak{} Para acessar o Real podemos utilizar teorias ou modelos,\allowbreak{} ambos são construções,\allowbreak{} criações humanas.\allowbreak{} Assim,\allowbreak{} os modelos em psicopatologia seriam formas de ficcionar um Real ao qual não temos acesso.\allowbreak{}\par{}A Psicopatologia Fundamental,\allowbreak{} segundo \textsuperscript{Pierre Fédida (\allowbreak{}1990)\allowbreak{}},\allowbreak{} trabalha com modelos provenientes de distintas teorias.\allowbreak{} Portanto,\allowbreak{} no processo de modelização do funcionamento humano,\allowbreak{} a psicopatologia recebe influência de modelos elaborados nos diversos campos científicos.\allowbreak{} A meta da psicopatologia fundamental é submeter esses modelos à prova crítica,\allowbreak{} confrontando-\allowbreak{}os entre si (\allowbreak{}\textsuperscript{Fédida \&\allowbreak{\allowbreak{}\allowbreak{}}\allowbreak{} Widlöcher,\allowbreak{} 1990}; \textsuperscript{Fédida,\allowbreak{} 1998)\allowbreak{}}.\allowbreak{}\par{}\textsuperscript{Marx \&\allowbreak{\allowbreak{}\allowbreak{}}\allowbreak{} Hillix (\allowbreak{}1978)\allowbreak{}} definem modelos como uma construção que funciona e que pode ser facilmente abandonada.\allowbreak{} É assim que Freud (\allowbreak{}1900)\allowbreak{} apresentava o aparelho psíquico,\allowbreak{} cuja forma provinha de analogias com outros campos da ciência,\allowbreak{} isto é,\allowbreak{} em comparação a um microscópio composto ou aparelho fotográfico.\allowbreak{} O modelo do aparelho psíquico (\allowbreak{}\textsuperscript{Freud,\allowbreak{} 1900\fshyp{}2012},\allowbreak{} p.\allowbreak{} 569)\allowbreak{} é apresentado na Figura 1.\allowbreak{}\par{}
%\par
%{
%\centering{
%\includegraphics[width=\maxwidth{0.5\textwidth}]{r1415-4714-rlpf-18-1-0152-gf01.tif}
%}
%\captionof{figure}{\textbf{Figura 1:} \textit{– O modelo do aparelho psíquico}} 
%}
\par
\par{}A estrutura do aparelho psíquico surge a partir de dois modelos:\allowbreak{} a estrutura espacial de aparelhos ópticos,\allowbreak{} constituída de lentes que se justapõem uma a outra; e o aparelho do arco-\allowbreak{}reflexo.\allowbreak{} A estrutura do aparelho psíquico será subdividida em três sistemas:\allowbreak{} primeiro,\allowbreak{} um sistema frontal,\allowbreak{} a terminação sensível que recebe as percepções.\allowbreak{} Atrás desse se encontra um segundo sistema,\allowbreak{} o qual transforma a excitação do primeiro em traços duradouros (\allowbreak{}trata-\allowbreak{}se aqui do inconsciente)\allowbreak{},\allowbreak{} que podem conservar elementos do material cru das lembranças; enfim,\allowbreak{} na terminação motora encontra-\allowbreak{}se outro sistema,\allowbreak{} o qual abre as represas da motilidade.\allowbreak{} Essa descrição e desenho do aparelho psíquico é tratada como uma representação auxiliar,\allowbreak{} como os andaimes de uma construção.\allowbreak{}\par{}O inconsciente está entre a terminação sensível e a motora,\allowbreak{} ou seja,\allowbreak{} em ruptura,\allowbreak{} em um “entre”.\allowbreak{} Destaca-\allowbreak{}se a ideia de inconsciente como outra cena,\allowbreak{} em lugar atemporal,\allowbreak{} em uma outra localidade,\allowbreak{} em um outro espaço.\allowbreak{} Como aponta \textsuperscript{Lacan (\allowbreak{}1985)\allowbreak{}},\allowbreak{} “é preciso que apreendamos o inconsciente em sua experiência de ruptura entre percepção e consciência” (\allowbreak{}p.\allowbreak{} 58)\allowbreak{}.\allowbreak{}\par{}Desde os estudos freudianos,\allowbreak{} a literatura aparece como referência metodológica nos ensaios metapsicológicos.\allowbreak{} As formulações de \textsuperscript{Freud (\allowbreak{}1919)\allowbreak{}} sobre o sentimento do inquietante (\allowbreak{}\textit{unheimlich})\allowbreak{} foram elaboradas a partir da análise de dois contos de E.\allowbreak{}T.\allowbreak{}A.\allowbreak{} Hoffmann.\allowbreak{} Ao lado dos modelos ópticos,\allowbreak{} teríamos também modelos literários?
\scieloSectionContainer{Lacan:\allowbreak{} da metáfora do estádio do espelho ao modelo óptico}
\par{}Ao formular a noção de Estádio do espelho como função exemplar que revela certas relações do sujeito com a própria imagem,\allowbreak{} enquanto forma primordial do Eu,\allowbreak{} \textsuperscript{Lacan (\allowbreak{}1998a)\allowbreak{}} trabalhava com a função metafórica do espelho.\allowbreak{} Essa noção consistia na releitura de um experimento científico de observação de bebês.\allowbreak{} Lacan destacou,\allowbreak{} daquele experimento,\allowbreak{} uma questão:\allowbreak{} por que os bebês se interessam pela sua própria imagem? Foi essa a pergunta que orientou a elaboração da concepção (\allowbreak{}teórica)\allowbreak{} de estádio do espelho.\allowbreak{} É quando consegue elaborar uma apresentação óptica do estádio do espelho que a constituição do sujeito obtém uma explicação,\allowbreak{} ou seja,\allowbreak{} a metáfora do estádio do espelho encontra as relações de homologia necessárias para fazer do modelo óptico um modelo para teorização.\allowbreak{}\par{}Para essa elaboração,\allowbreak{} \textsuperscript{Lacan (\allowbreak{}1979)\allowbreak{}} inicia pelo estudo da óptica,\allowbreak{} que se funda sobre uma teoria matemática,\allowbreak{} segundo a qual a todo ponto dado no espaço real corresponde um ponto,\allowbreak{} e só um num outro espaço,\allowbreak{} que é o espaço imaginário.\allowbreak{} É a hipótese estrutural fundamental (\allowbreak{}p.\allowbreak{} 93)\allowbreak{}.\allowbreak{}\par{}O “experimento do buquê invertido”(\allowbreak{}Figura 2)\allowbreak{} será a base de um modelo “para a relação entre o mundo imaginário e mundo real na economia psíquica” (\allowbreak{}\textsuperscript{Lacan,\allowbreak{} 1979},\allowbreak{} p.\allowbreak{} 95)\allowbreak{}.\allowbreak{} Esse modelo permite diferenciar o espaço imaginário e o espaço real,\allowbreak{} os quais podem se confundir em alguns fenômenos físicos,\allowbreak{} como um arco-\allowbreak{}íris.\allowbreak{} Podemos ver o arco-\allowbreak{}íris,\allowbreak{} mas ele não está lá,\allowbreak{} mas um aparelho fotográfico pode registrar as imagens.\allowbreak{} Será que o psiquismo,\allowbreak{} também registraria? Essa é a analogia proposta por \textsuperscript{Freud (\allowbreak{}1900\fshyp{}2012)\allowbreak{}}:\allowbreak{}\par{}
\par
%{
%\centering{
%\includegraphics[width=\maxwidth{0.5\textwidth}]{r1415-4714-rlpf-18-1-0152-gf02.tif}
%}
%\captionof{figure}{\textbf{Figura 2:} \textit{– O “Experimento óptico” de Bouasse}} 
%}
%\par
\begin{quote}\par{}O lugar psíquico corresponde então a um lugar no interior de um aparelho em que se forma um dos estágios prévios da imagem.\allowbreak{} No caso do microscópio e do telescópio,\allowbreak{} como se sabe,\allowbreak{} tais lugares são em parte lugares ideais,\allowbreak{} regiões em que não há nenhum componente palpável do aparelho.\allowbreak{} (\allowbreak{}p.\allowbreak{} 564)\allowbreak{}\end{quote}\par{}A clareza de Freud quanto ao uso do modelo depreende-\allowbreak{}se do uso das seguintes adjetivações:\allowbreak{} lugares ideais,\allowbreak{} componentes não palpáveis.\allowbreak{} Freud insiste:\allowbreak{} “no fundo,\allowbreak{} não precisamos fazer a suposição de um arranjo realmente espacial dos sistemas psíquicos” (\allowbreak{}p.\allowbreak{} 564)\allowbreak{}.\allowbreak{} O que Freud extraíra dos instrumentos ópticos para dar forma a seu aparelho psíquico era a estrutura matemática,\allowbreak{} mais especificamente,\allowbreak{} as leis da geometria,\allowbreak{} as quais \textsuperscript{Lacan (\allowbreak{}1953-\allowbreak{}1954\fshyp{}1979)\allowbreak{}} explicitará:\allowbreak{}\begin{quote}\par{}Sabemos que um espelho esférico pode produzir,\allowbreak{} de um objeto situado no ponto de seu centro de curvatura,\allowbreak{} uma imagem que lhe é simétrica,\allowbreak{} mas sobre a qual o importante é que ela é uma imagem real (\allowbreak{}.\allowbreak{}.\allowbreak{}.\allowbreak{})\allowbreak{} Em certas condições,\allowbreak{} essa imagem pode ser fitada pelo olho em sua realidade (\allowbreak{}.\allowbreak{}.\allowbreak{}.\allowbreak{})\allowbreak{} É o caso da chamada ilusão do buquê invertido,\allowbreak{} que encontraremos descrita em \textit{L’Optique et photométrie dites géometriques }de Bouasse.\allowbreak{} (\allowbreak{}\textsuperscript{Lacan,\allowbreak{} [1961]1966\fshyp{}1998b},\allowbreak{} p.\allowbreak{} 679)\allowbreak{}\end{quote}\par{}Lacan introduz um elemento novo no experimento óptico de modo a elaborar um modelo teórico para “as relações do Eu Ideal com o Ideal do Eu” (\allowbreak{}p.\allowbreak{} 679)\allowbreak{}.\allowbreak{} A montagem lacaniana que completará o aparelho será a introdução de um espelho plano (\allowbreak{}Figura 3)\allowbreak{}.\allowbreak{}\par{}
\par
%{
%\centering{
%\includegraphics[width=\maxwidth{0.5\textwidth}]{r1415-4714-rlpf-18-1-0152-gf03.tif}
%}
%\captionof{figure}{\textbf{Figura 3:} \textit{– A figura 3 apresenta a montagem do modelo óptico}} 
%}
%\par
\begin{quote}\par{}É que as ligações que ali irão aparecer,\allowbreak{} à maneira analógica,\allowbreak{} relacionam-\allowbreak{}-\allowbreak{}se claramente com estruturas (\allowbreak{}intra-\allowbreak{})\allowbreak{} subjetivas como tais,\allowbreak{} representando a relação com o outro e permitindo distinguir nela a dupla incidência do imaginário e do simbólico.\allowbreak{} (\allowbreak{}\textsuperscript{Lacan,\allowbreak{} 1961\fshyp{}1998b},\allowbreak{} pp.\allowbreak{} 680-\allowbreak{}681)\allowbreak{}\end{quote}\par{}Segue,\allowbreak{} então,\allowbreak{} a explicação do modelo óptico de Lacan.\allowbreak{}
\begin{enumerate}[label=\arabic*,leftmargin=*]
\item \par{}O vaso invertido no interior da caixa e sua imagem real vêm a circundar com seu gargalo o buquê de flores já montado acima dele,\allowbreak{} o qual desempenhará,\allowbreak{} para o olho,\allowbreak{} o papel de suporte de acomodação necessário para que se produza a ilusão.\allowbreak{}
\item \par{}Para que um observador situado na borda espelho esférico veja sua imagem no espelho A,\allowbreak{} é necessário que sua própria imagem venha,\allowbreak{} no espaço real (\allowbreak{}ao qual corresponde ponto a ponto o espaço virtual gerado por um espelho plano)\allowbreak{},\allowbreak{} situar-\allowbreak{}se no interior do cone que delimita a possibilidade da ilusão [campo x’y’ na figura 3)\allowbreak{}].\allowbreak{}
\end{enumerate}


\scieloSubSectionContainer{Do modelo óptico à teoria psicanalítica}
\par{}\textsuperscript{Eidelsztein (\allowbreak{}1992)\allowbreak{}} extrai,\allowbreak{} da comparação entre o modelo freudiano e o modelo lacaniano,\allowbreak{} a diferenciação teórica entre um Eu objeto em Freud e um Eu imagem em Lacan.\allowbreak{} É em função dessa questão que se torna produtivo o uso da analogia com o experimento óptico em psicanálise.\allowbreak{} Em óptica,\allowbreak{} as imagens são de dois tipos:\allowbreak{} imagens reais e imagens virtuais.\allowbreak{} As \textit{imagens reais }são as produzidas,\allowbreak{} por exemplo,\allowbreak{} por um espelho côncavo,\allowbreak{} ou seja,\allowbreak{} algo parecido à superfície interna e bem polida de uma esfera oca.\allowbreak{} Chamam-\allowbreak{}se imagens reais porque para o sujeito que percebe,\allowbreak{} elas se comportam como objetos e não como imagens,\allowbreak{} implicam uma ilusão óptica,\allowbreak{} isto é,\allowbreak{} o observador é enganado.\allowbreak{} As \textit{imagens virtuais} são as imagens cotidianas produzidas por um espelho plano (\allowbreak{}de uso cotidiano)\allowbreak{} e não implicam ilusão óptica alguma,\allowbreak{} já que para o sujeito observador essas imagens se comportam como tais,\allowbreak{} ou seja,\allowbreak{} como imagens.\allowbreak{}\par{}No modelo óptico,\allowbreak{}\scieloBodyFn{1}{O leitor pode se reportar à .\allowbreak{}}{1}{} a imagem especular (\allowbreak{}a imagem no espelho plano)\allowbreak{} serve como metáfora para indicar que o sujeito não se funda a si mesmo.\allowbreak{} Mas o Outro,\allowbreak{} cujo correspondente,\allowbreak{} no modelo óptico,\allowbreak{} é o espelho plano,\allowbreak{} é o mediador pelo qual o sujeito encontra sua “própria” imagem,\allowbreak{} porém é também o que separa o sujeito de sua imagem.\allowbreak{}\par{}Em vez de autonomia,\allowbreak{} há alienação,\allowbreak{} ou seja,\allowbreak{} o sujeito se reconhece por meio de uma imagem que ele não é,\allowbreak{} e onde não está.\allowbreak{} Apesar da ilusão de se conhecer através do espelho,\allowbreak{} o eu só se (\allowbreak{}re)\allowbreak{}conhece no espelho como uma imagem alienada.\allowbreak{} Para a distinção entre eu e não eu será preciso que,\allowbreak{} virtualmente,\allowbreak{} eu me veja projetado em i’(\allowbreak{}a)\allowbreak{},\allowbreak{} com a possibilidade de me situar em i(\allowbreak{}a)\allowbreak{},\allowbreak{} como origem da projeção.\allowbreak{} Por meio da forma i(\allowbreak{}a)\allowbreak{},\allowbreak{} a minha imagem,\allowbreak{} minha presença no Outro,\allowbreak{} não tem resto.\allowbreak{} Se não tem resto,\allowbreak{} não consigo ver o que perco ali.\allowbreak{} “É esse o sentido do estádio do espelho” (\allowbreak{}\textsuperscript{Lacan,\allowbreak{} 1962-\allowbreak{}1963\fshyp{}2005},\allowbreak{} p.\allowbreak{} 277)\allowbreak{}.\allowbreak{}\par{}Há um resto,\allowbreak{} no entanto,\allowbreak{} chamado de “a”,\allowbreak{} que não terá seu correspondente especular,\allowbreak{} representado metaforicamente pela alma,\allowbreak{} como veremos a seguir na análise dos contos “O espelho”.\allowbreak{} Esse resto produzirá a ilusão (\allowbreak{}x)\allowbreak{} que alimenta narcisicamente essa imagem não eu,\allowbreak{} i’(\allowbreak{}a)\allowbreak{},\allowbreak{} da qual me aproprio e onde me reconheço.\allowbreak{} E o modelo óptico demonstra matematicamente esse conceito pois,\allowbreak{} simetricamente,\allowbreak{} a cada ponto em i(\allowbreak{}a)\allowbreak{} corresponde um ponto em i’(\allowbreak{}a)\allowbreak{}.\allowbreak{} \textsuperscript{Soler (\allowbreak{}2012)\allowbreak{}} nos chama a atenção para a forma escrita da imagem egoizável:\allowbreak{} i(\allowbreak{}a)\allowbreak{},\allowbreak{} cujos parênteses indicam que há um vazio enquadrado em uma imagem.\allowbreak{}\par{}Enquadrado o vazio,\allowbreak{} posso encontrar a imagem como minha,\allowbreak{} ela se torna narcísica,\allowbreak{} pois agora que a vejo através do Outro (\allowbreak{}na analogia com o espelho plano)\allowbreak{} ela é apenas virtual.\allowbreak{} Através do espelho plano,\allowbreak{} o eu se (\allowbreak{}re)\allowbreak{}conhece como uma imagem alienada; em vez de autonomia,\allowbreak{} há alienação,\allowbreak{} ou seja,\allowbreak{} o sujeito se reconhece através de uma imagem que ele não é,\allowbreak{} e onde não está (\allowbreak{}\textsuperscript{D’Agord et al.\allowbreak{},\allowbreak{} 2013},\allowbreak{} p.\allowbreak{} 483)\allowbreak{}.\allowbreak{}\par{}A seguir apresentamos um quadro resumo para visualizar a relação entre o experimento óptico da física e o modelo óptico elaborado por Lacan:\allowbreak{}\par{}\end{multicols}
\ctable[
  width=\textwidth, pos = ht, left, long
]
{p{0.29\textwidth}p{0.27\textwidth}p{0.41\textwidth}}
{}
{ \\\hline
\multicolumn{1}{C{0.29\textwidth}}{Noções de Óptica} & \multicolumn{1}{C{0.27\textwidth}}{Experimento Óptico} & \multicolumn{1}{C{0.41\textwidth}}{Modelo Óptico:\allowbreak{} a teorização} \\\hline \multicolumn{1}{L{0.29\textwidth}}{O objeto real ocultado} & \multicolumn{1}{L{0.27\textwidth}}{Buquê de flores invertido e oculto sob uma caixa} & \multicolumn{1}{L{0.41\textwidth}}{A origem da projeção:\allowbreak{} o eu não se conhece.\allowbreak{} O buquê de flores invertido:\allowbreak{} i(\allowbreak{}a)\allowbreak{}:\allowbreak{} a origem da projeção desconhecida do eu.\allowbreak{}} \\\hline \multicolumn{1}{L{0.29\textwidth}}{Imagem Real:\allowbreak{} aquela que aparece fora do espelho.\allowbreak{} É uma imagem que pode ser tomada por objeto.\allowbreak{}} & \multicolumn{1}{L{0.27\textwidth}}{O buquê de flores aparece ilusoriamente inserido no gargalo do vaso.\allowbreak{}} & \multicolumn{1}{L{0.41\textwidth}}{Corresponde ao que causa o desejo,\allowbreak{} pois é a origem da imagem ilusória que aparece no espelho plano.\allowbreak{}} \\\hline \multicolumn{1}{L{0.29\textwidth}}{Espelho Plano} & \multicolumn{1}{L{0.27\textwidth}}{A introdução,\allowbreak{} por Lacan,\allowbreak{} de uma modificação no experimento de Bouasse:\allowbreak{} o Espelho plano.\allowbreak{}} & \multicolumn{1}{L{0.41\textwidth}}{O Outro é comparado ao espelho plano.\allowbreak{} No estádio do espelho,\allowbreak{} o Outro era representado pelas palavras do adulto que segurava o bebê no colo na frente de um espelho.\allowbreak{}} \\\hline \multicolumn{1}{L{0.29\textwidth}}{Imagem Virtual:\allowbreak{} Imagem puramente subjetiva.\allowbreak{}} & \multicolumn{1}{L{0.27\textwidth}}{O vaso invertido é refletido pelo espelho côncavo e visto em posição normal no espelho plano.\allowbreak{}} & \multicolumn{1}{L{0.41\textwidth}}{A função de criar imagem i’(\allowbreak{}a)\allowbreak{} da qual me aproprio e onde me reconheço,\allowbreak{} a imagem investida narcisicamente:\allowbreak{} Imaginário.\allowbreak{}} \\\hline 
}
\begin{multicols}{2}


\scieloSectionContainer{O espelho na Literatura}
\par{}As obras homônimas,\allowbreak{} “O espelho” (\allowbreak{}1882)\allowbreak{} de Machado de Assis (\allowbreak{}1839-\allowbreak{}1908)\allowbreak{} e “O espelho” (\allowbreak{}1962)\allowbreak{} de Guimarães Rosa (\allowbreak{}1908-\allowbreak{}1967)\allowbreak{} abordam o tema “espelho” como metáfora para a constituição da imagem de si a partir do olhar do outro.\allowbreak{} Ambos os contos trabalham a questão do esvaziamento da própria imagem quando essa alteridade existencial desaparece (\allowbreak{}em Machado de Assis)\allowbreak{} ou quando é extraída com a finalidade de controle em uma experiência científica (\allowbreak{}em Guimarães Rosa)\allowbreak{}.\allowbreak{}\par{}No conto de Machado de Assis,\allowbreak{} o espelho é uma figura para a demonstração da teoria da alma:\allowbreak{} uma metáfora apresentada pelo personagem-\allowbreak{}narrador Jacobina.\allowbreak{} Portanto,\allowbreak{} o espelho faz parte da ficção dentro da ficção.\allowbreak{} No conto de Guimarães Rosa,\allowbreak{} o narrador utiliza o espelho em uma investigação em busca de si.\allowbreak{} Entretanto,\allowbreak{} o foco em si,\allowbreak{} como objeto,\allowbreak{} impede a consideração ao subjetivo.\allowbreak{}\par{}Temos,\allowbreak{} portanto,\allowbreak{} dois narradores,\allowbreak{} um,\allowbreak{} o Jacobina de Machado de Assis,\allowbreak{} que,\allowbreak{} por valorizar o que viveu,\allowbreak{} criou uma teoria para explicar um acontecimento psicológico:\allowbreak{} aos 25 anos ele vivera uma despersonalização ao não poder captar momentaneamente a sua própria imagem no espelho:\allowbreak{} sensação de que a sua imagem no espelho havia se tornado vaga por decomposição dos contornos.\allowbreak{} Já o narrador de Guimarães Rosa relata um experimento de investigação sobre si,\allowbreak{} mas sem relacionar as suas vivências pessoais com sua atividade científica de investigação.\allowbreak{} Assim,\allowbreak{} o narrador relata que “estava amando”,\allowbreak{} mas não reconhece nesse sentimento as condições para acessar a alma.\allowbreak{} Eis a função crítica sob a forma da ironia.\allowbreak{}
\scieloSectionContainer{A dissimetria entre a alma exterior e a alma interior}
\par{}“Cada criatura humana traz duas almas consigo:\allowbreak{} uma que olha de dentro para fora,\allowbreak{} outra que olha de fora para dentro” (\allowbreak{}\textsuperscript{Machado de Assis,\allowbreak{} 1882\fshyp{}1998},\allowbreak{} p.\allowbreak{} 28)\allowbreak{}.\allowbreak{} Onde se encontrariam essas duas almas que caminham em sentidos contrários? Seria na superfície plana de um espelho? A seguir sintetizamos a teoria da alma apresentada pelo personagem Jacobina:\allowbreak{} a alma que olha de dentro para fora é a primeira.\allowbreak{} A alma exterior é a que olha de fora para dentro.\allowbreak{} Pode ser um espírito,\allowbreak{} um fluido,\allowbreak{} um homem,\allowbreak{} muitos homens,\allowbreak{} um objeto.\allowbreak{} Muda de natureza e de estado.\allowbreak{} O narrador Jacobina agrega seu testemunho como recurso retórico para convencer seu público de sua teoria:\allowbreak{} “Eu mesmo tenho experimentado dessas trocas” (\allowbreak{}p.\allowbreak{} 29)\allowbreak{}.\allowbreak{} “O alferes eliminou o homem.\allowbreak{} Durante alguns dias as duas naturezas equilibraram-\allowbreak{}se; mas não tardou que a primitiva cedesse à outra” (\allowbreak{}p.\allowbreak{} 32)\allowbreak{}.\allowbreak{}\par{}O testemunho de Jacobina seria um apólogo que ensinaria a diferenciar a essência do ilusório? No entanto,\allowbreak{} o tema da alma dividida não poderia ser considerado como universal? O que nos chama a atenção é a seguinte observação de Jacobina:\allowbreak{} “E casos há,\allowbreak{} não raros,\allowbreak{} em que a perda da alma exterior implica a da existência inteira.\allowbreak{}” Em relação ao conceito de constituição do sujeito,\allowbreak{} enquanto contribuição lacaniana à teoria psicanalítica,\allowbreak{} podemos trabalhar essa perda sob dois aspectos.\allowbreak{}\par{}Em primeiro lugar,\allowbreak{} a constituição subjetiva como dependente da relação ao Outro como tesouro dos significantes.\allowbreak{} Assim,\allowbreak{} o sujeito convocado ao lugar do Outro (\allowbreak{}o espelho plano)\allowbreak{} pode se reconhecer,\allowbreak{} isto é,\allowbreak{} ver a si mesmo aparecer no campo do Outro,\allowbreak{} mas essa relação é dissimétrica e não recíproca.\allowbreak{} Se o sujeito encontra no Outro a reciprocidade e a simetria,\allowbreak{} está tomando o Outro pelo outro.\allowbreak{} Nesses casos,\allowbreak{} a fala do outro pode tomar a forma de uma injunção,\allowbreak{} isto é,\allowbreak{} uma convocação supergoica em vez de uma convocação simbólica.\allowbreak{}\par{}Em segundo lugar,\allowbreak{} o tema da perda é trabalhado por \textsuperscript{Lacan (\allowbreak{}1964\fshyp{}1985)\allowbreak{}} no contexto de uma alienação constitutiva comparada à operação de disjunção inclusiva,\allowbreak{} uma forma de operação lógica que se sustenta na forma lógica da reunião (\allowbreak{}união)\allowbreak{}.\allowbreak{} Ou seja,\allowbreak{} um operador lógico chamado de “ou inclusivo” (\allowbreak{}vel)\allowbreak{} que se comporta como um operador de união.\allowbreak{} O dilema:\allowbreak{} “A bolsa ou a vida” foi utilizado como ilustração.\allowbreak{} Nesse dilema,\allowbreak{} há um elemento que,\allowbreak{} se escolhido,\allowbreak{} trará por consequência um “\textit{nem um,\allowbreak{} nem outro}” (\allowbreak{}p.\allowbreak{} 200)\allowbreak{}.\allowbreak{} Tal elemento seria a bolsa,\allowbreak{} pois se escolhida,\allowbreak{} perde-\allowbreak{}se a vida e,\allowbreak{} logo,\allowbreak{} também a bolsa.\allowbreak{} Escolhendo-\allowbreak{}se a vida,\allowbreak{} perde-\allowbreak{}se a bolsa e fica-\allowbreak{}se com uma vida privada da bolsa.\allowbreak{} Assim,\allowbreak{} a escolha é resumida a manter ou não a vida.\allowbreak{}\par{}Comparamos a alma exterior,\allowbreak{} do conto de Machado de Assis,\allowbreak{} à vida,\allowbreak{} do dilema acima.\allowbreak{} Assim,\allowbreak{} aqueles casos nos quais a perda da alma exterior implica a perda da existência inteira,\allowbreak{} ou seja,\allowbreak{} sem a alma exterior não haveria existência.\allowbreak{} Sem o Outro,\allowbreak{} sem o espelho plano,\allowbreak{} que vida haveria? Existir é ser reconhecido a partir da mediação simbólica.\allowbreak{} Esta é a alienação constitutiva do sujeito:\allowbreak{} a ideia de que a imagem do eu se forma a partir da imagem do outro,\allowbreak{} ou seja,\allowbreak{} da antecipação da totalidade,\allowbreak{} a imagem sempre estará à frente do próprio sujeito — eu sei que não estou ali,\allowbreak{} mas essa imagem me representa.\allowbreak{}
\scieloSectionContainer{A ciência sem consciência}
\par{}Em \textit{O espelho} de \textsuperscript{Guimarães Rosa (\allowbreak{}1962\fshyp{}2005)\allowbreak{}},\allowbreak{} o narrador dirige-\allowbreak{}se ao leitor para dar o testemunho de uma experiência científica de busca de si mesmo.\allowbreak{} Desse relato,\allowbreak{} o leitor capta a ironia do autor em relação à neutralidade científica.\allowbreak{} Destacamos a seguir três momentos:\allowbreak{}\par{}Primeiro,\allowbreak{} o narrador situa dois campos antagônicos:\allowbreak{} “Saiba que eu perseguia uma realidade experimental,\allowbreak{} não uma hipótese imaginária” (\allowbreak{}p.\allowbreak{} 117)\allowbreak{}.\allowbreak{} Há um jogo de inversões com os pares significantes,\allowbreak{} pois o método científico usa “a hipótese experimental” como uma ficção,\allowbreak{} como componente central de um modelo.\allowbreak{} Uma “realidade experimental” seria um paradoxo,\allowbreak{} pois a pesquisa,\allowbreak{} o experimento,\allowbreak{} é meio e não finalidade.\allowbreak{} O sujeito buscaria conhecer a si mesmo como se o conhecimento de si fosse uma realidade e não uma ficção.\allowbreak{} Como se pudesse experimentar o si mesmo e esse experimento não fosse imaginário.\allowbreak{} Logo,\allowbreak{} o si mesmo é um outro.\allowbreak{} Na ficção experimental,\allowbreak{} o experimento é um “modelo” da situação “real”,\allowbreak{} mas não é realidade.\allowbreak{}\par{}No segundo momento,\allowbreak{} o narrador relata que prosseguiu seu experimento eliminando tudo o que fosse contingente,\allowbreak{} ilusório,\allowbreak{} imaginário,\allowbreak{} hereditário,\allowbreak{} as paixões manifestas ou latentes.\allowbreak{} Até que um dia:\allowbreak{} “simplesmente lhe digo que me olhei no espelho e não me vi” (\allowbreak{}p.\allowbreak{} 118)\allowbreak{}.\allowbreak{} Quando descarta todo o ilusório,\allowbreak{} chega a um “transparente contemplador”,\allowbreak{} mas a transparência não se deixa ver,\allowbreak{} por isso,\allowbreak{} olha sem ver.\allowbreak{} Ao tentar acessar um deles,\allowbreak{} se perde o outro.\allowbreak{} Ou transparência ou imaginário.\allowbreak{} \textsuperscript{Magno (\allowbreak{}1985)\allowbreak{}} compara a objetividade não especularizável,\allowbreak{} por isso transparente,\allowbreak{} ao conceito de \textit{caput mortuum}:\allowbreak{} que era como os antigos nomeavam o resíduo inútil que sobrava de um experimento alquímico.\allowbreak{}\par{}Em um terceiro momento,\allowbreak{} o narrador apresenta um acontecimento posterior ao experimento,\allowbreak{} um só-\allowbreak{}depois (\allowbreak{}\textit{après-\allowbreak{}coup})\allowbreak{}.\allowbreak{} “Mais tarde,\allowbreak{} anos,\allowbreak{} (\allowbreak{}.\allowbreak{}.\allowbreak{}.\allowbreak{})\allowbreak{} o espelho mostrou-\allowbreak{}-\allowbreak{}me.\allowbreak{}.\allowbreak{}.\allowbreak{} perdoe-\allowbreak{}me o detalhe,\allowbreak{} eu já amava”.\allowbreak{} Quando o narrador traz uma lembrança pessoal,\allowbreak{} tratada como algo fortuito.\allowbreak{} Um detalhe,\allowbreak{} o “eu já amava” é referido sem que seja estabelecida relação com o experimento.\allowbreak{} Para o narrador,\allowbreak{} trata-\allowbreak{}se de uma separação entre os estados da alma e a atividade experimental.\allowbreak{} Contudo,\allowbreak{} o autor,\allowbreak{} Guimarães Rosa,\allowbreak{} convida o leitor a pensar que sim,\allowbreak{} há uma relação,\allowbreak{} o amor produz efeitos no experimento,\allowbreak{} por mais que o cientista não admita.\allowbreak{} Guimarães Rosa estaria questionando a ciência sem consciência? Referimo-\allowbreak{}nos ao aforismo “ciência sem consciência é a ruína da alma” (\allowbreak{}de Rabelais)\allowbreak{} citado por Lacan (\allowbreak{}1973-\allowbreak{}1974)\allowbreak{} para destacar que a própria ciência é sem consciência.\allowbreak{}
\scieloSectionContainer{Não fazemos ciência sem ficções}
\par{}O esvaziamento da imagem de si pela opacificação do espelho em um experimento científico em Guimarães Rosa,\allowbreak{} ou pela falta de uma fala no outro em Machado de Assis,\allowbreak{} encontra-\allowbreak{}se com a leitura psicanalítica da relação do sujeito com sua imagem.\allowbreak{} A alma seria como um impossível de ser conhecido,\allowbreak{} pois quando estamos na posição de olhar,\allowbreak{} não somos vistos e,\allowbreak{} quando somos vistos,\allowbreak{} não somos quem olha.\allowbreak{} O que é chamado de alma enquanto nada (\allowbreak{}o que o espelho não reflete em Guimarães Rosa)\allowbreak{},\allowbreak{} ausência de reflexo pela falta da alma exterior (\allowbreak{}em Machado de Assis)\allowbreak{},\allowbreak{} poderia ser então comparado ao objeto “a”? O que equivaleria,\allowbreak{} como articula \textsuperscript{Abreu (\allowbreak{}2009)\allowbreak{}},\allowbreak{} “ao encontro com o estranho em cada um de nós” (\allowbreak{}p.\allowbreak{} 169)\allowbreak{}.\allowbreak{}\par{}Propomos,\allowbreak{} então,\allowbreak{} uma comparação entre esse estranho e a imagem real no modelo óptico de Lacan.\allowbreak{} Uma,\allowbreak{} a imagem real,\allowbreak{} provém de um modelo tomado de empréstimo à ciência,\allowbreak{} outro,\allowbreak{} o sentimento do estranho,\allowbreak{} é metáfora,\allowbreak{} e ambos convergem para a conceituação de objeto “a”.\allowbreak{} Para conceber um conceito que não tem um correspondente empírico,\allowbreak{} a ilusão óptica e a metáfora literária são modelos fundamentais.\allowbreak{}\par{}A ciência não é isenta da ficção e a ficção não é isenta da ciência.\allowbreak{} Como vimos com o modelo óptico de Lacan,\allowbreak{} um modelo é uma ficção produtiva para a teoria psicanalítica.\allowbreak{} Assim como os modelos são ficções científicas,\allowbreak{} as metáforas científicas podem ser produtivas na literatura,\allowbreak{} como atesta a nossa leitura do conto de Guimarães Rosa.\allowbreak{}

\medskip\par\noindent
{}\textbf{Conflito de interesses\fshyp{}Conflict of interest}:\allowbreak{} Os autores declaram que não há conflito de interesses \fshyp{} The authors have no conflict of interest to declare.\allowbreak{}
\medskip\par\noindent
\footnotesize{\textit{Recebido em}: 24/1/2014.} \par \noindent
\footnotesize{\textit{Aceito em}: 19/3/2014.} \begin{scieloReferencesContainer}[Referências]
\scieloReferencesItem{Abreu,}\allowbreak{} D.\allowbreak{} N.\allowbreak{} (\allowbreak{}2009)\allowbreak{}.\allowbreak{} Os destinos dos reflexos:\allowbreak{} do sintoma ao \textit{sinthome},\allowbreak{} de Machado a Guimarães.\allowbreak{} In M.\allowbreak{}M.\allowbreak{} de Lima,\allowbreak{} \&\allowbreak{\allowbreak{}\allowbreak{}}\allowbreak{} M.\allowbreak{}A.\allowbreak{}C.\allowbreak{} Jorge (\allowbreak{}Orgs.\allowbreak{})\allowbreak{},\allowbreak{} \textit{Saber fazer com o Real:\allowbreak{} diálogos entre psicanálise e arte} (\allowbreak{}pp.\allowbreak{} 163-\allowbreak{}174)\allowbreak{}.\allowbreak{} Rio de Janeiro:\allowbreak{} Cia.\allowbreak{} de Freud.\allowbreak{}
\scieloReferencesItem{Bion,}\allowbreak{} W.\allowbreak{}R.\allowbreak{} (\allowbreak{}1990)\allowbreak{}.\allowbreak{} \textit{Bion’s Brazilian Lectures}.\allowbreak{} London:\allowbreak{} Karnac.\allowbreak{} (\allowbreak{}Trabalho original publicado em 1974)\allowbreak{}.\allowbreak{}
\scieloReferencesItem{Bunge,}\allowbreak{} M.\allowbreak{} (\allowbreak{}1975)\allowbreak{}.\allowbreak{} \textit{La ciencia,\allowbreak{} su método y su filosofía}.\allowbreak{} Buenos Aires:\allowbreak{} Ediciones Siglo Veinte.\allowbreak{}
\scieloReferencesItem{D’Agord,}\allowbreak{} M.\allowbreak{}R.\allowbreak{} de L.\allowbreak{}; Barbosa,\allowbreak{} M.\allowbreak{}R.\allowbreak{}O.\allowbreak{}; Hasan,\allowbreak{} R.\allowbreak{},\allowbreak{} \&\allowbreak{\allowbreak{}\allowbreak{}}\allowbreak{} Neves,\allowbreak{} R.\allowbreak{}C.\allowbreak{} (\allowbreak{}2013,\allowbreak{} set.\allowbreak{})\allowbreak{}.\allowbreak{} O duplo como fenômeno psíquico.\allowbreak{} \textit{Rev.\allowbreak{} Latinoam.\allowbreak{} Psicopat.\allowbreak{} Fund}.\allowbreak{},\allowbreak{} São Paulo,\allowbreak{} \textit{16}(\allowbreak{}3)\allowbreak{},\allowbreak{} 475-\allowbreak{}488.\allowbreak{}
\scieloReferencesItem{Eidelsztein,}\allowbreak{} A.\allowbreak{} (\allowbreak{}1992)\allowbreak{}.\allowbreak{} \textit{Modelos,\allowbreak{} esquemas y grafos en la enseñanza de Lacan}.\allowbreak{} Buenos Aires:\allowbreak{} Manantial.\allowbreak{}
\scieloReferencesItem{Fédida,}\allowbreak{} P.\allowbreak{} (\allowbreak{}1998,\allowbreak{} set.\allowbreak{})\allowbreak{}.\allowbreak{} De uma psicopatologia geral a uma psicopatologia fundamental.\allowbreak{} Nota sobre a noção de paradigma.\allowbreak{} \textit{Revista Latinoamericana de Psicopatologia Fundamental},\allowbreak{} \textit{I}(\allowbreak{}3)\allowbreak{},\allowbreak{} 107-\allowbreak{}121.\allowbreak{}
\scieloReferencesItem{Fédida,}\allowbreak{} P.\allowbreak{},\allowbreak{} \&\allowbreak{\allowbreak{}\allowbreak{}}\allowbreak{} Widlöcher,\allowbreak{} D.\allowbreak{} (\allowbreak{}1990)\allowbreak{}.\allowbreak{} \textit{Présentation}.\allowbreak{} \textit{Revue Internationale de Psychopathologie},\allowbreak{} \textit{1}(\allowbreak{}1)\allowbreak{},\allowbreak{} 3-\allowbreak{}4.\allowbreak{}
\scieloReferencesItem{Freud,}\allowbreak{} S.\allowbreak{} (\allowbreak{}2010)\allowbreak{}.\allowbreak{} O inquietante.\allowbreak{} In \textit{Obras Completas} (\allowbreak{}Vol.\allowbreak{} 14,\allowbreak{} pp.\allowbreak{} 328-\allowbreak{}376)\allowbreak{}.\allowbreak{} São Paulo:\allowbreak{} Companhia das Letras.\allowbreak{} (\allowbreak{}Trabalho original publicado em 1919)\allowbreak{}.\allowbreak{}
\scieloReferencesItem{Freud,}\allowbreak{} S.\allowbreak{} (\allowbreak{}2012)\allowbreak{}.\allowbreak{} \textit{A interpretação dos sonhos}.\allowbreak{} Vol.\allowbreak{} II.\allowbreak{} Porto Alegre:\allowbreak{} L\&\allowbreak{\allowbreak{}\allowbreak{}}\allowbreak{}PM.\allowbreak{} (\allowbreak{}Trabalho original publicado em 1900)\allowbreak{}.\allowbreak{}
\scieloReferencesItem{Granger,}\allowbreak{} G.\allowbreak{}G.\allowbreak{} (\allowbreak{}1994)\allowbreak{}.\allowbreak{} \textit{A ciência e as ciências}.\allowbreak{} São Paulo:\allowbreak{} Editora UNESP.\allowbreak{}
\scieloReferencesItem{Guimarães Rosa,}\allowbreak{} J.\allowbreak{} (\allowbreak{}2005)\allowbreak{}.\allowbreak{} O espelho.\allowbreak{} In \textit{Primeiras estórias }(\allowbreak{}pp.\allowbreak{} 113-\allowbreak{}120)\allowbreak{}.\allowbreak{} Rio de Janeiro:\allowbreak{} Nova Fronteira.\allowbreak{} (\allowbreak{}Trabalho original publicado em 1962)\allowbreak{}.\allowbreak{}
\scieloReferencesItem{Lacan,}\allowbreak{} J.\allowbreak{} (\allowbreak{}1979)\allowbreak{}.\allowbreak{} \textit{O seminário.\allowbreak{} Livro 1.\allowbreak{} Os escritos técnicos de Freud}.\allowbreak{} Rio de Janeiro:\allowbreak{} Jorge Zahar.\allowbreak{} (\allowbreak{}Trabalho original publicado em 1953-\allowbreak{}1954)\allowbreak{}.\allowbreak{}
\scieloReferencesItem{Lacan,}\allowbreak{} J.\allowbreak{} (\allowbreak{}1985)\allowbreak{}.\allowbreak{} \textit{O seminário.\allowbreak{} Livro 11.\allowbreak{} Os quatro conceitos fundamentais da psicanálise.\allowbreak{}} Rio de Janeiro:\allowbreak{} Jorge Zahar.\allowbreak{} (\allowbreak{}Trabalho original publicado em 1964)\allowbreak{}.\allowbreak{}
\scieloReferencesItem{Lacan,}\allowbreak{} J.\allowbreak{} (\allowbreak{}1998a)\allowbreak{}.\allowbreak{} O estádio do espelho como formador da função do eu.\allowbreak{} In \textit{Escritos} (\allowbreak{}pp.\allowbreak{} 96-\allowbreak{}103)\allowbreak{}.\allowbreak{} Rio de Janeiro:\allowbreak{} Jorge Zahar.\allowbreak{} (\allowbreak{}Trabalho original publicado em [1949]1966)\allowbreak{}.\allowbreak{}
\scieloReferencesItem{Lacan,}\allowbreak{} J.\allowbreak{} (\allowbreak{}1998b)\allowbreak{}.\allowbreak{} Observação sobre o relatório de Daniel Lagache:\allowbreak{} psicanálise e estrutura da personalidade.\allowbreak{} In \textit{Escritos}(\allowbreak{}pp.\allowbreak{} 653-\allowbreak{}691)\allowbreak{}.\allowbreak{} Rio de Janeiro:\allowbreak{} Jorge Zahar.\allowbreak{} (\allowbreak{}Trabalho original publicado em [1961]1966)\allowbreak{}.\allowbreak{}
\scieloReferencesItem{Lacan,}\allowbreak{} J.\allowbreak{} (\allowbreak{}2005)\allowbreak{}.\allowbreak{} \textit{O seminário.\allowbreak{} Livro 10.\allowbreak{} A angústia}.\allowbreak{} Rio de Janeiro:\allowbreak{} Jorge Zahar.\allowbreak{} (\allowbreak{}Trabalho original publicado em 1962-\allowbreak{}1963)\allowbreak{}.\allowbreak{}
\scieloReferencesItem{Lacan,}\allowbreak{} J.\allowbreak{} (\allowbreak{}s\fshyp{}d.\allowbreak{})\allowbreak{}.\allowbreak{} \textit{Le séminaire.\allowbreak{} Livre 1.\allowbreak{} Les }\textit{écrits}\textit{ técniques de Freud – 1953-\allowbreak{}1954.\allowbreak{}} Recuperado em 20\fshyp{}11\fshyp{}2013 de <www.\allowbreak{}staferla.\allowbreak{}free.\allowbreak{}fr>.\allowbreak{}
\scieloReferencesItem{Lacan,}\allowbreak{} J.\allowbreak{} (\allowbreak{}s\fshyp{}d.\allowbreak{})\allowbreak{}.\allowbreak{} \textit{Le}\textit{séminaire.\allowbreak{} Livre 10.\allowbreak{}}\textit{L’Angoisse} –\textit{ 1962-\allowbreak{}1963}.\allowbreak{} Recuperado em 20\fshyp{}11\fshyp{}2013 de <www.\allowbreak{}staferla.\allowbreak{}free.\allowbreak{}fr>.\allowbreak{}
\scieloReferencesItem{Lacan,}\allowbreak{} J.\allowbreak{} (\allowbreak{}s\fshyp{}d.\allowbreak{})\allowbreak{}.\allowbreak{} \textit{Le séminaire.\allowbreak{} Livre 21.\allowbreak{} Les non dupes errent – 1973-\allowbreak{}1974}.\allowbreak{} Recuperado em 20\fshyp{}11\fshyp{}2013 de <www.\allowbreak{}staferla.\allowbreak{}free.\allowbreak{}fr>.\allowbreak{}
\scieloReferencesItem{Lévi-\allowbreak{}Strauss,}\allowbreak{} C.\allowbreak{} (\allowbreak{}2005)\allowbreak{}.\allowbreak{} \textit{De perto e de longe}.\allowbreak{} São Paulo:\allowbreak{} Cosac Naify.\allowbreak{}
\scieloReferencesItem{Lévi-\allowbreak{}Strauss,}\allowbreak{} C.\allowbreak{} (\allowbreak{}2009)\allowbreak{}.\allowbreak{} \textit{As estruturas elementares do parentesco} (\allowbreak{}5\textsuperscript{a} ed.\allowbreak{})\allowbreak{}.\allowbreak{} Petrópolis:\allowbreak{} Vozes.\allowbreak{} (\allowbreak{}Trabalho original publicado em 1949)\allowbreak{}.\allowbreak{}
\scieloReferencesItem{Machado de Assis,}\allowbreak{} J.\allowbreak{} M.\allowbreak{} (\allowbreak{}1998)\allowbreak{}.\allowbreak{} O espelho.\allowbreak{} In \textit{Contos}(\allowbreak{}pp.\allowbreak{} 26-\allowbreak{}39)\allowbreak{}.\allowbreak{} Porto Alegre:\allowbreak{} L\&\allowbreak{\allowbreak{}\allowbreak{}}\allowbreak{}PM.\allowbreak{} (\allowbreak{}Trabalho original publicado em 1882)\allowbreak{}.\allowbreak{}
\scieloReferencesItem{Magno,}\allowbreak{} M.\allowbreak{}D.\allowbreak{} (\allowbreak{}1985)\allowbreak{}.\allowbreak{} \textit{Rosa Rosae}.\allowbreak{} Rio de Janeiro:\allowbreak{} Editora Aoutra.\allowbreak{}
\scieloReferencesItem{Marx,}\allowbreak{} M.\allowbreak{} H.\allowbreak{},\allowbreak{} \&\allowbreak{\allowbreak{}\allowbreak{}}\allowbreak{} Hillix,\allowbreak{} W.\allowbreak{} A.\allowbreak{} (\allowbreak{}1978)\allowbreak{}.\allowbreak{} \textit{Sistemas e teorias em psicologia}.\allowbreak{} São Paulo:\allowbreak{} Cultrix.\allowbreak{}
\scieloReferencesItem{Soler,}\allowbreak{} C.\allowbreak{} (\allowbreak{}2012)\allowbreak{}.\allowbreak{} \textit{Declinações da angústia}.\allowbreak{} São Paulo:\allowbreak{} Escuta.\allowbreak{}
\end{scieloReferencesContainer}

\end{multicols}

\scieloLicenseContainer{This is an Open Access article distributed under the terms of the Creative Commons Attribution Non-Commercial License, which permits unrestricted non-commercial use, distribution, and reproduction in any medium, provided the original work is properly cited.}
