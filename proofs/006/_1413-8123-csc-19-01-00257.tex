% Generated by jats2tex@0.11.1.0
\documentclass{article}
\usepackage[T1]{fontenc}
\usepackage[utf8]{inputenc} %% *
\usepackage[portuges,spanish,english,german,italian,russian]{babel} %% *
\usepackage{amstext}
\usepackage{authblk}
\usepackage{unicode-math}
\usepackage{multirow}
\usepackage{graphicx}
\usepackage{etoolbox}
\usepackage{xtab}
\usepackage{enumerate}
\usepackage{hyperref}
\usepackage{penalidades}
\usepackage[footnotesize,bf,hang]{caption}
\usepackage[nodayofweek,level]{datetime}
\setmainfont{Linux Libertine O}
\renewcommand*{\thefootnote}{\alph{footnote}}
\makeatletter
\newcommand{\fn}{\afterassignment\fn@aux\count0=}
\newcommand{\fn@aux}{\csname fn\the\count0\endcsname}
\makeatother

\newcommand{\journalid}{Cien Saude Colet}
\newcommand{\journaltitle}{Ciência \& Saúde Coletiva}
\newcommand{\abbrevjournaltitle}{Ciênc. saúde coletiva}
\newcommand{\issnppub}{1413-8123}
\newcommand{\issnepub}{1678-4561}
\newcommand{\publishername}{\textsc{abrasco} - Associação Brasileira de Saúde Coletiva}
\newcommand\publisherid{S1413-81232014000100026}
\newcommand\articleid{\textsc{doi} 10.1590/1413-81232014191.2007}
\def\subject{Article}
\title{Fatores que influenciam a adoção de ferramentas de \textsc{tic} nos experimentos
de bioinformática de organizações biofarmacêuticas\titlegroup{}}

\newcommand{\subtitlestyle}[1]{-- \emph{#1}\medskip}
\newcommand{\transtitlestyle}[1]{\par\medskip\Large #1}
\newcommand{\transsubtitlestyle}[1]{-- \Large\emph{ #1}}

\newcommand{\titlegroup}{
	\ifdef{\subtitle}{\subtitlestyle{\subtitle}}{}
	\ifdef{\transtitle}{\transtitlestyle{\transtitle}}{}
	\ifdef{\transsubtitle}{\transsubtitlestyle{\transsubtitle}}{}}

\newcommand{\subtitle}{um estudo de caso no Instituto Nacional do Câncer}
%\newcommand{\transtitle}{Factors affecting the adoption of \textsc{ict} tools in experiments with bioinformatics in biopharmaceutical organizations}
%\newcommand{\transsubtitle}{a case study in the Brazilian Cancer Institute}
\author{Pitassi, Claudio}
\author{Gonçalves, Antonio Augusto}
\author{Moreno, Valter de Assis}
\affil{Universidade Estácio de Sá}
\affil{Instituto Nacional do Câncer}
\affil{Faculdades Ibmec}
\date{ 01 2014}
\def\volume{19}
\def\issue{01}
\def\fpage{257}
\def\lpage{268}
\def\permissions{All the contents of this journal, except where otherwise noted,
is licensed under a Creative Commons Attribution License}
\newcommand{\kwdgroup}{\textsc{tic}, Inovação tecnológica, Biologia molecular,
Bioinformática, Biofarmacêutica}
\newcommand{\kwdgroupen}{\textsc{ict}, Technological innovation, Molecular biology,
Bioinformatics, Biopharmaceuticals}

\begin{document}
\selectlanguage{portuges}
\section*{Metadados não aplicados}
\begin{itemize}
\item[\textbf{língua do artigo}]{Português}
\ifdef{\journalid}{\item[\textbf{journalid}] \journalid}{}
\ifdef{\journaltitle}{\item[\textbf{journaltitle}] \journaltitle}{}

\ifdef{\journalsubtitle}{\item[\textbf{journalsubtitle}] \journaltitle}{}
\ifdef{\transjournaltitle}{\item[\textbf{journaltitle}] \journaltitle}{}
\ifdef{\transjournalsubtitle}{\item[\textbf{journalsubtitle}] \journaltitle}{}

\ifdef{\abbrevjournaltitle}{\item[\textbf{abbrevjournaltitle}]
\abbrevjournaltitle}{}
\ifdef{\issnppub}{\item[\textbf{issnppub}] \issnppub}{}
\ifdef{\issnepub}{\item[\textbf{issnepub}] \issnepub}{}
\ifdef{\publishername}{\item[\textbf{publishername}] \publishername}{}
\ifdef{\publisherid}{\item[\textbf{publisherid}] \publisherid}{}
\ifdef{\subject}{\item[\textbf{subject}] \subject}{}
\ifdef{\transtitle}{\item[\textbf{transtitle}] \transtitle}{}
\ifdef{\authornotes}{\item[\textbf{authornotes}] \authornotes}{}
\ifdef{\articleid}{\item[\textbf{articleid}] \articleid}{}
\ifdef{\volume}{\item[\textbf{volume}] \volume}{}
\ifdef{\issue}{\item[\textbf{issue}] \issue}{}
\ifdef{\fpage}{\item[\textbf{fpage}] \fpage}{}
\ifdef{\lpage}{\item[\textbf{lpage}] \lpage}{}
\ifdef{\permissions}{\item[\textbf{permissions}] \permissions}{}
\end{itemize}
\maketitle

\begingroup

\begin{abstract}

O objetivo deste artigo é identificar e analisar os fatores que influenciaram a
adoção de ferramentas de Tecnologias de Informação e de Comunicação (\textsc{tic}) nos
experimentos de Bioinformática do Instituto Nacional do Câncer (Inca). Trata-se
de um estudo de campo único descritivo e exploratório, dentro da tradição
qualitativa. As evidências foram coletadas principalmente em entrevistas de
fundo com os gestores de áreas da Coordenação Geral Técnico-Científica e da
Divisão de Tecnologia da Informação do Inca. As respostas foram tratadas pelo
método de análise de conteúdo do tipo categorial. As categorias de análise foram
definidas a partir da revisão da literatura e consolidadas nos sete fatores do
Modelo Tecnologia-Organização-Ambiente (\textsc{toe}) adaptado para este estudo. O modelo
proposto permitiu demonstrar como atuam no caso do Inca os fatores que impactam
a adoção das complexas \textsc{tic} usadas nos experimentos de Bioinformática,
contribuindo para investigações em duas áreas de importância crescente para o
Complexo Econômico-Industrial de Saúde brasileiro: a inovação tecnológica e a
Biotecnologia. Com base nas evidências coletadas, uma questão é formulada: em
que medida o alinhamento dos fatores pertinentes à adoção das \textsc{tic} nos
experimentos de Bioinformática pode aumentar a capacidade de inovar de uma
organização biofarmacêutica brasileira?

\iflanguage{portuges}{\medskip\noindent\textbf{Palavras-chave:} \kwdgroup}{}
\iflanguage{english}{\medskip\noindent\textbf{Keywords:} \kwdgroupen}{}
\iflanguage{spanish}{\medskip\noindent\textbf{Palavras claves:} \kwdgroupes}{}
\iflanguage{french}{\medskip\noindent\textbf{Mots clés:} \kwdgroupfr}{}
\end{abstract}
\endgroup

\begingroup
\renewcommand{\section}[1]{\subsection*{#1}}
\begin{otherlanguage}{english}

\begin{abstract}

The scope of this article is to identify and analyze the factors that influence
the adoption of \textsc{ict} tools in experiments with bioinformatics at the Brazilian
Cancer Institute (\textsc{inca}). It involves a descriptive and exploratory qualitative
field study. Evidence was collected mainly based on in-depth interviews with the
management team at the Research Center and the \textsc{it} Division. The answers were
analyzed using the categorical content method. The categories were selected from
the scientific literature and consolidated in the
Technology-Organization-Environment (\textsc{toe}) framework created for this study. The
model proposed made it possible to demonstrate how the factors selected impacted
\textsc{inca}´s adoption of bioinformatics systems and tools, contributing to the
investigation of two critical areas for the development of the health industry
in Brazil, namely technological innovation and bioinformatics. Based on the
evidence collected, a research question was posed: to what extent can the
alignment of the factors related to the adoption of \textsc{ict} tools in experiments
with bioinformatics increase the innovation capacity of a Brazilian
biopharmaceutical organization?

\ifdef{\kwdgroupen}{\medskip\noindent\textbf{Keywords:} \kwdgroupen}{}
\end{abstract}
\end{otherlanguage}
\endgroup

\section{Introdução}

Os estudos em Genômica, Proteômica e Dinâmica Molecular geraram uma imensa
quantidade de dados relativos a mapas genéticos e estruturas de
proteínas\textsuperscript{1}. A complexidade crescente dos estudos em Biologia fez emergir a disciplina de
Bioinformática, campo de conhecimento que foi se conformando ao longo das
últimas décadas como resultado da interdependência entre a evolução das
Tecnologias de Informação e de Comunicação (\textsc{tic}) e da Biologia
Molecular\textsuperscript{2}.

Grande parte das pesquisas sobre a difusão e o uso das \textsc{tic} tem por foco
aplicativos mais simples e estáveis, tipicamente concebidos para dar suporte aos
processos operacionais básicos nas organizações. No que diz respeito à P\&D em
organizações biofarmacêuticas, a literatura examinada não deixa claro se há
aspectos específicos que influenciam a adoção das complexas \textsc{tic} que se
desenvolveram em apoio à Biologia Molecular. O presente estudo está centrado na
seguinte questão de pesquisa: \textit{como se dá o processo de adoção de \textsc{tic} em
contextos voltados para o desenvolvimento de drogas em organizações
biofarmacêuticas?}
Esta pesquisa amplia o conhecimento científico ao identificar e analisar os
fatores que influenciaram a adoção de \textsc{tic} nos experimentos de Bioinformática do
Instituto Nacional do Câncer (Inca).

\section{Referencial teórico}

Há na literatura diferentes interpretações do conceito de
Bioinformática\textsuperscript{3}. Na sua definição ampla, a Bioinformática envolve a aplicação de \textsc{tic} nas
análises de qualquer área da Biologia\textsuperscript{3}. De forma mais restrita, a Bioinformática é a aplicação de informática aos
experimentos de Biologia Molecular, ou mais especificamente, de
Genômica\textsuperscript{4}. Também se advoga que Bioinformática aplica os princípios da Ciência da
Informação para interpretar dados biológicos, enquanto a Biologia Computacional
aplica os algoritmos matemáticos e computacionais aos experimentos
biológicos\textsuperscript{5}. Como este artigo foca uma organização biofarmacêutica voltada à pesquisa em
câncer, define-se Bioinformática como os experimentos de Genômica, Proteômica e
Dinâmica Molecular que se apoiam em ferramentas de \textsc{tic} para conduzir suas
análises.

As características geograficamente dispersas e cada vez mais
colaborativas\textsuperscript{6}
e abertas\textsuperscript{7}
das pesquisas em ciências da vida alavancaram o potencial de utilização de
modelos de \textit{e.sciences\textsuperscript{8}}. Uma infraestrutura computacional de \textit{grid information technology}
(\textsc{git}) permite o compartilhamento das capacidades de \textsc{ti} de cada parceiro nos
experimentos de Bioinformática, de modo a construir uma estrutura virtual e não
centralizada de P\&D\textsuperscript{9}. Um sistema de gerenciamento de fluxo de trabalho (\textit{workflow}
) científico define a sequência de tarefas e os parâmetros de execução dos
experimentos de Bioinformática entre as áreas internas da P\&D de uma empresa,
assim como entre estas e os parceiros externos, contornando as dificuldades
impostas por informações de baixa qualidade e pela diversidade de plataformas e
conceitos\textsuperscript{8}.

Os softwares usados em Bioinformática envolvem a construção de armazéns de
dados, apoiando-se em web semântica, ontologias e inteligência artificial para
capturar e tratar dados distribuídos por inúmeras fontes\textsuperscript{10-12}. A existência de base dados clínicos permite a aplicação de poderosas técnicas
de mineração em análises epidemiológicas, testes clínicos e Genômica, permitindo
estabelecer associações entre doenças e novos tratamentos para grupo de
pacientes com padrões genéticos específicos\textsuperscript{13}. Experimentos em Genômica, Proteômica e Dinâmica Molecular estão constantemente
produzindo novos dados, correlacionando-os, por meio de análise, aos dados já
existentes, o que requer sofisticadas ferramentas de busca e
\textit{screening\textsuperscript{11}.}

Estudos de Bioinformática requerem a análise de múltiplos sequenciamentos
genéticos, tornando a manipulação manual de dados praticamente
impossível\textsuperscript{14}. O uso no ramo biofarmacêutico de sistemas computadorizados acoplados à
modelagem matemática permitiu a realização de experimentos que processam uma
grande quantidade de dados em curto espaço de tempo\textsuperscript{1}. Ferramentas de \textit{High Throughtput Screening (\textsc{hts})}
são consideradas habilitadoras importantes na descoberta de novas drogas e são
definidas como uma coleção de tecnologias, geralmente envolvendo softwares
matemáticos, banco de dados, robótica, ótica, instrumentos para manipulação de
líquidos e técnicas de visualização\textsuperscript{15}, aplicadas nas fases "hit-to-lead" das pesquisas \textit{in
vitro\textsuperscript{16}}.

Como argumenta Lenoir\textsuperscript{17}, o desenvolvimento das técnicas de simulação e prototipação de estruturas
moleculares permitem a realização virtual de experimentos em Biotecnologia.
Estas \textsc{tic} fazem com que as empresas reduzam os custos com as técnicas
laboratoriais tradicionais, já que compostos provados inócuos "in silico" podem
ser de antemão descartados, o que amplia consideravelmente as oportunidades de
inovação pela melhor utilização dos recursos disponíveis na empresa.

Recorrer às comunidades de código aberto na Web é o comportamento usual nos
experimentos em ramos das ciências da vida\textsuperscript{18}. Estes portais são utilizados para padronização de conceitos, para depósito de
informações sobre sequenciamento genético e para disponibilização de aplicativos
usados em apoio aos experimentos de Genômica, de Proteômica e de Dinâmica
Molecular. A criação de equipes virtuais de P\&D, incluindo parceiros externos,
facilitada por aplicativos baseados na Internet é utilizada por grandes
farmacêuticas globais, tais como Roche e Schering (hoje
Bayer)\textsuperscript{19}. Por fim, as organizações também podem se valer de empresas especializadas na
condução de etapas de P\&D sob contrato, tais como as \textit{Contract Research
Organizations}
(\textsc{cro}s), para acessar capacidades complementares ou acelerar os experimentos.
Estes infomediarios se apóiam fortemente em tecnologias baseadas na \textsc{web} para a
obtenção, em escala global, de conhecimentos voltados à
inovação\textsuperscript{14}.

O Quadro ~\ref{fig:f01}
resume as ferramentas de \textsc{tic} que suportam a troca de informações e conhecimento,
a comunicação, a organização de fluxos de trabalho, além das próprias análises
realizadas pelos pesquisadores nos experimentos de Bioinformática.

Diferentes níveis de análise e tradições teóricas vêm sendo aplicados aos
estudos dos fatores que influenciam a adoção de \textsc{tic} pelas
organizações\textsuperscript{20}. Dado o caráter multifacetado e colaborativo do processo de inovação
tecnológica em Bio logia, o estudo dos fatores que influenciam a adoção de \textsc{tic}
em experimentos de Bioinformática deve ter por base uma abordagem capaz de
integrar diferentes níveis de análise e considerar diferentes dimensões. Tal
requisito é atendido pelo arcabouço proposto por Tornatzky e
Fleischer\textsuperscript{21}, retratado na Figura ~\ref{fig:f02}, denominado de Modelo Tecnologia-Organização-Ambiente (\textsc{toe}).

O ambiente delimita o contexto externo no qual as decisões de P\&D de uma
organização são tomadas. Devem ser consideradas as características do ramo, com
destaque para a intensidade da competição, a estrutura de apoio ao
desenvolvimento tecnológico e o arcabouço jurídico-institucional. O contexto
organizacional investiga como as características da própria organização
influenciam o processo de adoção da tecnologia. Envolve, entre outros aspectos,
o tamanho e o tipo de organização, a posse de recursos, o modelo de gestão e os
mecanismos de governança. A dimensão da tecnologia estuda como característi cas
inerentes às \textsc{tic} podem influenciar a sua adoção pela organização. A tecnologia
envolve produtos, processos e \textsc{si} e considera o grau em que estas tecnologias
estão acessíveis à organização.

\section{Método}

Dada à natureza do que se quer estudar, optou-se por realizar um estudo de campo
descritivo e exploratório, dentro da tradição qualitativa\textsuperscript{22}. Descritivo porque procura apresentar como o Inca está usando as \textsc{tic} em seus
experimentos de Bioinformática. Exploratório porque examina fenômenos
institucionais, organizacionais e tecnológicos subjacentes à P\&D em câncer,
ainda sem contornos definidos. A opção pelo estudo de caso como método
justifica-se pelo fato de os pesquisadores quererem, deliberadamente, considerar
as condições contextuais em que ele está circunscrito\textsuperscript{23}. A escolha do Inca como caso único\textsuperscript{23}
deve-se ao fato da organização ser a melhor capacitada no Brasil para realizar
todas as etapas da pesquisa na classe terapêutica de Oncologia e aos esforços de
incorporação das ferramentas de \textsc{tic} usadas na Bioinformática. A escolha de casos
emblemáticos é um procedimento científico adequado na medida em que permite
entender dinâmicas específicas, mas que podem trazer importantes lições para
outras organizações\textsuperscript{23}.

A Figura ~\ref{fig:f03}
apresenta uma estrutura simplificada do Inca, demarcando em negrito as áreas em
foco neste artigo. As unidades centrais de análise são as ferramentas de \textsc{tic}
usadas nos experimentos de Bioinformática conduzidos por um laboratório da
Coordenação de Pesquisa Clínica e Incorporação Tecnológica. A inclusão de
pesquisadores da Coordenação de Pesquisa Básica e Translacional permitiu
entender melhor a interdependência entre os experimentos ali conduzidos e as
pesquisas clínicas, avaliando o papel da Bioinformática nessa integração. A
inclusão da \textsc{dti} permitiu avaliar como a organização estava lidando com a adoção
das \textsc{tic} aqui analisadas sob a ótica dos gestores dos recursos de \textsc{ti} do Inca.

As evidências foram coletadas por meio da análise documental, das pesquisas em
sítios especializados e, principalmente, da observação direta e das entrevistas
de fundo com os gestores. Os sujeitos da pesquisa alinham-se ao grupo de
pesquisadores que lideravam as transformações em curso nos últimos anos na P\&D
do Inca, não refletindo, portanto, a opinião de todos os pesquisadores da \textsc{cgtc}
do Inca. A Tabela 1 traz informações sobre as entrevistas realizadas na sede do
Inca entre os meses de março a outubro de 2011.

1413-8123-csc-19-01-00257-gt01

As entrevistas foram gravadas e as respostas obtidas foram tratadas pelo método
de análise de conteúdo do tipo temático ou categorial\textsuperscript{34}. As categorias de análise foram consolidadas nos sete fatores do Modelo \textsc{toe}
adaptado para os objetivos desta pesquisa. Essas categorias foram definidas a
partir da triangulação das evidências coletadas da pesquisa bibliográfica, da
análise documental e das entrevistas de campo.

A Figura ~\ref{fig:f04}
apresenta o modelo \textsc{toe} adaptado para o estudo da adoção de ferramentas \textsc{tic}
utilizadas em experimentos de Bioinformática no Inca. Os próximos parágrafos
apresentam e discutem os resultados do estudo de caso.

A dinâmica evolutiva do \textit{ramo biofarmacêutico}
é direcionada pela P\&D\textsuperscript{35}. A pressão permanente para a descoberta de novos princípios ativos e a
velocidade com que as C\&T evoluem geram alta volatilidade no que tange às
terminologias, às bases de dados e aos aplicativos. Como atesta um gestor
\textit{Em biotecnologia você não pode ficar preso a uma rota tecnológica. Se
você não acompanhar o mercado, está fora}. Quanto maior o nível de competição na indústria, maior será, como argumentam
Chau e Tam\textsuperscript{36}, o estímulo para a adoção de \textsc{tic} complexas, ou seja, aquelas que implicam o
redesenho das práticas administrativas e acarretam profundas mudanças de
infraestrutura e de governança de \textsc{ti}. No caso do Inca, as evidencias indicam que
esse fator influenciou a decisão da \textsc{cgtc} de comprar equipamentos de
bioinformática no estado da arte.

A Biotecnologia alterou significativamente o peso relativo dos ativos da cadeia
produtiva farmacêutica que determinam o sucesso do lançamento de uma nova droga.
Nas condições atuais, o domínio das grandes farmacêuticas sobre a propriedade
intelectual (\textsc{pi}) à montante (\textit{upstream)}
não é condição de sucesso para o desenvolvimento de uma nova droga por métodos
de biotecnologia. Por outro lado\textit{,}
o valor estratégico dos ativos complementares a jusante que estas empresas ainda
detêm, tais como testes clínicos, plantas de produção e canais de
distribuição\textsuperscript{37}, foi fortalecido em comparação aos ativos de conhecimento detidos pelas
pequenas empresas de Biotecnologia.

As particularidades do \textit{arcabouço jurídico-institucional}
brasileiro têm forte impacto na P\&D das organizações biofarmacêutica do
país\textsuperscript{38}. Historicamente, os projetos de pesquisa das grandes farmacêuticas globais,
realizados nos países-sede, focam o desenvolvimento de fármacos destinados às
necessidades das populações dos países avançados, como é caso de medicamentos
para doenças crônico-degenerativas\textsuperscript{39}. Isto faz com que as economias em desenvolvimento sofram com a ausência de
medicamentos destinados às necessidades epidemiológicas típicas destas
sociedades, as chamadas doenças negligenciadas. No Brasil, as farmacêuticas
nacionais ainda apresentam baixas capacidades tecnológicas para suprir esta
lacuna por meio da inovação\textsuperscript{38}. Logo, uma característica marcante do Sistema Nacional de Inovação (\textsc{sni}) do
país é a presença de um parque público de laboratórios voltados para suprir por
meio da P\&D uma parcela das necessidades dos programas de saúde
pública\textsuperscript{39}. No entanto, as severas restrições a que são submetidas estas organizações as
impede de cumprir este papel na velocidade exigida no ramo biofarmacêutico:
\textit{Como o Inca vai levar a frente um projeto desta envergaduracom as
restrições institucionais impostas?}

Outra característica marcante da \textsc{sni} brasileiro diz respeito à relação precária
entre a Universidade e as empresas. Ainda que a literatura de inovação aponte
para a necessidade de mudanças comportamentais nos pesquisadores brasileiros no
sentido de valorizar as pesquisas voltadas para a necessidade do mercado, ainda
incidem sobre estes profissionais estímulos contraditórios, advindos de
diferentes instituições que influenciam o rumo das suas pesquisas: \textit{Tem
esta esquizofrenia, estimular a geração de patentes, mas ter a valorização do
trabalho do pesquisador pelos órgãos de fomento ainda vinculada à produção de
papers}.

O \textit{posicionamento tecnológico}
de uma organização reflete-se na intensidade dos seus esforços de
inovação\textsuperscript{40}. Na época deste trabalho, o Inca era um órgão auxiliar do Ministério da Saúde
que tinha como foco principal formular e executar as políticas públicas de
combate ao câncer visando atender, prioritariamente, o Sistema Único de Saúde
(\textsc{sus}). Embora ainda pudesse ser considerada uma organização que mantinha fortes
traços assistencialistas e acadêmicos, o papel da P\&D vinha, desde meados da
década passada, ganhando corpo na instituição. As evidências coletadas sugerem
que a possível configuração da \textsc{cgtc} como um centro de desenvolvimento
tecnológico é um processo em construção, que depende: : i) por um lado, de um
novo arcabouço jurídico-institucional para as atividades de P\&D dos institutos
de pesquisa no Brasil; ii) por outro, de uma definição estratégica interna,
cujos delineamentos estavam sendo discutidos na instituição.

O Inca era a referência na América Latina na realização de testes clínicos nas
fases 1, 2 e 3 para a classe terapêutica de oncologia. No momento em que a
pesquisa de campo foi realizada, cerca de 80\% destes estudos eram patrocinados
pelas grandes farmacêuticas globais, que os utilizam em testes multicêntricos de
novas drogas. Os demais 20\% eram demandados pelos pesquisadores do próprio Inca
ou pela Rede Nacional de Pesquisa Clinica. Há cerca de cinco anos, a \textsc{cgtc} do
Inca criou um laboratório de Bioinformática para conduzir análises relacionadas
aos experimentos da pesquisa básica em genômica e em proteômica. Também contava
com um pesquisador em nível de pós-doutorado realizando experimentos de Dinâmica
Molecular. A criação da área de pesquisa translacional era considerada um passo
crítico para a busca de maior integração entre a pesquisa básica e os testes
clínicos no Inca. Embora esta integração ainda estivesse fragmentada e não se
constituísse, na visão da bioinformática, \textit{no dia-a-dia da instituição}, já havia experimentos em curso tanto nos laboratórios físicos da pesquisa
básica quanto em Dinâmica Molecular propostos pela pesquisa translacional,
gerados de levantamentos epidemiológicos da incidência de determinadas lesões
moleculares observadas em pacientes do Instituto.

Na construção deste novo posicionamento, os gestores entrevistados viam a \textsc{cgtc}
do Inca concentrada nas atividades \textit{à montante}
do contínuo de inovação\textit{: O interesse do Inca não é produzir mercadoria.
O interesse do Inca é produzir conhecimento.}
Durante a execução desta pesquisa, o Inca estava em vias de concluir, com o
apoio do \textsc{bndes} e outras instituições de pesquisa, tais como a Fundação Oswaldo
Cruz (Fio cruz) e o Laboratório Nacional de Biociências (\textsc{lnbio}), as negociações
para a criação da Rede Nacional de Desenvolvimento de Fármacos Anti-Câncer
(Redefac). A construção deste arranjo colaborativo era visto pelos gestores
entrevistados como um passo decisivo para a reconfiguração da \textsc{cgtc} do Inca:

\textit{Temos de favorecer a posição do Inca como um articulador das
competências necessárias ao desenvolvimento de fármacos em câncer. O Inca monta
a cadeia com parceiros, contando com uma estrutura de inovação dentro do Inca.}

Neste arranjo, as empresas farmacêuticas poderiam assumir, via licenciamento do
composto desenvolvido pela Inca ou pela Redefac, o papel de produção do
fármaco\textit{.}
Mas os executivos entrevistados reconheceram que um posicionamento tecnológico
inovador exigirá uma melhor gestão da \textsc{pi}:

\textit{O Inca ainda não tem um escritório dedicado para isto (\textsc{pi}), que eu acho
que seria uma coisa fundamental para que a gente começasse a fazer esta parte de
desenvolvimento próprio de novas drogas. Se você não tem isso, você a princípio
já não está em condição de negociar com as grandes indústrias.}

Para Bell e Pavitt\textsuperscript{41}, as \textit{capacidades tecnológicas}
(\textsc{ct}) dividem-se em capacidades de produção e as capacidades para gerar e gerir
inovações tecnológicas. Estas últimas, as verdadeiras fontes de vantagem
competitiva em indústrias dinâmicas, são fundamentalmente diferentes das
primeiras, exigindo uma estratégia distinta da organização, já que não podem ser
alcançadas pela simples experiência acumulada em operar tecnologias compradas.
Embora o Inca realizasse experimentos nos segmentos \textit{a montante}, de maior complexidade tecnológica, os gestores entrevistados reconheceram que
a \textsc{cgtc} ainda não tinha acumulado \textsc{ct} para gerar novas moléculas em seus
experimentos de Bioinformática.

A existência de recursos acima das necessidades usuais (\textit{slack resources}
) desempenha um papel crucial na gestão da inovação tecnológica, pois permite
que a empresa realize projetos que não seriam feitos em contextos
organizacionais marcados por restrições de recursos\textsuperscript{21}. Houve um amplo consenso entre os entrevistados de que a falta destes recursos
constituía-se em um dos maiores entraves para que a \textsc{cgtc} do Inca se
transformasse em um centro de desenvolvimento tecnológico:

\textit{Um projeto para fazer o screening de novas moléculas é caro. Eu acredito
que o Inca vai se tornar um centro de referência em Biotecnologia, mas precisa
ter um aporte de recursos muito grande.}

Outro limitador importante identificado no campo era o número de pesquisadores,
fortemente contingenciado pelos critérios atuais de contratação. Na época da
realização desta pesquisa, o laboratório de Bioinformática contava com oito
pesquisadores, sendo que seis deles eram alunos de programas de mestrado e
doutorado em ciências da vida realizando seus projetos de pesquisa no Instituto:
\textit{Não é só comprar equipamento, precisa ter pessoal. E isso eu acho que
via ser um grande desafio para o Inca.}

Os ativos reputacionais de uma organização são vistos como chave para a
mobilização dos recursos institucionais necessários à inovação
tecnológica\textsuperscript{42}. A legitimidade alcançada pelo Inca, reconhecidamente uma organização de
excelência em saúde, era vista pelos gestores entrevistados como um elemento
importante para a construção de um novo arcabouço jurídico em apoio à inovação
na classe terapêutica de oncologia no Brasil:

\textit{Nós somos uma instituição de pesquisa clínica em Oncologia que temos
capacidade de fato de fazer e coordenar estudos clínicos não só
multi-institucionais em todo o Brasil, mas até multinacionais, vamos dizer
assim.}

Nos últimos anos, a \textsc{dti} tinha direcionado seus esforços para a implantação de
uma infraestrutura de \textsc{ti}, padronizando equipamentos e \textit{softwares}
em apoio aos processos de gestão e de assistência médica. O laboratório de
Bioinformática do Inca, criado durante este periodo, construiu uma área de
Informática própria, usando a maior autonomia que os financiamentos oriundos dos
orgãos de fomento à pesquisa lhe conferia para a compra de \textit{hardware e
software}. A importância de um parque computacional para a independência nos experimentos
de Bioinformática foi recorrentemente destacada pelo gestor do laboratório de
Bionformática:

\textit{Quando você tem um hardware dedicado para uma instituição, e isto eu não
digo só para o Inca, qual que é a finalidade disso? É estimular a pesquisa que é
feita dentro da sua Instituição. Quando você precisa, ela tá lá.}

Uma vez atendidas as necessidades das camadas de gestão e de informática médica,
os gestores da \textsc{dti} indicaram que o grande potencial de contribuição dos \textsc{si} para
o Inca nos próximos anos poderia estar na \textsc{cgtc}: \textit{Eu sinto que a
pesquisatem um potencial muito grande de oportunidades. A gente vai entrar
fortemente agora na pesquisa com um ambiente de gestão de conhecimento.}

Para Weill e Ross\textsuperscript{43}, os modelos de governança de \textsc{ti} definem os direitos e deveres entre as áreas
fim e a área de \textsc{ti} da organização no sentido de encorajar o uso efetivo das \textsc{tic}.
Segundo estes autores, modelos descentralizados são mais recomendados quando se
busca a inovação. No Inca, as ferramentas de Bioinformática ficavam sob a
responsabilidade da \textsc{cgtc} e os sistemas administrativos, de informática médica e
de tele-medicina, com a \textsc{dti}. Esta última posicionava-se como um prestador de
serviços à \textsc{cgtc}, o que era amplamente reconhecido pelos pesquisadores
entrevistados. A pesquisa de campo indicou que o conflito potencial entre o
laboratório de Bioinformática e a Informática Corporativa era contornado pela
postura das duas áreas:

\textit{Eu dei muita sorte, porque eu conheço outros exemplos, outras
instituições em que a \textsc{ti} da instituição não vê com bons olhos, porque acha que
alguém vai roubar o cargo de alguém, não entende que são coisas... Não é toda a
instituição que tem profissionais que entendem isto}.

A \textsc{dti}, ao ser indagada sobre a existência desta área de Informática fora dos
seus domínios diretos, explicou:

\textit{Até por não ser a nossa praia, e isto para mim está muito claro, a minha
equipe não domina este tipo de tecnologia, ela é muito específica. Open source,
geralmente focado em genética. Então a gente atua só na parte de apoio.}

A natureza das informações tratadas pelas ferramentas \textsc{tic} usadas nos
experimentos de Bioinformática requer que os pesquisadores desenvolvam alta
competência tanto em Biologia Molecular quanto em Informática. Por esse motivo,
entende-se que estas ferramentas não valem muito sem os recursos humanos
capacitados adequadamente nas técnicas de tratamento e análise das informações,
como atesta a fala de um entrevistado: \textit{Existiu assim um grande boom na
utilização deste tipo de ferramenta sem que a gente tivesse uma comunidade
preparada para analisar os dados que são gerados com elas.}

A padronização dos parâmetros e das interfaces, bem como a garantia da origem
das informações são críticos nos experimentos de
Bioinformática\textsuperscript{24}, o que favorece o uso de sistemas \textit{open source}. Vale lembrar que uma parte expressiva das informações geradas em experimentos
de sequenciamento genético não é passível de apropria ção\textsuperscript{6}. De fato, a inclusão dos dados gerados em experimentos de Bioinformática - e
dos aplicativos desenvolvidos em apoio a eles - nas bases de dados públicas é
praticamente mandatória para a publicação de artigos científicos em periódicos
de renome nesta área de conhecimento\textsuperscript{18}.

A dinâmica do ramo biofarmacêutico foi voz corrente entre os entrevistados
quando justificaram a necessidade de mudanças na \textsc{cgtc} do Inca. Entre as
preocupações manifestadas, destaca-se a competitividade do Complexo Econômico e
Industrial de Saúde brasileiro frente aos grandes grupos farmacêuticos globais:

\textit{Como é que o pessoal de Farma está fazendo, a Indústria? O cara tem um
time de 50 pessoas, onde 2 são administrativos, ou seja, 48 são cientistas.
Destes 48, 10 são químicos. Cada químico produz por semana uma biblioteca de
2000 compostos, e uma plataforma robotizada testa... Então assim, não tem como
competir}.

Os entrevistados concordaram que o posicionamento tecnológico que o Inca adotava
à época, junto com a escassez de recursos para realizar estas análises,
explicava o ainda baixo envolvimento do laboratório de Bioinformática em estudos
derivados das pesquisas clínicas. \textit{Um projeto para fazer o screening de
novas moléculas são milhões de reais. Você pega um edital do \textsc{cnpq}, o teto de um
projeto que você pode mandar são 100 mil reais.}
Explica também em parte a não utilização de \textsc{git} e de workflows científicos pela
organização, já que: \textit{o que falta ao Inca é ser mais aberto}. Porém, os entrevistados concordaram que o uso de \textsc{git} em estratégias de
\textit{e.science}
futuras poderia ser adotado:

\textit{Se realmente o interesse institucional do Inca é o desenvolvimento de
novas drogas, e isto vai ser feito em uma parceria, compartilharemos a
capacidade computacional que a gente tem, inclusive usando a deles}. \textit{Acho que isto é bem possível.}

Como a \textsc{cgtc} ainda não detinha \textsc{ct} para a descoberta de novos compostos, não havia
na Bioinformática do Inca a utilização de ferramentas de mineração de dados para
a realização de experimentos de reconhecimento de padrões, pois: \textit{...
para fazer data minning, você precisa ter geração de dados, e a gente ainda está
gerando os dados.}
Mas, com um posicionamento tecnológico mais agressivo, \textit{... com certeza a
gente vai fazer mineração de dados no futuro.}
Pelos mesmos motivos apontados acima, a \textsc{cgtc} do Inca ainda não havia investido
na compra de ferramentas de \textsc{hts}. É importante frisar que as redes de
codesenvolvimento, tais como a Redefac, eram vistas como um mecanismo mais
efetivo de acesso a estas ferramentas.

Os sistemas computacionais de experimentação usados nas análises em Genômica são
desenvolvidos em plataformas abertas. Perguntado sobre a razão desta
preferência, o entrevistado explicou porque este é um comportamento dominante:.\textit{..isto porque a bioinformática foi desenvolvida no ambiente acadêmico,
os sistemas estão publicados em revistas científicas, com ampla aceitação. Não
faz sentido desenvolver de novo.}

Para a análise de dados em Proteômica, o laboratório de Bioinformática
desenvolveu o Sistema de Gestão de Informações de Laboratório (p\textsc{lims}). Além das
informações básicas do laboratório, o sistema armazenava imagens dos
experimentos de análises bidimensionais de proteínas em electroforese em gel
(\textit{2D/1D protein gel electrophoresis}
). Embora existissem inúmeros softwares disponíveis na Internet, a
Bioinformática do Inca optou por desenvolver seu próprio sistema. Ao ser
indagado, o entrevistado da \textsc{bion} concordou que a \textsc{dti} poderia ter ajudado, dado o
escopo do sistema, embora a ajuda não tenha sido solicitada. Seguindo o
comportamento dominante já descrito anteriormente, o software foi
disponibilizado gratuitamente no sítio do Instituto
(http://lbbc.Inca.gov.br/proteo mics). A \textsc{cgtc} também utilizava aplicativos de
\textit{open source}
para os experimentos de simulação da Dinâmica Molecular (Ex.: Gromacs) e para o
\textit{docking}
(Ex.: Auto-Docking).

Durante a realização desta pesquisa, o Inca havia acabado de receber um novo
sequenciador de \textsc{dna} e estava finalizando a aquisição de uma solução de
\textit{High Performance Computing}
(\textsc{hpc}). Tais equipamentos iriam colocar o Instituto no estado da arte em
ferramentas de genômica, Proteômica e modelagem e simulação de proteínas. Há
indícios de que uma boa governança parece ter influenciado a decisão, tomada no
decorrer dessa pesquisa, de passar estes equipamentos para a \textsc{dti}, reconhecida
pela \textsc{bion} como melhor capacitada para a gestão da nova infraestrutura. Porém, as
limitações institucionais para a contratação de pesquisadores eram vistas como
um sério obstáculo para a realização de experimentos voltados à descoberta de
novas moléculas, ainda que o hardware permitisse:

\textit{É um circulo vicioso. Ele só vai ter dinheiro para fazer isso, para me
cobrar, para ter uma estrutura que permita a ele fazer este serviço, o dia em
que ele puder contratar um técnico.}

Dadas às características predominantes no ramo biofarmacêutico, observou-se uma
busca permanente no Inca pela padronização dos conceitos trabalhados em seus
experimentos, visto que isso auxiliaria o compartilhamento de informações com
outras Instituições. Para tal, o laboratório de Bioinformática e a área de
Dinâmica Molecular recorriam sistematicamente ao \textit{Gene Ontology}, um consórcio estabelecido para a padronização de atributos dos genes, e ao
\textit{Protein Data Bank}, sítio na web que deposita, processa e distribui gratuitamente dados de
estruturas macromoleculares.

Quanto ao uso de \textsc{cro}s, ficaram evidentes nos depoimentos as limitações que as
fontes de financiamento à pesquisa impunham à contratação destes prestadores de
serviço. No caso da realização de pesquisas clínicas financiadas pelas grandes
farmacêuticas, esta limitação não estava presente. A preocupação com a relação
ao custo-benefício, ou mesmos os riscos, de se tentar obter a \textsc{pi} das informações
geradas nos sistemas que apoiam os experimentos de Bioinformática foi
recorrentemente ressaltada:

\textit{Apesar de eu ser favorável ao depósito de patentes de proteção do
conhecimento, mas também eu entendo que tem algumas questões que se você
ponderar vale mais a pena você deixar ter livre acesso para que as pessoas usem
e abusem do conhecimento do que fazer uma proteção patentária.}

\section{Conclusões}

Os autores deste artigo entendem que a questão colocada nesta pesquisa foi
respondida na medida em que a aplicação do Modelo \textsc{toe}, adaptado aos objetivos do
estudo aqui empreendido, ajudou a demostrar, dentro das limitações de espaço do
artigo, como atuaram no caso do Inca os fatores que impactam a adoção de
ferramentas \textsc{tic} usadas em experimentos de Bionformática.

No nível acadêmico, o modelo proposto contribuiu para investigações em duas
áreas de importância crescente para o Complexo Econômico-Industrial de Saúde
brasileiro, a inovação tecnológica e a biotecnologia. Com base nas evidências
coletadas, uma questão é formulada para aprofundamento em pesquisas futuras: em
que medida o alinhamento dos fatores pertinentes à adoção das \textsc{tic} nos
experimentos de Bioinformática pode aumentar a capacidade de inovar de uma
organização biofarmacêutica brasileira?

O estudo mostrou a importância crescente da Ciência da Informação para a
Bioinformática do Inca, mas também trouxe evidências da possibilidade de uma
substancial melhora no que diz respeito: i) ao uso de data mining para
reconhecimento de padrões, estruturas e regras; ii) à integração dos sistemas de
gestão de laboratório aos demais aplicativos de uma organização de saúde; iii)
às estratégias \textit{e.science}. No caso do Inca, nem todas as áreas da \textsc{cgtc} mostraram o mesmo conhecimento de
informática e tinham a mesma proximidade à \textsc{dti} que o laboratório de
Bioinformática. Mesmo nesta última, há evidencias de que técnicas de gestão da
informação mais sofisticadas poderiam ter sido adotadas. Um exemplo do potencial
de contribuição da \textsc{ti} foi o desenvolvimento, a pedido da \textsc{cgtc}, do \textsc{sisbnt},
sistema que gerencia o Banco Nacional de Tumores, um biobanco que ajudou a mudar
o patamar das pesquisas na instituição\textit{.}

Do ponto de vista da prática gerencial, considerando o fato de que muitas
organizações brasileiras em ciências da vida, especialmente os institutos de
pesquisa públicos em busca de \textsc{ct} inovadoras, enfrentam desafios semelhantes aos
identificados na \textsc{cgtc} do Inca, podem-se extrair lições relevantes das discussões
aqui realizadas. O caso trouxe evidências de que a opção do Inca pela criação de
uma área de Bioinformática, quando mediada por mecanismos de governança
efetivos, contribuiu decisivamente para a adoção das ferramentas em foco neste
estudo. A criação da Redefac, dentro da qual o Inca espera desempenhar um papel
central, indicou caminhos alternativos interessantes para o enfrentamento do
desafio da produção nacional de fármacos em câncer. Mas as evidências obtidas
mostram que a conversão da \textsc{cgtc} do Inca em um centro de desenvolvimento
tecnológico exigirá profundas mudanças organizacionais e institucionais. Dentre
os pontos mais destacados pelos entrevistados estão uma maior coesão interna em
torno da visão do Inca como um centro de C\&T\&I e a necessidade de aumentar o
número e a senioridade dos pesquisadores envolvidos nesta tarefa.

Por fim, o caráter multidisciplinar dos experimentos em Bioinformática
examinados no contexto do Inca deixou evidente o risco de perspectivas
reducionistas ou reificantes das ferramentas de \textsc{tic} aqui discutidas. Por
conseguinte, defende-se neste estudo que o uso adequado destas ferramentas é
parte indissociável das competências que um pesquisador em Bioinformática deve
possuir para fazer avançar o conhecimento científico em Biologia Molecular,
particularmente nas organizações que lidam com pesquisa de ponta em câncer.

\section{Colaboradores}

C Pitassi participou da concepção e estruturação da pesquisa, na escolha do
referencial teórico, na metodologia, na coleta e análise dos dados e na redação
final do artigo. \textsc{va} Moreno participou da estruturação da pesquisa, na revisão
crítica dos resultados e na redação final do artigo. \textsc{aa} Gonçalves participou na
estruturação da pesquisa, na coleta e análise de dados e na redação final do
artigo.

\section*{Referências}
\begin{itemize}

\item[1] Macmullen \textsc{wj}, Denn S. Information problems in molecular biology and
Bioinformatics. J of the Amer Soc for Info Sci \& Tech 2005; 56(5):447-456.

\item[2] Attwood \textsc{tk}, Gisel A, Eriksson N-E, Bongcam-Rudloff E. Concepts,
historical milestones and the central place of bioinformatics in modern biology:
a European perspective. In: Mahdavi \textsc{ma}, editor. Bioinformatics-Trends and
Methodologies. Rijeka: Intech Online Publishers; 2011.

\item[3] Costa \textsc{lf}. Bioinformatics: perspectives for the future. Gen Mol Res
2004; 3(4):564-574.

\item[4] Vogt C. Bioinformática, genes e inovação. R ComCiência 2003; 46.

\item[5] Catanho M, De Miranda \textsc{ab}, Degrave W. Comparando genomas: bancos de
dados e ferramentas computacionais para a análise comparativa de genomas
procarióticos. R Eletr de Com Infor \& Inov 2007; 1(2):335-358.

\item[6] Powell W, White D, Koput K, Owen-Smith J. Network Dynamics and field
evolution: the growth of interorganizational collaboration in the life sciences.
\textsc{ajs} 2005; 110(4):1132-1205.

\item[7] Chiaroni D, Chiesa V, Frattini F. Patterns of collaboration along
the Bio-Pharmaceutical innovation process. J of Bus Chem 2008; 5(1):7-22.

\item[8] Curcin V, Ghanem M. Scientific workflow systems - can one size fit
all? Proceedings of the Cairo Biomedical Engineering Conference. Cairo: 2008;
1-9.

\item[9] Maqueira \textsc{jm}, Bruque S. Grid information technology as a new
technological tool for e.science, healthcare and life science. J of Tech Mangt
\& Innov 2007; 2(2):95-113.

\item[10] Critchlow T, Musick R, Slezak T. Experience applying meta-data to
Bioinformatics. Info Sci 2001; 139(1-2):3-17.

\item[11] Heath \textsc{sl}, Ramakrishnan N. The emerging landscape of Bioinformatics
software systems. \textsc{ieee} C 2005; 35(7):41-45.

\item[12] Stevens R, Goble \textsc{ca}, Bechhofer S. Ontology-based knowledge
representation for bioinformatics. B in Bioinf 2000; 1(4):398-414.

\item[13] Febles Rodriguez \textsc{jp}, Gonzalez Perez A. Aplicación de la minería de
datos en la Bioinformática. \textsc{acimed} 2002; 10(2):69-76.

\item[14] Verona G, Prandelli E, Sawhney M. Innovation and virtual
environment: toward virtual knowledge brokers. Org Studies 2006; 27(6):765-788.

\item[15] Persidis A. High-throughput screening. Nat Biotech 1998;
16(5):488-489.

\item[] 16 Bleicher \textsc{kh}, Böhm H-J, Müller K, Alanine \textsc{ai}. Hit and lead generation:
beyond high-throughput screening. Nat R Drug Disc 2003; 2(5):369-378.

\item[17] Lenoir T. Shaping biomedicine as an information science. In:
Bowden \textsc{me}, Hahn \textsc{tb}, Willians \textsc{rv}, editors. Proceedings of the 1998 Conference on
the History and Heritage of Science Information Systems. \textsc{asis} Monograph Series.
Medford: Information Today; 1998. p. 27-45.

\item[18] Brown C. The changing face of scientific discourse: analysis of
Genomic and Proteomic database usage and acceptance, J of the Amer Soct for Info
Sci \& Tech 2003; 54(10):926-938.

\item[19] Gassmann O, von Zedtwitz M. Trends and determinants of managing
virtual R \& D teams. R \& D Mangt 2003; 33(3):243-262.

\item[20] Venkatesh V, Morris \textsc{mg}, Davis \textsc{gb}, Davis \textsc{fd}. User acceptance of
information technology: toward a unified view. \textsc{mis} Quart 2003; 27(3):425-478.

\item[21] Tornatzky \textsc{lg}, Fleischer M. The processes of technological
innovation. Massachusetts: Lexington Books; 1990.

\item[22] Denzin \textsc{nk}, Lincoln \textsc{ys}. Handbook of qualitative research.
California: Sage Publications; 2000.

\item[23] Yin \textsc{rk}. Case study research - design and methods. London: \textsc{sage}
Publications; 1994.

\item[24] Digiampietri \textsc{la}. Gerenciamento de workflows científicos em
Bioinformática [tese]. Campinas: Unicamp; 2007.

\item[25] Neubauer F, Hoheisel A, Geiler J. Workflow-based grid
applications. Fut Gen Comp Syst 2006; 22(1-2):6-15.

\item[26] Bose R. Knowledge management-enabled health care management
systems: capabilities, infrastructure, and decision-support. Exp Syst with
Applic 2003; 24(1):59-71.

\item[27] Naznin F, Sarker R, Essam D. Vertical decomposition with genetic
algorithm for multiple sequence alignment. \textsc{bmc} Bioinf 2011; 12:353-379.

\item[28] Thurow K, Göde B, Dingerdissen U, Stoll N. Laboratory information
management systems for life science applications. Org Proc Res Develop 2004;
8(6): 970-982.

\item[29] Bare \textsc{jc}, Koide T, Reiss \textsc{dj}, Tenenbaum D, Baliga \textsc{ns}. Integration
and visualization of systems biology data in context of the genome. \textsc{bmc} Bioinf
2010; 11:382-390.

\item[30] Thomke \textsc{sh}. Experimentation matters: unlocking the potential of new
technologies for innovation. Boston: \textsc{hbs} Press; 2003.

\item[31] Pisano \textsc{gp}, Verganti R. Which kind of collaboration is right for
you? \textsc{hbr} 2008; 12:78-86.

\item[32] Stajich \textsc{je}, Lapp H. Open source tools and toolkits for
Bioinformatics: significance, and where are we? B in Bioinformatics 2006;
7(3):287-296.

\item[33] Hacievliyagil \textsc{nk}. The impact of open innovation on technology
transfers at Philips and \textsc{dsm} [thesis]. Delft: Delft University of Technology;
2007.

\item[34] Bardin L. Análise de conteúdo. Lisboa: Edições 70; 1979.

\item[35] Malerba F, Orsenigo L. Innovation and market structure in dynamics
of the pharmaceutical industry and Biotechnology: towards a history friendly
model. Ind Corp Change 2001; 11(4):667-703.

\item[36] Chau \textsc{pyk}, Tam \textsc{ky}. Factors affecting the adoption of open systems:
an exploratory study. \textsc{mis} Quart 1997; 21(1):1-24.

\item[37] Pisano \textsc{gp}. Profiting from innovation and the intellectual property
revolution. R Policy 2006; 35:1122-1130.

\item[38] Vieira \textsc{vmm}, Ohayon P. Inovação em fármacos e medicamentos:
estado-da-arte no Brasil e políticas de P \& D. R Econ \& Gestão 2007;
6(13):1-23.

\item[39] Oliveira \textsc{ea}, Labra \textsc{ma}, Bermudez J. A produção pública de
medicamento no Brasil: uma visão geral. Cad Saude Publica 2006;
22(11):2379-2389.

\item[40] Freeman C, Soete L. The economics of industrial innovation.
London: Frances Pinter; 1997.

\item[41] Bell M, Pavitt K. The development of technological capabilities.
In: Haque \textsc{iu}, editor. Trade, technology and international competitiveness.
Washington: World Bank; 1995.

\item[42] Teece D, Pisano G, Shuen A. Dynamic capabilities and strategic
management. Strat Mangt J 1997; 18(7): 509-533.

\item[43] Weill P, Ross \textsc{jw}. \textsc{it} governance. Boston: \textsc{hbs} Press; 2004.

\end{itemize}

\end{document}
