% Generated by jats2tex@0.11.1.0
\documentclass{article}
\usepackage{scielo}

\newcommand{\journalid}{Rev Saude Publica}
\newcommand{\journaltitle}{Revista de Saúde Pública}
\newcommand{\abbrevjournaltitle}{Rev. Saúde Pública}
\newcommand{\issnppub}{0034-8910}
\newcommand{\issnepub}{1518-8787}
\newcommand{\publishername}{Faculdade de Saúde Pública da Universidade de São Paulo}
\newcommand\articleid{S0034-8910.2013047004402}
\newcommand\articledoi{\textsc{doi} 10.1590/S0034-8910.2013047004402}
\def\subject{Prática de Saúde Pública}
\title{Necessidade de implantar programa nacional de segurança do
          paciente no Brasil\titlegroup{}} 
\author[% <xref ref-type="aff" rid="aff1">
I]{% <name >
% <surname >
Capucho, % <given-names >
Helaine Carneiro}
\author[% <xref ref-type="aff" rid="aff2">
II]{% <name >
% <surname >
Cassiani, % <given-names >
Silvia Helena De Bortoli}
\affil[% <label >
I]{% <institution content-type="orgname">
Ministério da Saúde}
\affil[% <label >
II]{% <institution content-type="orgname">
Universidade de São Paulo}
\def\authornotes{%tag
% <corresp >
%tag
% <bold >
Correspondência | Correspondence%endelem
 : Helaine Carneiro Capucho. Esplanada dos
          Ministérios. Bloco G Edifício Sede 9º andar Sala 949. 70058-900 Brasília, DF, Brasil.
          E-mail: helaine.capucho@saude.gov.br%endelem
%tag
% <fn fn-type="conflict">
%tag
% <p >

Os autores declaram não haver conflito de interesses.
%endelem
%endelem}
\date{% <day >
21 % <month >
08 % <year >
2013}
\def\volume{47}
\def\issue{4}
\def\fpage{791}
\def\lpage{798}
\newcommand{\cclicense}{\ccbync}
\newcommand{\kwdgroup}{% <kwd >, Segurança do Paciente, % <kwd >, Avaliação de Programas e Projetos de Saúde, % <kwd >, Sistema Único de Saúde, % <kwd >, Garantia da Qualidade dos Cuidados de Saúde}
\newcommand{\kwdgroupes}{% <kwd >, Segurança do Paciente, % <kwd >, Avaliação de Programas e Projetos de Saúde, % <kwd >, Sistema Único de Saúde, % <kwd >, Garantia da Qualidade dos Cuidados de Saúde}
\newcommand{\kwdgroupen}{% <kwd >, Segurança do Paciente, % <kwd >, Avaliação de Programas e Projetos de Saúde, % <kwd >, Sistema Único de Saúde, % <kwd >, Garantia da Qualidade dos Cuidados de Saúde}
%%% Nota fn1 %%%%%%%%%%%%%%%%%%%%%%%%%%%%%%%%%%%%%%%%%%%%%%%%%%%%%%%%
\expandafter\newcommand\csname fn1\endcsname{% <p >

World Health Organization. \textsc{apps} web-based registration mechanism open. Geneva; 2012
          [citado 2012 jun 2]. Disponível em: http://www.who.int/patientsafety/en/}
%%% Nota fn2 %%%%%%%%%%%%%%%%%%%%%%%%%%%%%%%%%%%%%%%%%%%%%%%%%%%%%%%%
\expandafter\newcommand\csname fn2\endcsname{% <p >

Capucho HC. Sistemas manuscrito e informatizado de notificação voluntária de incidentes
          em saúde como base para a cultura de segurança do paciente [tese de doutorado]. São Paulo:
          Escola de Enfermagem de Ribeirão Preto da \textsc{usp}; 2012.}
%%% Nota fn3 %%%%%%%%%%%%%%%%%%%%%%%%%%%%%%%%%%%%%%%%%%%%%%%%%%%%%%%%
\expandafter\newcommand\csname fn3\endcsname{% <p >

Ministério da Saúde. Portaria GM/MS nº 529, de 1 de abril de 2013. Institui o Programa
          Nacional de Segurança do Paciente (\textsc{pnsp}). %tag
% <italic >
\textit{Diario Oficial Uniao}
%endelem
 . 2 abr
          2013;Seção1:43-4.}
%%% Nota fn4 %%%%%%%%%%%%%%%%%%%%%%%%%%%%%%%%%%%%%%%%%%%%%%%%%%%%%%%%
\expandafter\newcommand\csname fn4\endcsname{% <p >

Petramale CA. O projeto dos hospitais sentinela e a gerência de risco sanitário
          hospitalar. In: Capucho HC, Carvalho FD, Cassiani \textsc{shb}. Farmacovigilância - Gerenciamento
          de Riscos da Terapia Medicamentosa para a Segurança do Paciente. São Caetano do Sul:
          Editora Yendis. 2001. p. 191-224.}
%%% Nota fn5 %%%%%%%%%%%%%%%%%%%%%%%%%%%%%%%%%%%%%%%%%%%%%%%%%%%%%%%%
\expandafter\newcommand\csname fn5\endcsname{% <p >

Ministério da Saúde. Portaria nº 396, de 4 de março de 2011 Institui o projeto de
          formação e melhoria da qualidade de rede de saúde (Quali-\textsc{sus}-Rede) e suas diretrizes
          operacionais gerais. %tag
% <italic >
\textit{Diario Oficial Uniao}
%endelem
 . 9 mar 2011.}
%%% Nota fn6 %%%%%%%%%%%%%%%%%%%%%%%%%%%%%%%%%%%%%%%%%%%%%%%%%%%%%%%%
\expandafter\newcommand\csname fn6\endcsname{% <p >

Ministério da Saúde. Índice de Desempenho do \textsc{sus} – \textsc{idsus}. Brasília (DF); 2011 [citado
          2012 jun 2]. Disponível em:
          http://portal.saude.gov.br/portal/saude/area.cfm?id\_{}area=1080}
%%% Nota fn7 %%%%%%%%%%%%%%%%%%%%%%%%%%%%%%%%%%%%%%%%%%%%%%%%%%%%%%%%
\expandafter\newcommand\csname fn7\endcsname{% <p >

Brasil. Lei nº 12.401, de 28 de abril de 2011. Altera a Lei nº 8.080, de 19 de setembro
          de 1990, para dispor sobre a assistência terapêutica e a incorporação de tecnologia em
          saúde no âmbito do Sistema Único de Saúde - \textsc{sus}. %tag
% <italic >
\textit{Diario Oficial Uniao}
%endelem
 .
          29 abr 2011:1.}
%%% Nota fn8 %%%%%%%%%%%%%%%%%%%%%%%%%%%%%%%%%%%%%%%%%%%%%%%%%%%%%%%%
\expandafter\newcommand\csname fn8\endcsname{% <p >

World Health Organization. \textsc{who} launches ‘Nine patient safety solutions. Geneva; 2007
          [citado 2012 jun 2]. Disponível em:
          http://www.who.int/mediacentre/news/releases/2007/pr22/en/index.html}

\begin{document}
\selectlanguage{portuges}
\newcommand{\lingua}{Português}
\maketitle
\tableofcontents

%tag
% <front >
%tag
% <journal-meta >
%tag
% <journal-id journal-id-type="nlm-ta">
%endelem
%tag
% <journal-title-group >
%tag
% <journal-title >
%endelem
%tag
% <abbrev-journal-title abbrev-type="publisher">
%endelem
%endelem
%tag
% <issn pub-type="ppub">
%endelem
%tag
% <issn pub-type="epub">
%endelem
%tag
% <publisher >
%tag
% <publisher-name >
%endelem
%endelem
%endelem
%tag
% <article-meta >
%tag
% <article-id pub-id-type="publisher-id">
%endelem
%tag
% <article-id pub-id-type="doi">
%endelem
%tag
% <article-categories >
%tag
% <subj-group subj-group-type="heading">
%tag
% <subject >
%endelem
%endelem
%endelem
%tag
% <title-group >
%tag
% <article-title xml:lang="pt">
%endelem
%tag
% <trans-title-group xml:lang="es">
%tag
% <trans-title >
Necesidad de implantar programa nacional de seguridad del paciente en
            Brasil%endelem
%endelem
%endelem
%tag
% <contrib-group >
%tag
% <contrib contrib-type="author">
%endelem
%tag
% <contrib contrib-type="author">
%endelem
%endelem
%tag
% <aff id="aff1">
%endelem
%tag
% <aff id="aff2">
%endelem
%tag
% <author-notes >
%endelem
%tag
% <pub-date pub-type="epub-ppub">
%endelem
%tag
% <volume >
%endelem
%tag
% <issue >
%endelem
%tag
% <fpage >
%endelem
%tag
% <lpage >
%endelem
%tag
% <history >
%tag
% <date date-type="received">
%endelem
%tag
% <date date-type="accepted">
%endelem
%endelem
%tag
% <permissions >
%tag
% <license license-type="open-access" xlink:href="http://creativecommons.org/licenses/by-nc/3.0/">
%endelem
%endelem
%tag
% <trans-abstract xml:lang="en">

\begingroup
\renewcommand{\section}[1]{\subsection*{#1}}
\begin{otherlanguage}{english}
\renewcommand{\abstractname}{% <title >
\textsc{abstract}}
\begin{abstract}
% <sec >
\section{% <title >
\textsc{objective}}
% <p >

 To estimate the prevalence of arterial hypertension and obesity and the population attributable fraction of hypertension that is due to obesity in Brazilian adolescents.

% <sec >
\section{% <title >
\textsc{methods}}
% <p >

 Data from participants in the Brazilian Study of Cardiovascular Risks in Adolescents (\textsc{erica}), which was the first national school-based, cross-section study performed in Brazil were evaluated. The sample was divided into 32 geographical strata and clusters from 32 schools and classes, with regional and national representation. Obesity was classified using the body mass index according to age and sex. Arterial hypertension was defined when the average systolic or diastolic blood pressure was greater than or equal to the 95%tag
% <sup >\textsuperscript{th}
%endelem
 percentile of the reference curve. Prevalences and 95\% confidence intervals (95\%CI) of arterial hypertension and obesity, both on a national basis and in the macro-regions of Brazil, were estimated by sex and age group, as were the fractions of hypertension attributable to obesity in the population.

% <sec >
\section{% <title >
\textsc{results}}
% <p >

 We evaluated 73,399 students, 55.4\% female, with an average age of 14.7 years (SD = 1.6). The prevalence of hypertension was 9.6\% (95\%CI 9.0-10.3); with the lowest being in the North, 8.4\% (95\%CI 7.7-9.2) and Northeast regions, 8.4\% (95\%CI 7.6-9.2), and the highest being in the South, 12.5\% (95\%CI 11.0-14.2). The prevalence of obesity was 8.4\% (95\%CI 7.9-8.9), which was lower in the North region and higher in the South region. The prevalences of arterial hypertension and obesity were higher in males. Obese adolescents presented a higher prevalence of hypertension, 28.4\% (95\%CI 25.5-31.2), than overweight adolescents, 15.4\% (95\%CI 17.0-13.8), or eutrophic adolescents, 6.3\% (95\%CI 5.6-7.0). The fraction of hypertension attributable to obesity was 17.8\%.

% <sec >
\section{% <title >
\textsc{conclusions}}
% <p >

 \textsc{erica} was the first nationally representative Brazilian study providing prevalence estimates of hypertension in adolescents. Regional and sex differences were observed. The study indicates that the control of obesity would lower the prevalence of hypertension among Brazilian adolescents by 1/5.

\ifdef{\kwdgroupen}{\medskip\noindent\textbf{Keywords:} \kwdgroupen}{}
\end{abstract}
\end{otherlanguage}
\endgroup
%endelem
%tag
% <abstract xml:lang="pt">

\begingroup
\renewcommand{\section}[1]{\subsection*{#1}}

\begin{abstract}
% <p >

O objetivo do estudo foi suscitar reflexão acerca da necessidade de se criar um sistema
          nacional de notificações sobre incidentes como base para um programa de segurança do
          paciente. Incidentes em saúde acarretam danos aos pacientes e oneram o sistema de saúde.
          Embora tenha lançado recentemente um programa de avaliação da qualidade nas instituições
          de saúde, o Ministério da Saúde, Brasil, ainda não possui um programa que avalie
          sistematicamente os resultados negativos da assistência. Discute-se a necessidade de se
          implementar programa brasileiro de segurança do paciente, a fim de promover a cultura pela
          segurança do paciente e da qualidade em saúde no Sistema Único de Saúde.

\ifdef{\kwdgroup}{\iflanguage{portuges}{\medskip\noindent\textbf{Palavras-chave:} \kwdgroup}{}}{}
\ifdef{\kwdgroupen}{\iflanguage{english}{\medskip\noindent\textbf{Keywords:} \kwdgroupen}{}}{}
\ifdef{\kwdgroupes}{\iflanguage{spanish}{\medskip\noindent\textbf{Palavras claves:} \kwdgroupes}{}}{}
\ifdef{\kwdgroupfr}{\iflanguage{french}{\medskip\noindent\textbf{Mots clés:} \kwdgroupfr}{}}{}
\end{abstract}
\endgroup
%endelem
%tag
% <trans-abstract xml:lang="es">

\begingroup
\renewcommand{\section}[1]{\subsection*{#1}}
\begin{otherlanguage}{spanish}

\begin{abstract}
% <p >

El objetivo del estudio fue suscitar reflexión sobre la necesidad de crear un sistema
          nacional de notificaciones sobre incidentes como base para un programa brasileño de
          seguridad del paciente. Incidentes en salud generaron daños a los pacientes y sobrecargan
          el sistema de salud. A pesar de que se haya lanzado recientemente un programa de
          evaluación de la calidad en las instituciones de salud, el Ministerio de Salud Brasileño
          no posee aún un programa que evalúe sistemáticamente los resultados negativos de la
          asistencia. Se discute la necesidad de implementar un programa brasileño de seguridad del
          paciente, con el fin de promover la cultura por la seguridad del paciente y la calidad de
          la salud en el Sistema Único de Salud.

\ifdef{\kwdgroupes}{\medskip\noindent\textbf{Palavras claves:} \kwdgroupes}{}
\end{abstract}
\end{otherlanguage}
\endgroup
%endelem
%tag
% <kwd-group xml:lang="pt">
%endelem
%tag
% <kwd-group xml:lang="es">
%endelem
%tag
% <kwd-group xml:lang="en">
%endelem
%tag
% <counts >
%tag
% <fig-count count="2">
%endelem
%tag
% <ref-count count="23">
%endelem
%tag
% <page-count count="8">
%endelem
%endelem
%endelem
%endelem
%tag
% <body >
%tag
% <sec sec-type="intro">
\section{% <title >
\textsc{introdução}}
% <p >

Desde há muito tempo os resultados da assistência são utilizados para avaliação da
        qualidade dos serviços de saúde. Os babilônicos pagavam pelos serviços médicos mediante os
        resultados obtidos e, na Idade Média, os médicos que obtivessem resultados negativos na
        prestação de assistência tinham parte de seus corpos mutilados. %tag
% <xref ref-type="bibr" rid="B17">\textsuperscript{[}%tag
% <sup >\textsuperscript{17}
%endelem\textsuperscript{]}  
%endelem

% <p >

Os resultados negativos em saúde são conhecidos principalmente como eventos adversos ou
        qualquer tipo de incidente com potencial para causar danos aos pacientes %tag
% <xref ref-type="bibr" rid="B20">\textsuperscript{[}%tag
% <sup >\textsuperscript{20}
%endelem\textsuperscript{]}  
%endelem
 e que pode fornecer importantes informações para a construção de um sistema de saúde
        mais seguro. %tag
% <xref ref-type="bibr" rid="B14">\textsuperscript{[}%tag
% <sup >\textsuperscript{14}
%endelem\textsuperscript{]}  
%endelem
 Os incidentes podem ser sem dano, com dano (evento adverso), ou %tag
% <italic >
\textit{near
          misses}
%endelem, também denominado de potencial evento adverso. %tag
% <xref ref-type="bibr" rid="B4">\textsuperscript{[}%tag
% <sup >\textsuperscript{4}
%endelem\textsuperscript{]}  
%endelem
%tag
% <sup >\textsuperscript{,}
%endelem
%tag
% <xref ref-type="bibr" rid="B23">\textsuperscript{[}%tag
% <sup >\textsuperscript{23}
%endelem\textsuperscript{]}  
%endelem

% <p >

Resultados negativos em saúde foram relatados pelo %tag
% <italic >
\textit{Institute of Medicine}
%endelem

        (\textsc{iom}) em 1999, %tag
% <xref ref-type="bibr" rid="B11">\textsuperscript{[}%tag
% <sup >\textsuperscript{11}
%endelem\textsuperscript{]}  
%endelem
 que estimou entre 44.000 a 98.000 mortes por ano nos Estados Unidos devido a erros
        na assistência ao paciente. Desde então, os resultados ou desfechos em saúde têm sido objeto
        de estudo, pois estão relacionados diretamente à qualidade e à segurança do paciente. A
        segurança do paciente é definida como o ato de evitar, prevenir ou melhorar os resultados
        adversos ou as lesões originadas no processo de atendimento médico-hospitalar. %tag
% <xref ref-type="bibr" rid="B21">\textsuperscript{[}%tag
% <sup >\textsuperscript{21}
%endelem\textsuperscript{]}  
%endelem

% <p >

Diante da mobilização mundial após a publicação desse impactante relatório, a Organização
        Mundial da Saúde (\textsc{oms}) lançou a Aliança Mundial para a Segurança do Paciente em 2004. %tag
% <xref ref-type="fn" rid="fn1">\footnote{\fn1}  
%endelem
 Isso despertou os países membros para o compromisso de desenvolver políticas
        públicas e práticas voltadas para a segurança do paciente, incluindo o Brasil.
% <p >

Na Europa, estimou-se que 10,8\% dos pacientes hospitalizados foram acometidos por eventos
        adversos, dos quais 46\% poderiam ter sido prevenidos. %tag
% <xref ref-type="bibr" rid="B22">\textsuperscript{[}%tag
% <sup >\textsuperscript{22}
%endelem\textsuperscript{]}  
%endelem
 No Brasil, estudo conduzido em hospitais do Rio de Janeiro estimou incidência de
        7,6\% desses eventos. %tag
% <xref ref-type="bibr" rid="B13">\textsuperscript{[}%tag
% <sup >\textsuperscript{13}
%endelem\textsuperscript{]}  
%endelem

% <p >

Apesar de o Ministério da Saúde e a Agência Nacional de Vigilância Sanitária (\textsc{anvisa})
        promoverem iniciativas da Aliança Mundial para a Segurança do Paciente da \textsc{oms}, como a
        campanha para introdução do protocolo de cirurgia segura nos hospitais, a adesão por parte
        dos serviços é baixa, justamente por não terem uma cultura institucional voltada para a
        segurança do paciente. Isso se reflete na alta ocorrência de eventos adversos evitáveis em
        hospitais brasileiros, que corresponde a cerca de 67\% de todos os eventos adversos. %tag
% <xref ref-type="bibr" rid="B13">\textsuperscript{[}%tag
% <sup >\textsuperscript{13}
%endelem\textsuperscript{]}  
%endelem
%tag
% <sup >\textsuperscript{,}
%endelem
%tag
% <xref ref-type="bibr" rid="B20">\textsuperscript{[}%tag
% <sup >\textsuperscript{20}
%endelem\textsuperscript{]}  
%endelem

% <p >

Embora o sistema de saúde brasileiro tenha aspectos positivos como a cobertura universal de
        vacinação e o sistema nacional de transplantes, a alta frequência de eventos adversos
        relacionados a medicamentos e infecções hospitalares é motivo de preocupação. %tag
% <xref ref-type="bibr" rid="B13">\textsuperscript{[}%tag
% <sup >\textsuperscript{13}
%endelem\textsuperscript{]}  
%endelem
%tag
% <sup >\textsuperscript{,}
%endelem
%tag
% <xref ref-type="bibr" rid="B20">\textsuperscript{[}%tag
% <sup >\textsuperscript{20}
%endelem\textsuperscript{]}  
%endelem
 Esses eventos são atribuídos à falta de políticas governamentais %tag
% <xref ref-type="fn" rid="fn2">\footnote{\fn2}  
%endelem
%tag
% <sup >\textsuperscript{,}
%endelem
%tag
% <xref ref-type="fn" rid="fn3">\footnote{\fn3}  
%endelem
%tag
% <sup >\textsuperscript{,}
%endelem
%tag
% <xref ref-type="fn" rid="fn4">\footnote{\fn4}  
%endelem
 que incentivem as instituições de saúde a participar de programas de qualidade e
        acreditação. %tag
% <xref ref-type="bibr" rid="B15">\textsuperscript{[}%tag
% <sup >\textsuperscript{15}
%endelem\textsuperscript{]}  
%endelem
%tag
% <sup >\textsuperscript{,}
%endelem
%tag
% <xref ref-type="bibr" rid="B16">\textsuperscript{[}%tag
% <sup >\textsuperscript{16}
%endelem\textsuperscript{]}  
%endelem
 Atualmente, há hospitais brasileiros que ainda são prestadores de serviços que atuam
        sem avaliar seus processos de trabalho ou usar seus resultados para a melhoria contínua da
        qualidade. %tag
% <xref ref-type="fn" rid="fn4">\footnote{\fn4}  
%endelem

% <p >

Faz-se necessário, portanto, conhecer a realidade brasileira quanto à ocorrência de
        incidentes, o que pode ser obtido com o envolvimento das instituições de saúde para que
        monitorizem essa ocorrência e o tratamento das informações pertinentes, além de notificá-las
        aos órgãos governamentais. Entretanto, a simples existência de um fluxo de informações
        organizado não gera conhecimento por si só. Esse só se dará por meio da ação de atores
        interdisciplinares e que interajam entre si. %tag
% <xref ref-type="bibr" rid="B14">\textsuperscript{[}%tag
% <sup >\textsuperscript{14}
%endelem\textsuperscript{]}  
%endelem

% <p >

O objetivo deste estudo foi suscitar reflexão acerca da utilização de um sistema nacional
        de notificações sobre incidentes como base para um programa brasileiro de segurança do
        paciente.

%endelem
%tag
% <sec >
\section{% <title >
Qualidade no \textsc{sus} e segurança do paciente}
% <p >

Em 2011, o Ministério da Saúde lançou um Projeto de Formação e Melhoria da Qualidade de
        Rede de Atenção à Saúde, o Quali\textsc{sus} Rede. %tag
% <xref ref-type="fn" rid="fn5">\footnote{\fn5}  
%endelem
 Apesar de ser um importante avanço para o desenvolvimento da qualidade do Sistema
        Único de Saúde (\textsc{sus}), o projeto não contempla incentivo à adoção de um programa de
        acreditação hospitalar e também não contempla objetivo estratégico diretamente relacionado à
        segurança do paciente, item considerado essencial para a qualidade, segundo o \textsc{iom} e a
        \textsc{oms}.
% <p >

Outra iniciativa do Ministério da Saúde é a monitorização do Índice de Desempenho do \textsc{sus}
        (\textsc{idsus}), que tem como objetivo aferir o desempenho do sistema de saúde quanto ao acesso –
        potencial ou obtido – e à efetividade da atenção básica, das atenções ambulatorial e
        hospitalar, e de urgências e emergência na esfera nacional. %tag
% <xref ref-type="fn" rid="fn6">\footnote{\fn6}  
%endelem
 Essa aferição é feita por meio de indicadores de qualidade.
% <p >

Dentre os indicadores estabelecidos no \textsc{idsus}, não há nenhum relacionado diretamente à
        segurança do paciente, como a taxa de incidentes ocorridos no atendimento de urgência e
        emergência. Por outro lado, no \textsc{idsus} os indicadores são abordados como a proporção de óbitos
        nas internações por infarto agudo do miocárdio, que calcula o risco de morrer por essa
        condição após a internação por tal causa, e estima, indiretamente, o atraso do atendimento
        pré-hospitalar e no diagnóstico. %tag
% <xref ref-type="fn" rid="fn4">\footnote{\fn4}  
%endelem
 Ainda que sejam poucos, pode-se considerar um avanço que esse tipo de indicador
        esteja sendo utilizado em um programa oficial do governo brasileiro.
% <p >

O projeto prevê repasse de verba diferenciado para aquelas regiões que atingirem níveis
        mais elevados de qualidade. Esse tipo de programa já é realizado com sucesso em outros
        países. Na Inglaterra e nos Estados Unidos, por exemplo, além de compartilhar os indicadores
        sobre segurança do paciente entre as instituições do país, com o objetivo de conhecer e
        estabelecer níveis de qualidade e segurança nas organizações hospitalares, aquelas que
        atingem níveis mais elevados são recompensadas com remuneração diferenciada. %tag
% <xref ref-type="bibr" rid="B3">\textsuperscript{[}%tag
% <sup >\textsuperscript{3}
%endelem\textsuperscript{]}  
%endelem
%tag
% <sup >\textsuperscript{,}
%endelem
%tag
% <xref ref-type="bibr" rid="B9">\textsuperscript{[}%tag
% <sup >\textsuperscript{9}
%endelem\textsuperscript{]}  
%endelem

% <p >

Esse modelo de pagamento por qualidade é conhecido como %tag
% <italic >
\textit{pay for
          performance}
%endelem
 (P4P) %tag
% <xref ref-type="bibr" rid="B7">\textsuperscript{[}%tag
% <sup >\textsuperscript{7}
%endelem\textsuperscript{]}  
%endelem
 e é alternativo àquele amplamente utilizado no Brasil, a remuneração
          %tag
% <italic >
\textit{fee-for-service}
%endelem
 (pagamento por serviço executado), que estimula a
        sobreutilização de recursos, especialmente as tecnologias em saúde, e que não traz garantias
        de que o custo adicional e a facilidade de acesso resultem numa efetiva melhoria da
        qualidade do nível de saúde da população atendida. %tag
% <xref ref-type="bibr" rid="B7">\textsuperscript{[}%tag
% <sup >\textsuperscript{7}
%endelem\textsuperscript{]}  
%endelem

% <p >

O P4P está em desenvolvimento em muitos países, incluindo o Brasil. Na Inglaterra, país
        modelo para o uso dessa ferramenta, os pagamentos contribuem com 30\% da renda de algumas
        clínicas. %tag
% <xref ref-type="bibr" rid="B7">\textsuperscript{[}%tag
% <sup >\textsuperscript{7}
%endelem\textsuperscript{]}  
%endelem
%tag
% <sup >\textsuperscript{,}
%endelem
%tag
% <xref ref-type="bibr" rid="B12">\textsuperscript{[}%tag
% <sup >\textsuperscript{12}
%endelem\textsuperscript{]}  
%endelem
 O que se espera do P4P é que os próprios usuários passem a escolher o serviço pelo
        qual desejam ser atendidos, com base em relatórios públicos de indicadores de desempenho.
        Constance et al %tag
% <xref ref-type="bibr" rid="B6">\textsuperscript{[}%tag
% <sup >\textsuperscript{6}
%endelem\textsuperscript{]}  
%endelem
 têm mostrado que a publicação desses relatórios é um bom mecanismo para a melhoria
        da qualidade na saúde.
% <p >

O Brasil ainda tem como desafios a alta rotatividade de profissionais de saúde nos serviços
        públicos, além da limitação qualitativa dos recursos humanos, do uso indevido das
        tecnologias e da baixa continuidade da atenção prestada aos pacientes. %tag
% <xref ref-type="bibr" rid="B1">\textsuperscript{[}%tag
% <sup >\textsuperscript{1}
%endelem\textsuperscript{]}  
%endelem
%tag
% <sup >\textsuperscript{,}
%endelem
%tag
% <xref ref-type="bibr" rid="B20">\textsuperscript{[}%tag
% <sup >\textsuperscript{20}
%endelem\textsuperscript{]}  
%endelem
 Ainda, pequeno número de hospitais brasileiros se dedica ao ensino e à pesquisa e
        não influencia a melhoria das práticas assistenciais em razão da desarticulação entre
        ensino, pesquisa e assistência, e da discreta utilização da saúde baseada em evidências na
        assistência ao paciente e das pesquisas sobre segurança do paciente que estão restritas a
        ilhas de excelência. %tag
% <xref ref-type="bibr" rid="B1">\textsuperscript{[}%tag
% <sup >\textsuperscript{1}
%endelem\textsuperscript{]}  
%endelem
%tag
% <sup >\textsuperscript{,}
%endelem
%tag
% <xref ref-type="bibr" rid="B16">\textsuperscript{[}%tag
% <sup >\textsuperscript{16}
%endelem\textsuperscript{]}  
%endelem
%tag
% <sup >\textsuperscript{,}
%endelem
%tag
% <xref ref-type="bibr" rid="B20">\textsuperscript{[}%tag
% <sup >\textsuperscript{20}
%endelem\textsuperscript{]}  
%endelem

% <p >

Diante do cenário exposto, no qual as políticas implementadas pelo Ministério da Saúde não
        têm sido suficientes para estimular o olhar crítico para a segurança do paciente, com
        estabelecimento de metas específicas para prevenir danos evitáveis e minimizar riscos de
        incidentes, propõe-se o desenvolvimento de um programa nacional de segurança do paciente que
        esteja vinculada aos programas de qualidade do governo federal. Tal programa deve envolver,
        no mínimo, o Ministério da Saúde, a \textsc{anvisa}, a Agência Nacional de Saúde Suplementar (\textsc{ans}) e
        o Ministério da Educação, sendo o último um importante aliado para a formação de
        profissionais de saúde, especialmente nos hospitais de ensino.
% <p >

O programa nacional de segurança do paciente faz-se necessário porque vem ao encontro do
        moderno conceito em saúde de prevenção quaternária, que objetiva a detecção de indivíduos em
        risco de intervencionismo excessivo em saúde, que implica atividades desnecessárias, e
        sugerir-lhes alternativas eticamente aceitáveis, atenuando ou evitando efeitos adversos.
          %tag
% <xref ref-type="bibr" rid="B2">\textsuperscript{[}%tag
% <sup >\textsuperscript{2}
%endelem\textsuperscript{]}  
%endelem
%tag
% <sup >\textsuperscript{,}
%endelem
%tag
% <xref ref-type="bibr" rid="B19">\textsuperscript{[}%tag
% <sup >\textsuperscript{19}
%endelem\textsuperscript{]}  
%endelem

% <p >

Essa abordagem é particularmente importante no Brasil, que teve o crescimento exponencial
        de novas tecnologias disponíveis no mercado de saúde na última década, especialmente após a
        criação da \textsc{anvisa}, tem marco legal muito recente sobre incorporação de tecnologias baseada
        em evidências %tag
% <xref ref-type="fn" rid="fn7">\footnote{\fn7}  
%endelem
 e está investindo em um modelo humanizado e orientado para a saúde. %tag
% <xref ref-type="bibr" rid="B1">\textsuperscript{[}%tag
% <sup >\textsuperscript{1}
%endelem\textsuperscript{]}  
%endelem

% <p >

O programa de segurança do paciente deve ser difundido nas diferentes instituições que
        compõem o sistema de saúde em todos os estados da federação a fim de que conheçam e
        compartilhem o conhecimento acerca dos resultados obtidos na assistência, incluindo os
        resultados negativos. Portanto, a implantação de um sistema nacional de notificações de
        incidentes deve ser uma das ações prioritárias de um programa nacional de segurança do
        paciente que contemple, minimamente, metas para gestão dos riscos envolvendo a assistência à
        saúde, tais como a identificação correta de pacientes, redução de infecções hospitalares,
        erros em procedimentos como cirurgias e medicação, que estão entre as chamadas nove soluções
        para a segurança do paciente, segundo a \textsc{oms}. %tag
% <xref ref-type="fn" rid="fn8">\footnote{\fn8}  
%endelem

%endelem
%tag
% <sec >
\section{% <title >
Sistema de notificações de incidentes}
% <p >

As notificações por parte dos profissionais de saúde, pacientes e seus cuidadores são
        importantes para a identificação de incidentes em saúde, especialmente por ser um método de
        baixo custo e, principalmente, por envolver os profissionais que prestam assistência em uma
        política de melhoria contínua centrada no paciente.
% <p >

Para garantir a produção de informação nas instituições de saúde para a tomada de decisões
        e a responsabilização com a melhoria de qualidade, é condição essencial que sejam feitos
        investimentos no desenvolvimento de capacidades locais e nos sistemas de informação já
        existentes. %tag
% <xref ref-type="bibr" rid="B10">\textsuperscript{[}%tag
% <sup >\textsuperscript{10}
%endelem\textsuperscript{]}  
%endelem

% <p >

A experiência da \textsc{anvisa} com a Rede Sentinela é um bom exemplo de que a interação entre
        governo e instituições de saúde é possível e pode promover o desenvolvimento dos serviços em
        prol da segurança e da qualidade, seja pela cultura do relato voluntário, seja pela adesão
        aos programas de qualidade.
% <p >

O número de notificações encaminhadas pelos hospitais integrantes da Rede Sentinela ao
        Sistema de Notificações em Vigilância Sanitária (\textsc{notivisa}) aumentou em 48,8\% após o primeiro
        ano de implantação do sistema, quando comparado ao ano anterior. O estímulo da \textsc{anvisa} para
        que os hospitais da Rede participassem de programas de qualidade, acreditação ou similar
        influenciou para que 30\% desses hospitais estivessem participando de algum programa desse
        tipo em 2008. %tag
% <xref ref-type="fn" rid="fn3">\footnote{\fn3}  
%endelem

% <p >

Embora a iniciativa da \textsc{anvisa} tenha sido importante para o estímulo da qualidade nos
        hospitais, ela é uma pequena parcela entre as mais de 8.000 instituições hospitalares
        brasileiras, correspondendo apenas a cerca de 13\% dos leitos hospitalares no País. %tag
% <xref ref-type="fn" rid="fn3">\footnote{\fn3}  
%endelem

% <p >

Por esse motivo, o papel dos órgãos governamentais que recebem as informações sobre os
        resultados em saúde é fundamental, cabendo a eles ações que promovam a melhoria em curto
        espaço de tempo a fim de evitar danos aos pacientes. A utilização de sistemas informatizados
        em plataforma web, ou seja, disponíveis na internet para envio e recebimento imediatos, é
        passo fundamental para que um país de grande extensão territorial como o Brasil desenvolva
        um programa nacional de segurança do paciente. Além disso, deve-se estabelecer um modelo
        brasileiro de pagamento por desempenho a fim de beneficiar as instituições que estejam
        comprometidas como o modelo de melhoria contínua da qualidade.
% <p >

O modelo para o sistema nacional de notificações de incidentes pode ser útil no
        desenvolvimento da cultura de segurança do paciente no \textsc{sus}, conforme mostra a %tag
% <xref ref-type="fig" rid="f01">
Figura~\ref{fig:f01}
%endelem
 .
% <p >

%tag
% <fig id="f01">
\begin{figure}
\includegraphics[width=\textwidth]{% <graphic xlink:href="0034-8910-rsp-47-04-0791-gf01">
0034-8910-rsp-47-04-0791-gf01}
\caption{}\label{fig:f01}
\end{figure}
%endelem

% <p >

O desenvolvimento e implementação de um sistema informatizado único para receber
        notificações de todas as instituições de saúde deverá ter como objetivos facilitar e
        agilizar o processo de envio e de tomada de decisões a partir da notificação, minimizando
        riscos e evitando eventos adversos, ampliando a qualidade da assistência e a segurança dos
        pacientes em todos os níveis, da menor clínica de atenção à saúde ao sistema de saúde
        brasileiro; ampliar o conhecimento sobre os riscos e incidentes que ocorrem nas instituições
        brasileiras, direcionando o planejamento de ações dos gestores de saúde; melhorar a
        qualidade dos dados encaminhados; garantir a legibilidade das informações disponíveis;
        preservar a confidencialidade dos notificadores e dados relatados; e, por fim, reduzir
        custos do processo de notificação.
% <p >

À medida que o sistema seja utilizado com maior frequência e eficácia para o envio
        voluntário de notificações de incidentes, especialmente os potenciais eventos adversos, a
        tomada de decisões quanto às intervenções necessárias para evitar a ocorrência de danos
        poderá ser mais rápida, o que pode reduzir gastos desnecessários com tratamento de eventos
        adversos que poderiam ter sido prevenidos.
% <p >

Para tanto, a autonomia e pró-atividade das instituições de saúde deve ser estimulada. Até
        que haja a tomada de decisão por parte do governo, as instituições também devem realizar
        ações de melhoria internas, visando à promoção da segurança do paciente e à qualidade da
        atenção. Nesse sentindo, os estabelecimentos de saúde deverão ter acesso ao sistema
        informatizado não somente para o envio de notificações da gerência de riscos para o sistema
        nacional, mas também para que essa gerência se utilize do sistema para receber as
        notificações da equipe de saúde de sua instituição. Adicionalmente, o sistema deve permitir
        que a instituição acompanhe o andamento da análise das informações por ela encaminhadas ao
        sistema nacional. \textsc{imageminline}.
% <p >

O sistema informatizado é uma importante estratégia de promoção da qualidade aliada à
        sustentabilidade, pois, ao deixar de utilizar papéis, reduz gastos com materiais de consumo
        e geração de resíduos como os próprios papéis, cartuchos de impressoras e canetas.
        Adicionalmente, há outros aspectos que justificam a implantação de sistemas informatizados
        de notificação, a saber: %tag
% <xref ref-type="bibr" rid="B5">\textsuperscript{[}%tag
% <sup >\textsuperscript{5}
%endelem\textsuperscript{]}  
%endelem

% <p >

elimina a necessidade de utilização de sistemas de envio de documentos internos às
            instituições e destas para o sistema nacional de notificações, o que reduz o tempo da
            chegada da informação e reduz gastos com o envio delas;
% <p >

pode-se eliminar a possibilidade de extravio e perda de informações, especialmente se
            estas forem preservadas em bancos de dados redundantes e cópias de segurança, sem
            necessidade de espaço para arquivo físico, além de permitir o manuseio ágil das
            informações e a análise de indicadores de gestão;
% <p >

é possível requerer mais informações sobre os incidentes sem dificultar a coleta dos
            dados, melhorando a qualidade das informações e ampliando a participação dos
            profissionais de saúde, o que não é possível com o sistema manuscrito.
% <p >

Quanto aos aspectos sociais da sustentabilidade, reduzir o tempo gasto para o envio do
        relato aumenta a participação dos profissionais de saúde com as notificações, bem como sua
        disponibilidade junto aos pacientes possibilitam o envolvimento do paciente e seus
        cuidadores no processo de monitorização de riscos e incidentes em saúde. Em uma política
        nacional, esses atores são fontes importantes de notificação voluntária. Com o sistema
        informatizado de notificações em plataforma web, é possível que qualquer pessoa com acesso à
        internet faça uma notificação.
% <p >

Países que já possuem política nacional de segurança do paciente, como Inglaterra, Estados
        Unidos, Austrália e Canadá, já permitem que os usuários do sistema e seus cuidadores façam
        notificações sobre riscos e incidentes que vivenciaram ou perceberam em serviços de saúde,
        sendo fundamentais para promoção da qualidade da assistência.
% <p >

O sistema nacional de notificações deveria, ainda, estar hospedado em um site interativo
        que disponibilizasse gratuitamente notícias, dicas de segurança, informações sobre eventos
        adversos, protocolos sobre como implementar um programa de segurança nos serviços de saúde,
        cursos e palestras %tag
% <italic >
\textit{online}
%endelem, como o %tag
% <italic >
\textit{Institute for Health
          Improvement}
%endelem
 tem feito nos Estados Unidos para os hospitais americanos. Esse
        portal teria como dupla função manter e estimular a adesão das instituições, e difundir e
        estimular a adoção de práticas seguras por meio do intercâmbio entre elas.
% <p >

A implantação do sistema informatizado de notificações sobre incidentes na saúde como base
        para a cultura de segurança do paciente no sistema de saúde brasileiro parece ser uma
        estratégia viável e necessária para a qualificação da assistência, com a qual os gestores
        conhecerão os incidentes que ocorrem na prestação de assistência aos usuários do sistema, em
        instituições públicas e privadas, de forma sistematizada, sem depender de que pesquisas
        sejam realizadas exclusivamente para esse fim. Desse modo, nortear-se-á o delineamento de
        estratégias de gestão de riscos para a segurança do paciente, ampliando a qualidade dos
        serviços ofertados à população brasileira.

% <list list-type="bullet">
\begin{itemize}
%tag
% <list-item >
\item %tag
% <p >

elimina a necessidade de utilização de sistemas de envio de documentos internos às
            instituições e destas para o sistema nacional de notificações, o que reduz o tempo da
            chegada da informação e reduz gastos com o envio delas;
%endelem

%endelem
%tag
% <list-item >
\item %tag
% <p >

pode-se eliminar a possibilidade de extravio e perda de informações, especialmente se
            estas forem preservadas em bancos de dados redundantes e cópias de segurança, sem
            necessidade de espaço para arquivo físico, além de permitir o manuseio ágil das
            informações e a análise de indicadores de gestão;
%endelem

%endelem
%tag
% <list-item >
\item %tag
% <p >

é possível requerer mais informações sobre os incidentes sem dificultar a coleta dos
            dados, melhorando a qualidade das informações e ampliando a participação dos
            profissionais de saúde, o que não é possível com o sistema manuscrito.
%endelem

%endelem
 \end{itemize}

% <fig id="f01">
\begin{figure}
\includegraphics[width=\textwidth]{% <graphic xlink:href="0034-8910-rsp-47-04-0791-gf01">
0034-8910-rsp-47-04-0791-gf01}
\caption{}\label{fig:f01}
\end{figure}

%endelem
%endelem
%tag
% <back >
%tag
% <app-group >
%tag
% <p >

\textsc{destaques}
%endelem
%tag
% <p >

Recentemente, foi lançado pelo Ministério da Saúde o Programa Nacional de Segurança do
        Paciente para que ações de segurança do paciente fossem promovidas no âmbito do Sistema
        Único de Saúde. É louvável essa iniciativa e foi o maior foco de discussão do artigo, o qual
        aborda importantes pontos que não foram incluídos no citado Programa. Esses pontos
        referem-se à forma com que o Governo pretende remunerar as instituições que obtiverem os
        melhores resultados na prestação de serviços e também como serão tratadas as informações
        advindas de notifi cações, de forma que o conhecimento gerado promova a melhoria efetiva dos
        serviços prestados pelo Sistema Único de Saúde.
%endelem
%tag
% <p >

O artigo faz refl exão à luz da melhoria da política de saúde para a segurança do paciente,
        onde foi possível verificar que ações para a cultura de segurança podem proliferar e gerar
        bons resultados também em hospitais públicos. Assim, traz à luz discussões importantes para
        todas as instituições e especialmente para gestores do sistema de forma a aprimorar o
        Programa Nacional e desencadear a melhoria contínua dos serviços centrada no paciente.
%endelem
%tag
% <p >

Profa. Rita de Cássia Barradas Barata
%endelem
%tag
% <p >

Editora Científica
%endelem
%endelem
%tag
% <ref-list >

\section*{% <title >
\textsc{referências}}
\begin{itemize}
% <ref id="B1">

\item[% <label >
1] % <mixed-citation >. Almeida-Filho N. Ensino superior e os serviços de saúde no Brasil.
            %tag
% <italic >
\textit{Lancet}
%endelem
 . 2011;6-7.  
% <ref id="B2">

\item[% <label >
2] % <mixed-citation >. Bentzen N. \textsc{wonca} dictionary of general/family practice. Copenhagen:
          Maanedskift Lager; 2003.  
% <ref id="B3">

\item[% <label >
3] % <mixed-citation >. Berlowitz D, Burgess Jr JF, Young GJ. Improving quality of care: emerging
          evidence on pay-for-performance. %tag
% <italic >
\textit{Med Care Res Rev}
%endelem
 . 2006;63(1
          Suppl):73S-95S.  
% <ref id="B4">

\item[% <label >
4] % <mixed-citation >. Capucho HC. Near miss: quase erro ou potencial evento adverso? %tag
% <italic >
\textit{Rev
            Latino-Am Enferm}
%endelem
 . 2011;19(5):1272-3.
          \textsc{doi}:10.1590/S0104-11692011000500027  
% <ref id="B5">

\item[% <label >
5] % <mixed-citation >. Capucho HC, Arnas ER, Cassiani \textsc{shbd}. Segurança do paciente: comparação
          entre notificações voluntárias manuscritas e informatizadas sobre incidentes em saúde.
            %tag
% <italic >
\textit{Rev Gaucha Enferm}
%endelem
 . 2013;34(1):164-72.
          \textsc{doi}:10.1590/S1983-14472013000100021  
% <ref id="B6">

\item[% <label >
6] % <mixed-citation >. Constance HF, Yee Wei L, Mattke S, Damberg C, Shekelle PG. Systematic
          Review: The Evidence That Publishing Patient Care Performance Data Improves Quality of
          Care. %tag
% <italic >
\textit{Ann Intern Med}
%endelem
 . 2008;15(148):111-123.  
% <ref id="B7">

\item[% <label >
7] % <mixed-citation >. Escrivao Jr A, Koyama MF. O relacionamento entre hospitais e operadoras de
          planos de saúde no âmbito do Programa de Qualificação da Saúde Suplementar da \textsc{ans}.
            %tag
% <italic >
\textit{Cienc Saude Coletiva}
%endelem
 . 2007;12(4):903-14.
          \textsc{doi}:10.1590/S1413-81232007000400012  
% <ref id="B8">

\item[% <label >
8] % <mixed-citation >. Fisher ES. Paying for Performance - Risks and Recommendations. %tag
% <italic >
\textit{New
            Eng J Med}
%endelem
 . 2006;355(18):1845-7. \textsc{doi}:10.1056/\textsc{nejm}p068221  
% <ref id="B9">

\item[% <label >
9] % <mixed-citation >. Fung CH, Lim YW, Mattke S, Damberg C, Shekelle PG. Systematic Review: The
          Evidence That Publishing Patient Care Performance Data Improves Quality of Care.
            %tag
% <italic >
\textit{Ann Intern Med}
%endelem
 . 2008;148(2):111-23.
          \textsc{doi}:10.7326/0003-4819-148-2-200801150-00006  
% <ref id="B10">

\item[% <label >
10] % <mixed-citation >. Gouvêa \textsc{csdd}, Travassos C. Indicadores de segurança do paciente para
          hospitais de pacientes agudos: revisão sistemática. %tag
% <italic >
\textit{Cad Saude Publica}
%endelem
 .
          2010;26(6):1061-78. \textsc{doi}:10.1590/S0102-311X2010000600002  
% <ref id="B11">

\item[% <label >
11] % <mixed-citation >. Kohn LT, Corrigan JM, Donaldson MS. To err is human: building a safer
          health system. 2.ed. Washington: National Academy of Sciences; 1999.  
% <ref id="B12">

\item[% <label >
12] % <mixed-citation >. McDonald R, Roland M. Pay for performance in primary care in England and
          California: comparison of unintended consequences. %tag
% <italic >
\textit{Ann Fam Med}
%endelem
 .
          2009;7(2):121-7. \textsc{doi}:10.1370/afm.946  
% <ref id="B13">

\item[% <label >
13] % <mixed-citation >. Mendes W, Martins M, Rozenfeld S, Travassos C. The assessment of adverse
          events in hospitals in Brazil. %tag
% <italic >
\textit{Int J Qual Health Care}
%endelem
 .
          2009;21(4):279-84. \textsc{doi}:10.1093/intqhc/mzp022  
% <ref id="B14">

\item[% <label >
14] % <mixed-citation >. Miasso, AI, Grou CR, Cassiani \textsc{shb}, Silva \textsc{aebc}, Fakih FT. Erros de
          medicação: tipos, fatores causais e providencias em quatro hospitais brasileiros.
            %tag
% <italic >
\textit{Rev Esc Enferm \textsc{usp}}
%endelem
 . 2006:40(4):524-32.
          \textsc{doi}:10.1590/S0080-62342006000400011  
% <ref id="B15">

\item[% <label >
15] % <mixed-citation >. Novaes HM. O processo de acreditação dos serviços de saúde. %tag
% <italic >
\textit{Rev
            Adm Saude}
%endelem
 . 2007;9(37):133-40.  
% <ref id="B16">

\item[% <label >
16] % <mixed-citation >. Paim J, Travassos C, Almeida C, Bahia L, Macinko J. O sistema de saúde
          brasileiro: história, avanços e desafios. %tag
% <italic >
\textit{Lancet}
%endelem
 .
          2011;11-31.  
% <ref id="B17">

\item[% <label >
17] % <mixed-citation >. Shoyer AL, London MJ, VillaNueva CB, Sethi GK, Marshall G, Moritz TE, et
          al. The processes, structures, and outcomes of care in cardiac surgery study an overview.
            %tag
% <italic >
\textit{Med Care}
%endelem
 . 1995;33(10):OS1-4.
          \textsc{doi}:10.1097/00005650-199510001-00001  
% <ref id="B18">

\item[% <label >
18] % <mixed-citation >. Thomas AN, Panchagnula U. Medication-related patient safety incidents in
          critical care: a review of reports to the UK National Patient Safety Agency.
            %tag
% <italic >
\textit{Anaesthesia}
%endelem
 . 2008;63(7):726-33.
          \textsc{doi}:10.1111/j.1365-2044.2008.05485.x  
% <ref id="B19">

\item[% <label >
19] % <mixed-citation >. Unruh LY, Zhang NJ. Nurse Staffing and patient safety in hospitals: new
          variable and longitudinal approaches. %tag
% <italic >
\textit{Nurs Res}
%endelem
 . 2012;61(1):3-12.
          \textsc{doi}:10.1097/\textsc{nnr}.0b013e3182358968  
% <ref id="B20">

\item[% <label >
20] % <mixed-citation >. Victora CG, Barreto ML, Leal MC, Monteiro CA, Schmidt MI, Paim J, et al.
          Condições de saúde e inovações nas políticas de saúde no Brasil: o caminho a percorrer.
            %tag
% <italic >
\textit{Lancet}
%endelem
 . 2011;90-102.  
% <ref id="B21">

\item[% <label >
21] % <mixed-citation >. Vincent C. Segurança do paciente. Orientações para evitar eventos
          adversos. São Caetano do Sul: Editora Yendis; 2009.  
% <ref id="B22">

\item[% <label >
22] % <mixed-citation >. Vincent C, Woloshynowych M. Adverse events in British hospitals:
          preliminary retrospective record review. %tag
% <italic >
\textit{\textsc{bmj}}
%endelem
 . 2001;322(7285):517-9.
          \textsc{doi}:10.1136/bmj.322.7285.517  
% <ref id="B23">

\item[% <label >
23] % <mixed-citation >. World Health Organization. The conceptual framework for the international
          classification for patient safety. Version 1.1. Final technical report. Chapter 3. The
          international classification for patient safety. Key concepts and preferred terms. Geneva;
          2009 [citado 2011 jul 04]. Disponível em:
          http://www.who.int/patientsafety/taxonomy/icps\_{}chapter3.pdf  
 
\end{itemize}
%endelem
%tag
% <fn-group >
%endelem
%endelem
%tag
% <sub-article article-type="translation" id="TR01" xml:lang="en">
%endelem

   \section*{Metadados não aplicados}
    \begin{itemize}
    \ifdef{\lingua}{\item[\textbf{língua do artigo}] \lingua}{}
    \ifdef{\journalid}{\item[\textbf{journalid}] \journalid}{}
    \ifdef{\journaltitle}{\item[\textbf{journaltitle}] \journaltitle}{}
    \ifdef{\journalsubtitle}{\item[\textbf{journalsubtitle}] \journalsubtitle}{}
    \ifdef{\historydateaccepted}{\item[\textbf{historydateaccepted}] \historydateaccepted}{}
    \ifdef{\historydatereceived}{\item[\textbf{historydatereceived}] \historydatereceived}{}
    \ifdef{\ack}{\item[\textbf{ack}] \ack}{}
    \ifdef{\transjournaltitle}{\item[\textbf{journaltitle}] \journaltitle}{}
    \ifdef{\transjournalsubtitle}{\item[\textbf{journalsubtitle}] \journaltitle}{}
    \ifdef{\abbrevjournaltitle}{\item[\textbf{abbrevjournaltitle}] \abbrevjournaltitle}{}
    \ifdef{\issnppub}{\item[\textbf{issnppub}] \issnppub}{}
    \ifdef{\issnepub}{\item[\textbf{issnepub}] \issnepub}{}
    \ifdef{\alttitleauthor}{\item[\textbf{alttitle}] \alttitleauthor}{}
    \ifdef{\alttitle}{\item[\textbf{alttitleauthor}] \alttitle}{}
    \ifdef{\publishername}{\item[\textbf{publishername}] \publishername}{}
    \ifdef{\publisherid}{\item[\textbf{publisherid}] \publisherid}{}
    \ifdef{\subject}{\item[\textbf{subject}] \subject}{} 
    \ifdef{\transtitle}{\item[\textbf{transtitle}] \transtitle}{}
    \ifdef{\authornotes}{\item[\textbf{authornotes}] \authornotes}{}
    \ifdef{\articleid}{\item[\textbf{articleid}] \articleid}{}
    \ifdef{\articledoi}{\item[\textbf{articledoi}] \articledoi}{}
    \ifdef{\volume}{\item[\textbf{volume}] \volume}{}
    \ifdef{\issue}{\item[\textbf{issue}] \issue}{}
    \ifdef{\fpage}{\item[\textbf{fpage}] \fpage}{}
    \ifdef{\lpage}{\item[\textbf{lpage}] \lpage}{}
    \ifdef{\permissions}{\item[\textbf{permissions}] \permissions}{}
    \ifdef{\copyrightyear}{\item[\textbf{copyrightyear}] \copyrightyear}{}

    \end{itemize}
\end{document}%tag
% </ transfer-MimeType="text/xml" transfer-Status="200" transfer-Message="OK" transfer-\textsc{uri}="file:///home/editorial/Dropbox/\textsc{scielo}/\textsc{sps}+TeX/proofs/001/0034-8910-rsp-47-04-0791.xml" source="0034-8910-rsp-47-04-0791.xml" transfer-Encoding="\textsc{utf}-8">
%endelem
