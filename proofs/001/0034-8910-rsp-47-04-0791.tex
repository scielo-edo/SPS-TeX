% Generated by jats2tex@0.11.1.0
\documentclass{article}

\usepackage[T1]{fontenc}
\usepackage[utf8]{inputenc} %% *
\usepackage[portuges,spanish,english]{babel} %% *
\usepackage{amstext}
\usepackage{authblk}
\usepackage{unicode-math}
\usepackage{multirow}
\usepackage{graphicx}
\usepackage{etoolbox}
\usepackage[footnotesize,bf,hang]{caption}
\usepackage[nodayofweek,level]{datetime}
\selectlanguage{portuges}

\newcommand{\journalid}{Rev Saude Publica}
\newcommand{\journaltitle}{Revista de Saúde Pública}
\newcommand{\abbrevjournaltitle}{Rev. Saúde Pública}
\newcommand{\issnppub}{0034-8910}
\newcommand{\issnepub}{1518-8787}
\newcommand{\publishername}{Faculdade de Saúde Pública da Universidade de São
Paulo}
\def\articleid{DOI S0034-8910.2013047004402}
\def\articleid{DOI 10.1590/S0034-8910.2013047004402}
\def\subject{Prática de Saúde Pública}
\title{Necessidade de implantar programa nacional de segurança do
paciente no Brasil}
\newcommand{\transtitle}{Necesidad de implantar programa nacional de seguridad
del paciente en
Brasil}
\author[I]{Capucho, Helaine Carneiro
I}
\author[II]{Cassiani, Silvia Helena De Bortoli
II}\affil[I]{Departamento de Gestão e Incorporação de Tecnologias em
SaúdeSecretaria de Ciência, Tecnologia e Insumos
EstratégicosMinistério da Saúde}\affil[II]{Departamento de Enfermagem Geral e
EspecializadaEscola de Enfermagem de Ribeirão PretoUniversidade de São Paulo}
\def\authornotes{Correspondência | Correspondence : Helaine Carneiro Capucho.
Esplanada dos
Ministérios. Bloco G Edifício Sede 9º andar Sala 949. 70058-900 Brasília, DF,
Brasil.
E-mail: helaine.capucho@saude.gov.br
Os autores declaram não haver conflito de interesses.
}
\date{\selectlanguage{portuges}\formatdate{21}{08}{2013}}
\def\volume{47}
\def\issue{4}
\def\fpage{791}
\def\lpage{798}
\def\permissions{This is an Open Access article distributed under the terms of
the Creative
Commons Attribution Non-Commercial License, which permits unrestricted
non-commercial
use, distribution, and reproduction in any medium, provided the original work is
properly cited.}
\newcommand{\kwdgrouptrans}{
\selectlanguage{portuges}
\medskip\noindent\textbf{Palavras-chave}: {Segurança do Paciente, Avaliação de
Programas e Projetos de Saúde, Sistema Único de Saúde, Garantia da Qualidade dos
Cuidados de Saúde}}
\def\subject{Public Health Practice}
\title{The need to establish a national patient safety program in
Brazil}
\author[I]{Capucho, Helaine Carneiro
I}
\author[II]{Cassiani, Silvia Helena De Bortoli
II}\affil[I]{}\affil[II]{}
\def\authornotes{Correspondence : Helaine Carneiro Capucho Esplanada dos
Ministérios Bloco G
Edifício Sede 9º andar Sala 949 70058-900 Brasília, DF, Brasil E-mail:
helaine.capucho@saude.gov.br
The authors declare that there are no conflicts of interest.
}
\newcommand{\kwdgroup}{
\selectlanguage{portuges}
\medskip\noindent\textbf{Palavras-chave}: {Patient Safety, Program Evaluation,
Unified Health System, Quality Assurance, Health Care}}

\begin{document}

%%%%%%%%%%%%%%%%%%%%%%%%%%%%%%%%%
\section*{Metadados não aplicados}
\begin{itemize}
\ifdef{\journalid}{\item[\textbf{journalid}] \journalid}{}
\ifdef{\journaltitle}{\item[\textbf{journaltitle}] \journaltitle}{}
\ifdef{\abbrevjournaltitle}{\item[\textbf{abbrevjournaltitle}]
\abbrevjournaltitle}{}
\ifdef{\issnppub}{\item[\textbf{issnppub}] \issnppub}{}
\ifdef{\issnepub}{\item[\textbf{issnepub}] \issnepub}{}
\ifdef{\publishername}{\item[\textbf{publishername}] \publishername}{}
\ifdef{\subject}{\item[\textbf{subject}] \subject}{}
\ifdef{\transtitle}{\item[\textbf{transtitle}] \transtitle}{}
\ifdef{\authornotes}{\item[\textbf{authornotes}] \authornotes}{}
\ifdef{\articleid}{\item[\textbf{articleid}] \articleid}{}
\ifdef{\volume}{\item[\textbf{volume}] \volume}{}
\ifdef{\issue}{\item[\textbf{issue}] \issue}{}
\ifdef{\fpage}{\item[\textbf{fpage}] \fpage}{}
\ifdef{\lpage}{\item[\textbf{lpage}] \lpage}{}
\ifdef{\permissions}{\item[\textbf{permissions}] \permissions}{}
\end{itemize}
%%%%%%%%%%%%%%%%%%%%%%%%%%%%%%%%%

\maketitle

\selectlanguage{portuges}
% \renewcommand{\abstractname}{}
\begin{abstract}

O objetivo do estudo foi suscitar reflexão acerca da necessidade de se criar um
sistema
nacional de notificações sobre incidentes como base para um programa de
segurança do
paciente. Incidentes em saúde acarretam danos aos pacientes e oneram o sistema
de saúde.
Embora tenha lançado recentemente um programa de avaliação da qualidade nas
instituições
de saúde, o Ministério da Saúde, Brasil, ainda não possui um programa que avalie
sistematicamente os resultados negativos da assistência. Discute-se a
necessidade de se
implementar programa brasileiro de segurança do paciente, a fim de promover a
cultura pela
segurança do paciente e da qualidade em saúde no Sistema Único de Saúde.

\ifdef{\kwdgroup}{\kwdgroup}{}
\end{abstract}
%%%%%%%%%%%%%%%%%%%%%%%%%%%%%%%%%%%%%%%%%%%%%%%%%%%%%%%%%%%

\selectlanguage{spanish}
% \renewcommand{\abstractname}{}
\begin{abstract}

El objetivo del estudio fue suscitar reflexión sobre la necesidad de crear un
sistema
nacional de notificaciones sobre incidentes como base para un programa brasileño
de
seguridad del paciente. Incidentes en salud generaron daños a los pacientes y
sobrecargan
el sistema de salud. A pesar de que se haya lanzado recientemente un programa de
evaluación de la calidad en las instituciones de salud, el Ministerio de Salud
Brasileño
no posee aún un programa que evalúe sistemáticamente los resultados negativos de
la
asistencia. Se discute la necesidad de implementar un programa brasileño de
seguridad del
paciente, con el fin de promover la cultura por la seguridad del paciente y la
calidad de
la salud en el Sistema Único de Salud.

\ifdef{\kwdgrouptrans}{\kwdgrouptrans}{}
\end{abstract}
%%%%%%%%%%%%%%%%%%%%%%%%%%%%%%%%%%%%%%%%%%%%%%%%%%%%%%%%%%%

\selectlanguage{portuges}
% \renewcommand{\abstractname}{}
\begin{abstract}

The aim of the study was to promote reflection on the need to create a national
incident
notification system based on a brazilian patient safety program. Incidents in
health care
harm patients and encumber the health care system. Although a quality assessment
program
has been recently launched in health care institutions, the Brazilian Ministry
of Health
does not yet have a program which systematically assesses negative outcomes of
health
care. This article discusses the need to establish a national patient safety
program in
Brazil, aiming to promote a culture of patient safety and quality health care in
the
Brazilian Unified Health System.

\ifdef{\kwdgroup}{\kwdgroup}{}
\end{abstract}
%%%%%%%%%%%%%%%%%%%%%%%%%%%%%%%%%%%%%%%%%%%%%%%%%%%%%%%%%%%

\end{document}
