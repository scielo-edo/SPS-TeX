% Generated by jats2tex@0.11.1.0
\documentclass{article}

\usepackage[T1]{fontenc}
\usepackage[utf8]{inputenc} %% *
\usepackage[portuges,spanish,english]{babel} %% *
\usepackage{amstext}
\usepackage{authblk}
\usepackage{unicode-math}
\usepackage{multirow}
\usepackage{graphicx}
\usepackage{etoolbox}
\usepackage[footnotesize,bf,hang]{caption}
\usepackage[nodayofweek,level]{datetime}
\selectlanguage{portuges}

\newcommand{\journalid}{Rev Saude Publica}
\newcommand{\journaltitle}{Revista de Saúde Pública}
\newcommand{\abbrevjournaltitle}{Rev. Saúde Pública}
\newcommand{\issnppub}{0034-8910}
\newcommand{\issnepub}{1518-8787}
\newcommand{\publishername}{Faculdade de Saúde Pública da Universidade de São
Paulo}
\def\articleid{DOI S0034-8910.2013047004402}
\def\articleid{DOI 10.1590/S0034-8910.2013047004402}
\def\subject{Prática de Saúde Pública}
\title{Necessidade de implantar programa nacional de segurança do
paciente no Brasil}
\newcommand{\transtitle}{Necesidad de implantar programa nacional de seguridad
del paciente en
Brasil}
\author[I]{Capucho, Helaine Carneiro
I}
\author[II]{Cassiani, Silvia Helena De Bortoli
II}\affil[I]{Departamento de Gestão e Incorporação de Tecnologias em
SaúdeSecretaria de Ciência, Tecnologia e Insumos
EstratégicosMinistério da Saúde}\affil[II]{Departamento de Enfermagem Geral e
EspecializadaEscola de Enfermagem de Ribeirão PretoUniversidade de São Paulo}
\def\authornotes{Correspondência | Correspondence : Helaine Carneiro Capucho.
Esplanada dos
Ministérios. Bloco G Edifício Sede 9º andar Sala 949. 70058-900 Brasília, DF,
Brasil.
E-mail: helaine.capucho@saude.gov.br
Os autores declaram não haver conflito de interesses.
}
\date{\selectlanguage{portuges}\formatdate{21}{08}{2013}}
\def\volume{47}
\def\issue{4}
\def\fpage{791}
\def\lpage{798}
\def\permissions{This is an Open Access article distributed under the terms of
the Creative
Commons Attribution Non-Commercial License, which permits unrestricted
non-commercial
use, distribution, and reproduction in any medium, provided the original work is
properly cited.}
\newcommand{\kwdgrouptrans}{
\selectlanguage{portuges}
\medskip\noindent\textbf{Palavras-chave}: {Segurança do Paciente, Avaliação de
Programas e Projetos de Saúde, Sistema Único de Saúde, Garantia da Qualidade dos
Cuidados de Saúde}}

\begin{document}

%%%%%%%%%%%%%%%%%%%%%%%%%%%%%%%%%
\section*{Metadados não aplicados}
\begin{itemize}
\item[\textbf{journalid}] \journalid
\item[\textbf{journaltitle}] \journaltitle
\item[\textbf{abbrevjournaltitle}] \abbrevjournaltitle
\item[\textbf{issnppub}] \issnppub
\item[\textbf{issnepub}] \issnepub
\item[\textbf{publishername}] \publishername
\item[\textbf{subject}] \subject
\item[\textbf{transtitle}] \transtitle
\item[\textbf{authornotes}] \authornotes
\item[\textbf{articleid}] \articleid
\item[\textbf{volume}] \volume
\item[\textbf{issue}] \issue
\item[\textbf{fpage}] \fpage
\item[\textbf{lpage}] \lpage
\item[\textbf{permissions}] \permissions
\end{itemize}
%%%%%%%%%%%%%%%%%%%%%%%%%%%%%%%%%

\maketitle

\selectlanguage{portuges}
% \renewcommand{\abstractname}{}
\begin{abstract}

O objetivo do estudo foi suscitar reflexão acerca da necessidade de se criar um
sistema
nacional de notificações sobre incidentes como base para um programa de
segurança do
paciente. Incidentes em saúde acarretam danos aos pacientes e oneram o sistema
de saúde.
Embora tenha lançado recentemente um programa de avaliação da qualidade nas
instituições
de saúde, o Ministério da Saúde, Brasil, ainda não possui um programa que avalie
sistematicamente os resultados negativos da assistência. Discute-se a
necessidade de se
implementar programa brasileiro de segurança do paciente, a fim de promover a
cultura pela
segurança do paciente e da qualidade em saúde no Sistema Único de Saúde.

\ifdef{\kwdgroup}{\kwdgroup}{}
\end{abstract}
%%%%%%%%%%%%%%%%%%%%%%%%%%%%%%%%%%%%%%%%%%%%%%%%%%%%%%%%%%%

\selectlanguage{spanish}
% \renewcommand{\abstractname}{}
\begin{abstract}

El objetivo del estudio fue suscitar reflexión sobre la necesidad de crear un
sistema
nacional de notificaciones sobre incidentes como base para un programa brasileño
de
seguridad del paciente. Incidentes en salud generaron daños a los pacientes y
sobrecargan
el sistema de salud. A pesar de que se haya lanzado recientemente un programa de
evaluación de la calidad en las instituciones de salud, el Ministerio de Salud
Brasileño
no posee aún un programa que evalúe sistemáticamente los resultados negativos de
la
asistencia. Se discute la necesidad de implementar un programa brasileño de
seguridad del
paciente, con el fin de promover la cultura por la seguridad del paciente y la
calidad de
la salud en el Sistema Único de Salud.

\ifdef{\kwdgrouptrans}{\kwdgrouptrans}{}
\end{abstract}
%%%%%%%%%%%%%%%%%%%%%%%%%%%%%%%%%%%%%%%%%%%%%%%%%%%%%%%%%%%
\section{INTRODUÇÃO}

Desde há muito tempo os resultados da assistência são utilizados para avaliação
da
qualidade dos serviços de saúde. Os babilônicos pagavam pelos serviços médicos
mediante os
resultados obtidos e, na Idade Média, os médicos que obtivessem resultados
negativos na
prestação de assistência tinham parte de seus corpos mutilados. 17

Os resultados negativos em saúde são conhecidos principalmente como eventos
adversos ou
qualquer tipo de incidente com potencial para causar danos aos pacientes 20 e
que pode fornecer importantes informações para a construção de um sistema de
saúde
mais seguro. 14 Os incidentes podem ser sem dano, com dano (evento adverso), ou
near
misses , também denominado de potencial evento adverso. 4,23

Resultados negativos em saúde foram relatados pelo Institute of Medicine
(IOM) em 1999, 11 que estimou entre 44.000 a 98.000 mortes por ano nos Estados
Unidos devido a erros
na assistência ao paciente. Desde então, os resultados ou desfechos em saúde têm
sido objeto
de estudo, pois estão relacionados diretamente à qualidade e à segurança do
paciente. A
segurança do paciente é definida como o ato de evitar, prevenir ou melhorar os
resultados
adversos ou as lesões originadas no processo de atendimento médico-hospitalar.
21

Diante da mobilização mundial após a publicação desse impactante relatório, a
Organização
Mundial da Saúde (OMS) lançou a Aliança Mundial para a Segurança do Paciente em
2004. a Isso despertou os países membros para o compromisso de desenvolver
políticas
públicas e práticas voltadas para a segurança do paciente, incluindo o Brasil.

Na Europa, estimou-se que 10,8\% dos pacientes hospitalizados foram acometidos
por eventos
adversos, dos quais 46\% poderiam ter sido prevenidos. 22 No Brasil, estudo
conduzido em hospitais do Rio de Janeiro estimou incidência de
7,6\% desses eventos. 13

Apesar de o Ministério da Saúde e a Agência Nacional de Vigilância Sanitária
(ANVISA)
promoverem iniciativas da Aliança Mundial para a Segurança do Paciente da OMS,
como a
campanha para introdução do protocolo de cirurgia segura nos hospitais, a adesão
por parte
dos serviços é baixa, justamente por não terem uma cultura institucional voltada
para a
segurança do paciente. Isso se reflete na alta ocorrência de eventos adversos
evitáveis em
hospitais brasileiros, que corresponde a cerca de 67\% de todos os eventos
adversos. 13,20

Embora o sistema de saúde brasileiro tenha aspectos positivos como a cobertura
universal de
vacinação e o sistema nacional de transplantes, a alta frequência de eventos
adversos
relacionados a medicamentos e infecções hospitalares é motivo de preocupação.
13,20 Esses eventos são atribuídos à falta de políticas governamentais b,c,d que
incentivem as instituições de saúde a participar de programas de qualidade e
acreditação. 15,16 Atualmente, há hospitais brasileiros que ainda são
prestadores de serviços que atuam
sem avaliar seus processos de trabalho ou usar seus resultados para a melhoria
contínua da
qualidade. d

Faz-se necessário, portanto, conhecer a realidade brasileira quanto à ocorrência
de
incidentes, o que pode ser obtido com o envolvimento das instituições de saúde
para que
monitorizem essa ocorrência e o tratamento das informações pertinentes, além de
notificá-las
aos órgãos governamentais. Entretanto, a simples existência de um fluxo de
informações
organizado não gera conhecimento por si só. Esse só se dará por meio da ação de
atores
interdisciplinares e que interajam entre si. 14

O objetivo deste estudo foi suscitar reflexão acerca da utilização de um sistema
nacional
de notificações sobre incidentes como base para um programa brasileiro de
segurança do
paciente.

\section{Qualidade no SUS e segurança do paciente}

Em 2011, o Ministério da Saúde lançou um Projeto de Formação e Melhoria da
Qualidade de
Rede de Atenção à Saúde, o QualiSUS Rede. e Apesar de ser um importante avanço
para o desenvolvimento da qualidade do Sistema
Único de Saúde (SUS), o projeto não contempla incentivo à adoção de um programa
de
acreditação hospitalar e também não contempla objetivo estratégico diretamente
relacionado à
segurança do paciente, item considerado essencial para a qualidade, segundo o
IOM e a
OMS.

Outra iniciativa do Ministério da Saúde é a monitorização do Índice de
Desempenho do SUS
(IDSUS), que tem como objetivo aferir o desempenho do sistema de saúde quanto ao
acesso –
potencial ou obtido – e à efetividade da atenção básica, das atenções
ambulatorial e
hospitalar, e de urgências e emergência na esfera nacional. f Essa aferição é
feita por meio de indicadores de qualidade.

Dentre os indicadores estabelecidos no IDSUS, não há nenhum relacionado
diretamente à
segurança do paciente, como a taxa de incidentes ocorridos no atendimento de
urgência e
emergência. Por outro lado, no IDSUS os indicadores são abordados como a
proporção de óbitos
nas internações por infarto agudo do miocárdio, que calcula o risco de morrer
por essa
condição após a internação por tal causa, e estima, indiretamente, o atraso do
atendimento
pré-hospitalar e no diagnóstico. d Ainda que sejam poucos, pode-se considerar um
avanço que esse tipo de indicador
esteja sendo utilizado em um programa oficial do governo brasileiro.

O projeto prevê repasse de verba diferenciado para aquelas regiões que atingirem
níveis
mais elevados de qualidade. Esse tipo de programa já é realizado com sucesso em
outros
países. Na Inglaterra e nos Estados Unidos, por exemplo, além de compartilhar os
indicadores
sobre segurança do paciente entre as instituições do país, com o objetivo de
conhecer e
estabelecer níveis de qualidade e segurança nas organizações hospitalares,
aquelas que
atingem níveis mais elevados são recompensadas com remuneração diferenciada. 3,9

Esse modelo de pagamento por qualidade é conhecido como pay for
performance (P4P) 7 e é alternativo àquele amplamente utilizado no Brasil, a
remuneração
fee-for-service (pagamento por serviço executado), que estimula a
sobreutilização de recursos, especialmente as tecnologias em saúde, e que não
traz garantias
de que o custo adicional e a facilidade de acesso resultem numa efetiva melhoria
da
qualidade do nível de saúde da população atendida. 7

O P4P está em desenvolvimento em muitos países, incluindo o Brasil. Na
Inglaterra, país
modelo para o uso dessa ferramenta, os pagamentos contribuem com 30\% da renda
de algumas
clínicas. 7,12 O que se espera do P4P é que os próprios usuários passem a
escolher o serviço pelo
qual desejam ser atendidos, com base em relatórios públicos de indicadores de
desempenho.
Constance et al 6 têm mostrado que a publicação desses relatórios é um bom
mecanismo para a melhoria
da qualidade na saúde.

O Brasil ainda tem como desafios a alta rotatividade de profissionais de saúde
nos serviços
públicos, além da limitação qualitativa dos recursos humanos, do uso indevido
das
tecnologias e da baixa continuidade da atenção prestada aos pacientes. 1,20
Ainda, pequeno número de hospitais brasileiros se dedica ao ensino e à pesquisa
e
não influencia a melhoria das práticas assistenciais em razão da desarticulação
entre
ensino, pesquisa e assistência, e da discreta utilização da saúde baseada em
evidências na
assistência ao paciente e das pesquisas sobre segurança do paciente que estão
restritas a
ilhas de excelência. 1,16,20

Diante do cenário exposto, no qual as políticas implementadas pelo Ministério da
Saúde não
têm sido suficientes para estimular o olhar crítico para a segurança do
paciente, com
estabelecimento de metas específicas para prevenir danos evitáveis e minimizar
riscos de
incidentes, propõe-se o desenvolvimento de um programa nacional de segurança do
paciente que
esteja vinculada aos programas de qualidade do governo federal. Tal programa
deve envolver,
no mínimo, o Ministério da Saúde, a ANVISA, a Agência Nacional de Saúde
Suplementar (ANS) e
o Ministério da Educação, sendo o último um importante aliado para a formação de
profissionais de saúde, especialmente nos hospitais de ensino.

O programa nacional de segurança do paciente faz-se necessário porque vem ao
encontro do
moderno conceito em saúde de prevenção quaternária, que objetiva a detecção de
indivíduos em
risco de intervencionismo excessivo em saúde, que implica atividades
desnecessárias, e
sugerir-lhes alternativas eticamente aceitáveis, atenuando ou evitando efeitos
adversos.
2,19

Essa abordagem é particularmente importante no Brasil, que teve o crescimento
exponencial
de novas tecnologias disponíveis no mercado de saúde na última década,
especialmente após a
criação da ANVISA, tem marco legal muito recente sobre incorporação de
tecnologias baseada
em evidências g e está investindo em um modelo humanizado e orientado para a
saúde. 1

O programa de segurança do paciente deve ser difundido nas diferentes
instituições que
compõem o sistema de saúde em todos os estados da federação a fim de que
conheçam e
compartilhem o conhecimento acerca dos resultados obtidos na assistência,
incluindo os
resultados negativos. Portanto, a implantação de um sistema nacional de
notificações de
incidentes deve ser uma das ações prioritárias de um programa nacional de
segurança do
paciente que contemple, minimamente, metas para gestão dos riscos envolvendo a
assistência à
saúde, tais como a identificação correta de pacientes, redução de infecções
hospitalares,
erros em procedimentos como cirurgias e medicação, que estão entre as chamadas
nove soluções
para a segurança do paciente, segundo a OMS. h

\section{Sistema de notificações de incidentesFluxo simplificado para o Sistema
Nacional de Notificações de Incidentes em
Saúde.}

As notificações por parte dos profissionais de saúde, pacientes e seus
cuidadores são
importantes para a identificação de incidentes em saúde, especialmente por ser
um método de
baixo custo e, principalmente, por envolver os profissionais que prestam
assistência em uma
política de melhoria contínua centrada no paciente.

Para garantir a produção de informação nas instituições de saúde para a tomada
de decisões
e a responsabilização com a melhoria de qualidade, é condição essencial que
sejam feitos
investimentos no desenvolvimento de capacidades locais e nos sistemas de
informação já
existentes. 10

A experiência da ANVISA com a Rede Sentinela é um bom exemplo de que a interação
entre
governo e instituições de saúde é possível e pode promover o desenvolvimento dos
serviços em
prol da segurança e da qualidade, seja pela cultura do relato voluntário, seja
pela adesão
aos programas de qualidade.

O número de notificações encaminhadas pelos hospitais integrantes da Rede
Sentinela ao
Sistema de Notificações em Vigilância Sanitária (NOTIVISA) aumentou em 48,8\%
após o primeiro
ano de implantação do sistema, quando comparado ao ano anterior. O estímulo da
ANVISA para
que os hospitais da Rede participassem de programas de qualidade, acreditação ou
similar
influenciou para que 30\% desses hospitais estivessem participando de algum
programa desse
tipo em 2008. c

Embora a iniciativa da ANVISA tenha sido importante para o estímulo da qualidade
nos
hospitais, ela é uma pequena parcela entre as mais de 8.000 instituições
hospitalares
brasileiras, correspondendo apenas a cerca de 13\% dos leitos hospitalares no
País. c

Por esse motivo, o papel dos órgãos governamentais que recebem as informações
sobre os
resultados em saúde é fundamental, cabendo a eles ações que promovam a melhoria
em curto
espaço de tempo a fim de evitar danos aos pacientes. A utilização de sistemas
informatizados
em plataforma web, ou seja, disponíveis na internet para envio e recebimento
imediatos, é
passo fundamental para que um país de grande extensão territorial como o Brasil
desenvolva
um programa nacional de segurança do paciente. Além disso, deve-se estabelecer
um modelo
brasileiro de pagamento por desempenho a fim de beneficiar as instituições que
estejam
comprometidas como o modelo de melhoria contínua da qualidade.

O modelo para o sistema nacional de notificações de incidentes pode ser útil no
desenvolvimento da cultura de segurança do paciente no SUS, conforme mostra a
Figura .

\begin{figure}
\caption{Fluxo simplificado para o Sistema Nacional de Notificações de
Incidentes em
Saúde.}
\end{figure}

O desenvolvimento e implementação de um sistema informatizado único para receber
notificações de todas as instituições de saúde deverá ter como objetivos
facilitar e
agilizar o processo de envio e de tomada de decisões a partir da notificação,
minimizando
riscos e evitando eventos adversos, ampliando a qualidade da assistência e a
segurança dos
pacientes em todos os níveis, da menor clínica de atenção à saúde ao sistema de
saúde
brasileiro; ampliar o conhecimento sobre os riscos e incidentes que ocorrem nas
instituições
brasileiras, direcionando o planejamento de ações dos gestores de saúde;
melhorar a
qualidade dos dados encaminhados; garantir a legibilidade das informações
disponíveis;
preservar a confidencialidade dos notificadores e dados relatados; e, por fim,
reduzir
custos do processo de notificação.

À medida que o sistema seja utilizado com maior frequência e eficácia para o
envio
voluntário de notificações de incidentes, especialmente os potenciais eventos
adversos, a
tomada de decisões quanto às intervenções necessárias para evitar a ocorrência
de danos
poderá ser mais rápida, o que pode reduzir gastos desnecessários com tratamento
de eventos
adversos que poderiam ter sido prevenidos.

Para tanto, a autonomia e pró-atividade das instituições de saúde deve ser
estimulada. Até
que haja a tomada de decisão por parte do governo, as instituições também devem
realizar
ações de melhoria internas, visando à promoção da segurança do paciente e à
qualidade da
atenção. Nesse sentindo, os estabelecimentos de saúde deverão ter acesso ao
sistema
informatizado não somente para o envio de notificações da gerência de riscos
para o sistema
nacional, mas também para que essa gerência se utilize do sistema para receber
as
notificações da equipe de saúde de sua instituição. Adicionalmente, o sistema
deve permitir
que a instituição acompanhe o andamento da análise das informações por ela
encaminhadas ao
sistema nacional.

O sistema informatizado é uma importante estratégia de promoção da qualidade
aliada à
sustentabilidade, pois, ao deixar de utilizar papéis, reduz gastos com materiais
de consumo
e geração de resíduos como os próprios papéis, cartuchos de impressoras e
canetas.
Adicionalmente, há outros aspectos que justificam a implantação de sistemas
informatizados
de notificação, a saber: 5

elimina a necessidade de utilização de sistemas de envio de documentos internos
às
instituições e destas para o sistema nacional de notificações, o que reduz o
tempo da
chegada da informação e reduz gastos com o envio delas;

pode-se eliminar a possibilidade de extravio e perda de informações,
especialmente se
estas forem preservadas em bancos de dados redundantes e cópias de segurança,
sem
necessidade de espaço para arquivo físico, além de permitir o manuseio ágil das
informações e a análise de indicadores de gestão;

é possível requerer mais informações sobre os incidentes sem dificultar a coleta
dos
dados, melhorando a qualidade das informações e ampliando a participação dos
profissionais de saúde, o que não é possível com o sistema manuscrito.

Quanto aos aspectos sociais da sustentabilidade, reduzir o tempo gasto para o
envio do
relato aumenta a participação dos profissionais de saúde com as notificações,
bem como sua
disponibilidade junto aos pacientes possibilitam o envolvimento do paciente e
seus
cuidadores no processo de monitorização de riscos e incidentes em saúde. Em uma
política
nacional, esses atores são fontes importantes de notificação voluntária. Com o
sistema
informatizado de notificações em plataforma web, é possível que qualquer pessoa
com acesso à
internet faça uma notificação.

Países que já possuem política nacional de segurança do paciente, como
Inglaterra, Estados
Unidos, Austrália e Canadá, já permitem que os usuários do sistema e seus
cuidadores façam
notificações sobre riscos e incidentes que vivenciaram ou perceberam em serviços
de saúde,
sendo fundamentais para promoção da qualidade da assistência.

O sistema nacional de notificações deveria, ainda, estar hospedado em um site
interativo
que disponibilizasse gratuitamente notícias, dicas de segurança, informações
sobre eventos
adversos, protocolos sobre como implementar um programa de segurança nos
serviços de saúde,
cursos e palestras online , como o Institute for Health
Improvement tem feito nos Estados Unidos para os hospitais americanos. Esse
portal teria como dupla função manter e estimular a adesão das instituições, e
difundir e
estimular a adoção de práticas seguras por meio do intercâmbio entre elas.

A implantação do sistema informatizado de notificações sobre incidentes na saúde
como base
para a cultura de segurança do paciente no sistema de saúde brasileiro parece
ser uma
estratégia viável e necessária para a qualificação da assistência, com a qual os
gestores
conhecerão os incidentes que ocorrem na prestação de assistência aos usuários do
sistema, em
instituições públicas e privadas, de forma sistematizada, sem depender de que
pesquisas
sejam realizadas exclusivamente para esse fim. Desse modo, nortear-se-á o
delineamento de
estratégias de gestão de riscos para a segurança do paciente, ampliando a
qualidade dos
serviços ofertados à população brasileira.

\end{document}
